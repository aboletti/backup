\chapter{Analysis introduction}
\label{sec:theo}

In this thesis, I will present the angular analysis of the \BtoKstmumu decay, performed with the data collected by the Compact Muon Solenoid (CMS) Experiment in the 2012 run of proton-proton (pp) collisions at a centre-of-mass energy $\sqrt{s}=8\TeV$, and corresponding to an integrated luminosity of \SI{20.5}{\per\femto\barn}.

Two analyses of the \BtoKstmumu decay have been performed by the CMS Collaboration, using the data described above.
The first analysis~\cite{CMS:2012} was aiming to measure the branching fraction and to perform a partial measurement of the angular distribution.
The \Ks longitudinal polarisation fraction, $F_L$, and the forward-backward asymmetry of the dimuon system, $A_{FB}$, were estimated as a function of the dimuon invariant mass squared, $q^2$.
A combination with the results obtained with pp collision data collected in 2011, corresponding to an integrated luminosity of \SI{5.2}{\per\femto\barn} at $\sqrt{s}=7\TeV$, was also performed.
The results of this analysis are in good agreement with the Standard Model (SM) predictions and with the results from previous experiments: BaBar~\cite{Aubert:2008bi}, Belle~\cite{Wei:2009zv}, CDF~\cite{Aaltonen:2011ja}, and LHCb~\cite{Aaij:2013iag}.

The first complete angular analysis was performed by the LHCb Collaboration~\cite{Aaij:2013qta}, with the data collected in 2011 pp collisions at $\sqrt{s}=7\TeV$.
In this analysis the measurement has been extended to the full set of angular parameters, which will be defined in Section~\ref{sec:decRate}.
A tension with respect to the SM predictions has been found in the measurement of one of these parameters, $P_5'$.
The significance of this tension, without taking into account the look-elsewhere effect, is reported to be 3.7 standard deviations.

The LHCb collaboration published also a complete angular analysis~\cite{Aaij:2015oid}, combining the data collected in 2011 and 2012 pp collisions for an overall integrated luminosity of \SI{3}{\per\femto\barn}.
The results for the $P_5'$ parameter were still showing a discrepancy with respect to the predictions, as shown in Figure~\ref{fig:LHCb}.
The tension is localised in the $q^2$ region between $4\GeV^2$ and $8\GeV^2$, and its significance is evaluated, using the full set of parameters, to be of about 3.4 standard deviations.

\begin{figure}[!hbt]
  \centering
  \includegraphics[width=0.85\textwidth]{PBasis_P5pPad.pdf}
  \caption{Results of $P_5'$ angular parameter in bins of $q^2$ from LHCb Run~1 data, compared with SM-based predictions.
    Image from Reference~\cite{Aaij:2015oid}.}
  \label{fig:LHCb}
\end{figure}

After these results, the Belle Collaboration performed an extension~\cite{BelleP5p} of its original analysis, to measure the full set of angular parameters, in particular $P_5'$.
The results are in agreement with the measurements of LHCb, but with a larger uncertainty that prevent to obtain a significant discrepancy of the $P_5'$ results with respect to the SM predictions.

The second analysis from the CMS Collaboration~\cite{Sirunyan:2017dhj} is an extension of the first analysis, with the goal of measuring the $P_5'$ angular parameter to provide a new independent experimental result.
The expected precision of this measurement is not as good as the latest result of the LHCb Collaboration, given the lower number of signal events and worse signal over background ratio.
Anyway, it will have a smaller uncertainty and a $q^2$ bin structure more similar to the LHCb one, with respect to the result from Belle Collaboration.
For this reason, these results could have an important role in the process of shedding light on the matter.

In this thesis I will describes the details of this second analysis of the CMS Collaboration.
Chapter~\ref{sec:theo} will continue with the description of the theoretical framework used to described the decays like \BtoKstmumu, and with the formulation of the angular distribution and few considerations about it.
In Chapter~\ref{sec:detect} the Large Hadron Collider (LHC) and the CMS experiment will be described, focusing in particular on the aspects more related to the analyses of Flavour Physics.
The selection criteria used to reduce the contamination from background events will be presented in Chapter~\ref{sec:selection}, while in Chapter~\ref{sec:fit} the procedure used to extract the angular parameters from the selected data will be explained in details.
Chapter~\ref{sec:eff} will be dedicated to the methods used to describe the efficiency of the selection criteria on signal events.
Several tests have been performed to validate the analysis procedure; they will be described in Chapter~\ref{sec:validation}.
The sources of systematic uncertainty that have been investigated will be listed in Chapter~\ref{sec:syst}, while the study of the statistical uncertainty and the results of the analysis will be reported in Chapter~\ref{sec:result}.
Finally, I will conclude in Chapter~\ref{sec:FutConcl} by describing the future prospects for this analysis and focusing in particular on the ongoing efforts to improve it for the data from LHC Run~2 collisions.

\section{Flavour Physics}
\label{sec:flav}

The sector of the Standard Model of particle physics in which the precision of the experimental probes reached higher levels is the Flavour Physics.

In particular, it is the realm of the Cabibbo-Kobayashi-Maskawa (CKM) matrix, i.e. the $3\times3$ unitary matrix that rules the charge current weak interactions between quarks.
This matrix was proposed by Kobayashi and Maskawa about 45 years ago, extending the Cabibbo $2\times2$ matrix to allow a source of CP-violation to be included in the model through a non-vanishing phase (that is not possible in a $2\times2$ unitary matrix).
It allowed to foresee the existence of the three heavier quarks before discovering them in the experimental facilities.
Its parameters, that link the basis of weak eigenstates with the one of Yukawa eigenstates, have been determined in the past decades with great precision.

The unitarity of the CKM matrix is the real testing ground of the completeness of the theory.
An observation of non-unitary behaviours in the CKM parameters would open the gates to scenarios beyond the SM, suggesting the presence of a forth generation of quarks.
However, after many years of B-factory probes, and few years of analyses at LHC, no cracks have been found in the paradigm.
This robust long-standing structure is not promising for searches of phenomena of New Physics (NP).

On the other hand, few recent results from the LHCb Collaboration raised the interest for the world of the so-called rare decays.
One result have been already mentioned before: the tension of the $P_5'$ parameter measured in the angular analysis of the flavour-changing neutral current (FCNC) \BtoKstmumu decay.
Another, very interesting, result that is showing a discrepancy with respect to the predictions is the measurement~\cite{Aaij:2017vbb} of $R(\mathrm{K}^{(*)})$.
These parameters are defined as the branching-fraction ratios:
\begin{equation}\label{eq:RK-def}
  \begin{split}
    R(\mathrm{K}^{*}) &= \frac{\BR(\PBz\to\Ks\mu^+\mu^-)}{\BR(\PBz\to\Ks e^+e^-)} \\
    R(\mathrm{K})     &= \frac{\BR(\PBp\to\PKp\mu^+\mu^-)}{\BR(\PBp\to\PKp e^+e^-)}
  \end{split}
\end{equation}
expressed as a function of the dimuon invariant mass squared, $q^2$.

In the latest LHCb results for both these ratios, a hint for tension with respect to the SM is present.
Even if this discrepancy is less significant than the one on $P_5'$, the theoretical prediction for this variables are not affected at all by hadronic uncertainties, and it makes them very robust.
One very interesting aspect of these discrepancies is that they could be related to a common cause, affecting one of the coefficients of the effective Hamiltonian, the so-called Wilson coefficients.

In the following section, a brief description of the theoretical framework used to described these FCNC decays is given.

\subsection[Flavour-changing neutral currents b to s l l]{Flavour-changing neutral currents $b\to s\ell^+\ell^-$}
\label{sec:FCNC}

The penguin-mediated FCNC $b\to s\ell^+\ell^-$ are a set of $b$-hadron semileptonic decays with a pair of non-resonant charged leptons in the final state.
These decay channels are doubly suppressed within the SM:
\begin{itemize}
\item they are forbidden at tree-level, since there is no neutral current in SM allowed to violate the flavour, and the leading-order Feynman diagram that mediates them is a weak penguin loop;
\item the leading-order diagram is Cabibbo suppressed, since it is proportional to the CKM elements $|V_{ts}V_{tb}|\sim10^{-2}$.
\end{itemize}
These causes of suppression result in small branching fractions and they create an ideal environment for NP searches.
Any potential contribution from physics beyond the SM that enters in the loop diagram could produce sizeable effects in the decay branching fractions or in their angular distributions.

On the other hand, the theoretical predictions of the angular distributions and branching fractions of these decays are quite susceptible to hadronic uncertainties due to long-distance Quantum Chromodynamic (QCD) processes.
In order to separate, in the theoretical framework, the effect from long-distance QCD and from short-distance effects, which are related to QCD and electroweak interactions but also to NP processes, an effective Hamiltonian is defined as follow:
\begin{equation} \label{eq:Heff}
  {\cal H}_{\eff} = - \frac{4\,G_F}{\sqrt{2}}\left(\lambda_t {\cal H}_{\eff}^{(t)} + \lambda_u {\cal H}_{\eff}^{(u)}\right)
\end{equation}
where $\lambda_i=V_{ib}V_{is}^*$ and
\begin{eqnarray*}
  {\cal H}_{\eff}^{(t)} & = & C_1 \mathcal O_1^c + C_2 \mathcal O_2^c + \sum_{i=3}^{6} C_i \mathcal O_i + \sum_{i=7,8,9,10,P,S} (C_i \mathcal O_i + C'_i \mathcal O'_i)\,, \\
  {\cal H}_{\eff}^{(u)} & = & C_1 (\mathcal O_1^c-\mathcal O_1^u)  + C_2(\mathcal O_2^c-\mathcal O_2^u)\,.
\end{eqnarray*}

The operators $\mathcal O_i^{(a)}$ for $i<7$ are equal to the $P_i$ defined in the Reference~\cite{Bobeth:1999mk}, while for $i\geq7$ they are defined accordingly to Reference~\cite{Altmannshofer:2008dz}:
\begin{align}
  {\mathcal{O}}_{7} &= \frac{e}{g^2} m_b (\bar{s} \sigma_{\mu \nu} P_R b) F^{\mu \nu} &
  {\mathcal{O}}_{7}^\prime &= \frac{e}{g^2} m_b (\bar{s} \sigma_{\mu \nu} P_L b) F^{\mu \nu} \label{eq:O7}\\
  {\mathcal{O}}_{8} &= \frac{1}{g} m_b (\bar{s} \sigma_{\mu \nu} T^a P_R b) G^{\mu \nu \, a} &
  {\mathcal{O}}_{8}^\prime &= \frac{1}{g} m_b (\bar{s} \sigma_{\mu \nu} T^a P_L b) G^{\mu \nu \, a} \label{eq:O8}\\
  {\mathcal{O}}_{9} &= \frac{e^2}{g^2} (\bar{s} \gamma_{\mu} P_L b)(\bar{\mu} \gamma^\mu \mu) &
  {\mathcal{O}}_{9}^\prime &= \frac{e^2}{g^2} (\bar{s} \gamma_{\mu} P_R b)(\bar{\mu} \gamma^\mu \mu) \label{eq:O9}\\
  {\mathcal{O}}_{10} &=\frac{e^2}{g^2} (\bar{s}  \gamma_{\mu} P_L b)(\bar{\mu} \gamma^\mu \gamma_5 \mu) &
  {\mathcal{O}}_{10}^\prime &=\frac{e^2}{g^2} (\bar{s}  \gamma_{\mu} P_R b)(\bar{\mu} \gamma^\mu \gamma_5 \mu) \label{eq:O10}\\
  {\mathcal{O}}_{S} &=\frac{e^2}{16\pi^2} m_b (\bar{s} P_R b)(\bar{\mu} \mu) &
  {\mathcal{O}}_{S}^\prime &=\frac{e^2}{16\pi^2} m_b (\bar{s} P_L b)(\bar{\mu} \mu) \label{eq:OS}\\
  {\mathcal{O}}_{P} &=\frac{e^2}{16\pi^2} m_b (\bar{s} P_R b)(\bar{\mu} \gamma_5 \mu) &
  {\mathcal{O}}_{P}^\prime &=\frac{e^2}{16\pi^2} m_b (\bar{s} P_L b)(\bar{\mu} \gamma_5 \mu) \label{eq:OP}
\end{align}
where $g$ is the strong coupling constant, $m_b$ is the running $b$ quark mass in the $\overline{\mathrm{MS}}$ scheme, and $P_{L,R}=(1\mp\gamma_5)/2$ are the chirality projectors.

The set of Wilson coefficients $C^{(\prime)}_i$ encodes the contribution from short-distance physics and could contain NP effects.
These coefficients can be expanded and calculated in perturbation theory in powers of $\alpha_s\equiv g^2/4\pi$:
\begin{equation}\label{WilExpansion}
  C_i = C_i^{(0)} + \frac{\alpha_s}{4\pi}\, C_i^{(1)} + \left(\frac{\alpha_s}{4\pi}\right)^2 C_i^{(2)} + O(\alpha_s^3)\,,
\end{equation}
Their predicted values of these expanded coefficients, both within SM and for NP scenarios, are calculated at the scale $\mu=m_W$ and then evolved down to $\mu\sim m_b$.

According to the SM, several of the terms in the effective Hamiltonian are expected to vanish or be highly suppressed; this is the case for all the primed operators, as well as for the scalar and pseudoscalar ones, ${\mathcal{O}}_{S}$ and ${\mathcal{O}}_{P}$.

%%%%%%%%%%%%%%%%%%%%
If we restrict the study to a $\mathrm{B}^{(0,\pm)}\to\mathrm{K}^{*(0,\pm)}\ell^+\ell^-$ decay, the matrix elements of the effective Hamiltonian operators ${\mathcal{O}}^{(\prime)}_{7,9,10,S,P}$ can be described as a function of seven form factors, $A_i(q^2)$, $V(q^2)$, and $T_j(q^2)$, where $0\le i\le2$ and $1\le j\le2$, and $q^2$ is the squared momentum carried by the pair of leptons.
In literature the form factor $A_3(q^2)$, which is a linear combination of $A_1(q^2)$ and $A_2(q^2)$, is also used.

The theoretical predictions of the form factors are computed using the QCD sum rules on the light-cone (LCSRs)~\cite{Khodjamirian:2001bj}.
With this technique, the form factors are expanded in powers of $m_b m_{\mathrm{K}^*}/(m_b^2-q^2)$.
This expansion works well for values of $q^2$ lower than about $8\,\mathrm{GeV}^2$, but for higher values the high-order terms become more relevant and the uncertainty associated to the approximated prediction grows accordingly.

%%%%%%%%%%%%%%%%%%
In the $\mathrm{B}^{(0,\pm)}\to\mathrm{K}^{*(0,\pm)}\ell^+\ell^-$ decay amplitude, not only the terms proportional to the form factors are contributing.
It contains additional ``non-factorisable effects'', which are related to the matrix elements of the operators not considered when defining the form factors: the purely hadronic operators ${\mathcal{O}}_{1,2,3,4,5,6}$ and the chromomagnetic operators ${\mathcal{O}}^{(\prime)}_{8}$, where an additional virtual photon emission is needed to produce the lepton pair.

The contribution of these effects cannot be expressed as a function of a new set of form factors.
In the combined heavy-quark and large-energy limits, they can be calculated using the QCD factorisation technique~\cite{Beneke:2000wa,Beneke:2001at,Beneke:2004dp}.

%%%%%%%%%%%%%%%%%%%%

\section[The decay B0 to K*0 mu+ mu-]{The decay $\mathrm{B}^0_d \to \mathrm{K}^{*0} \mu^+ \mu^-$}
\label{sec:Kstmm}

The FCNC decay \BtoKstmumudecay is an optimal laboratory to probe the flavour sector of the SM.
From the experimental point of view, it has a fully charged final state, with two muons, which are easy to identify in a multi-purpose particle detector, and two charged hadronic particles.
Furthermore, the charges of the hadrons in the final state determine the CP-state of the decay, i.e. whether it is \BtoKstmumudecay or \BtoKstmumuconjdecay.
From the phenomenological point of view, the angular analysis of this decay can provide information on several components of the effective Hamiltonian described in Section~\ref{sec:FCNC}: the electromagnetic and semileptonic operators, $\mathcal{O}_{7,9,10}$, and their chirality-flipped counterparts, $\mathcal{O}_{7^\prime,9^\prime,10^\prime}$, together with scalar and pseudoscalar operators, $\mathcal{O}_{S,P,S^\prime,P^\prime}$.

%%%%%%%%%%%%%%%%%%%%%%%

The decay in the four-body final state is completely described by a set of four kinematical variables, \TL, \TK, \PHI, and $q^2$.
The variable \TL is defined as the angle between the momentum of the \Pgmp (\Pgmm) and the direction opposite to the \PBz (\PaBz) momentum, in the dimuon rest frame:
\begin{equation} \label{eq:TLdef}
  \begin{aligned}
    \cTL & = \left( p_{\Pgmp}^{(\Pgmp\Pgmm)} \right) \cdot \left( -p_{\PBz }^{(\Pgmp\Pgmm)} \right)\, && \mathrm{for}\:\PBz\:\mathrm{decay} \\
    \cTL & = \left( p_{\Pgmm}^{(\Pgmp\Pgmm)} \right) \cdot \left( -p_{\PaBz}^{(\Pgmp\Pgmm)} \right)\, && \mathrm{for}\:\PaBz\:\mathrm{decay} \\
  \end{aligned}
\end{equation}
%% \begin{alignat}{3} \label{eq:TLdef}
%%   \cTL = & \left( p_{\Pgmp}^{(\Pgmp\Pgmm)} \right) \cdot \left( -p_{\PBz }^{(\Pgmp\Pgmm)} \right)\, & \mathrm{for \PBz decay;} \\
%%   \cTL = & \left( p_{\Pgmm}^{(\Pgmp\Pgmm)} \right) \cdot \left( -p_{\PaBz}^{(\Pgmp\Pgmm)} \right)\, & \mathrm{for \PaBz decay.}
%% \end{alignat}
The variable \TK is defined as the angle between the direction of the kaon and the direction opposite that of the \PBz (\PaBz), in the \cPKstz (\cPAKstz) rest frame
\begin{equation} \label{eq:TKdef}
  \begin{aligned}
    \cTK&=\left( p_{\PKp}^{(\cPKstz) } \right) \cdot \left( -p_{\PBz }^{(\cPKstz) } \right)\, && \mathrm{for}\:\PBz\:\mathrm{decay;} \\
    \cTK&=\left( p_{\PKm}^{(\cPAKstz)} \right) \cdot \left( -p_{\PaBz}^{(\cPAKstz)} \right)\, && \mathrm{for}\:\PaBz\:\mathrm{decay.}
  \end{aligned}
\end{equation}
The variable \PHI is defined as the angle between the plane containing the \Pgmp and \Pgmm momenta and the plane containing the momenta of the kaon and the pion, in the \PBz rest frame, with the following sign conventions:
\begin{equation} \label{eq:PHIdef}
  \begin{aligned}
    \cos(\PHI) &= \left( p_{\Pgmp}^{(\PBz)} \times p_{\Pgmm}^{(\PBz)} \right) \cdot \left( p_{\PKp}^{(\PBz)} \times p_{\Pgpm}^{(\PBz)} \right) \, && \mathrm{and} \\
    \sin(\PHI) &= \left[ \left( p_{\Pgmp}^{(\PBz)} \times p_{\Pgmm}^{(\PBz)} \right) \times \left( p_{\PKp}^{(\PBz)} \times p_{\Pgpm}^{(\PBz)} \right) \right] \cdot p_{\cPKstz}^{(\PBz)}\, && \mathrm{for}\:\PBz\:\mathrm{decay;} \\[8pt]
    \cos(\PHI) &= \left( p_{\Pgmm}^{(\PaBz)} \times p_{\Pgmp}^{(\PaBz)} \right) \cdot \left( p_{\PKm}^{(\PaBz)} \times p_{\Pgpp}^{(\PaBz)} \right) \, && \mathrm{and} \\
    \sin(\PHI) &=-\left[ \left( p_{\Pgmm}^{(\PaBz)} \times p_{\Pgmp}^{(\PaBz)} \right) \times \left( p_{\PKm}^{(\PaBz)} \times p_{\Pgpp}^{(\PaBz)} \right) \right] \cdot p_{\cPAKstz}^{(\PaBz)}\, && \mathrm{for}\:\PaBz\:\mathrm{decay.} \\
  \end{aligned}
\end{equation}
Finally, the variable $q^2$ is defined as the invariant mass squared of the dimuon system.
In the Equations~\ref{eq:TLdef},~\ref{eq:TKdef},~\ref{eq:PHIdef} the notation $p_a^{(b)}$ indicates the momentum of the particle $a$ in the rest frame of the particle $b$.

The definition of the kinematical variables is coherent with the one used in the previous experimental analyses performed on this decay channel.
The definition of \TL differs from the one used in some phenomenological papers.
A graphic representation of the definition of the three angular variables is shown in Figure~\ref{fig:ske}.

\begin{figure*}[tb]
  \begin{center}
    \includegraphics[width=0.99\textwidth]{SketchDecay.pdf}
    \caption{Illustration of the angular variables \TL (left), \TK (middle), and \PHI (right) for the decay \BtoKstmumudecay.}
    \label{fig:ske}
  \end{center}
\end{figure*}

The range of definition of the angular variables are $\left[0,\pi\right]$, for \TL and \TK, and $\left[-\pi,\pi\right]$, for \PHI.

%%%%%%%%%%%%%%%%%%%%%%%%%555

\subsection{The angular decay rate}
\label{sec:decRate}

The matrix element for the \BtoKstmumudecay, obtained from the effective Hamiltonian in Equation~\ref{eq:Heff}, is:
\begin{equation}\label{eq:matrixelement}
  \begin{split}
    {\mathcal M}\ =& \frac{G_F\alpha}{\sqrt{2}\pi}V_{tb}^{}V_{ts}^*\bigg\{\bigg[ \braket{K \pi|\bar{s}\gamma^{\mu}({C_9^\text{eff}P_L+C_9^{\prime\text{eff}} P_R})b|\bar B} \\
      &-\frac{2m_b}{q^2}\braket{K\pi|\bar{s}i\sigma^{\mu\nu}q_{\nu}(C_7^\text{eff} P_R+C_7^{\prime\text{eff}}  P_L)b|\bar B}\bigg](\bar{\mu}\gamma_{\mu}\mu)\\
    &+ \braket{K\pi|\bar{s}\gamma^{\mu}({C_{10}^\text{eff} P_L+C_{10}^{\prime\text{eff}} P_R})b|\bar B}(\bar{\mu}\gamma_{\mu}\gamma_5 \mu) \\
    & {+\braket{K \pi|\bar{s} ({C_S P_R+C_S^\prime P_L}) b|\bar B}(\bar{\mu}\mu)}{+\braket{K \pi|\bar{s} ({C_P P_R+C_P^\prime P_L}) b|\bar B}(\bar{\mu}\gamma_5\mu)} \bigg \}.
  \end{split}
\end{equation}
where the naive factorisation has been applied, by ignoring the non-factorisable terms, and a set of effective Wilson coefficients, $C_{i}^\text{eff}$, is used.
These effective coefficients are linear combinations of the set defined in Section~\ref{sec:FCNC}, and their definition is described in Reference~\cite{Altmannshofer:2008dz}.

The differential decay distribution as a function of the kinematical variables can be obtained by squaring Equation~\ref{eq:matrixelement}.
Projecting the result on a basis of combinations of spherical harmonics of the angular variables, one obtains the following expression for the \BtoKstmumuconjdecay decay:
\begin{small}
  \begin{align} \label{eq:angulardist}
    \frac{d^4\Gamma}{dq^2\, d\cTL\, d\cTK\, d\PHI} &=\frac{9}{32\pi}\bigg[ I_1^s\sin^2\TK + I_1^c\cos^2\TK \nonumber\\
      & + (I_2^s\sin^2\TK + I_2^c\cos^2\TK)\cos2\TL + I_3 \sin^2\TK \sin^2\TL \cos 2\PHI \nonumber \\
      & + I_4 \sin 2\TK \sin 2\TL \cos\PHI + I_5 \sin 2\TK \sin\TL \cos\PHI \nonumber \\
      & + (I_6^s \sin^2\TK + {I_6^c \cos^2\TK})  \cos\TL + I_7 \sin 2\TK \sin\TL \sin\PHI \nonumber \\
    & + I_8 \sin 2\TK \sin 2\TL \sin\PHI + I_9 \sin^2\TK \sin^2\TL \sin 2\PHI \bigg]
  \end{align}
\end{small}
where the complex angular coefficients $I_i^{(a)}$ depend only on $q^2$.
The differential distribution of the opposite CP-state decay, \BtoKstmumudecay, has the same expression, but it is function of the weak-phase conjugated coefficients $\bar{I}_i^{(a)}$.

The set of angular coefficients $I_i^{(a)}$ has been expressed as a function of the effective Wilson coefficients and of the form factors, under the assumptions of naive factorisation of the matrix element and adding corrections to describe the effect of non-factorisable terms.
In literature they are also expressed as a function of eight \Ks transversity amplitudes.

Unlike Wilson coefficients, form factors, and transversity amplitudes, the angular coefficients $I_i^{(a)}$ are physical observable and they can be experimentally measured by the angular analysis of the \BtoKstmumudecay decay.
However, the theoretical predictions of these coefficients are prone to hadronic uncertainties, derived from their strong dependency on the form factors.
For this reasons, more sophisticated bases of angular coefficients have been defined in order to be independent from form factors at leading order of the effective-theory expansion.

Firstly, two sets of coefficients are defined, the CP-averages:
\begin{equation}
  S^{(a)}_i = \frac{ I^{(a)}_i + \bar I^{(a)}_i }{\frac{d\Gamma}{dq^2} + \frac{d\bar\Gamma}{dq^2}}
  \label{eq:Ss}
\end{equation}
and the CP asymmetries:
\begin{equation}
  A^{(a)}_i = \frac{ I^{(a)}_i - \bar I^{(a)}_i }{\frac{d\Gamma}{dq^2}+\frac{d\bar\Gamma}{dq^2}}
  \label{eq:As}
\end{equation}
that are useful to disentangle potential NP effects that introduce new sources of CP-violation from the others.
Since this analysis is aiming the measurement of two CP-averaged angular parameters, I will not spend time describing the basis of CP-violating coefficients but I will focus on the set of $S^{(a)}_i$ parameters.

The twelve real angular coefficients in this set can be reduced to eight, in the approximation of negligible lepton mass with respect to $q$.
This is true in this analysis, since only candidates with $q$ value greater than 1\GeV are used, as described in Section~\ref{sec:q2}.
In the massless-muon limit, the following conditions are valid:
\begin{gather}
  S_1^s = 3 S_2^s \nonumber\\
  S_1^c = - S_2^c \nonumber\\
  \frac{3}{4}\left(2S_1^s+S_1^c\right)-\frac{1}{4}\left(2S_2^s+S_2^c\right) = 1 \nonumber\\
  S_6^c = 0 \label{eq:massless-reductions}
\end{gather}
where the latter condition is always true within the SM, even without the massless-muon limit, since the $S_6^c$ term is generated by the scalar operator, which does not exist in the SM.

The $S_1^c$ coefficient corresponds to the fraction of \Ks produced with longitudinal polarisation, and it is usually referred to as $F_L$.
In the same way, the coefficient $S_6^s$ is proportional to the forward-backward asymmetry of the muon system, thus the parameter $A_{FB}=\frac{3}{4}S_6^s$ is used.

The so-called $P$-primed basis of angular parameters, clean from leading-order hadronic uncertainties, is defined as:
\begin{equation}\label{eq:defPPrime}
  \begin{split}
    P_1 &= \frac{2S_3}{1-F_L}\\
    P_2 &= \frac{2}{3}\frac{A_{FB}}{1-F_L}\\
    P_3 &= \frac{-S_9}{1-F_L}\\
    P'_{4,5,8} &= \frac{S_{4,5,8}}{\sqrt{F_L(1-F_L)}}\\
    P'_6 &= \frac{S_7}{\sqrt{F_L(1-F_L)}}\\
  \end{split}
\end{equation}

Using the $P_i^{(\prime)}$ basis, with $F_L$ and its complementary $F_T=1-F_L$, the differential angular distribution can be written as:
\begin{equation} \label{eq:Angular}
  \begin{split}
    \frac{1}{\mathrm{d}\Gamma/\mathrm{d}q^2}&\frac{\mathrm{d}^4\Gamma}{\mathrm{d}q^2 \mathrm{d}\cos\theta_l \mathrm{d}\cos\theta_\mathrm{K} \mathrm{d}\phi} =\frac{9}{32\pi}\bigg[\frac{3}{4}F_\mathrm{T}\sin^2\theta_\mathrm{K} + F_\mathrm{L}\cos^2\theta_\mathrm{K} \\
      &\left.+\left(\frac{1}{4}F_\mathrm{T}\sin^2\theta_\mathrm{K}-F_\mathrm{L}\cos^2\theta_\mathrm{K}\right)\cos2\theta_l+\frac{1}{2}P_1F_\mathrm{T}\sin^2\theta_\mathrm{K}\sin^2\theta_l\cos 2\phi \right.\\
      &+\sqrt{F_\mathrm{T}F_\mathrm{L}}\left(\frac{1}{2}P_4'\sin2\theta_\mathrm{K}\sin2\theta_l\cos\phi+P_5'\sin2\theta_\mathrm{K}\sin\theta_l\cos\phi \right)\\
      &-\sqrt{F_\mathrm{T}F_\mathrm{L}}\left(P_6'\sin2\theta_\mathrm{K}\sin\theta_l\sin\phi-\frac{1}{2}P_8'\sin2\theta_\mathrm{K}\sin2\theta_l\sin\phi \right)\\
      &+2P_2F_\mathrm{T}\sin^2\theta_\mathrm{K}\cos\theta_l-P_3F_\mathrm{T}\sin^2\theta_\mathrm{K}\sin^2\theta_l\sin2\phi \bigg]
  \end{split}
\end{equation}

\subsubsection{S-wave contamination}
\label{sec:S-waveform}

Although the $K^+\pi^-$ invariant mass must be consistent with a $\text{K}^{*0}$, there can be contributions from a spinless (S-wave) $K^+\pi^-$ combination.
The presence of a $K^+\pi^-$ system in an S-wave configuration, due to a non-resonant contribution or to feed through from $K^+\pi^-$ scalar resonances, results in additional terms in the different decay rate.

Denoting the right-hand side of Equation~\ref{eq:Angular} by $W_p$, the differential decay rate takes the form:
\begin{equation} \label{eq:S-wave}
  \frac{1}{\mathrm{d}\Gamma/\mathrm{d}q^2}\frac{\mathrm{d}^4\Gamma}{\mathrm{d}q^2 \mathrm{d}\cos\theta_l \mathrm{d}\cos\theta_\mathrm{K} \mathrm{d}\phi}\bigg|_{S+P} = (1-F_\mathrm{S})W_p + (W_s + W_{sp})
\end{equation}
where 
\begin{equation} \label{eq:S-wave0}
  W_s = \frac{3}{16\pi} F_\mathrm{S}\sin^2\theta_l
\end{equation}
%% and $W_{sp}$ is given from Eq.(44) in~\cite{Genon:Swave}. 
and $W_{sp}$ is:
\begin{equation} \label{eq:S-wave1}
  \begin{split}
    W_{sp}=\frac{3}{16 \pi}\left[&A_\mathrm{S}\sin^2\theta_l\cos\theta_\mathrm{K}+ A_\mathrm{S}^4\sin\theta_\mathrm{K}\sin2\theta_l\cos\phi+A^5_\mathrm{S}\sin\theta_\mathrm{K}\sin\theta_l\cos\phi\right.\\
      &+\left.A_\mathrm{S}^7\sin\theta_\mathrm{K}\sin\theta_l\sin\phi+A_\mathrm{S}^8\sin\theta_\mathrm{K}\sin2\theta_l\sin\phi\right]
  \end{split}
\end{equation}
where $F_\mathrm{S}$ is the fraction of the S-wave component in the $\text{K}^{*0}$ mass window, and $A_\mathrm{S}^i$ are the interference amplitudes between the S-wave and the P-wave decays\cite{Genon:Swave}.
%%%%%%%%%%%%%%%%%%%%%%%%%%%%%%%%%%

\subsection{The angular folding}
\label{sec:folding}

%% Because of the limited number of signal candidates in the data set, we didn't fit the data to full differential distribution of Equation~\ref{eq:Angular}.
%% To retrieve the interesting variables more effectively and to reduce the number of fitting parameters, we performed the following transformations of the decay-rate formulation.

The differential decay distributions, described above, contains 8 parameters for the P-wave component, and 6 parameters for the S-wave and interference contamination.
As mentioned in the introduction, the primary goal of this angular analysis is to measure the angular parameter $P_5'$.
In order to perform this measurement in the best possible environment, it has been decided to simplify the angular analysis by reducing the number of measured parameters.
This is done with the \textit{angular folding} method.
%% , as already used and explained in References~\cite{LHCb2,Matias2012}.

In the following, I will use the expression ``folding a variable $x$ around $a$'' referring to the redefinition of that variable in the following way:
\begin{equation} \label{eq:folding-def}
  \begin{aligned}
    & x \to x \quad & \textrm{for}\: x \ge a \\ 
    & x \to 2a-x \quad & \textrm{for}\: x < a \\ 
  \end{aligned}
\end{equation}

The key of this method is to exploit the even and odd symmetries of the terms forming the differential distribution, around some values of the angular variables.
By folding a variable around one of these values, all the terms with an even symmetry will be unchanged after the transformation, while all the terms with an odd symmetry will be become equal to zero.
Any folding applied around a value for which there is no symmetry, in at least one term of the distribution, would modify the expression of that term in a non-trivial way.
In this analysis, only foldings corresponding to symmetries of the angular decay rate are applied.

In order to preserve the term proportional to $P_5'$ in the angular distribution and to reduce as much as possible the total number of parameters, two foldings have been applied.
The \PHI variable has been folded around 0, reducing its range from $\left[-\pi,\pi\right]$ to $\left[0,\pi\right]$.
Then, the \TL variable has been folded around $\pi/2$, reducing its range from $\left[0,\pi\right]$ to $\left[0,\pi/2\right]$.

Note that, according to the definition in Equation~\ref{eq:folding-def}, the resulting range of \TL should have been $\left[\pi/2,\pi\right]$.
However, for convenience this second folding has been applied in the other way:
\begin{equation} \label{eq:folding-def-alt}
  \begin{aligned}
    & x \to 2a-x \quad & \textrm{for}\: x \ge a \\
    & x \to x \quad & \textrm{for}\: x < a \\
  \end{aligned}
\end{equation}
The effect of this alternative definition on the differential distribution is identical to the original one.

%% , to reduce the number of parameters in the fit, we "fold" the data twice.
%% ``Folding'' means that we divide the decay rate into different parts, calculate them separately according to some symmetries and then add them together to obtain the equivalent decay rates.
%% If we take consecutive steps of ``folding'', the similar expansions are used to get the full \pdfs.

%% Let us take the first folding as an example.
%% The first folding is at $ \phi=0$ (for $\phi<0,\phi\rightarrow-\phi$, the $\phi$'s domain is reduced to (0,$\pi$)).
%% To be more clear, we divide the decay rate $d\Gamma$ into two parts corresponding to $\phi>0$ and $\phi<0$, i.e. $d\Gamma(\phi;\phi>0)$, and $d\Gamma(\phi;\phi<0)$:
%% \begin{equation} \label{eq:folding}
%%   \begin{split}
%%     d\hat{\Gamma} &= d\Gamma(\phi|\phi<0) + d\Gamma(\phi|\phi>0) \\
%%     & = f_0(\phi|\phi\rightarrow-\phi) + f_0(\phi|\phi>0) \\
%%     & = f_0(\cos\phi, -\sin\phi) + f_0(\cos\phi, \sin\phi)
%%   \end{split}
%% \end{equation}

%% According to trigonometric identities $\cos(-\phi) = \cos\phi $, $\sin(-\phi) = -\sin\phi $, we can cancel the terms that are odd under this transformation.
%% %% containing $\sin\phi$
%% Equation~\ref{eq:Angular} now reads:
%% \begin{equation} \label{eq:fold1}
%%   \begin{split}
%%     \frac{1}{\mathrm{d}\Gamma/\mathrm{d}q^2}&\frac{\mathrm{d}^4\Gamma}{\mathrm{d}q^2 \mathrm{d}\cos\theta_l \mathrm{d}\cos\theta_\mathrm{K} \mathrm{d}\phi} = \frac{9}{16\pi}\left[\frac{3}{4}F_\mathrm{T}\sin^2\theta_\mathrm{K} + F_\mathrm{L}\cos^2\theta_\mathrm{K} \right.\\
%%       &\left.+(\frac{1}{4}F_\mathrm{T}\sin^2\theta_\mathrm{K}-F_\mathrm{L}\cos^2\theta_\mathrm{K})\cos2\theta_l+\frac{1}{2}P_1F_\mathrm{T}\sin^2\theta_\mathrm{K}\sin^2\theta_l\cos 2\phi \right.\\
%%       &+\sqrt{F_\mathrm{T}F_\mathrm{L}}(\frac{1}{2}P_4'\sin2\theta_\mathrm{K}\sin2\theta_l\cos\phi+P_5'\sin2\theta_\mathrm{K}\sin\theta_l\cos\phi )\\
%%       &\left.+2P_2F_\mathrm{T}\sin^2\theta_\mathrm{K}\cos\theta_l \right]
%%   \end{split}
%% \end{equation}

%% The second folding is performed at $\theta_l = \pi/2$ (for $\theta_l>\pi/2,\theta_l\rightarrow \pi- \theta_l$).
%% The domain of $\theta_l$ is reduced to (0,$\pi$/2).
%% According to $\cos(\pi-\theta_l) = -\cos\theta_l$ and $\sin(\pi-\theta_l) = \sin\theta_l$, we can cancel the terms that are odd under this transformation.
%% %% proportional to $P_4'$, which contains $\sin 2\theta_l$.

%% \begin{equation} \label{eq:fold2}
%%   \begin{split}
%%     \frac{1}{\mathrm{d}\Gamma/\mathrm{d}q^2}&\frac{\mathrm{d}^4\Gamma}{\mathrm{d}q^2 \mathrm{d}\cos\theta_l \mathrm{d}\cos\theta_\mathrm{K} \mathrm{d}\phi} = \frac{9}{8\pi}\left[\frac{3}{4}F_\mathrm{T}\sin^2\theta_\mathrm{K} + F_\mathrm{L}\cos^2\theta_\mathrm{K} \right.\\
%%       &\left.+(\frac{1}{4}F_\mathrm{T}\sin^2\theta_\mathrm{K}-F_\mathrm{L}\cos^2\theta_\mathrm{K})\cos2\theta_l+\frac{1}{2}P_1F_\mathrm{T}\sin^2\theta_\mathrm{K}\sin^2\theta_l\cos 2\phi \right.\\
%%       &\left.+\sqrt{F_\mathrm{T}F_\mathrm{L}}P_5'\sin2\theta_\mathrm{K}\sin\theta_l\cos\phi  \right]
%%   \end{split}
%% \end{equation}
%% %%%%%%%%%%%%%%%%%%%%%%%%%%%%%%%%

%% For S-wave and the interference terms, we do the same transformation as P-wave, after the first ``folding'', it can reads:
%% \begin{equation} \label{eq:S-fold1}
%%   \begin{split}
%%     \frac{1}{\mathrm{d}\Gamma/\mathrm{d}q^2}&\frac{\mathrm{d}^4\Gamma}{\mathrm{d}q^2 \mathrm{d}\cos\theta_l \mathrm{d}\cos\theta_\mathrm{K} \mathrm{d}\phi} = \frac{3}{8\pi}\left[F_\mathrm{S}\sin^2\theta_l+ A_\mathrm{S}\sin^2\theta_l\cos\theta_\mathrm{K}\right.\\
%%       &+\left. A_\mathrm{S}^4\sin\theta_\mathrm{K}\sin2\theta_l\cos\phi + A^5_\mathrm{S}\sin\theta_\mathrm{K}\sin\theta_l\cos\phi\right]
%%   \end{split}
%% \end{equation}

%% After the second ``folding'', it reads:
%% \begin{equation} \label{eq:S-fold2}
%%   \begin{split}
%%     \frac{1}{\mathrm{d}\Gamma/\mathrm{d}q^2}&\frac{\mathrm{d}^4\Gamma}{\mathrm{d}q^2 \mathrm{d}\cos\theta_l \mathrm{d}\cos\theta_\mathrm{K} \mathrm{d}\phi} = \\
%%     &\frac{3}{4\pi}\left[F_\mathrm{S}\sin^2\theta_l+A_\mathrm{S}\sin^2\theta_l\cos\theta_\mathrm{K}+A^5_\mathrm{S}\sin\theta_\mathrm{K}\sin\theta_l\cos\phi\right]
%%   \end{split}
%% \end{equation}
%%%%%%%%%%%%%%%%%%%%%%%%%%%%%%%%%%5

After the application of the two foldings, the differential decay distribution can be written as:
\begin{equation} \label{eq:PDF-f2}
  \begin{split}
    \frac{1}{\mathrm{d}\Gamma/\mathrm{d}q^2}&\frac{\mathrm{d}^4\Gamma}{\mathrm{d}q^2 \mathrm{d}\cos\theta_l \mathrm{d}\cos\theta_\mathrm{K} \mathrm{d}\phi} = \\
    &\frac{9}{8\pi}\left\{\frac{2}{3}\left[ (F_\mathrm{S}+A_\mathrm{S}\cos\theta_\mathrm{K})\left(1-\cos^2\theta_l\right) + A^5_\mathrm{S}\sqrt{1-\cos^2\theta_\mathrm{K}}\sqrt{1-\cos^2\theta_l}\cos\phi \right] \right.\\
    & + \left(1 - F_\mathrm{S}\right)\left[2F_\mathrm{L}\cos^2\theta_\mathrm{K}\left(1-\cos^2\theta_l\right)+\frac{1}{2}\left(1-F_\mathrm{L}\right)\left(1-\cos^2\theta_\mathrm{K}\right)\left(1+\cos^2\theta_l\right) \right.\\
      & + \frac{1}{2}P_1(1-F_\mathrm{L})(1-\cos^2\theta_\mathrm{K})(1-\cos^2\theta_l)\cos 2\phi \\
      & \left.\left. + 2P_5'\cos\theta_\mathrm{K}\sqrt{F_\mathrm{L}\left(1-F_\mathrm{L}\right)}\sqrt{1-\cos^2\theta_\mathrm{K}}\sqrt{1-\cos^2\theta_l}\cos\phi\right]\right\}
  \end{split}
\end{equation}

Now it contains 6 angular parameters: $F_\mathrm{L}$, $P_1$, and $P_5'$ for the P-wave component, $F_S$ for the S-wave component, $A_\mathrm{S}$ and $A^5_\mathrm{S}$ for the interference component.

\subsection{Parameter constrains}
\label{sec:bound}

\subsubsection{Range of definition of interference terms}
\label{sec:As5.range}
Due to their nature, the values of the interference terms $A_s$ and $A_s^5$ are limited by the amplitude of the pure P-wave and S-wave components~\cite{Genon:Swave}.
Their allowed ranges are the following:
\begin{equation} \label{eq:As.range}
  |A_s|<2\sqrt{3}\sqrt{F_S(1-F_S)F_L}*F_{theo}
\end{equation}
\begin{equation} \label{eq:As5.range}
  |A^5_s|<\sqrt{3}\sqrt{F_S(1-F_S)F_T(1+P_1)}*F_{theo}
\end{equation}
where $F_{theo}$ is a constant factor that depends on the selection cuts applied to the $\mathrm{K}\pi$ system mass.
For the selection criteria used in this analysis, as described in Section~\ref{sec:offsel}, according to the theoretical prescriptions~\cite{Genon:Swave} the value used for $F_{theo}$ is 0.89.

In order to make sure that the fitted value of the $A_s^5$ parameter is contained in this range, it has been substituted in the \pdf by:
\begin{equation} \label{eq:As5.subst}
  A^5_s\to f\sqrt{3}\sqrt{F_S(1-F_S)F_T(1+P_1)}*F_{theo}
\end{equation}
where $f$ is a placeholder parameter defined in the range [-1;1].
The fit will be performed with respect to $f$, constrained in his range of validity, and the resulting values of $A_s^5$ will be obtained from $f$ by reversing Equation~\ref{eq:As5.subst}.
In this thesis this additional step will be kept implicit: when it will be stated that the likelihood is maximised as a function of $A_s^5$, it will be intended that it is maximised as a function of $f$.

\subsubsection{\pdf validity in the parameter space}
\label{sec:phys.bound}
Using the \pdf parameterisation described above it is not guaranteed that it is physical, i.e. positive in the whole ($\cos\theta_K$,$\cos\theta_l$,$\phi$) space.

In order to have a working fit sequence and reliable results, the values of the angular parameters that allow the \pdf to be physical need to be identified.
As explained in Section~\ref{sec:fitseq}, the angular parameters that are free to float in the fit to the data are $P_1$, $P_5'$, and $A^5_\mathrm{S}$, while the others, $F_\mathrm{L}$, $F_S$, and $A_\mathrm{S}$, are kept fixed to the results of the previous CMS analysis.
Thus, the region of validity of the \pdf is computed only in the three-dimensional space of the floating parameters, assuming the values of the others fixed like in the fit procedure.

The analytic boundaries for the pure P-wave component can be found in literature~\cite{Matias:2014jua} and, after the reduction to the $P_1$, $P_5'$ parameter space, they are:
\begin{equation} \label{eq:anal.bound}
  (P_5')^2 - 1 < P_1 < 1
\end{equation}
without any dependence on the value of $F_\mathrm{L}$.

The pure S-wave component is always positive, as can be derived from Equation~\ref{eq:S-wave0} and from the range of definition of $F_S$, from 0 to 1.
In general, the interference terms can be also negative, so additional constraints are needed to guarantee the positiveness of the full \pdf.
Since these constraints are not available in literature in an analytic form, a numerical computation is needed to describe them.

%% This operation can be done analytically, probing the \pdf for some selected values of the angular variables.
%% From this procedure it is possible to get that the P1 physical range is [-1,1]; but no boundary can be extracted for $P_5'$ and $A^5_\mathrm{S}$, for which a numerical computation is needed.

For each bin, eight different physical boundaries in the ($P_1$,$P_5'$) parameter space have been computed:
\begin{itemize}
\item one boundary is obtained by requiring that only the P-wave component is positive; this is just a cross-check of the validity of Equation~\ref{eq:anal.bound}, since the same result is expected;
\item seven boundaries are obtained by requiring the whole \pdf to be positive, for seven different values of $A_s^5$; according the convention defined in Section~\ref{sec:As5.range}, the seven values of the placeholder parameter $f$ are chosen to be [-1, -2/3, -1/3, 0, 1/3, 2/3, 1].
\end{itemize}

To compute numerically each of these physical boundaries, the $P_1$,$P_5'$ space has been scanned with a grid of step 0.01 in both directions.
For each point of this \textit{parameter} grid, the values of $\cos\theta_K$, $\cos\theta_l$ and $\phi$ are moved on a three-dimensional grid with step 0.02 in each direction; if the \pdf is positive for all of the points of this \textit{angle} grid, the point in the $P_1$,$P_5'$ space is defined to be inside the physical region, otherwise it is outside.
%% The resulting region is equivalent to a $P_1$-dependent upper boundary for the absolute value of $P_5'$.

The boundaries of the resulting physical regions, plotted in the negative $P_5'$ sector only, are shown from Figure~\ref{fig:bound0} to Figure~\ref{fig:bound8}.
The physical region is the one with $P_5'$ values smaller, in module, than the plotted boundaries, and it is included between them and their projections in the positive $P_5'$ sector. 
The P-wave boundary is symmetrically reflected with respect to $P_5'=0$; the boundaries assigned to a value of $A_s^5$ can be reflected to the positive $P_5'$ sector, but the reflected boundary refers to the opposite value of $A_s^5$, as can be derived from the symmetries in Equation~\ref{eq:PDF-f2}.

\begin{figure}[!hbt]
  \centering
  \includegraphics[width=0.85\textwidth]{Figures/boundaries/bound_b0.pdf}
  \caption{Physical boundaries of the negative $P_5'$ sector of $q^2$ bin 0.
    Accordingly to the description in Section~\ref{sec:phys.bound}, the magenta line is the boundary of the P-wave physical region and the set of grey-scale lines are the boundaries of the total-PDF physical region, for different $A_s^5$ values (black for $f=-1$, lightest grey for $f=1$).}
  \label{fig:bound0}
\end{figure}

\begin{figure}[!hbt]
  \centering
  \includegraphics[width=0.85\textwidth]{Figures/boundaries/bound_b1.pdf}
  \caption{Physical boundaries of the negative $P_5'$ sector of $q^2$ bin 1. Accordingly to the description in Sec.~\ref{sec:phys.bound}, the magenta line is the boundary of the P-wave physical region and the set of grey-scale lines are the boundaries of the total-PDF physical region, for different $A_s^5$ values (black for $f=-1$, lightest grey for $f=1$).}
  \label{fig:bound1}
\end{figure}

\begin{figure}[!hbt]
  \centering
  \includegraphics[width=0.85\textwidth]{Figures/boundaries/bound_b2.pdf}
  \caption{Physical boundaries of the negative $P_5'$ sector of $q^2$ bin 2. Accordingly to the description in Sec.~\ref{sec:phys.bound}, the magenta line is the boundary of the P-wave physical region and the set of grey-scale lines are the boundaries of the total-PDF physical region, for different $A_s^5$ values (black for $f=-1$, lightest grey for $f=1$).}
  \label{fig:bound2}
\end{figure}

\begin{figure}[!hbt]
  \centering
  \includegraphics[width=0.85\textwidth]{Figures/boundaries/bound_b3.pdf}
  \caption{Physical boundaries of the negative $P_5'$ sector of $q^2$ bin 3. Accordingly to the description in Sec.~\ref{sec:phys.bound}, the magenta line is the boundary of the P-wave physical region and the set of grey-scale lines are the boundaries of the total-PDF physical region, for different $A_s^5$ values (black for $f=-1$, lightest grey for $f=1$).}
  \label{fig:bound3}
\end{figure}

\begin{figure}[!hbt]
  \centering
  \includegraphics[width=0.85\textwidth]{Figures/boundaries/bound_b5.pdf}
  \caption{Physical boundaries of the negative $P_5'$ sector of $q^2$ bin 5. Accordingly to the description in Sec.~\ref{sec:phys.bound}, the magenta line is the boundary of the P-wave physical region and the set of grey-scale lines are the boundaries of the total-PDF physical region, for different $A_s^5$ values (black for $f=-1$, lightest grey for $f=1$).}
  \label{fig:bound5}
\end{figure}

\begin{figure}[!hbt]
  \centering
  \includegraphics[width=0.85\textwidth]{Figures/boundaries/bound_b7.pdf}
  \caption{Physical boundaries of the negative $P_5'$ sector of $q^2$ bin 7. Accordingly to the description in Sec.~\ref{sec:phys.bound}, the magenta line is the boundary of the P-wave physical region and the set of grey-scale lines are the boundaries of the total-PDF physical region, for different $A_s^5$ values (black for $f=-1$, lightest grey for $f=1$).}
  \label{fig:bound7}
\end{figure}

\begin{figure}[!hbt]
  \centering
  \includegraphics[width=0.85\textwidth]{Figures/boundaries/bound_b8.pdf}
  \caption{Physical boundaries of the negative $P_5'$ sector of $q^2$ bin 8. Accordingly to the description in Sec.~\ref{sec:phys.bound}, the magenta line is the boundary of the P-wave physical region and the set of grey-scale lines are the boundaries of the total-PDF physical region, for different $A_s^5$ values (black for $f=-1$, lightest grey for $f=1$).}
  \label{fig:bound8}
\end{figure}
