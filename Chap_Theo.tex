\chapter{Analysis introduction}
\label{sec:theo}

temp

\section{Flavour Physics}
\label{sec:flav}

temp

\subsection[Flavour-changing neutral currents b to s l l]{Flavour-changing neutral currents $b\to s\ell^+\ell^-$}
\label{sec:FCNC}

The penguin-mediated flavour-changing neutral currents (FCNC) $b\to s\ell^+\ell^-$ are a set of $b$-hadron semileptonic decays with a pair of non-resonant charged leptons in the final state.
These decay channels are doubly suppressed within the SM:
\begin{itemize}
\item they are forbidden at tree-level, since there is no neutral current in SM allowed to violate the flavour, and the leading-order Feynman diagram that mediate them is a weak penguin loop;
\item the leading-order diagram is Cabibbo suppressed, since it is proportional to $|V_{ts}V_{tb}|\sim10^{-2}$.
\end{itemize}
These suppressions result in in small branching fractions and they create an ideal environment for New Physics (NP) searches.
Any potential contrubution from physics beyond the SM that enters in the loop diagram could produce sizeable effects in the decay branching fractions or in their angular distributions.

On the other hand, the theoretical predictions of the angular distributions and branching fractions of these decays are quite suscettible to hadronic uncertainties due to long-distance Quantum Chromodynamic (QCD) processes.
In order to separate, in the theoretical framework, the effect from long-distance QCD and from short-disctance effects, which are related to QCD and electroweak interactions but also to NP processes, an effective Hamiltonian is defined as follow:
\begin{equation} \label{eq:Heff}
  {\cal H}_{\eff} = - \frac{4\,G_F}{\sqrt{2}}\left(\lambda_t {\cal H}_{\eff}^{(t)} + \lambda_u {\cal H}_{\eff}^{(u)}\right)
\end{equation}
where $\lambda_i=V_{ib}V_{is}^*$ and
\begin{eqnarray*}
  {\cal H}_{\eff}^{(t)} & = & C_1 \mathcal O_1^c + C_2 \mathcal O_2^c + \sum_{i=3}^{6} C_i \mathcal O_i + \sum_{i=7,8,9,10,P,S} (C_i \mathcal O_i + C'_i \mathcal O'_i)\,, \\
  {\cal H}_{\eff}^{(u)} & = & C_1 (\mathcal O_1^c-\mathcal O_1^u)  + C_2(\mathcal O_2^c-\mathcal O_2^u)\,.
\end{eqnarray*}

The of opearators $\mathcal O_i^{(a)}$ for $i<7$ are defined in the Reference~\cite{oper}, while the other ones are defined accordingly to Reference~\cite{Altmannshofer:2008dz}:
\begin{align}
  {\mathcal{O}}_{7} &= \frac{e}{g^2} m_b (\bar{s} \sigma_{\mu \nu} P_R b) F^{\mu \nu} &
  {\mathcal{O}}_{7}^\prime &= \frac{e}{g^2} m_b (\bar{s} \sigma_{\mu \nu} P_L b) F^{\mu \nu} \label{eq:O7}\\
  {\mathcal{O}}_{8} &= \frac{1}{g} m_b (\bar{s} \sigma_{\mu \nu} T^a P_R b) G^{\mu \nu \, a} &
  {\mathcal{O}}_{8}^\prime &= \frac{1}{g} m_b (\bar{s} \sigma_{\mu \nu} T^a P_L b) G^{\mu \nu \, a} \label{eq:O8}\\
  {\mathcal{O}}_{9} &= \frac{e^2}{g^2} (\bar{s} \gamma_{\mu} P_L b)(\bar{\mu} \gamma^\mu \mu) &
  {\mathcal{O}}_{9}^\prime &= \frac{e^2}{g^2} (\bar{s} \gamma_{\mu} P_R b)(\bar{\mu} \gamma^\mu \mu) \label{eq:O9}\\
  {\mathcal{O}}_{10} &=\frac{e^2}{g^2} (\bar{s}  \gamma_{\mu} P_L b)(\bar{\mu} \gamma^\mu \gamma_5 \mu) &
  {\mathcal{O}}_{10}^\prime &=\frac{e^2}{g^2} (\bar{s}  \gamma_{\mu} P_R b)(\bar{\mu} \gamma^\mu \gamma_5 \mu) \label{eq:O10}\\
  {\mathcal{O}}_{S} &=\frac{e^2}{16\pi^2} m_b (\bar{s} P_R b)(\bar{\mu} \mu) &
  {\mathcal{O}}_{S}^\prime &=\frac{e^2}{16\pi^2} m_b (\bar{s} P_L b)(\bar{\mu} \mu) \label{eq:OS}\\
  {\mathcal{O}}_{P} &=\frac{e^2}{16\pi^2} m_b (\bar{s} P_R b)(\bar{\mu} \gamma_5 \mu) &
  {\mathcal{O}}_{P}^\prime &=\frac{e^2}{16\pi^2} m_b (\bar{s} P_L b)(\bar{\mu} \gamma_5 \mu) \label{eq:OP}
\end{align}
where $g$ is the strong coupling constant, $m_b$ is therunning $b$ quark mass in the $\bar{\mathrm{MS}}$ scheme, and $P_{L,R}=(1\mp\gamma_5)/2$ are the chirality projectors.

The set of Wilson coefficients $C^{(\prime)}_i$ encodes the contribution from short-distance physics and could contain NP effects.
These coefficients can be expanded and calculated in perturbation theory in powers of $\alpha_s$:
\begin{equation}\label{WilExpansion}
  C_i = C_i^{(0)} + \frac{\alpha_s}{4\pi}\, C_i^{(1)} + \left(\frac{\alpha_s}{4\pi}\right)^2 C_i^{(2)} + O(\alpha_s^3)\,,
\end{equation}
Their predicted values of these expanded coefficients, both within SM and for NP scenarios, are calculated at the scale $\mu=m_W$ and then evolved down to $\mu\sim m_b$.

According to the SM, several of the terms in the effective Hamiltonian are expected to vanish or be highly suppressed; this is the case for all the primed operators, as well as for the scalar and pseudoscalar ones, ${\mathcal{O}}_{S}$ and ${\mathcal{O}}_{P}$.

%%%%%%%%%%%%%%%%%%%%
If we restrict the study to a $\mathrm{B}^{(0,\pm)}\to\K^{*(0,\pm)}\ell^+\ell^-$ decay, the matrix elements of the effective Hamiltonian operators ${\mathcal{O}}^{(\prime)}_{7,9,10,S,P}$ can be descibed as a function of seven form factors, $A_i(q^2)$, $V(q^2)$, and $T_j(q^2)$, where $0\le i\le2$ and $1\le j\le2$, and $q^2$ is the squared muomentum carried by the pair of leptons.
In literature is also used the form factor $A_3(q^2)$, which is a linear combination of $A_1(q^2)$ and $A_2(q^2)$.

The theoretical predictions of the form factors are computed using the QCD sum rules on the light-cone (LCSRs)~\cite{Khodjamirian:2001bj}.
With this technique, the form factors are expanded as a function of $m_b m_{\mathrm{K}^*}/(m_b^2-q^2)$.
The expansion works well for values of $q^2$ lower than about $6\,\mathrm{GeV}^2$, but for higher values the high-order terms of the expansion become more relevant and the uncertainty associated to the approximated prediction grows accordingly.

%%%%%%%%%%%%%%%%%%
In the $\mathrm{B}^{(0,\pm)}\to\K^{*(0,\pm)}\ell^+\ell^-$ decay amplitude, not only the terms proportional to the form factors are contributing.







\section[The decay B0 to K*0 mu+ mu-]{The decay $\mathrm{B}^0_d \to \mathrm{K}^{*0} \mu^+ \mu^-$}
\label{sec:Kstmm}

The FCNC decay \BtoKstmumudecay is an optimal laboratory to probe the flavour sector of the SM.
From the experimental point of view, it has a fully charged final state, with two muons, which are easy to identify in a multi-purpose particle detector, and two charged hadronic particles.
Furthermore, the charges of the hadrons in the final state determine the CP-state of the decay, i.e. whether it is \BtoKstmumudecay or \BtoKstmumuconjdecay.
From the phenomenological point of view, the angular analysis of this decay can provide information on several components of the weak effective Hamiltonian: the electromagnetic and semileptonic operators, $\mathcal{O}_{7,9,10}$, and their chirality-flipped counterparts, $\mathcal{O}_{7^\prime,9^\prime,10^\prime}$, together with scalar and pseudoscalar operators, $\mathcal{O}_{S,P,S^\prime,P^\prime}$.

%%%%%%%%%%%%%%%%%%%%%%%

The decay in the four-body final state is completely described by a set of four kinematical variables, \TL, \TK, \PHI, and $q^2$.
The variable \TL is defined as the angle between the momentum of the \Pgmp (\Pgmm) and the direction opposite to the \PBz (\PaBz) momentum, in the dimuon rest frame:
\begin{equation} \label{eq:TLdef}
  \begin{aligned}
    \cTL & = \left( p_{\Pgmp}^{(\Pgmp\Pgmm)} \right) \cdot \left( -p_{\PBz }^{(\Pgmp\Pgmm)} \right)\, && \mathrm{for}\:\PBz\:\mathrm{decay} \\
    \cTL & = \left( p_{\Pgmm}^{(\Pgmp\Pgmm)} \right) \cdot \left( -p_{\PaBz}^{(\Pgmp\Pgmm)} \right)\, && \mathrm{for}\:\PaBz\:\mathrm{decay} \\
  \end{aligned}
\end{equation}
%% \begin{alignat}{3} \label{eq:TLdef}
%%   \cTL = & \left( p_{\Pgmp}^{(\Pgmp\Pgmm)} \right) \cdot \left( -p_{\PBz }^{(\Pgmp\Pgmm)} \right)\, & \mathrm{for \PBz decay;} \\
%%   \cTL = & \left( p_{\Pgmm}^{(\Pgmp\Pgmm)} \right) \cdot \left( -p_{\PaBz}^{(\Pgmp\Pgmm)} \right)\, & \mathrm{for \PaBz decay.}
%% \end{alignat}
The variable \TK is defined as the angle between the direction of the kaon and the direction opposite that of the \PBz (\PaBz), in the \cPKstz (\cPAKstz) rest frame
\begin{equation} \label{eq:TKdef}
  \begin{aligned}
    \cTK&=\left( p_{\PKp}^{(\cPKstz) } \right) \cdot \left( -p_{\PBz }^{(\cPKstz) } \right)\, && \mathrm{for}\:\PBz\:\mathrm{decay;} \\
    \cTK&=\left( p_{\PKm}^{(\cPAKstz)} \right) \cdot \left( -p_{\PaBz}^{(\cPAKstz)} \right)\, && \mathrm{for}\:\PaBz\:\mathrm{decay.}
  \end{aligned}
\end{equation}
The variable \PHI is defined as the angle between the plane containing the \Pgmp and \Pgmm momenta and the plane containing the momenta of the kaon and the pion, in the \PBz rest frame, with the following sign conventions:
\begin{equation} \label{eq:PHIdef}
  \begin{aligned}
    \cos(\PHI) &= \left( p_{\Pgmp}^{(\PBz)} \times p_{\Pgmm}^{(\PBz)} \right) \cdot \left( p_{\PKp}^{(\PBz)} \times p_{\Pgpm}^{(\PBz)} \right) \, && \mathrm{and} \\
    \sin(\PHI) &= \left[ \left( p_{\Pgmp}^{(\PBz)} \times p_{\Pgmm}^{(\PBz)} \right) \times \left( p_{\PKp}^{(\PBz)} \times p_{\Pgpm}^{(\PBz)} \right) \right] \cdot p_{\cPKstz}^{(\PBz)}\, && \mathrm{for}\:\PBz\:\mathrm{decay;} \\[8pt]
    \cos(\PHI) &= \left( p_{\Pgmm}^{(\PaBz)} \times p_{\Pgmp}^{(\PaBz)} \right) \cdot \left( p_{\PKm}^{(\PaBz)} \times p_{\Pgpp}^{(\PaBz)} \right) \, && \mathrm{and} \\
    \sin(\PHI) &=-\left[ \left( p_{\Pgmm}^{(\PaBz)} \times p_{\Pgmp}^{(\PaBz)} \right) \times \left( p_{\PKm}^{(\PaBz)} \times p_{\Pgpp}^{(\PaBz)} \right) \right] \cdot p_{\cPAKstz}^{(\PaBz)}\, && \mathrm{for}\:\PaBz\:\mathrm{decay.} \\
  \end{aligned}
\end{equation}
Finally, the variable $q^2$ is defined as the invariant mass squared of the dimuon system.
In the Equations~\ref{eq:TLdef},~\ref{eq:TKdef},~\ref{eq:PHIdef} the notation $p_a^{(b)}$ indicates the momentum of the particle $a$ in the rest frame of the particle $b$.

The definition of the kinematical variables is coherent with the one used in the previous analyses performed on this decay channel.
The definition of \TL differs from the one used in some phenomenological papers.
A graphic representation of the definition of the three angular variables is shown in Figure~\ref{fig:ske}.

\begin{figure*}[!tbh]
  \begin{center}
    \includegraphics[width=0.99\textwidth]{SketchDecay.pdf}
    \caption{Illustration of the angular variables \TL (left), \TK (middle), and \PHI (right) for the decay \BtoKstmumudecay.}
    \label{fig:ske}
  \end{center}
\end{figure*}

The range of definition of the angular variables are $\left[0,\pi\right]$, for \TL and \TK, and $\left[-\pi,\pi\right]$, for \PHI.

%%%%%%%%%%%%%%%%%%%%%%%%%555

\subsection{The angular decay rate}
\label{sec:decRate}

The matrix element for the \BtoKstmumudecay can be obtained from the effective Hamiltonian in Equation~\ref{eq:Heff}:
\begin{equation}\label{eq:matrixelement}
  \begin{split}
    {\mathcal M}\ =& \frac{G_F\alpha}{\sqrt{2}\pi}V_{tb}^{}V_{ts}^*\bigg\{\bigg[ \braket{K \pi}{\bar{s}\gamma^{\mu}({C_9^\text{eff}P_L+C_9^{\prime\text{eff}} P_R})b}{\bar B} \\
      &-\frac{2m_b}{q^2}\braket{K\pi}{\bar{s}i\sigma^{\mu\nu}q_{\nu}(C_7^\text{eff} P_R+C_7^{\prime\text{eff}}  P_L)b}{\bar B}\bigg](\bar{\mu}\gamma_{\mu}\mu)\\
    &+ \braket{K\pi}{\bar{s}\gamma^{\mu}({C_{10}^\text{eff} P_L+C_{10}^{\prime\text{eff}} P_R})b}{\bar B}(\bar{\mu}\gamma_{\mu}\gamma_5 \mu) \\
    & {+\braket{K \pi}{\bar{s} ({C_S P_R+C_S^\prime P_L}) b}{\bar B}(\bar{\mu}\mu)}{+\braket{K \pi}{\bar{s} ({C_P P_R+C_P^\prime P_L}) b}{\bar B}(\bar{\mu}\gamma_5\mu)} \bigg \}.
  \end{split}
\end{equation}
where the naive factorisation has been applied and a set of effective Wilson coefficients, $C_{i}^\text{eff}$, is used.
These effective coefficients are linear combinations of the set defined in Section~\ref{sec:FCNC}, and their definition is described in Reference~\cite{Altmannshofer:2008dz}.

The differential decay distribution as a function of the kinematical variables can be obtained by squaring Equation~\ref{eq:matrixelement}.
Projecting the result on a basis of combinations of spherical harmonics of the angular variables, one obtains the following expression for the \BtoKstmumuconjdecay decay:
\begin{small}
  \begin{align} \label{eq:angulardist}
    \frac{d^4\Gamma}{dq^2\, d\cTL\, d\cTK\, d\PHI} &=\frac{9}{32\pi}\bigg[ I_1^s\sin^2\TK + I_1^c\cos^2\TK \nonumber\\
      & + (I_2^s\sin^2\TK + I_2^c\cos^2\TK)\cos2\TL + I_3 \sin^2\TK \sin^2\TL \cos 2\PHI \nonumber \\
      & + I_4 \sin 2\TK \sin 2\TL \cos\PHI + I_5 \sin 2\TK \sin\TL \cos\PHI \nonumber \\
      & + (I_6^s \sin^2\TK + {I_6^c \cos^2\TK})  \cos\TL + I_7 \sin 2\TK \sin\TL \sin\PHI \nonumber \\
    & + I_8 \sin 2\TK \sin 2\TL \sin\PHI + I_9 \sin^2\TK \sin^2\TL \sin 2\PHI \bigg]
  \end{align}
\end{small}
where the complex angular coefficients $I_i^{(a)}$ depend only on $q^2$.
The differential distribution of the opposite CP-state decay, \BtoKstmumudecay, has the same expression, but it is function of the complex conjugated coefficients $\bar{I}_i^{(a)}$.

The set of angular coefficients $I_i^{(a)}$ has been expressed as a function of the effective Wilson coefficients and of the form factors, under the assumptions of naive factorisation of the matrix element and adding corrections to describe the effect of non-factorisable terms.
In literature they are also expressed as a function of eight \Ks transversity amplitudes.

Unlike Wilson coefficients, form factors, and transversity amplitudes, the angular coefficients $I_i^{(a)}$ are physical observable and they can be experimentally measured by the angular analysis of the \BtoKstmumudecay decay.
However, the theoretical predictions of these coefficients are prone to hadronic uncertainties, derived from their strong dependency on the form factors.
For this reasons, more sophisticated bases of angular coefficients have been defined such that they are independent from form factors at leading order of the effective-theory expansion.

Firstly, two sets of coefficients are defined, the CP-averages:
\begin{equation}
  S^{(a)}_i = \frac{ I^{(a)}_i + \bar I^{(a)}_i }{\frac{d\Gamma}{dq^2} + \frac{d\bar\Gamma}{dq^2}}
  \label{eq:Ss}
\end{equation}
and the CP asymmetries:
\begin{equation}
  A^{(a)}_i = \frac{ I^{(a)}_i - \bar I^{(a)}_i }{\frac{d\Gamma}{dq^2}+\frac{d\bar\Gamma}{dq^2}}
  \label{eq:As}
\end{equation}
that are useful to disentangle potential NP effects that introduce new sources of CP-violation from the others.
Since this analysis is aiming the measurement of two CP-averaged angular parameters, I will not spend time describing the basis of CP-violating coefficients but I will focus on the set of $S^{(a)}_i$ parameters.

The twelve real angular coefficients in this set can be reduced to eight, in the approximation of negligible lepton mass with respect to $q$.
This is true in this analysis, since only candidates with $q$ value greater than 1\GeV are used, as described in Section~\ref{sec:q2}.
In the massless-muon limit, the following conditions are valid:
\begin{gather}
  S_1^s = 3 S_2^s \nonumber\\
  S_1^c = - S_2^c \nonumber\\
  \frac{3}{4}\left(2S_1^s+S_1^c\right)-\frac{1}{4}\left(2S_2^s+S_2^c\right) = 1 \nonumber\\
  S_6^c = 0 \label{eq:massless-reductions}
\end{gather}
where the latter condition is always true within the SM, even without the massless-muon limit, since it is generated by the scalar operator, which does not exist in the SM.

The $S_1^c$ coefficient corresponds to the fraction of \Ks produced with longitudinal polarisation, and it is usually referred to as $F_L$.
In the same way, the coefficient $S_6^s$ is proportional to the forward-backward asymmetry of the muon system, thus the parameter $A_{FB}=\frac{3}{4}S_6^s$ is used.

The so-called $P$-primed basis of angular parameters, clean from leading-order hadronic uncertainties, is defined as:
\begin{equation}\label{eq:defPPrime}
  \begin{split}
    P_1 &= \frac{2S_3}{1-F_L}\\
    P_2 &= \frac{2}{3}\frac{A_{FB}}{1-F_L}\\
    P_3 &= \frac{-S_9}{1-F_L}\\
    P'_{4,5,8} &= \frac{S_{4,5,8}}{\sqrt{F_L(1-F_L)}}\\
    P'_6 &= \frac{S_7}{\sqrt{F_L(1-F_L)}}\\
  \end{split}
\end{equation}

Using the $P_i^{(\prime)}$ basis, with $F_L$ and its complementary $F_T=1-F_L$, the differential angular distribution can be written as:
\begin{equation} \label{eq:Angular}
  \begin{split}
    \frac{1}{\mathrm{d}\Gamma/\mathrm{d}q^2}&\frac{\mathrm{d}^4\Gamma}{\mathrm{d}q^2 \mathrm{d}\cos\theta_l \mathrm{d}\cos\theta_\mathrm{K} \mathrm{d}\phi} =\frac{9}{32\pi}\left[\frac{3}{4}F_\mathrm{T}\sin^2\theta_\mathrm{K} + F_\mathrm{L}\cos^2\theta_\mathrm{K} \right.\\
      &\left.+\left(\frac{1}{4}F_\mathrm{T}\sin^2\theta_\mathrm{K}-F_\mathrm{L}\cos^2\theta_\mathrm{K}\right)\cos2\theta_l+\frac{1}{2}P_1F_\mathrm{T}\sin^2\theta_\mathrm{K}\sin^2\theta_l\cos 2\phi \right.\\
      &+\sqrt{F_\mathrm{T}F_\mathrm{L}}\left(\frac{1}{2}P_4'\sin2\theta_\mathrm{K}\sin2\theta_l\cos\phi+P_5'\sin2\theta_\mathrm{K}\sin\theta_l\cos\phi \right)\\
      &-\sqrt{F_\mathrm{T}F_\mathrm{L}}\left(P_6'\sin2\theta_\mathrm{K}\sin\theta_l\sin\phi-\frac{1}{2}P_8'\sin2\theta_\mathrm{K}\sin2\theta_l\sin\phi \right)\\
      &\left.+2P_2F_\mathrm{T}\sin^2\theta_\mathrm{K}\cos\theta_l-P_3F_\mathrm{T}\sin^2\theta_\mathrm{K}\sin^2\theta_l\sin2\phi \right]
  \end{split}
\end{equation}

\subsubsection{S-wave contamination}
\label{sec:S-waveform}

Although the $K^+\pi^-$ invariant mass must be consistent with a $\text{K}^{*0}$, there can be contributions from a spinless (S-wave) $K^+\pi^-$ combination.
The presence of a $K^+\pi^-$ system in an S-wave configuration, due to a non-resonant contribution or to feed through from $K^+\pi^-$ scalar resonances, results in additional terms in the different angular distribution.
Denoting the right-hand side of Equation~\ref{eq:Angular} by $W_p$, the differential decay rate takes the form:
\begin{equation} \label{eq:S-wave}
  \begin{split}
    (1-F_\mathrm{S})W_p + (W_s + W_{sp})
  \end{split}
\end{equation}
where 
\begin{equation} \label{eq:S-wave0}
  \begin{split}
    W_s = \frac{3}{16\pi} F_\mathrm{S}\sin^2\theta_l
  \end{split}
\end{equation}
%% and $W_{sp}$ is given from Eq.(44) in~\cite{Genon:Swave}. 
and $W_{sp}$ is~\cite{Genon:Swave}:
\begin{equation} \label{eq:S-wave1}
  \begin{split}
    W_{sp}= &\frac{3}{16 \pi}\left[ A_\mathrm{S}\sin^2\theta_l\cos\theta_\mathrm{K}+ A_\mathrm{S}^4\sin\theta_\mathrm{K}\sin2\theta_l\cos\phi\right.\\
      &+\left.A^5_\mathrm{S}\sin\theta_\mathrm{K}\sin\theta_l\cos\phi+A_\mathrm{S}^7\sin\theta_\mathrm{K}\sin\theta_l\sin\phi+A_\mathrm{S}^8\sin\theta_\mathrm{K}\sin2\theta_l\sin\phi\right]
  \end{split}
\end{equation}
where $F_\mathrm{S}$ is the fraction of the S-wave component in the $\text{K}^{*0}$ mass window, and $W_{sp}$ contains all the interference terms, $A_\mathrm{S}^i$ are the interference amplitudes between the S-wave and the P-wave decays\cite{Genon:Swave}.
%%%%%%%%%%%%%%%%%%%%%%%%%%%%%%%%%%

\subsection{The \pdf folding}
\label{sec:folding}

%% The angular distribution of the decay
%% $\text{B}^0 \rightarrow \text{K}^{*0} \mu^+ \mu^-$ can be described by
%% three angles ($\theta_l $, $\theta_\mathrm{K} $ and $\phi$) and the
%% invariant mass squared of the dimuon system ($q^2$).
Because of the limited number of signal candidates in the data set, we didn't fit the data to full differential distribution of Equation~\ref{eq:Angular}.
To retrieve the interesting variables more effectively and to reduce the number of fitting parameters, we performed the following transformations of the decay-rate formulation.

Inspired by the derivations in ref \cite{LHCb2}\cite{Matias2012}, to reduce the number of parameters in the fit, we "fold" the data twice.
``Folding'' means that we divide the decay rate into different parts, calculate them separately according to some symmetries and then add them together to obtain the equivalent decay rates.
If we take consecutive steps of ``folding'', the similar expansions are used to get the full \pdfs.

Let us take the first folding as an example.
The first folding is at $ \phi=0$ (for $\phi<0,\phi\rightarrow-\phi$, the $\phi$'s domain is reduced to (0,$\pi$)).
To be more clear, we divide the decay rate $d\Gamma$ into two parts corresponding to $\phi>0$ and $\phi<0$, i.e. $d\Gamma(\phi;\phi>0)$, and $d\Gamma(\phi;\phi<0)$:
\begin{equation} \label{eq:folding}
  \begin{split}
    d\hat{\Gamma} &= d\Gamma(\phi|\phi<0) + d\Gamma(\phi|\phi>0) \\
    & = f_0(\phi|\phi\rightarrow-\phi) + f_0(\phi|\phi>0) \\
    & = f_0(\cos\phi, -\sin\phi) + f_0(\cos\phi, \sin\phi)
  \end{split}
\end{equation}

According to trigonometric identities $\cos(-\phi) = \cos\phi $, $\sin(-\phi) = -\sin\phi $, we can cancel the terms that are odd under this transformation.
%% containing $\sin\phi$
Equation~\ref{eq:Angular} now reads:
\begin{equation} \label{eq:fold1}
  \begin{split}
    \frac{1}{\mathrm{d}\Gamma/\mathrm{d}q^2}&\frac{\mathrm{d}^4\Gamma}{\mathrm{d}q^2 \mathrm{d}\cos\theta_l \mathrm{d}\cos\theta_\mathrm{K} \mathrm{d}\phi} = \frac{9}{16\pi}\left[\frac{3}{4}F_\mathrm{T}\sin^2\theta_\mathrm{K} + F_\mathrm{L}\cos^2\theta_\mathrm{K} \right.\\
      &\left.+(\frac{1}{4}F_\mathrm{T}\sin^2\theta_\mathrm{K}-F_\mathrm{L}\cos^2\theta_\mathrm{K})\cos2\theta_l+\frac{1}{2}P_1F_\mathrm{T}\sin^2\theta_\mathrm{K}\sin^2\theta_l\cos 2\phi \right.\\
      &+\sqrt{F_\mathrm{T}F_\mathrm{L}}(\frac{1}{2}P_4'\sin2\theta_\mathrm{K}\sin2\theta_l\cos\phi+P_5'\sin2\theta_\mathrm{K}\sin\theta_l\cos\phi )\\
      &\left.+2P_2F_\mathrm{T}\sin^2\theta_\mathrm{K}\cos\theta_l \right]
  \end{split}
\end{equation}

The second folding is performed at $\theta_l = \pi/2$ (for $\theta_l>\pi/2,\theta_l\rightarrow \pi- \theta_l$).
The domain of $\theta_l$ is reduced to (0,$\pi$/2).
According to $\cos(\pi-\theta_l) = -\cos\theta_l$ and $\sin(\pi-\theta_l) = \sin\theta_l$, we can cancel the terms that are odd under this transformation.
%% proportional to $P_4'$, which contains $\sin 2\theta_l$.

\begin{equation} \label{eq:fold2}
  \begin{split}
    \frac{1}{\mathrm{d}\Gamma/\mathrm{d}q^2}&\frac{\mathrm{d}^4\Gamma}{\mathrm{d}q^2 \mathrm{d}\cos\theta_l \mathrm{d}\cos\theta_\mathrm{K} \mathrm{d}\phi} = \frac{9}{8\pi}\left[\frac{3}{4}F_\mathrm{T}\sin^2\theta_\mathrm{K} + F_\mathrm{L}\cos^2\theta_\mathrm{K} \right.\\
      &\left.+(\frac{1}{4}F_\mathrm{T}\sin^2\theta_\mathrm{K}-F_\mathrm{L}\cos^2\theta_\mathrm{K})\cos2\theta_l+\frac{1}{2}P_1F_\mathrm{T}\sin^2\theta_\mathrm{K}\sin^2\theta_l\cos 2\phi \right.\\
      &\left.+\sqrt{F_\mathrm{T}F_\mathrm{L}}P_5'\sin2\theta_\mathrm{K}\sin\theta_l\cos\phi  \right]
  \end{split}
\end{equation}
%%%%%%%%%%%%%%%%%%%%%%%%%%%%%%%%

For S-wave and the interference terms, we do the same transformation as P-wave, after the first ``folding'', it can reads:
\begin{equation} \label{eq:S-fold1}
  \begin{split}
    \frac{1}{\mathrm{d}\Gamma/\mathrm{d}q^2}&\frac{\mathrm{d}^4\Gamma}{\mathrm{d}q^2 \mathrm{d}\cos\theta_l \mathrm{d}\cos\theta_\mathrm{K} \mathrm{d}\phi} = \frac{3}{8\pi}\left[F_\mathrm{S}\sin^2\theta_l+ A_\mathrm{S}\sin^2\theta_l\cos\theta_\mathrm{K}\right.\\
      &+\left. A_\mathrm{S}^4\sin\theta_\mathrm{K}\sin2\theta_l\cos\phi + A^5_\mathrm{S}\sin\theta_\mathrm{K}\sin\theta_l\cos\phi\right]
  \end{split}
\end{equation}

After the second ``folding'', it reads:
\begin{equation} \label{eq:S-fold2}
  \begin{split}
    \frac{1}{\mathrm{d}\Gamma/\mathrm{d}q^2}&\frac{\mathrm{d}^4\Gamma}{\mathrm{d}q^2 \mathrm{d}\cos\theta_l \mathrm{d}\cos\theta_\mathrm{K} \mathrm{d}\phi} = \\
    &\frac{3}{4\pi}\left[F_\mathrm{S}\sin^2\theta_l+A_\mathrm{S}\sin^2\theta_l\cos\theta_\mathrm{K}+A^5_\mathrm{S}\sin\theta_\mathrm{K}\sin\theta_l\cos\phi\right]
  \end{split}
\end{equation}
%%%%%%%%%%%%%%%%%%%%%%%%%%%%%%%%%%5

After the two folding are performed, the angular distribution can be written, using  Equation~\ref{eq:fold2} and Equation~\ref{eq:S-fold2} as:
\begin{equation} \label{eq:PDF-f2}
  \begin{split}
    \frac{1}{\mathrm{d}\Gamma/\mathrm{d}q^2}&\frac{\mathrm{d}^4\Gamma}{\mathrm{d}q^2 \mathrm{d}\cos\theta_l \mathrm{d}\cos\theta_\mathrm{K} \mathrm{d}\phi} = \\
    &\frac{9}{8\pi}\left\{\frac{2}{3}\left[ (F_\mathrm{S}+A_\mathrm{S}\cos\theta_\mathrm{K})\left(1-\cos^2\theta_l\right) + A^5_\mathrm{S}\sqrt{1-\cos^2\theta_\mathrm{K}}\sqrt{1-\cos^2\theta_l}\cos\phi \right] \right.\\
    & + \left(1 - F_\mathrm{S}\right)\left[2F_\mathrm{L}\cos^2\theta_\mathrm{K}\left(1-\cos^2\theta_l\right)+\frac{1}{2}\left(1-F_\mathrm{L}\right)\left(1-\cos^2\theta_\mathrm{K}\right)\left(1+\cos^2\theta_l\right) \right.\\
      & + \frac{1}{2}P_1(1-F_\mathrm{L})(1-\cos^2\theta_\mathrm{K})(1-\cos^2\theta_l)\cos 2\phi \\
      & \left.\left. + 2P_5'\cos\theta_\mathrm{K}\sqrt{F_\mathrm{L}\left(1-F_\mathrm{L}\right)}\sqrt{1-\cos^2\theta_\mathrm{K}}\sqrt{1-\cos^2\theta_l}\cos\phi\right]\right\}
  \end{split}
\end{equation}

Now we have 6 parameters, they are $F_\mathrm{L}$, $F_S$, $P_1$, $P_5'$, $A_\mathrm{S}$ and $A^5_\mathrm{S}$.
%%%%%%%%%%%%%%%%%%%%%%%%%%%%%%%%%%%55

\subsection{Parameter constrains}
\label{sec:bound}

\subsubsection{Range of definition of interference terms}
\label{sec:As5.range}
Due to their nature, the values of the interference terms $A_s$ and $A_s^5$ are limited by the amplitude of the pure P-wave and S-wave components~\cite{Genon:Swave}.
Their allowed ranges are the following:
\begin{equation} \label{eq:As.range}
  |A_s|<2\sqrt{3}\sqrt{F_S(1-F_S)F_L}*F_{theo}
\end{equation}
\begin{equation} \label{eq:As5.range}
  |A^5_s|<\sqrt{3}\sqrt{F_S(1-F_S)F_T(1+P_1)}*F_{theo}
\end{equation}
where $F_{theo}$ is a constant factor that depends on the selection cuts applied to the $\mathrm{K}\pi$ system mass.
For the selection criteria used in this analysis, as described in Section~\ref{sec:offsel}, according to the theoretical prescriptions~\cite{Genon:Swave} the value used for $F_{theo}$ is 0.89.

In order to make sure that the fitted value of the $A_s^5$ parameter is contained in this range, it has been substituted in the \pdf by:
\begin{equation} \label{eq:As5.subst}
  A^5_s\to f\sqrt{3}\sqrt{F_S(1-F_S)F_T(1+P_1)}*F_{theo}
\end{equation}
where $f$ is a placeholder parameter defined in the range [-1;1].
The fit will be performed with respect to $f$, constrained in his range of validity, and the resulting values of $A_s^5$ will be obtained from $f$ by reversing Equation~\ref{eq:As5.subst}.
In this thesis this additional step will be kept implicit: when it will be stated that the likelihood is maximised as a function of $A_s^5$, it will be intended that it is maximised as a function of $f$.

\subsubsection{\pdf validity in the parameter space}
\label{sec:phys.bound}
Using the \pdf parameterisation described above it is not guaranteed that it is physical, i.e. positive in the whole ($\cos\theta_K$,$\cos\theta_l$,$\phi$) space.

In order to have a working fit sequence and reliable results, the values of the angular parameters that allow the \pdf to be physical need to be identified.
As explained in Section~\ref{sec:fitseq}, the angular parameters that are free to float in the fit to the data are $P_1$, $P_5'$, and $A^5_\mathrm{S}$, while the others, $F_\mathrm{L}$, $F_S$, and $A_\mathrm{S}$, are kept fixed to the results of the previous CMS analysis.
Thus, the region of validity of the \pdf is computed only in the three-dimensional space of the floating parameters, assuming the values of the others fixed like in the fit procedure.

The analytic boundaries for the pure P-wave component can be found in literature~\cite{Matias:2014jua}, and after the reduction to the $P_1$, $P_5'$ parameter space they are:
\begin{equation} \label{eq:anal.bound}
  (P_5')^2 - 1 < P_1 < 1
\end{equation}
without any dependence on the value of $F_\mathrm{L}$.

The pure S-wave component is always positive, as can be derived from Equation~\ref{eq:S-wave0} and from the range of definition of $F_S$, from 0 to 1.
In general, the interference terms can be also negative, so additional constraints are needed to guarantee the positiveness of the full \pdf.
Since these constraints are not available in literature in an analytically form, a numerical computation is needed to describe them.

%% This operation can be done analytically, probing the \pdf for some selected values of the angular variables.
%% From this procedure it is possible to get that the P1 physical range is [-1,1]; but no boundary can be extracted for $P_5'$ and $A^5_\mathrm{S}$, for which a numerical computation is needed.

For each bin, eight different physical boundaries in the ($P_1$,$P_5'$) parameter space have been computed:
\begin{itemize}
\item one boundary is obtained by requiring that only the P-wave component is positive; this is just a cross-check of the validity of Equation~\ref{eq:anal.bound}, since the same result is expected;
\item seven boundaries are obtained by requiring the whole \pdf to be positive, for seven different values of $A_s^5$; according the convention defined in Section~\ref{sec:As5.range}, the seven values of the placeholder parameter $f$ are chosen to be [-1, -2/3, -1/3, 0, 1/3, 2/3, 1].
\end{itemize}

To compute numerically each of these physical boundaries, the $P_1$,$P_5'$ space has been scanned with a grid of step 0.01 in both directions.
For each point of this \textit{parameter} grid, the values of $\cos\theta_K$, $\cos\theta_l$ and $\phi$ are moved on a three-dimensional grid with step 0.02 in each direction; if the \pdf is positive for all of the points of this \textit{angle} grid, the point in the $P_1$,$P_5'$ space is inside the physical region, otherwise it is outside.
%% The resulting region is equivalent to a $P_1$-dependent upper boundary for the absolute value of $P_5'$.

The boundaries of the resulting physical regions, plotted in the negative $P_5'$ sector only, are shown from Figure~\ref{fig:bound0} to Figure~\ref{fig:bound8}.
The physical region is the one with $P_5'$ values smaller, in module, than the plotted boundaries, and it is included between them and their projections in the positive $P_5'$ sector. 
The P-wave boundary is symmetrically reflected with respect to $P_5'=0$; the boundaries assigned to a value of $A_s^5$ can be reflected to the positive $P_5'$ sector, but the reflected boundary refers to the opposite value of $A_s^5$, as can be derived from the symmetries in Equation~\ref{eq:PDF-f2}.

\begin{figure}[!hbt]
  \centering
  \includegraphics[width=0.85\textwidth]{Figures/boundaries/bound_b0.pdf}
  \caption{Physical boundaries of the negative $P_5'$ sector of $q^2$ bin 0.
    Accordingly to the description in Section~\ref{sec:phys.bound}, the magenta line is the boundary of the P-wave physical region and the set of grey-scale lines are the boundaries of the total-PDF physical region, for different $A_s^5$ values (black for $f=-1$, lightest grey for $f=1$).}
  \label{fig:bound0}
\end{figure}

\begin{figure}[!hbt]
  \centering
  \includegraphics[width=0.85\textwidth]{Figures/boundaries/bound_b1.pdf}
  \caption{Physical boundaries of the negative $P_5'$ sector of $q^2$ bin 1. Accordingly to the description in Sec.~\ref{sec:phys.bound}, the magenta line is the boundary of the P-wave physical region and the set of grey-scale lines are the boundaries of the total-PDF physical region, for different $A_s^5$ values (black for $f=-1$, lightest grey for $f=1$).}
  \label{fig:bound1}
\end{figure}

\begin{figure}[!hbt]
  \centering
  \includegraphics[width=0.85\textwidth]{Figures/boundaries/bound_b2.pdf}
  \caption{Physical boundaries of the negative $P_5'$ sector of $q^2$ bin 2. Accordingly to the description in Sec.~\ref{sec:phys.bound}, the magenta line is the boundary of the P-wave physical region and the set of grey-scale lines are the boundaries of the total-PDF physical region, for different $A_s^5$ values (black for $f=-1$, lightest grey for $f=1$).}
  \label{fig:bound2}
\end{figure}

\begin{figure}[!hbt]
  \centering
  \includegraphics[width=0.85\textwidth]{Figures/boundaries/bound_b3.pdf}
  \caption{Physical boundaries of the negative $P_5'$ sector of $q^2$ bin 3. Accordingly to the description in Sec.~\ref{sec:phys.bound}, the magenta line is the boundary of the P-wave physical region and the set of grey-scale lines are the boundaries of the total-PDF physical region, for different $A_s^5$ values (black for $f=-1$, lightest grey for $f=1$).}
  \label{fig:bound3}
\end{figure}

\begin{figure}[!hbt]
  \centering
  \includegraphics[width=0.85\textwidth]{Figures/boundaries/bound_b5.pdf}
  \caption{Physical boundaries of the negative $P_5'$ sector of $q^2$ bin 5. Accordingly to the description in Sec.~\ref{sec:phys.bound}, the magenta line is the boundary of the P-wave physical region and the set of grey-scale lines are the boundaries of the total-PDF physical region, for different $A_s^5$ values (black for $f=-1$, lightest grey for $f=1$).}
  \label{fig:bound5}
\end{figure}

\begin{figure}[!hbt]
  \centering
  \includegraphics[width=0.85\textwidth]{Figures/boundaries/bound_b7.pdf}
  \caption{Physical boundaries of the negative $P_5'$ sector of $q^2$ bin 7. Accordingly to the description in Sec.~\ref{sec:phys.bound}, the magenta line is the boundary of the P-wave physical region and the set of grey-scale lines are the boundaries of the total-PDF physical region, for different $A_s^5$ values (black for $f=-1$, lightest grey for $f=1$).}
  \label{fig:bound7}
\end{figure}

\begin{figure}[!hbt]
  \centering
  \includegraphics[width=0.85\textwidth]{Figures/boundaries/bound_b8.pdf}
  \caption{Physical boundaries of the negative $P_5'$ sector of $q^2$ bin 8. Accordingly to the description in Sec.~\ref{sec:phys.bound}, the magenta line is the boundary of the P-wave physical region and the set of grey-scale lines are the boundaries of the total-PDF physical region, for different $A_s^5$ values (black for $f=-1$, lightest grey for $f=1$).}
  \label{fig:bound8}
\end{figure}
