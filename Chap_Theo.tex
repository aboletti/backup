\chapter{Analysis introduction}
\label{sec:theo}

%% temp

%% \section{Flavour Physics}
%% \label{sec:flav}

%% temp

%% \subsection{Flavour-changing neutral currents}
%% \label{sec:FCNC}

%% temp

\section{The decay B0 to K*0 mu+ mu-}
\label{sec:Kstmm}

%% temp

%%%%%%%%%%%%%%%%%%%%5


The angle $\theta_l$ is defined as the angle between the direction of the $\mu^+ $ ($\mu^- $) and the direction opposite that of the $\text{B}^0$ ($\bar{\text{B}^0}$) in the dimuon rest frame; the angle $\theta_\mathrm{K} $ is defined as the angle between the direction of the kaon and the direction opposite that of the $B^0$ ($\bar{\text{B}^0}$) in the $\text{K}^{*0}$ rest frame; the angle $\phi$ is the angle between the plane containing the $\mu^+ $ and $\mu^- $ and the plane containing the kaon and the pion from the $\PKst$ decay in the $\text{B}^0$ rest frame.
%%%%%%%%%%%%%%%%%%%%%%%%

Figure~\ref{fig:ske} illustrates the angular variables needed to describe the decay: $\theta_\ell$
is the angle between the positive (negative) muon momentum and the direction opposite to the
\PBz\ $\big(\PaBz\big)$ momentum in the dimuon rest frame,
$\theta_\PK$ is the angle between the kaon momentum and the direction opposite to
the \PBz\ $\big(\PaBz\big)$ momentum in the \cPKstz\ $\big(\cPAKstz\big)$ rest frame,
and $\varphi$ is the angle between the plane containing the two muons and the plane containing the
kaon and the pion in the \PBz\ rest frame.

\begin{figure*}[t]
  \begin{center}
    \includegraphics[width=0.99\textwidth]{SketchDecay.pdf}
    \caption{Illustration of the angular variables $\theta_\ell$ (left), $\theta_\PK$ (middle), and $\varphi$ (right) for the decay \BtoKstmumudecay.}
    \label{fig:ske}
  \end{center}
\end{figure*}

%%%%%%%%%%%%%%%%%%%%%%%%%555

\subsection{The angular decay rate}
\label{sec:decRate}

%%%%%%%%%%%%%%%%%%%%%%%%%%%%%%%%%%%%%%%
Using the notation of~\cite{Ball2009}, the differential angular distribution can be written as:
\begin{equation} \label{eq:Angular}
    \begin{split}
        \frac{1}{\mathrm{d}\Gamma/\mathrm{d}q^2}&\frac{\mathrm{d}^4\Gamma}{\mathrm{d}q^2 \mathrm{d}\cos\theta_l \mathrm{d}\cos\theta_\mathrm{K} \mathrm{d}\phi} =\frac{9}{32\pi}\left[\frac{3}{4}F_\mathrm{T}\sin^2\theta_\mathrm{K} + F_\mathrm{L}\cos^2\theta_\mathrm{K} \right.\\
            &\left.+\left(\frac{1}{4}F_\mathrm{T}\sin^2\theta_\mathrm{K}-F_\mathrm{L}\cos^2\theta_\mathrm{K}\right)\cos2\theta_l+\frac{1}{2}P_1F_\mathrm{T}\sin^2\theta_\mathrm{K}\sin^2\theta_l\cos 2\phi \right.\\
            &+\sqrt{F_\mathrm{T}F_\mathrm{L}}\left(\frac{1}{2}P_4'\sin2\theta_\mathrm{K}\sin2\theta_l\cos\phi+P_5'\sin2\theta_\mathrm{K}\sin\theta_l\cos\phi \right)\\
            &-\sqrt{F_\mathrm{T}F_\mathrm{L}}\left(P_6'\sin2\theta_\mathrm{K}\sin\theta_l\sin\phi-\frac{1}{2}P_8'\sin2\theta_\mathrm{K}\sin2\theta_l\sin\phi \right)\\
            &\left.+2P_2F_\mathrm{T}\sin^2\theta_\mathrm{K}\cos\theta_l-P_3F_\mathrm{T}\sin^2\theta_\mathrm{K}\sin^2\theta_l\sin2\phi \right]
    \end{split}
\end{equation}
where the $q^2$ dependent observables $P_i$ and $P'_i$ are optimized observables built via 
combinations of the $\text{K}^{*0}$ decay amplitudes, as defined in~\cite{Genon:Swave};
$F_\mathrm{L}$ is the longitudinal polarization fraction of the $\text{K}^{*0}$ and
$F_\mathrm{T}=(1-F_\mathrm{L})$.
%%%%%%%%%%%%%%%%%%%%%%%%%%%%%%%%%%%%5
\subsubsection{S-wave contamination}
\label{sec:S-waveform}
Although the $K^+\pi^-$ invariant mass must be consistent with a
$\text{K}^{*0}$, there can be contributions from a spinless (S-wave)
$K^+\pi^-$ combination. The presence of a $K^+\pi^-$
system in an S-wave configuration, due to a non-resonant contribution or
to feed through from $K^+\pi^-$ scalar resonances, results in additional
terms in the different angular distribution. Denoting the right-hand side
 of Eq.~\ref{eq:Angular} by $W_p$, the differential decay rate takes the form

\begin{equation} \label{eq:S-wave}
    \begin{split}
    (1-F_\mathrm{S})W_p + (W_s + W_{sp})
    \end{split}
\end{equation}

where 
\begin{equation} \label{eq:S-wave0}
    \begin{split}
      W_s = \frac{3}{16\pi} F_\mathrm{S}\sin^2\theta_l
    \end{split}
\end{equation}

and $W_{sp}$ is given from Eq.(44) in~\cite{Genon:Swave}. 
\begin{equation} \label{eq:S-wave1}
    \begin{split}
      W_{sp}= &\frac{3}{16 \pi}\left[ A_\mathrm{S}\sin^2\theta_l\cos\theta_\mathrm{K}+ A_\mathrm{S}^4\sin\theta_\mathrm{K}\sin2\theta_l\cos\phi\right.\\
            &+\left.A^5_\mathrm{S}\sin\theta_\mathrm{K}\sin\theta_l\cos\phi+A_\mathrm{S}^7\sin\theta_\mathrm{K}\sin\theta_l\sin\phi+A_\mathrm{S}^8\sin\theta_\mathrm{K}\sin2\theta_l\sin\phi\right]
    \end{split}
\end{equation}

where $F_\mathrm{S}$ is the fraction of the S-wave component in the
$\text{K}^{*0}$ mass window, and $W_{sp}$ contains all the
interference terms, $A_\mathrm{S}^i$ are the intererence amplitudes between the
S-wave and the P-wave decays\cite{Genon:Swave}.
%%%%%%%%%%%%%%%%%%%%%%%%%%%%%%%%%%


\subsection{The \pdf folding}
\label{sec:folding}

%% The angular distribution of the decay
%% $\text{B}^0 \rightarrow \text{K}^{*0} \mu^+ \mu^-$ can be described by
%% three angles ($\theta_l $, $\theta_\mathrm{K} $ and $\phi$) and the
%% invariant mass squared of the dimuon system ($q^2$).
Because of the
limited number of signal candidates in the data set, we didn't fit the
data to full differential distribution of Eq.~\ref{eq:Angular}. To
retrieve the interesting variables more effectively and to reduce the
number of fitting parameters, we performed the following transformations of the
decay-rate formulation.


Inspired by the derivations in ref \cite{LHCb2}\cite{Matias2012}, to
reduce the number of parameters in the fit, we "fold" the data
twice. ``Folding'' means that we divide the decay rate into different
parts, calculate them separately according to some symmetries and then
add them together to obtain the equivalent decay rates. If we take
consecutive steps of ``folding'', the similar expansions are used to
get the full PDFs.

Let us take the first folding as an example.
The first folding is at
$ \phi=0$ (for $\phi<0,\phi\rightarrow-\phi$, the $\phi$'s domain is reduced to
    (0,$\pi$)). To be more clear, we divide the decay rate $d\Gamma$
into two parts corresponding to $\phi>0$ and $\phi<0$,
i.e. $d\Gamma(\phi;\phi>0)$, and $d\Gamma(\phi;\phi<0)$:

\begin{equation} \label{eq:folding}
    \begin{split}
        d\hat{\Gamma} &= d\Gamma(\phi|\phi<0) + d\Gamma(\phi|\phi>0) \\
        & = f_0(\phi|\phi\rightarrow-\phi) + f_0(\phi|\phi>0) \\
        & = f_0(\cos\phi, -\sin\phi) + f_0(\cos\phi, \sin\phi)
    \end{split}
\end{equation}


According to trigonometric identities $\cos(-\phi) = \cos\phi $,
$\sin(-\phi) = -\sin\phi $, we can cancel the terms 
that are odd under this transformation
%% containing $\sin\phi$
. Eq.~\ref{eq:Angular} now reads:

\begin{equation} \label{eq:fold1}
    \begin{split}
        \frac{1}{\mathrm{d}\Gamma/\mathrm{d}q^2}&\frac{\mathrm{d}^4\Gamma}{\mathrm{d}q^2 \mathrm{d}\cos\theta_l \mathrm{d}\cos\theta_\mathrm{K} \mathrm{d}\phi} = \frac{9}{16\pi}\left[\frac{3}{4}F_\mathrm{T}\sin^2\theta_\mathrm{K} + F_\mathrm{L}\cos^2\theta_\mathrm{K} \right.\\
            &\left.+(\frac{1}{4}F_\mathrm{T}\sin^2\theta_\mathrm{K}-F_\mathrm{L}\cos^2\theta_\mathrm{K})\cos2\theta_l+\frac{1}{2}P_1F_\mathrm{T}\sin^2\theta_\mathrm{K}\sin^2\theta_l\cos 2\phi \right.\\
            &+\sqrt{F_\mathrm{T}F_\mathrm{L}}(\frac{1}{2}P_4'\sin2\theta_\mathrm{K}\sin2\theta_l\cos\phi+P_5'\sin2\theta_\mathrm{K}\sin\theta_l\cos\phi )\\
            &\left.+2P_2F_\mathrm{T}\sin^2\theta_\mathrm{K}\cos\theta_l \right]
    \end{split}
\end{equation}

The second folding is performed at $\theta_l = \pi/2$ (for
$\theta_l>\pi/2,\theta_l\rightarrow \pi- \theta_l$). The domain of
$\theta_l$ is reduced to (0,$\pi$/2). According to $\cos(\pi-\theta_l) = -
\cos\theta_l$ and $\sin(\pi-\theta_l) = \sin\theta_l$, we can cancel the terms that are odd under this transformation.
 %% proportional to $P_4'$, which contains $\sin 2\theta_l$.


\begin{equation} \label{eq:fold2}
    \begin{split}
        \frac{1}{\mathrm{d}\Gamma/\mathrm{d}q^2}&\frac{\mathrm{d}^4\Gamma}{\mathrm{d}q^2 \mathrm{d}\cos\theta_l \mathrm{d}\cos\theta_\mathrm{K} \mathrm{d}\phi} = \frac{9}{8\pi}\left[\frac{3}{4}F_\mathrm{T}\sin^2\theta_\mathrm{K} + F_\mathrm{L}\cos^2\theta_\mathrm{K} \right.\\
            &\left.+(\frac{1}{4}F_\mathrm{T}\sin^2\theta_\mathrm{K}-F_\mathrm{L}\cos^2\theta_\mathrm{K})\cos2\theta_l+\frac{1}{2}P_1F_\mathrm{T}\sin^2\theta_\mathrm{K}\sin^2\theta_l\cos 2\phi \right.\\
            &\left.+\sqrt{F_\mathrm{T}F_\mathrm{L}}P_5'\sin2\theta_\mathrm{K}\sin\theta_l\cos\phi  \right]
    \end{split}
\end{equation}
%%%%%%%%%%%%%%%%%%%%%%%%%%%%%%%%

 For S-wave and the
interference terms, we do the same transformation as P-wave, after the
first ``folding'', it can reads:

\begin{equation} \label{eq:S-fold1}
    \begin{split}
        \frac{1}{\mathrm{d}\Gamma/\mathrm{d}q^2}&\frac{\mathrm{d}^4\Gamma}{\mathrm{d}q^2 \mathrm{d}\cos\theta_l \mathrm{d}\cos\theta_\mathrm{K} \mathrm{d}\phi} = \frac{3}{8\pi}\left[F_\mathrm{S}\sin^2\theta_l+ A_\mathrm{S}\sin^2\theta_l\cos\theta_\mathrm{K}\right.\\
            &+\left. A_\mathrm{S}^4\sin\theta_\mathrm{K}\sin2\theta_l\cos\phi + A^5_\mathrm{S}\sin\theta_\mathrm{K}\sin\theta_l\cos\phi\right]
    \end{split}
\end{equation}

After the second ``folding'', it reads:
\begin{equation} \label{eq:S-fold2}
    \begin{split}
      \frac{1}{\mathrm{d}\Gamma/\mathrm{d}q^2}&\frac{\mathrm{d}^4\Gamma}{\mathrm{d}q^2 \mathrm{d}\cos\theta_l \mathrm{d}\cos\theta_\mathrm{K} \mathrm{d}\phi} = \\
      &\frac{3}{4\pi}\left[F_\mathrm{S}\sin^2\theta_l+A_\mathrm{S}\sin^2\theta_l\cos\theta_\mathrm{K}+A^5_\mathrm{S}\sin\theta_\mathrm{K}\sin\theta_l\cos\phi\right]
    \end{split}
\end{equation}
%%%%%%%%%%%%%%%%%%%%%%%%%%%%%%%%%%5

After the two folding are performed, the angular distribution can be
written, using  Eq.~\ref{eq:fold2} and Eq.~\ref{eq:S-fold2} as:

\begin{equation} \label{eq:PDF-f2}
  \begin{split}
    \frac{1}{\mathrm{d}\Gamma/\mathrm{d}q^2}&\frac{\mathrm{d}^4\Gamma}{\mathrm{d}q^2 \mathrm{d}\cos\theta_l \mathrm{d}\cos\theta_\mathrm{K} \mathrm{d}\phi} = \\
    &\frac{9}{8\pi}\left\{\frac{2}{3}\left[ (F_\mathrm{S}+A_\mathrm{S}\cos\theta_\mathrm{K})\left(1-\cos^2\theta_l\right) + A^5_\mathrm{S}\sqrt{1-\cos^2\theta_\mathrm{K}}\sqrt{1-\cos^2\theta_l}\cos\phi \right] \right.\\
    & + \left(1 - F_\mathrm{S}\right)\left[2F_\mathrm{L}\cos^2\theta_\mathrm{K}\left(1-\cos^2\theta_l\right)+\frac{1}{2}\left(1-F_\mathrm{L}\right)\left(1-\cos^2\theta_\mathrm{K}\right)\left(1+\cos^2\theta_l\right) \right.\\
      & + \frac{1}{2}P_1(1-F_\mathrm{L})(1-\cos^2\theta_\mathrm{K})(1-\cos^2\theta_l)\cos 2\phi \\
      & \left.\left. + 2P_5'\cos\theta_\mathrm{K}\sqrt{F_\mathrm{L}\left(1-F_\mathrm{L}\right)}\sqrt{1-\cos^2\theta_\mathrm{K}}\sqrt{1-\cos^2\theta_l}\cos\phi\right]\right\}
  \end{split}
\end{equation}

Now we have 6 parameters, they are $F_\mathrm{L}$, $F_S$, $P_1$, $P_5'$, $A_\mathrm{S}$
and $A^5_\mathrm{S}$.
%%%%%%%%%%%%%%%%%%%%%%%%%%%%%%%%%%%55

\subsection{Parameter constrains}
\label{sec:bound}

\subsubsection{Range of definition of interference terms}
\label{sec:As5.range}
Due to their nature, the value of the interference terms $A_s$ and $A_s^5$ is limited by the amplitude of the pure P-wave and S-wave components~\cite{Genon:Swave}. Their allowded ranges are the following:
\begin{equation} \label{eq:As.range}
  |A_s|<2\sqrt{3}\sqrt{F_S(1-F_S)F_L}*F_{theo}
\end{equation}
\begin{equation} \label{eq:As5.range}
  |A^5_s|<\sqrt{3}\sqrt{F_S(1-F_S)F_T(1+P_1)}*F_{theo}
\end{equation}
where $F_{theo}$ is a constant factor that depends on the selection cuts applied to the $K\pi$ system mass, and in this analysis is 0.89.

To make sure that the fitted value of $A_s^5$ is contained in this range, it has been substituted in the PDF by
\begin{equation} \label{eq:As5.subst}
  A^5_s\to f\sqrt{3}\sqrt{F_S(1-F_S)F_T(1+P_1)}*F_{theo}
\end{equation}
where $f$ is a placeholder parameter defined in the range [-1;1]. The fit is then performed with respect to $f$ instead of $A_s^5$.

\subsubsection{P.D.F validity in the parameter space}
\label{sec:phys.bound}
Using the P.D.F. parametrization described above, it is not guaranteed that it is physical (i.e. positive in the whole ($\cos\theta_K$,$\cos\theta_l$,$\phi$) space).

In order to have a working fit sequence and reliable results, we need to identify which values of the parameters allow the P.D.F. to be physical. Since, as explain in Sec.~\ref{sec:finalform}, the free parameters will be $P_1$, $P_5'$, $A^5_\mathrm{S}$, we will compute only their physical regions, keeping the other parameters fixed at the value from the BPH-13-010 analysis result.

This operation can be done analytically, probing the P.D.F. for some selected values of the angular variables. From this procedure it is possible to get that the P1 physical range is [-1,1]; but no boundary can be extracted for $P_5'$ and $A^5_\mathrm{S}$, for which a numerical computation is needed.

To compute numerically the boundary of the physical region, the $P_1/P_5'$ space has been scanned with a grid of step 0.01 in both directions. For each point of this grid, the values of $\cos\theta_K$, $\cos\theta_l$ and $\phi$ are moved on a 3D grid with step 0.02; if the PDF is positive for all of the points of this second grid, the point in the $P_1/P_5'$ space is inside the physical region, otherwise it is outside. The resulting region is equivalent to a $P_1$-dependent upper boundary for the absolute value of $P_5'$.

For each bin, this phisical region has been computed eight times:
\begin{itemize}
\item once, by requiring that only the P-wave component is positive; in this case the result is independent from the nuisance parameter $A_s^5$.
\item seven times, by requiring the whole PDF to be positive, for different values of $A_s^5$; according the convention defined in Sec.~\ref{sec:As5.range}, the seven values of the placeholder parameter $f$ are {-1, -2/3, -1/3, 0, 1/3, 2/3, 1}.
\end{itemize}

The boundaries of this regions, plotted in the negative $P_5'$ sector only (the boundaries are symmetrical with respect to $P_5'=0$), are shown from Figure~\ref{fig:bound0} to Figure~\ref{fig:bound8}.

\begin{figure}[!hbt]
  \centering
  \includegraphics[width=0.85\textwidth]{Figures/boundaries/bound_b0.pdf}
  \caption{Physical boundaries of the negative $P_5'$ sector of $q^2$ bin 0. Accordingly to the description in Sec.~\ref{sec:phys.bound}, the magenta line is the boundary of the P-wave physical region and the set of gray-scale lines are the boundaries of the total-PDF physical region, for different $A_s^5$ values (black for $f=-1$, lightest gray for $f=1$).}
  \label{fig:bound0}
\end{figure}

\begin{figure}[!hbt]
  \centering
  \includegraphics[width=0.85\textwidth]{Figures/boundaries/bound_b1.pdf}
  \caption{Physical boundaries of the negative $P_5'$ sector of $q^2$ bin 1. Accordingly to the description in Sec.~\ref{sec:phys.bound}, the magenta line is the boundary of the P-wave physical region and the set of gray-scale lines are the boundaries of the total-PDF physical region, for different $A_s^5$ values (black for $f=-1$, lightest gray for $f=1$).}
  \label{fig:bound1}
\end{figure}

\begin{figure}[!hbt]
  \centering
  \includegraphics[width=0.85\textwidth]{Figures/boundaries/bound_b2.pdf}
  \caption{Physical boundaries of the negative $P_5'$ sector of $q^2$ bin 2. Accordingly to the description in Sec.~\ref{sec:phys.bound}, the magenta line is the boundary of the P-wave physical region and the set of gray-scale lines are the boundaries of the total-PDF physical region, for different $A_s^5$ values (black for $f=-1$, lightest gray for $f=1$).}
  \label{fig:bound2}
\end{figure}

\begin{figure}[!hbt]
  \centering
  \includegraphics[width=0.85\textwidth]{Figures/boundaries/bound_b3.pdf}
  \caption{Physical boundaries of the negative $P_5'$ sector of $q^2$ bin 3. Accordingly to the description in Sec.~\ref{sec:phys.bound}, the magenta line is the boundary of the P-wave physical region and the set of gray-scale lines are the boundaries of the total-PDF physical region, for different $A_s^5$ values (black for $f=-1$, lightest gray for $f=1$).}
  \label{fig:bound3}
\end{figure}

\begin{figure}[!hbt]
  \centering
  \includegraphics[width=0.85\textwidth]{Figures/boundaries/bound_b5.pdf}
  \caption{Physical boundaries of the negative $P_5'$ sector of $q^2$ bin 5. Accordingly to the description in Sec.~\ref{sec:phys.bound}, the magenta line is the boundary of the P-wave physical region and the set of gray-scale lines are the boundaries of the total-PDF physical region, for different $A_s^5$ values (black for $f=-1$, lightest gray for $f=1$).}
  \label{fig:bound5}
\end{figure}

\begin{figure}[!hbt]
  \centering
  \includegraphics[width=0.85\textwidth]{Figures/boundaries/bound_b7.pdf}
  \caption{Physical boundaries of the negative $P_5'$ sector of $q^2$ bin 7. Accordingly to the description in Sec.~\ref{sec:phys.bound}, the magenta line is the boundary of the P-wave physical region and the set of gray-scale lines are the boundaries of the total-PDF physical region, for different $A_s^5$ values (black for $f=-1$, lightest gray for $f=1$).}
  \label{fig:bound7}
\end{figure}

\begin{figure}[!hbt]
  \centering
  \includegraphics[width=0.85\textwidth]{Figures/boundaries/bound_b8.pdf}
  \caption{Physical boundaries of the negative $P_5'$ sector of $q^2$ bin 8. Accordingly to the description in Sec.~\ref{sec:phys.bound}, the magenta line is the boundary of the P-wave physical region and the set of gray-scale lines are the boundaries of the total-PDF physical region, for different $A_s^5$ values (black for $f=-1$, lightest gray for $f=1$).}
  \label{fig:bound8}
\end{figure}
