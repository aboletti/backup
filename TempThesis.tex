%% \documentclass[12pt,oneside,a4paper]{report}
\documentclass[a4paper,11pt,twoside]{book}

\usepackage{amsmath}
\usepackage{amssymb}
\usepackage{siunitx}
\usepackage{xspace}
\usepackage{amsmath}
\usepackage{mathtools}
\usepackage{graphicx}
\usepackage{subfigure}
\usepackage{lineno}
%% \linenumbers

%%%%% NEW COMMANDS
\newcommand{\MeV}{\ensuremath{\,\textrm{MeV}}\xspace}
\newcommand{\GeV}{\ensuremath{\,\textrm{GeV}}\xspace}
\newcommand{\TeV}{\ensuremath{\,\textrm{TeV}}\xspace}

\newcommand{\stat}{\ensuremath{\,\textrm{(stat)}}\xspace}
\newcommand{\syst}{\ensuremath{\,\textrm{(syst)}}\xspace}

\newcommand{\pt}{\ensuremath{p_T}\xspace}
\newcommand{\met}{\ensuremath{\mathrm{m}E_T}\xspace}

\DeclarePairedDelimiter\abs{\lvert}{\rvert}%

\newcommand{\PaBz}{\ensuremath{\overline{\textrm{B}}{}^{0}}\xspace}
\newcommand{\PBz}{\ensuremath{{\textrm{B}}{}^{0}}\xspace}
\newcommand{\PBs}{\ensuremath{{\textrm{B}}{}^{0}_{\mathrm{s}}}\xspace}
\newcommand{\PBp}{\ensuremath{{\textrm{B}}{}^{+}}\xspace}
\newcommand{\PB}{\ensuremath{{\textrm{B}}}\xspace}
\newcommand{\PD}{\ensuremath{{\textrm{D}}}\xspace}
\newcommand{\Pgmp}{\ensuremath{\mu^{+}}\xspace}
\newcommand{\Pgmm}{\ensuremath{\mu^{-}}\xspace}
\newcommand{\Ppsi}{\ensuremath{\psi}\xspace}
\newcommand{\JPsi}{\ensuremath{\textrm{J}/\psi}\xspace}
\newcommand{\PJpsi}{\ensuremath{\textrm{J}/\psi}\xspace}
\newcommand{\cPJgy}{\ensuremath{\textrm{J}/\psi}\xspace}
\newcommand{\PKp}{\ensuremath{\textrm{K}^{+}}\xspace}
\newcommand{\PKm}{\ensuremath{\textrm{K}^{-}}\xspace}
\newcommand{\Pgpp}{\ensuremath{\pi^{+}}\xspace}
\newcommand{\Pgpm}{\ensuremath{\pi^{-}}\xspace}
\newcommand{\Pgf}{\ensuremath{\phi}\xspace}
\newcommand{\Pgg}{\ensuremath{\gamma}\xspace}
\newcommand{\PKpi}{\ensuremath{\textrm{K}\pi}\xspace}

\newcommand{\BR}{\ensuremath{\mathcal{B}r}} %
\newcommand{\Ks}{\ensuremath{\textrm{K}^{*0}}\xspace} %
\newcommand{\BKsmm}{\ensuremath{\PBz\to{\Ks\mu\mu}\,}\xspace} %
\newcommand{\KsKpi}{\ensuremath{{\Ks\to\textrm{K}^+\pi^-}\,}\xspace} %
\newcommand{\BKpimm}{\ensuremath{\PBz\to{\Ks(\to{\textrm{K}^+\pi^-})\mu\mu}\,}\xspace} %

\newcommand{\BKsJ}{\ensuremath{\PBz\to{\Ks(\to{\textrm{K}^+\pi^-})\JPsi(\to\mu\mu)}\,}\xspace} %
\newcommand{\BKsPsip}{\ensuremath{\PBz\to{\Ks(\to{\textrm{K}^+\pi^-})\psi'(\to\mu\mu)}\,}\xspace} %
\newcommand{\Pfp}{\ensuremath{P'_5}\xspace} %
\newcommand{\TL}{\ensuremath{\theta_l}\xspace} %
\newcommand{\TK}{\ensuremath{\theta_K}\xspace} %
\newcommand{\cTL}{\ensuremath{\cos\theta_l}\xspace} %
\newcommand{\cTK}{\ensuremath{\cos\theta_K}\xspace} %

\newcommand{\PHI}{\ensuremath{\phi}\xspace} %

\newcommand{\pdfKDE}{\ensuremath{\mathtt{pdf}_{\mathrm{KDE}}}\xspace} %
\newcommand{\pdfKDEx}{\ensuremath{\mathtt{pdf}_{\mathrm{KDE}}(x)}\xspace} %
\newcommand{\pdf}{\texttt{pdf}\xspace} %
\newcommand{\pdfs}{\texttt{pdf}s\xspace} %
\newcommand{\e}[1]{\ensuremath{\cdot{10}^{#1}}}%


%%% from the paper
\newcommand{\PKst}{\ensuremath{\textrm{K}^\ast}\xspace}
\newcommand{\cPKst}{\ensuremath{\textrm{K}^\ast}\xspace}
\newcommand{\cPKstz}{\ensuremath{\textrm{K}^{\ast0}}\xspace}
\newcommand{\cPAKstz}{\ensuremath{\overline{\textrm{K}}{}^{\ast0}}\xspace}
\newcommand{\BtoKstmumu}{\ensuremath{\PBz\to\cPKstz \Pgmp \Pgmm}\xspace}
\newcommand{\BtoKstJpsi}{\ensuremath{\PBz\to\cPJgy \cPKstz}\xspace}
\newcommand{\BtoKstpsip}{\ensuremath{\PBz\to\psi' \cPKstz}\xspace}
\newcommand{\BtoKstJpsimumu}{\ensuremath{\PBz\to\cPJgy(\Pgmp \Pgmm) \cPKstz}\xspace}
\newcommand{\BtoKstpsipmumu}{\ensuremath{\PBz\to\psi'(\Pgmp \Pgmm) \cPKstz}\xspace}
\newcommand{\BtoKstmumudecay}{\ensuremath{\PBz\to\cPKstz(\PKp \Pgpm) \Pgmp \Pgmm}\xspace}
\newcommand{\BtoKstJpsidecay}{\ensuremath{\PBz\to\cPJgy(\Pgmp \Pgmm) \cPKstz(\PKp \Pgpm)}\xspace}
\newcommand{\BtoKstpsipdecay}{\ensuremath{\PBz\to\psi'(\Pgmp \Pgmm) \cPKstz(\PKp \Pgpm)}\xspace}
\newcommand{\Kstmumudecay}{\ensuremath{\cPKstz(\PKp \Pgpm) \Pgmp \Pgmm}\xspace}
\newcommand{\KstJpsidecay}{\ensuremath{\cPJgy(\Pgmp \Pgmm) \cPKstz(\PKp \Pgpm)}\xspace}
\newcommand{\Kstpsipdecay}{\ensuremath{\psi'(\Pgmp \Pgmm) \cPKstz(\PKp \Pgpm)}\xspace}


\graphicspath{{./Note1/}{./Paper/}{./Master/}}

%%%%%%%%%%%%%%% INFOS

%% \title{Measurement of the P5' angular observable of  the decay B0 to K*mu+mu- at 8 TeV}
\title{Measurement of angular parameters from the decay $\mathrm{B}^0 \to \mathrm{K}^{*0} \mu^+ \mu^-$ in proton-proton collisions at $\sqrt{s}=8~\mathrm{TeV}$}

% >> Authors
%% \address[pdinfn]{INFN sez. Padova - Italy}
%% \address[pduni]{Universit\`a Padova and INFN - Italy}
%% \author[pdinfn,pduni]{Alessio Boletti}
\author{Alessio Boletti}

% >> Date
\date{\today}

% >> PDF Metadata
%% \hypersetup{%
%% pdfauthor={Alessio Boletti, Stefano Lacaprara, Geng Chen, Mauro Dinardo, Linwei Li, Dayong Wang},%
%% pdftitle={Full angular analysis B0 to Kstar muon-muon},%
%% pdfsubject={Efficiency for B0 to Ks mumu},%
%% pdfkeywords={CMS, B-physics, FCNC, B0}}


%%%%%%%% CONTENTS

\begin{document}

\begin{titlepage}
  %% \begin{center}
  {\center\includegraphics[width=0.2\textwidth,clip]{Images/unipd-logo.png}\\[1cm]}
  %% \textsc{\Large Universit\`{a} degli Studi di Padova}\\
  {\large Sede Amministrativa: Universit\`{a} degli Studi di Padova}\\
  %\textsc{\Large Facolt\`{a} di Scienze MM.FF.NN.}\\
  %% \textsc

  {\large Dipartimento di Fisica ed Astronomia ``G. Galilei''}\\

  \vspace{1 cm}
  
  %% \textsc
  {\large CORSO DI DOTTORATO DI RICERCA IN: FISICA}\\

  {\large CICLO XXX}\\

  \vspace{1 cm}
  
  %\vspace{0.4cm} 
  %\HRule \\[0.4cm]
  {\Large \center \bfseries Measurement of angular parameters\\from the decay $\mathrm{B}^0 \to \mathrm{K}^{*0} \mu^+ \mu^-$\\in proton-proton collisions at $\sqrt{s}=8~\mathrm{TeV}$}\\% Titolo

  \vspace{1cm}
  %\HRule \\[1.4cm]

  {\small Tesi redatta con il contributo finanziario dell'Istituto Nazionale di Fisica Nucleare}\\

  \vspace{1 cm}
  \begin{flushleft}
    \begin{minipage}{0.8\textwidth}
      \begin{flushleft} \large
        \textbf{Coordinatore:} Ch.mo Prof. Gianguido Dall'Agata\\
        \vspace{5mm}
        \textbf{Supervisore:} Dr. Paolo Ronchese\\
        \vspace{5mm}
        \textbf{Co-Supervisore:} Dr. Stefano Lacaprara\\
        %\vspace{0.5cm} %Relatore    MODIFICARE
        \vspace{1cm} %Relatore    MODIFICAR
        %\emph{Controrelatore:} \\
        %Prof. Sabino \textsc{Matarrese} %Correlatore    MODIFICARE 
      \end{flushleft}
    \end{minipage}
  \end{flushleft}
  \vspace{1cm}
  \begin{flushright}
    \begin{minipage}{0.5\textwidth}
      \begin{flushright}
        \large \textbf{Dottorando:} Alessio Boletti\\
      \end{flushright}
    \end{minipage}
  \end{flushright}
  % \begin{minipage}{0.4\textwidth}
  % \begin{flushright} \large
  % \end{flushright}
  % \end{minipage}
  %% \vfill
  %% \textsc{\Large Anno Accademico 2013/2014}
  %% \end{center}
\end{titlepage}

%% \maketitle %maketitle comes after all the front information has been supplied

%% % >> Abstract
%% \abstract{
%%   Does a thesis need an abstract?
%% }

\tableofcontents
\clearpage

\chapter*{Abstract}
 Angular distributions of the decay $\mathrm{B}^0 \to \mathrm{K}^{*0} \mu^ +\mu^-$ are studied using a sample of proton-proton collisions at $\sqrt{s}=8~\mathrm{TeV}$
 collected with the CMS detector at the LHC, corresponding to an integrated luminosity of $20.5~\mathrm{fb}^{-1}$.
 An angular analysis is performed to determine the $P_1$ and $P_5'$ parameters, where the $P_5'$ parameter is of particular interest because of recent measurements
 that indicate a potential discrepancy with the standard model predictions. Based on a sample of 1397 signal events, the $P_1$ and $P_5'$ parameters are determined
 as a function of the dimuon invariant mass squared. The measurements are in agreement with predictions based on the standard model.

\clearpage

%% \section{Introduction}

%%%%%%%%%%%%%%%%

%%%%%%%%%%%%%%55

%% The $q^2$ bins to be used in the analysis are defined in the Table~\ref{tab:q2 bins}.
%% They are chosen in such way to match the measurements performed in previous experiments.

%% Nine bins of $q^2$ are used in the analysis, including two which are dominated
%% by the control samples.
The $q^2$ range used in this analysis extends from 1\GeV$^2$ to 19\GeV$^2$ and it is divided in nine bins, defined in Table~\ref{tab:q2bins}.
The bin definition is the same used in the previous angular analysis on the same datset and it is the result of a compromise between being coherent with the definition used in the previous measurements and having an expected signal yield homogeneously distributed over the $q^2$ bins.

\begin{table}[!htb]
  \begin{center}
    \begin{small}
      \caption{Range definition of the dimuon invariant mass bins. Both $J/\psi$ and $\psi'$ regions, namely $q^2$ bins No.4 and No.6, are used as control channels.
        \label{tab:q2bins}}
      \begin{tabular}{c|l}
        Bin index & $q^2$ range ($Gev^2/c^4$) \\
        \hline
        0 & 1-2  \\
        1 & 2-4.3  \\
        2 & 4.3-6  \\
        3 & 6-8.68   \\
        4 & 8.68-10.09 ($J/\Psi$ region) \\
        5 & 10.09-12.86\\
        6 & 12.86-14.18 ($\Psi'$ region)\\
        7 & 14.18-16\\
        8 & 16-19\\
      \end{tabular}
    \end{small}
  \end{center}
\end{table}

The selection criteria and the analysis techniques are identical for any $q^2$ bin and in each of them the analysis is performed independently.
The two $q^2$ bins containing the control channel regions are not used to fit the signal events.


%% \clearpage

\chapter{Analysis introduction}
\label{sec:theo}

%% temp

%% \section{Flavour Physics}
%% \label{sec:flav}

%% temp

%% \subsection{Flavour-changing neutral currents}
%% \label{sec:FCNC}

%% temp

\section{The decay B0 to K*0 mu+ mu-}
\label{sec:Kstmm}

%% temp

%%%%%%%%%%%%%%%%%%%%5


The angle $\theta_l$ is defined as the angle between the direction of the $\mu^+ $ ($\mu^- $) and the direction opposite that of the $\text{B}^0$ ($\bar{\text{B}^0}$) in the dimuon rest frame; the angle $\theta_\mathrm{K} $ is defined as the angle between the direction of the kaon and the direction opposite that of the $B^0$ ($\bar{\text{B}^0}$) in the $\text{K}^{*0}$ rest frame; the angle $\phi$ is the angle between the plane containing the $\mu^+ $ and $\mu^- $ and the plane containing the kaon and the pion from the $\PKst$ decay in the $\text{B}^0$ rest frame.
%%%%%%%%%%%%%%%%%%%%%%%%

Figure~\ref{fig:ske} illustrates the angular variables needed to describe the decay: $\theta_\ell$
is the angle between the positive (negative) muon momentum and the direction opposite to the
\PBz\ $\big(\PaBz\big)$ momentum in the dimuon rest frame,
$\theta_\PK$ is the angle between the kaon momentum and the direction opposite to
the \PBz\ $\big(\PaBz\big)$ momentum in the \cPKstz\ $\big(\cPAKstz\big)$ rest frame,
and $\varphi$ is the angle between the plane containing the two muons and the plane containing the
kaon and the pion in the \PBz\ rest frame.

\begin{figure*}[t]
  \begin{center}
    \includegraphics[width=0.99\textwidth]{SketchDecay.pdf}
    \caption{Illustration of the angular variables $\theta_\ell$ (left), $\theta_\PK$ (middle), and $\varphi$ (right) for the decay \BtoKstmumudecay.}
    \label{fig:ske}
  \end{center}
\end{figure*}

%%%%%%%%%%%%%%%%%%%%%%%%%555

\subsection{The angular decay rate}
\label{sec:decRate}

%%%%%%%%%%%%%%%%%%%%%%%%%%%%%%%%%%%%%%%
Using the notation of~\cite{Ball2009}, the differential angular distribution can be written as:
\begin{equation} \label{eq:Angular}
    \begin{split}
        \frac{1}{\mathrm{d}\Gamma/\mathrm{d}q^2}&\frac{\mathrm{d}^4\Gamma}{\mathrm{d}q^2 \mathrm{d}\cos\theta_l \mathrm{d}\cos\theta_\mathrm{K} \mathrm{d}\phi} =\frac{9}{32\pi}\left[\frac{3}{4}F_\mathrm{T}\sin^2\theta_\mathrm{K} + F_\mathrm{L}\cos^2\theta_\mathrm{K} \right.\\
            &\left.+\left(\frac{1}{4}F_\mathrm{T}\sin^2\theta_\mathrm{K}-F_\mathrm{L}\cos^2\theta_\mathrm{K}\right)\cos2\theta_l+\frac{1}{2}P_1F_\mathrm{T}\sin^2\theta_\mathrm{K}\sin^2\theta_l\cos 2\phi \right.\\
            &+\sqrt{F_\mathrm{T}F_\mathrm{L}}\left(\frac{1}{2}P_4'\sin2\theta_\mathrm{K}\sin2\theta_l\cos\phi+P_5'\sin2\theta_\mathrm{K}\sin\theta_l\cos\phi \right)\\
            &-\sqrt{F_\mathrm{T}F_\mathrm{L}}\left(P_6'\sin2\theta_\mathrm{K}\sin\theta_l\sin\phi-\frac{1}{2}P_8'\sin2\theta_\mathrm{K}\sin2\theta_l\sin\phi \right)\\
            &\left.+2P_2F_\mathrm{T}\sin^2\theta_\mathrm{K}\cos\theta_l-P_3F_\mathrm{T}\sin^2\theta_\mathrm{K}\sin^2\theta_l\sin2\phi \right]
    \end{split}
\end{equation}
where the $q^2$ dependent observables $P_i$ and $P'_i$ are optimized observables built via 
combinations of the $\text{K}^{*0}$ decay amplitudes, as defined in~\cite{Genon:Swave};
$F_\mathrm{L}$ is the longitudinal polarization fraction of the $\text{K}^{*0}$ and
$F_\mathrm{T}=(1-F_\mathrm{L})$.
%%%%%%%%%%%%%%%%%%%%%%%%%%%%%%%%%%%%5
\subsubsection{S-wave contamination}
\label{sec:S-waveform}
Although the $K^+\pi^-$ invariant mass must be consistent with a
$\text{K}^{*0}$, there can be contributions from a spinless (S-wave)
$K^+\pi^-$ combination. The presence of a $K^+\pi^-$
system in an S-wave configuration, due to a non-resonant contribution or
to feed through from $K^+\pi^-$ scalar resonances, results in additional
terms in the different angular distribution. Denoting the right-hand side
 of Eq.~\ref{eq:Angular} by $W_p$, the differential decay rate takes the form

\begin{equation} \label{eq:S-wave}
    \begin{split}
    (1-F_\mathrm{S})W_p + (W_s + W_{sp})
    \end{split}
\end{equation}

where 
\begin{equation} \label{eq:S-wave0}
    \begin{split}
      W_s = \frac{3}{16\pi} F_\mathrm{S}\sin^2\theta_l
    \end{split}
\end{equation}

and $W_{sp}$ is given from Eq.(44) in~\cite{Genon:Swave}. 
\begin{equation} \label{eq:S-wave1}
    \begin{split}
      W_{sp}= &\frac{3}{16 \pi}\left[ A_\mathrm{S}\sin^2\theta_l\cos\theta_\mathrm{K}+ A_\mathrm{S}^4\sin\theta_\mathrm{K}\sin2\theta_l\cos\phi\right.\\
            &+\left.A^5_\mathrm{S}\sin\theta_\mathrm{K}\sin\theta_l\cos\phi+A_\mathrm{S}^7\sin\theta_\mathrm{K}\sin\theta_l\sin\phi+A_\mathrm{S}^8\sin\theta_\mathrm{K}\sin2\theta_l\sin\phi\right]
    \end{split}
\end{equation}

where $F_\mathrm{S}$ is the fraction of the S-wave component in the
$\text{K}^{*0}$ mass window, and $W_{sp}$ contains all the
interference terms, $A_\mathrm{S}^i$ are the intererence amplitudes between the
S-wave and the P-wave decays\cite{Genon:Swave}.
%%%%%%%%%%%%%%%%%%%%%%%%%%%%%%%%%%


\subsection{The \pdf folding}
\label{sec:folding}

%% The angular distribution of the decay
%% $\text{B}^0 \rightarrow \text{K}^{*0} \mu^+ \mu^-$ can be described by
%% three angles ($\theta_l $, $\theta_\mathrm{K} $ and $\phi$) and the
%% invariant mass squared of the dimuon system ($q^2$).
Because of the
limited number of signal candidates in the data set, we didn't fit the
data to full differential distribution of Eq.~\ref{eq:Angular}. To
retrieve the interesting variables more effectively and to reduce the
number of fitting parameters, we performed the following transformations of the
decay-rate formulation.


Inspired by the derivations in ref \cite{LHCb2}\cite{Matias2012}, to
reduce the number of parameters in the fit, we "fold" the data
twice. ``Folding'' means that we divide the decay rate into different
parts, calculate them separately according to some symmetries and then
add them together to obtain the equivalent decay rates. If we take
consecutive steps of ``folding'', the similar expansions are used to
get the full PDFs.

Let us take the first folding as an example.
The first folding is at
$ \phi=0$ (for $\phi<0,\phi\rightarrow-\phi$, the $\phi$'s domain is reduced to
    (0,$\pi$)). To be more clear, we divide the decay rate $d\Gamma$
into two parts corresponding to $\phi>0$ and $\phi<0$,
i.e. $d\Gamma(\phi;\phi>0)$, and $d\Gamma(\phi;\phi<0)$:

\begin{equation} \label{eq:folding}
    \begin{split}
        d\hat{\Gamma} &= d\Gamma(\phi|\phi<0) + d\Gamma(\phi|\phi>0) \\
        & = f_0(\phi|\phi\rightarrow-\phi) + f_0(\phi|\phi>0) \\
        & = f_0(\cos\phi, -\sin\phi) + f_0(\cos\phi, \sin\phi)
    \end{split}
\end{equation}


According to trigonometric identities $\cos(-\phi) = \cos\phi $,
$\sin(-\phi) = -\sin\phi $, we can cancel the terms 
that are odd under this transformation
%% containing $\sin\phi$
. Eq.~\ref{eq:Angular} now reads:

\begin{equation} \label{eq:fold1}
    \begin{split}
        \frac{1}{\mathrm{d}\Gamma/\mathrm{d}q^2}&\frac{\mathrm{d}^4\Gamma}{\mathrm{d}q^2 \mathrm{d}\cos\theta_l \mathrm{d}\cos\theta_\mathrm{K} \mathrm{d}\phi} = \frac{9}{16\pi}\left[\frac{3}{4}F_\mathrm{T}\sin^2\theta_\mathrm{K} + F_\mathrm{L}\cos^2\theta_\mathrm{K} \right.\\
            &\left.+(\frac{1}{4}F_\mathrm{T}\sin^2\theta_\mathrm{K}-F_\mathrm{L}\cos^2\theta_\mathrm{K})\cos2\theta_l+\frac{1}{2}P_1F_\mathrm{T}\sin^2\theta_\mathrm{K}\sin^2\theta_l\cos 2\phi \right.\\
            &+\sqrt{F_\mathrm{T}F_\mathrm{L}}(\frac{1}{2}P_4'\sin2\theta_\mathrm{K}\sin2\theta_l\cos\phi+P_5'\sin2\theta_\mathrm{K}\sin\theta_l\cos\phi )\\
            &\left.+2P_2F_\mathrm{T}\sin^2\theta_\mathrm{K}\cos\theta_l \right]
    \end{split}
\end{equation}

The second folding is performed at $\theta_l = \pi/2$ (for
$\theta_l>\pi/2,\theta_l\rightarrow \pi- \theta_l$). The domain of
$\theta_l$ is reduced to (0,$\pi$/2). According to $\cos(\pi-\theta_l) = -
\cos\theta_l$ and $\sin(\pi-\theta_l) = \sin\theta_l$, we can cancel the terms that are odd under this transformation.
 %% proportional to $P_4'$, which contains $\sin 2\theta_l$.


\begin{equation} \label{eq:fold2}
    \begin{split}
        \frac{1}{\mathrm{d}\Gamma/\mathrm{d}q^2}&\frac{\mathrm{d}^4\Gamma}{\mathrm{d}q^2 \mathrm{d}\cos\theta_l \mathrm{d}\cos\theta_\mathrm{K} \mathrm{d}\phi} = \frac{9}{8\pi}\left[\frac{3}{4}F_\mathrm{T}\sin^2\theta_\mathrm{K} + F_\mathrm{L}\cos^2\theta_\mathrm{K} \right.\\
            &\left.+(\frac{1}{4}F_\mathrm{T}\sin^2\theta_\mathrm{K}-F_\mathrm{L}\cos^2\theta_\mathrm{K})\cos2\theta_l+\frac{1}{2}P_1F_\mathrm{T}\sin^2\theta_\mathrm{K}\sin^2\theta_l\cos 2\phi \right.\\
            &\left.+\sqrt{F_\mathrm{T}F_\mathrm{L}}P_5'\sin2\theta_\mathrm{K}\sin\theta_l\cos\phi  \right]
    \end{split}
\end{equation}
%%%%%%%%%%%%%%%%%%%%%%%%%%%%%%%%

 For S-wave and the
interference terms, we do the same transformation as P-wave, after the
first ``folding'', it can reads:

\begin{equation} \label{eq:S-fold1}
    \begin{split}
        \frac{1}{\mathrm{d}\Gamma/\mathrm{d}q^2}&\frac{\mathrm{d}^4\Gamma}{\mathrm{d}q^2 \mathrm{d}\cos\theta_l \mathrm{d}\cos\theta_\mathrm{K} \mathrm{d}\phi} = \frac{3}{8\pi}\left[F_\mathrm{S}\sin^2\theta_l+ A_\mathrm{S}\sin^2\theta_l\cos\theta_\mathrm{K}\right.\\
            &+\left. A_\mathrm{S}^4\sin\theta_\mathrm{K}\sin2\theta_l\cos\phi + A^5_\mathrm{S}\sin\theta_\mathrm{K}\sin\theta_l\cos\phi\right]
    \end{split}
\end{equation}

After the second ``folding'', it reads:
\begin{equation} \label{eq:S-fold2}
    \begin{split}
      \frac{1}{\mathrm{d}\Gamma/\mathrm{d}q^2}&\frac{\mathrm{d}^4\Gamma}{\mathrm{d}q^2 \mathrm{d}\cos\theta_l \mathrm{d}\cos\theta_\mathrm{K} \mathrm{d}\phi} = \\
      &\frac{3}{4\pi}\left[F_\mathrm{S}\sin^2\theta_l+A_\mathrm{S}\sin^2\theta_l\cos\theta_\mathrm{K}+A^5_\mathrm{S}\sin\theta_\mathrm{K}\sin\theta_l\cos\phi\right]
    \end{split}
\end{equation}
%%%%%%%%%%%%%%%%%%%%%%%%%%%%%%%%%%5

After the two folding are performed, the angular distribution can be
written, using  Eq.~\ref{eq:fold2} and Eq.~\ref{eq:S-fold2} as:

\begin{equation} \label{eq:PDF-f2}
  \begin{split}
    \frac{1}{\mathrm{d}\Gamma/\mathrm{d}q^2}&\frac{\mathrm{d}^4\Gamma}{\mathrm{d}q^2 \mathrm{d}\cos\theta_l \mathrm{d}\cos\theta_\mathrm{K} \mathrm{d}\phi} = \\
    &\frac{9}{8\pi}\left\{\frac{2}{3}\left[ (F_\mathrm{S}+A_\mathrm{S}\cos\theta_\mathrm{K})\left(1-\cos^2\theta_l\right) + A^5_\mathrm{S}\sqrt{1-\cos^2\theta_\mathrm{K}}\sqrt{1-\cos^2\theta_l}\cos\phi \right] \right.\\
    & + \left(1 - F_\mathrm{S}\right)\left[2F_\mathrm{L}\cos^2\theta_\mathrm{K}\left(1-\cos^2\theta_l\right)+\frac{1}{2}\left(1-F_\mathrm{L}\right)\left(1-\cos^2\theta_\mathrm{K}\right)\left(1+\cos^2\theta_l\right) \right.\\
      & + \frac{1}{2}P_1(1-F_\mathrm{L})(1-\cos^2\theta_\mathrm{K})(1-\cos^2\theta_l)\cos 2\phi \\
      & \left.\left. + 2P_5'\cos\theta_\mathrm{K}\sqrt{F_\mathrm{L}\left(1-F_\mathrm{L}\right)}\sqrt{1-\cos^2\theta_\mathrm{K}}\sqrt{1-\cos^2\theta_l}\cos\phi\right]\right\}
  \end{split}
\end{equation}

Now we have 6 parameters, they are $F_\mathrm{L}$, $F_S$, $P_1$, $P_5'$, $A_\mathrm{S}$
and $A^5_\mathrm{S}$.
%%%%%%%%%%%%%%%%%%%%%%%%%%%%%%%%%%%55

\subsection{Parameter constrains}
\label{sec:bound}

\subsubsection{Range of definition of interference terms}
\label{sec:As5.range}
Due to their nature, the value of the interference terms $A_s$ and $A_s^5$ is limited by the amplitude of the pure P-wave and S-wave components~\cite{Genon:Swave}. Their allowded ranges are the following:
\begin{equation} \label{eq:As.range}
  |A_s|<2\sqrt{3}\sqrt{F_S(1-F_S)F_L}*F_{theo}
\end{equation}
\begin{equation} \label{eq:As5.range}
  |A^5_s|<\sqrt{3}\sqrt{F_S(1-F_S)F_T(1+P_1)}*F_{theo}
\end{equation}
where $F_{theo}$ is a constant factor that depends on the selection cuts applied to the $K\pi$ system mass, and in this analysis is 0.89.

To make sure that the fitted value of $A_s^5$ is contained in this range, it has been substituted in the PDF by
\begin{equation} \label{eq:As5.subst}
  A^5_s\to f\sqrt{3}\sqrt{F_S(1-F_S)F_T(1+P_1)}*F_{theo}
\end{equation}
where $f$ is a placeholder parameter defined in the range [-1;1]. The fit is then performed with respect to $f$ instead of $A_s^5$.

\subsubsection{P.D.F validity in the parameter space}
\label{sec:phys.bound}
Using the P.D.F. parametrization described above, it is not guaranteed that it is physical (i.e. positive in the whole ($\cos\theta_K$,$\cos\theta_l$,$\phi$) space).

In order to have a working fit sequence and reliable results, we need to identify which values of the parameters allow the P.D.F. to be physical. Since, as explain in Sec.~\ref{sec:finalform}, the free parameters will be $P_1$, $P_5'$, $A^5_\mathrm{S}$, we will compute only their physical regions, keeping the other parameters fixed at the value from the BPH-13-010 analysis result.

This operation can be done analytically, probing the P.D.F. for some selected values of the angular variables. From this procedure it is possible to get that the P1 physical range is [-1,1]; but no boundary can be extracted for $P_5'$ and $A^5_\mathrm{S}$, for which a numerical computation is needed.

To compute numerically the boundary of the physical region, the $P_1/P_5'$ space has been scanned with a grid of step 0.01 in both directions. For each point of this grid, the values of $\cos\theta_K$, $\cos\theta_l$ and $\phi$ are moved on a 3D grid with step 0.02; if the PDF is positive for all of the points of this second grid, the point in the $P_1/P_5'$ space is inside the physical region, otherwise it is outside. The resulting region is equivalent to a $P_1$-dependent upper boundary for the absolute value of $P_5'$.

For each bin, this phisical region has been computed eight times:
\begin{itemize}
\item once, by requiring that only the P-wave component is positive; in this case the result is independent from the nuisance parameter $A_s^5$.
\item seven times, by requiring the whole PDF to be positive, for different values of $A_s^5$; according the convention defined in Sec.~\ref{sec:As5.range}, the seven values of the placeholder parameter $f$ are {-1, -2/3, -1/3, 0, 1/3, 2/3, 1}.
\end{itemize}

The boundaries of this regions, plotted in the negative $P_5'$ sector only (the boundaries are symmetrical with respect to $P_5'=0$), are shown from Figure~\ref{fig:bound0} to Figure~\ref{fig:bound8}.

\begin{figure}[!hbt]
  \centering
  \includegraphics[width=0.85\textwidth]{Figures/boundaries/bound_b0.pdf}
  \caption{Physical boundaries of the negative $P_5'$ sector of $q^2$ bin 0. Accordingly to the description in Sec.~\ref{sec:phys.bound}, the magenta line is the boundary of the P-wave physical region and the set of gray-scale lines are the boundaries of the total-PDF physical region, for different $A_s^5$ values (black for $f=-1$, lightest gray for $f=1$).}
  \label{fig:bound0}
\end{figure}

\begin{figure}[!hbt]
  \centering
  \includegraphics[width=0.85\textwidth]{Figures/boundaries/bound_b1.pdf}
  \caption{Physical boundaries of the negative $P_5'$ sector of $q^2$ bin 1. Accordingly to the description in Sec.~\ref{sec:phys.bound}, the magenta line is the boundary of the P-wave physical region and the set of gray-scale lines are the boundaries of the total-PDF physical region, for different $A_s^5$ values (black for $f=-1$, lightest gray for $f=1$).}
  \label{fig:bound1}
\end{figure}

\begin{figure}[!hbt]
  \centering
  \includegraphics[width=0.85\textwidth]{Figures/boundaries/bound_b2.pdf}
  \caption{Physical boundaries of the negative $P_5'$ sector of $q^2$ bin 2. Accordingly to the description in Sec.~\ref{sec:phys.bound}, the magenta line is the boundary of the P-wave physical region and the set of gray-scale lines are the boundaries of the total-PDF physical region, for different $A_s^5$ values (black for $f=-1$, lightest gray for $f=1$).}
  \label{fig:bound2}
\end{figure}

\begin{figure}[!hbt]
  \centering
  \includegraphics[width=0.85\textwidth]{Figures/boundaries/bound_b3.pdf}
  \caption{Physical boundaries of the negative $P_5'$ sector of $q^2$ bin 3. Accordingly to the description in Sec.~\ref{sec:phys.bound}, the magenta line is the boundary of the P-wave physical region and the set of gray-scale lines are the boundaries of the total-PDF physical region, for different $A_s^5$ values (black for $f=-1$, lightest gray for $f=1$).}
  \label{fig:bound3}
\end{figure}

\begin{figure}[!hbt]
  \centering
  \includegraphics[width=0.85\textwidth]{Figures/boundaries/bound_b5.pdf}
  \caption{Physical boundaries of the negative $P_5'$ sector of $q^2$ bin 5. Accordingly to the description in Sec.~\ref{sec:phys.bound}, the magenta line is the boundary of the P-wave physical region and the set of gray-scale lines are the boundaries of the total-PDF physical region, for different $A_s^5$ values (black for $f=-1$, lightest gray for $f=1$).}
  \label{fig:bound5}
\end{figure}

\begin{figure}[!hbt]
  \centering
  \includegraphics[width=0.85\textwidth]{Figures/boundaries/bound_b7.pdf}
  \caption{Physical boundaries of the negative $P_5'$ sector of $q^2$ bin 7. Accordingly to the description in Sec.~\ref{sec:phys.bound}, the magenta line is the boundary of the P-wave physical region and the set of gray-scale lines are the boundaries of the total-PDF physical region, for different $A_s^5$ values (black for $f=-1$, lightest gray for $f=1$).}
  \label{fig:bound7}
\end{figure}

\begin{figure}[!hbt]
  \centering
  \includegraphics[width=0.85\textwidth]{Figures/boundaries/bound_b8.pdf}
  \caption{Physical boundaries of the negative $P_5'$ sector of $q^2$ bin 8. Accordingly to the description in Sec.~\ref{sec:phys.bound}, the magenta line is the boundary of the P-wave physical region and the set of gray-scale lines are the boundaries of the total-PDF physical region, for different $A_s^5$ values (black for $f=-1$, lightest gray for $f=1$).}
  \label{fig:bound8}
\end{figure}

\clearpage

\chapter[LHC and CMS detector]{The Large Hadron Collider and the Compact Muon Solenoid Experiment}
\label{sec:detect}

The Large Hadron Collider (LHC) \cite{LHC} is an accelerator located at the European Laboratory for Particle Physics Research (CERN) in Geneva. It has been conceived to collide proton beams at a centre-of-mass energy of $\sqrt{s} = 14$ TeV and
a nominal instantaneous luminosity of $\mathcal{L} = 10^{34}$ cm$^{-2}$ s$^{-1}$, representing a seven-fold increase in energy and a hundred-fold increase in integrated luminosity over the previous hadron collider experiments. Its main purpose is to search for rare processes like the production of Higgs or new particles with mass of 1 TeV and beyond. Two experiments have been installed around the LHC to pursue these results: ATLAS \cite{ATL} and CMS \cite{CMS}. Furthermore, the LHCb \cite{LHbr} experiment studies the properties of charm and beauty hadrons produced with large cross sections in the forward region in collisions at the LHC, and the ALICE \cite{ALI} experiment analyses the data from relativistic heavy ion collisions to study the hadronic matter in extreme temperature and density conditions (i.e. high quark-gluon density).

\section{The Large Hadron Collider}
The LHC has been installed in the same tunnel which hosted the $e^+e^-$ collider
LEP (Large Electron-Positron collider). Accelerated electrons and positrons suffer large
energy loss due to the synchrotron radiation, which is proportional to $E^4/(Rm^4)$,
where $E$ is the electron energy, $m$ its mass and $R$ the accelerator radius. To
obtain energies of the order of TeV, at the fixed accelerator radius, only massive
charged particles could have been used: protons and heavy nuclei. The energy loss
is reduced by a factor $(2000)^4$ for a given fixed energy $E$ for protons, respect to electrons.
Another important aspect of the LHC is the collision rate. To produce a sufficient
number of rare processes, the collision rate needs to be very high. Beam protons
are collected in packets called bunches. The collision rate is proportional to the
instantaneous luminosity of the accelerator, defined as:
\begin{displaymath}\quad
\mathcal{L}=\frac{fkn^2_p}{4\pi\sigma_x\sigma_y} \quad,
\end{displaymath}
where $f$ is the bunch revolution frequency, $k$ the number of bunches, $n_p$ the number
 of protons per bunch, $\sigma_x$ and $\sigma_y$ their transverse dispersion along the $x$ and $y$
axis. At the nominal 14 TeV LHC conditions ($\mathcal{L} = 10^{34}$ cm$^{-2}$ s$^{-1}$) the parameter
values are: $k$ = 2808, $n_p=1.5\cdot10^{11}$ and $\sigma_x\sigma_y = 16.6\,\mu\mathrm{m}^2$ (with $\sigma_z = 7.6$ cm along
the beam). The integrated luminosity is defined as $L =\int\mathcal{L}\mathrm{d}t$. For comparison
we can consider the Tevatron accelerator at Fermilab, which produced proton-antiproton
collisions since 1992. Its centre of mass energy was $1.8$ TeV until 1998 and $1.96$ TeV
since 2001. To increase $\mathcal{L}$ by two orders of magnitude, protons are injected in both
LHC beams. The antiprotons, in fact, are obtained by steering
proton beams onto a nickel target and represent only a small fraction of the wide
range of secondary particles produced in this interactions, thus have a production rate lower than the proton one.
\\

\begin{figure}
\centering
\includegraphics[width=0.8\columnwidth]{Images/beamline.png}
\caption{LHC dipole magnet section scheme.}
\label{beamline}
\end{figure}
The LHC is composed by 1232 super-conducting dipole magnets each 15 m
long, providing a $8.3$ T magnetic field to let the beams circulate inside their trajectories 
along the 27 km circumference. Two vacuum pipes are used to let beams
circulate in opposite directions. A scheme representing the transverse dipole magnet 
section is represented in figure \ref{beamline}. More than 8000 other magnets are utilized
for the beam injection, their collimation, trajectory correction, crossing. All the
magnets are kept cool by superfluid helium at $1.9$ K temperature.
The beams are accelerated from 450 GeV (the injection energy from the SPS) to 7
TeV with 16 Radio Frequency cavities (8 per beam) which raise the beam energy
by 16 MeV each round with an electric field of 5 MV/m oscillating at 400 MHz
frequency.\\
Before the injection into the LHC, the beams are produced and accelerated by
different components of the CERN accelerator complex. Being produced from
ionized hydrogen atoms, protons are accelerated by the linear accelerator LINAC,
Booster and the Proton Synchrotron (PS) up to 26 GeV energy, the bunches being
separated by 25 ns each. The beams are then injected into the Super Proton Synchrotron 
(SPS) where they are accelerated up to 450 GeV. They are then finally
transferred to the LHC and accelerated up to 7 TeV energy per beam. The CERN
accelerator complex is illustrated in figure \ref{acc}.
\begin{figure}
\centering
\includegraphics[width=0.7\columnwidth]{Images/acc.pdf}
\caption{Scheme representing the CERN accelerator complex.}
\label{acc}
\end{figure}

The LHC started its operations in December 2009 with centre of mass energy for the proton-proton collision
$\sqrt{s} = 0.9$ TeV. The centre of mass energy was set to $\sqrt{s} = 7$ TeV in the 2010 and 2011 runs and raised to $\sqrt{s} = 8$ TeV in the 2012 runs. Here are reported the CMS detected peak and integrated luminosities for proton-proton runs.
In 2010 the peak luminosity reached $\mathcal{L}=203.80\,\mathrm{Hz}/\mu\mathrm{b}$ and the integrated luminosity has been $L=40.76\,\mathrm{pb}^{-1}$, while during 2011 the peak luminosity increased to $\mathcal{L}=4.02\,\mathrm{Hz}/\mathrm{nb}$ and the integrated luminosity has been $L=5.55\,\mathrm{fb}^{-1}$.
In the 2012 runs the peak luminosity reached $\mathcal{L}=7.67\,\mathrm{Hz}/\mathrm{nb}$ and the integrated luminosity has been $L=21.79\,\mathrm{fb}^{-1}$, as graphically summarized in figure \ref{lumi_2012}.
\begin{figure}
\centering
\begin{tabular}{@{}p{0.5\columnwidth}@{} p{0.5\columnwidth}@{}}
\includegraphics[width=0.52\columnwidth]{Images/lumi_pea_2012.pdf}&
\includegraphics[width=0.52\columnwidth]{Images/lumi_int_2012.pdf}
\end{tabular}
\caption{LHC performance in 2012. Left: CMS detected peak
luminosity; right: CMS detected integrated luminosity.}
\label{lumi_2012}
\end{figure}

\section{CMS Experiment}
The Compact Muon Solenoid \cite{CMS} is a general purpose detector situated at interaction 
point 5 of the CERN Large Hadron Collider. It is designed around a 4 T 
solenoidal magnetic field provided by the largest superconducting solenoid ever
built. The structure of CMS is shown in figure \ref{CMS_sch}, where particular emphasis is
put on the volumes of the different subsystems: the Silicon Pixel Detector, the
Silicon Strip Tracker, the Electromagnetic and Hadronic Calorimeters, and Muon
Detectors.
\begin{figure}
\centering
\includegraphics[width=\columnwidth]{Images/CMS2.pdf}
\caption{Transverse (left) and longitudinal (right) cross sections of the CMS detector showing the volumes of the different detector subsystems. The transverse cross section is drawn for the central barrel, coaxial with the beam line, while complementary end-caps are shown in the longitudinal view.}
\label{CMS_sch}
\end{figure}

We can briefly summarize the aims of the CMS detector \cite{CMS1}. They are mainly:
\begin{itemize}
\item search for SM and MSSM Higgs boson decaying into photons, $b$ quarks, $\tau$
leptons, $W$ and $Z$ bosons,
\item search for additional heavy neutral gauge bosons predicted in many superstring-inspired 
theories or Great Unification Theories and decaying to muon pairs,
\item search for new Physics in various topologies: multilepton events, multijet
events, events with missing transverse energy\footnote{Missing transverse energy \met is the amount of energy which must be added to balance the modulus of the vector sum of the projections of the track momenta and calorimeter clusters in the plane perpendicular to beam axis. Its direction is opposite to this vector sum directions.} or momentum, any combination 
of the three above,
\item study of the $B$-hadron rare decay channels (like $B^0_{(s)}\to\mu\mu$) and of CP violation in the decay of the $B$ mesons (like $B^0_s\to J/\psi\phi\to\mu^+\mu^-K^+K^-$),
\item search for $B^0\to\mu^+\mu^-$ decays,
\item study of QCD and jet physics at the TeV scale,
\item study of top quark and EW physics.
\end{itemize}
CMS has been therefore designed as a multipurpose experiment, with particular 
focus on muon, photon, and displaced tracks
reconstruction. Superb performances have been achieved overall, in particular in:
\begin{itemize}
\item primary and secondary vertex localization,
\item charged particle momentum resolution and reconstruction efficiency in the
tracking volume,
\item electromagnetic energy resolution,
\item isolation of leptons and photons at high luminosities,
\item measurement of the direction of photons, rejection of $\pi^0\to\gamma\gamma$,
\item diphoton and dielectron mass resolution $\sim1\%$ at 100GeV,
\item measurement of the missing transverse energy \met and dijet mass with high
resolution,
\item muon identification over a wide range of momenta,
\item dimuon mass resolution $\sim1\%$ at 100 GeV,
\item unambiguously determining the charge of muons with $p_T$ up to 1 TeV,
\item triggering and offline tagging of $\tau$ leptons and $b$ jets.
\end{itemize}
\mbox{}\\

The reference frame used to describe the CMS detector and the collected events
has its origin in the geometrical centre of the solenoid. Different
types of global coordinates measured with respect to the origin\footnote{Global coordinates are measured in the CMS reference frame while local coordinates are measured in the reference frame of a specific sub-detector or sensitive element.} are used:
\begin{itemize}
\item cartesian coordinate system, $\hat{x}$ axis points towards the centre of LHC,
$\hat{y}$ points upwards, perpendicular to LHC plane, while $\hat{z}$ completes the
right-handed reference,
\item polar coordinate system, directions are defined with an azimuthal angle
$\tan\phi=y/x$ and a polar angle $\tan\theta=\rho/z$, where $\rho=\sqrt{x^2+y^2}$,
\item polar coordinate system, with instead of the polar angle the rapidity $y$ and the pseudorapidity $\eta$, obtained for any particle from
\begin{displaymath}\quad
y=\frac{1}{2}\ln\Bigg(\frac{E+p_z}{E-p_z}\Bigg) \quad,
\end{displaymath}
\begin{displaymath}\quad
\eta=-\ln\bigg(\tan\frac{\theta}{2}\bigg) \quad,
\end{displaymath}
where $E$ is the particle energy and $p_z$ the component of its momentum along the
beam direction.
\end{itemize}

\subsection{Magnet}
The whole CMS detector is designed around a $\sim4$ T superconducting solenoid \cite{mag}
$12.5$ m long and with inner radius of 3 m. The solenoid thickness is $3.9$ radiation
lengths and it can store up to $2.6$ GJ of energy.

The field is closed by a $10^4$ t iron return yoke made of five barrels and two
end-caps, composed of three layers each. The yoke is instrumented with four layers
of muon stations. The coil is cooled down to $4.8$ K by a helium refrigeration plant,
while insulation is given by two pumping stations providing vacuum on the 40 m$^3$
of the cryostat volume.

The magnet was designed in order to reach precise measurement of muon momenta. 
A high magnetic field is required to keep a compact spectrometer capable
to measure 100 GeV track momentum with percent precision. A solenoidal field
was chosen because it keeps the bending in the transverse plane, where an accuracy
better than $20\,\mu\mathrm{m}$ is achieved in vertex position measurements. The size of the
solenoid allows efficient track reconstruction up to a pseudorapidity of $2.4$. The
inner radius is large enough to accommodate both the Silicon Tracking System
and the calorimeters. During the 2012 acquisitions the magnet was operated at $3.8$ T.

\subsection{Tracking System}
The core of CMS is a Silicon Tracking System \cite{trs} with $2.5$ m diameter and
$5.8$ m length, designed to provide a precise and efficient measurement of the trajectories 
of charged particles emerging from LHC collisions and reconstruction of
secondary vertices.

\begin{figure}
\centering
\includegraphics[width=\columnwidth]{Images/tra_sch.pdf}
\caption{Layout of the CMS silicon tracker showing the relative position of
hybrid pixels, single-sided strips and double-sided strips. Figure from \cite{CMS}.}
\label{tra_sch}
\end{figure}
\begin{figure}
\centering
\includegraphics[width=\columnwidth]{Images/tren_sch.pdf}
\caption{Layout of the current CMS Pixel Detector. Figure from \cite{trs}.}
\label{tren_sch}
\end{figure}
The CMS Tracking System is composed of both silicon Pixel and Strip Detectors, 
as shown in figure \ref{tra_sch}. The Pixel Detector consists of 1440 pixel modules
arranged in three barrel layers and two disks in each end-cap as in figure \ref{tren_sch}. The
Strip Detector consists of an inner tracker with four barrel layers and three end-cap
disks and an outer tracker with six barrel layers and nine end-cap disks, housing a
total amount of 15148 strip modules of both single-sided and double-sided types.
Its active silicon surface of about 200 m$^2$ makes the CMS tracker the largest silicon
tracker ever built.

The LHC physics programme requires high reliability, efficiency and precision
in reconstructing the trajectories of charged particles with transverse momentum
larger than 1 GeV in the pseudorapidity range $|\eta|<2.5$. Heavy quark flavours
can be produced in many of the interesting channels and a precise measurement
of secondary vertices is therefore needed. The tracker completes the functionalities 
of ECAL and Muon System to identify electrons and muons. Also hadronic
decays of tau leptons need robust tracking to be identified in both the one-prong
and three-prongs topologies. Tracker information is heavily used in the High Level
Trigger of CMS to help reducing the event collection rate from the 40 MHz of
bunch crossing to the 100 Hz of mass storage.

\subsubsection{Silicon Pixel Detector}
The large number of particles produced in 25 pile-up events\footnote{Events that occur in the same bunch crossing.}, at nominal LHC
luminosity, results into a hit rate density of 1 MHz mm$^{-2}$ at 4 cm from the beamline,
decreasing down to 3 kHz mm$^{-2}$ at a radius of 115 cm. Pixel detectors are used
at radii below 10 cm to keep the occupancy below $1\%$. The chosen size for pixels,
$0.100\times0.150\,\mathrm{mm}^2$ in the transverse and longitudinal directions respectively, leads
to an occupancy of the order of $10^{-4}$. The layout of the Pixel Detector consists
of a barrel region (BPIX), with three barrels at radii of $4.4$, $7.3$ and $10.2$ cm,
complemented by two disks on each side (FPIX), at $34.5$ and $46.5$ cm from the
nominal interaction point. This layout provides about 66 million pixels covering a
total area of about 1 m$^2$ and measuring three high precision points on each charged
particle trajectory up to $|\eta|=2.5$. Detectors in FPIX disks are tilted by $20^\circ$ in a
turbine-like geometry to induce charge sharing and achieve a spatial resolution of
about $20\,\mu\mathrm{m}$.

\subsubsection{Silicon Strip Tracker}
In the inner Strip Tracker, which is housed between radii of 20 and 55 cm, the
reduced particle flux allows a typical cell size of $0.080\times100\,\mathrm{mm}^2$, resulting in a
$2\%$ occupancy per strip at design luminosity. In the outer region, the strip pitch
is increased to $0.180\times250\,\mathrm{mm}^2$ together with the sensor thickness which scales
from $0.320$ mm to $0.500$ mm. This choice compensates the larger capacitance of
the strip and the corresponding larger noise with the possibility to achieve a larger
depletion of the sensitive volume and a higher charge signal.

The Tracker Inner Barrel and Disks (TIB and TID) deliver up to 4 \mbox{($r$, $\phi$)} measurements 
on a trajectory using $0.320$ mm thick silicon strip sensors with strips
parallel to the beamline. The strip pitch is $0.080$ mm in the first two layers and
$0.120$ mm in the other two layers, while in the TID the mean pitch varies from
$0.100$ mm to $0.141$ mm. Single point resolution in the TIB is $0.023$ mm with the
finer pitch and $0.035$ mm with the coarser one. The Tracker Outer Barrel (TOB)
surrounds the TIB/TID and provides up to 6 $r-\phi$ measurements on a trajectory
using 0.500 mm thick sensors. The strip pitch varies from $0.183$ mm in the four
innermost layers to $0.122$ mm in the outermost two layers, corresponding to a resolution 
of $0.053$ mm and $0.035$ mm respectively. Tracker End-Caps (TEC) enclose
the previous sub-detectors at $124\mathrm{cm}<|z|<282\mathrm{cm}$ with 9 disks carrying 7 rings
of microstrips, 4 of them are $0.320$ mm thick while the remaining 3 are $0.500$ mm
thick. TEC strips are radially oriented and their pitch varies from $0.097$ mm to
$0.184$ mm.

As shown in figure \ref{tra_sch}, the first two layers and rings of TIB, TID and TOB, as
well as three out of the TEC rings, carry strips on both sides with a stereo angle
of 100 milliradians to measure the other coordinate: $z$ in barrels and $r$ in rings.
This layout ensures 9 hits in the silicon Strip Tracker in the full acceptance range
$|\eta|<2.4$, and at least four of them are two-dimensional. The total area of Strip
Tracker is about 198 m$^2$ read out by $9.3$ million channels.

\subsubsection{Trajectory Reconstruction}
Due to the magnetic field charged particles travel through the tracking detectors
on a helical trajectory which is described by 5 parameters: the curvature $\kappa$, the
track azimuthal angle $\phi$, the pseudorapidity $\eta$, the signed transverse impact parameter 
$d_0$ and the longitudinal impact parameter $z_0$. The transverse (longitudinal)
impact parameter of a track is defined as the transverse (longitudinal) distance of
closest approach of the track to the primary vertex. %as explained in Section ??.
The main standard algorithm used in CMS for track reconstruction is the Combinatorial 
Track Finder (CFT) algorithm \cite{CFT} which uses the reconstructed positions
of the passage of charged particles in the silicon detectors to determine the track
parameters. The CFT algorithm proceeds in three stages: track seeding, track
finding and track fitting. Track candidates are best seeded from hits in the pixel
detector because of the low occupancy, the high efficiency and the unambiguous
two-dimensional position information. The track finding stage is based on a standard 
Kalman filter pattern recognition approach which starts with the seed
parameters. The trajectory is extrapolated to the next tracker layer and compatible 
hits are assigned to the track on the basis of the $\chi^2$ between the predicted and
measured positions. At each stage the Kalman filter updates the track parameters
with the new hits.

The tracks are assigned a quality based on the $\chi^2$ and the number of missing
hits and only the best quality tracks are kept for further propagation. Ambiguities
between tracks are resolved during and after track finding. In case two tracks share
more than $50\%$ of their hits, the lower quality track is discarded. For each trajectory 
the finding stage results in an estimate of the track parameters. However,
since the full information is only available at the last hit and constraints applied
during trajectory building can bias the estimate of the track parameters, all valid
tracks are refitted with a standard Kalman filter and a second filter (smoother)
running from the exterior towards the beam line. The expected performance of
the track reconstruction is shown in figure \ref{part_eff} for muons, pions and hadrons. The
track reconstruction efficiency for high energy muons is about $99\%$ and drops at
$|\eta|>2.1$ due to the reduced coverage of the forward pixel detector. For pions and
hadrons the efficiency is in general lower because of interactions with the material
in the tracker.
\begin{figure}
\centering
\includegraphics[width=\columnwidth]{Images/part_eff.pdf}
\caption{Global track reconstruction effciency as a function of track pseudorapidity 
for muons (left) and pions (right) of transverse momenta of 1, 10 and 100
GeV. Figures from \cite{CMS}.}
\label{part_eff}
\end{figure}

The material budget is shown in figure \ref{mat_bud} as a function of pseudorapidity,
with the different contributions of sub-detectors and services.
\begin{figure}
\centering
\includegraphics[width=\columnwidth]{Images/mat_bud.pdf}
\caption{Material budget of the current CMS Tracker in units of radiation
length $X_0$ as a function of the pseudorapidity, showing the different contribution
of sub-detectors (left) and functionalities (right). Figures from \cite{CMS}.}
\label{mat_bud}
\end{figure}

\begin{figure}
\centering
\includegraphics[width=\columnwidth]{Images/part_res.pdf}
\caption{Resolution of several track parameters as a function of track pseudorapidity 
for single muons with transverse momenta of 1, 10 and 100 GeV: transverse
momentum (left), transverse impact parameter (middle) and longitudinal impact
parameter (right). Figures from \cite{CMS}.}
\label{part_res}
\end{figure}
The performance of the Silicon Tracker in terms of track reconstruction efficiency 
and resolution, of vertex and momentum measurement, are shown in figure \ref{part_eff}
and figure \ref{part_res} respectively. The first one, in particular, shows the difference in reconstruction 
efficiency for muons and pions, due to the larger interaction cross section
of pions, which cannot be assumed to be minimum-ionizing particles and therefore
are much more degraded by the amount of material.

\subsubsection{Vertex Reconstruction}
The reconstruction of interaction vertices allows CMS
to reject tracks coming from pile-up events. The primary vertex reconstruction is a
two-step process. Firstly the reconstructed tracks are grouped in vertex candidates
and their $z$ coordinates at the beam closest approach point are evaluated, retaining
only tracks with impact parameter respect to the vertex candidate less than 3 cm. Vertices are then reconstructed
through a recursive method for parameter estimation through a Kalman filter \cite{Kal1}
algorithm. For a given event, the primary vertices are ordered according to the
total transverse momentum of the associated tracks, $\sum p_T$. The vertex reconstruction 
efficiency is very close to $100\%$ and the position resolution is of the order of
$\mathcal{O}(10)\,\mu\mathrm{m}$ in all directions.

It is also possible to reconstruct the secondary vertices, for example those from
b-quark decays. The secondary vertex reconstruction uses tracks associated to
jets applying further selection cuts: the transverse impact parameter of the tracks
must be greater than $100\,\mu\mathrm{m}$, to avoid tracks coming from the primary vertex, and the longitudinal impact parameter
below 2 cm, to avoid tracks from pile-up events.

\subsection{Muon Spectrometer}
Detection of muons at CMS exploits different technologies and is performed by
a ``Muon System'' rather than a single detector \cite{muo}. Muons are the only particles 
able to reach the external muon chambers with a minimal energy loss when
traversing the calorimeters, the solenoid and the magnetic field return yoke. Muons can
provide strong indication of interesting signal events and are natural candidates
for triggering purposes. The CMS Muon System was designed to cope with three
major functions: robust and fast identification of muons, good resolution of momentum 
measurement and triggering.

The Muon System is composed of three types of gaseous detectors, located inside
the empty volumes of the iron yoke, and therefore arranged in barrel and end-cap
sections. The coverage of Muon System is shown in figure \ref{muo_sch}.
\begin{figure}
\centering
\includegraphics[width=1.1\columnwidth]{Images/muo_sch.pdf}
\caption{Transverse and longitudinal cross sections of the CMS detector showing 
the Muon System with particular emphasis on the different technologies used
for detectors; the ME/4/2 CSC layers in the end-cap were included in the design
but are not currently installed. Figures from \cite{CMS}.}
\label{muo_sch}
\end{figure}

In the barrel region the neutron-induced background is small and the muon
rate is low; moreover, the field is uniform and contained in the yoke. For these
reasons, standard drift chambers with rectangular cells are used. The barrel Drift
Tubes (DT) cover the $|\eta|<1.2$ region, are divided in five wheels in the beam direction and are organized in four stations housed among the yoke layers.
The first three stations contain 12 cell planes, arranged in two superlayers providing measurement along $r\phi$ and one superlayerlayer measuring along $z$, each of them containing four layers.
The fourth station provides measurement only in the transverse plane.

Both the muon rates and backgrounds are high in the forward region, where
the magnetic field is large and non uniform. The choice for muon detectors fell
upon cathode strip chambers (CSC) because of their fast response time, fine segmentation 
and radiation tolerance. Each end-cap is equipped with four stations
of CSCs. The CSCs cover the $0.9<|\eta|<2.4$ pseudorapidity range. The cathode 
strips are oriented radially and provide precise measurement in the bending
plane, the anode wires run approximately perpendicular to the strips and are read
out to measure the pseudorapidity and the beam-crossing time of a muon. The
muon reconstruction efficiency is typically $95-99\%$ except for the regions between
two barrel DT wheels or at the transition between DTs and CSCs, where the
efficiency drops.

Both the DTs and CSCs can trigger on muons with a Level 1 $p_T$ (see section \ref{TRG}) resolution
of $15\%$ and $25\%$, respectively. Additional trigger-dedicated muon detectors were
added to help measured the correct beam-crossing time. These are Resistive Plate
Chambers (RPC), gaseous detector operated in the avalanche mode, which can
provide independent and fast trigger with high segmentation and sharp $p_T$ threshold 
over a large portion of the pseudorapidity range. The overall $p_T$ resolution on
muons is shown in figure \ref{mu_reso}, with emphasis on the different contribution from the
Muon System and the Silicon Tracker.
\begin{figure}
\centering
\includegraphics[width=\columnwidth]{Images/mu_reso.pdf}
\caption{Resolution on $p_T$ measurement of muons with the Muon System, the
Silicon Tracker or both, in the barrel (left) and end-caps (right). Figures from \cite{CMS}.}
\label{mu_reso}
\end{figure}

\subsubsection{Muon Reconstruction}
Muon detection and reconstruction play a key role in the CMS physics program,
both for the discovery of New Physics and for precision measurements of SM
processes. CMS has been designed for a robust detection of muons over the entire
kinematic range of the LHC and in a condition of very high background. The
muon system allows an efficient and pure identification of muons, while the inner
tracker provides a very precise measurement of their properties. An excellent
muon momentum resolution is made possible by the high-field solenoidal magnet.
The steel flux return yoke provides additional bending power in the spectrometer,
and serves as hadron absorber to facilitate the muon identification. Several muon
reconstruction strategies are available in CMS, in order to fulfil the specific needs
of different analyses. The muon reconstruction consists of three main stages:
\begin{enumerate}
\item local reconstruction: in each muon chamber, the raw data from the detector
read-out are reconstructed as individual points in space; in CSC and DT
chambers, such points are then fitted to track segments;
\item stand-alone reconstruction: points and segments in the muon spectrometer
are collected and fitted to tracks, referred to as ``stand-alone muon tracks'';
\item global reconstruction: stand-alone tracks are matched to compatible tracks in
the inner tracker and a global fit is performed using the whole set of available
measurements: the resulting tracks are called ``global muon tracks''.
\end{enumerate}
Muon identification represents a complementary approach with respect to global
reconstruction: it starts from the inner tracker tracks and flags them as muons by
searching for matching segments in the muon spectrometer. The muon candidates
produced with this strategy are referred to as ``tracker muons''.
After the completion of both algorithms, the reconstructed stand-alone, global
and tracker muons are merged into a single software object, with the addition of
further information, like the energy collected in the matching calorimeter towers.
This information can be used for further identification, in order to achieve a balance
between efficiency and purity of the muon sample.

\subsection{Calorimetry}
Identification of electrons, photons, and hadrons relies on accurate calorimetry,
which is a destructive measurement of the energy of a particle. As in most of
the particle physics experiments, a distinction is made between electromagnetic
calorimetry and hadron calorimetry. Electromagnetic calorimetry is based on the
production of EM showers inside a high-Z absorber, while hadron calorimetry
measures the effects of hadron inelastic scattering with heavy nuclei, including
production of photons from neutral pions and muons, and neutrinos from weak
decays. Calorimetry must be precise and hermetic also to measure any imbalance
of momenta in the transverse plane which can signal the presence of undetected
particles such as high-$p_T$ neutrinos.

\begin{figure}
\centering
\includegraphics[width=0.7\columnwidth]{Images/ecal_fig.pdf}
\caption{Cut-away view of the CMS ECAL showing the hierarchical structure
of crystals arranged in supercystals and modules and the orientation of crystals
whose major axis is always directed to the origin of the reference frame.}
\label{ecal_fig}
\end{figure}
The electromagnetic calorimeter of CMS, ECAL, is a homogeneous calorimeter, 
where the absorber material is the same as the sensitive one \cite{ECA}. ECAL
is composed of 61200 lead tungstate (PbWO$_4$) crystals in the barrel region and
7324 crystals in the end-caps, as shown in figure \ref{ecal_fig}. The crystal cross-section is
$22\times22\,\mathrm{mm}^2$ at the front face, while the length is 230 mm. End-caps are equipped
with a preshower detector. Lead tungstate was chosen because of its high density, $8.28$
g cm$^{-3}$, short radiation length, $0.89$ cm, and small Molière radius, $2.2$ cm. This way,
the calorimeter can be kept compact with fine granularity, while scintillation and
optical properties of PbWO$_4$ make it fast and radiation tolerant. Signal transmission 
exploits total internal reflection. Scintillation light detection relies on two
different technologies. Avalanche photodiodes (APD) are used in the barrel region,
mounted in pairs on each crystals, while vacuum phototriodes (VPT) are used in
the end-caps. The preshower detector is a sampling calorimeter composed of lead
radiators and silicon strips detectors, and it is used to identify neutral pions in
the forward region. The nominal energy resolution, measured with electron beams
having momenta between 20 and 250 GeV, is
\begin{displaymath}\quad
\bigg(\frac{\sigma_E}{E}\bigg)^2=\Bigg(\frac{2.8\%}{\sqrt{E}}\Bigg)^2+\Bigg(\frac{0.12}{E}\Bigg)^2+(0.30\%)^2 \quad,
\end{displaymath}
where all the energies are in GeV and the different contributions are respectively: the stochastic one (S), due to fluctuations 
in the lateral shower containment and in the energy released in the
preshower, that due to electronics (N), digitization and pile-up, and the constant term (C),
due to intercalibration errors, energy leakage from the back of the crystal and non-
uniformity in light collection.
\\

\begin{figure}
\centering
\includegraphics[width=0.8\columnwidth]{Images/hcal_sch.pdf}
\caption{Cross section of the CMS HCAL showing the tower segmentation.
Figure from \cite{HCA}.}
\label{hcal_sch}
\end{figure}
The hadron calorimeter of CMS, HCAL, is a sampling calorimeter employed
for the measurement of hadron jets and neutrinos or exotic particles resulting in
apparent missing transverse energy \cite{HCA}. A longitudinal view of HCAL is shown
in figure \ref{hcal_sch}. The hadron calorimeter size is constrained in the barrel region,
$|\eta|<1.3$, by the maximum radius of ECAL and the inner radius of the solenoid
coil. Because of this, the total amount of the absorber material is limited and
an outer calorimeter layer is located outside of the solenoid to collect the tail
of the showers. The pseudorapidity coverage is extended in the $3<|\eta|<5.2$ by
forward Cherenkov-based calorimeters. The barrel part, HB, consists of 36 wedges,
segmented into 4 azimuthal sectors each, and made out of flat brass absorber
layers, enclosed between two steel plates and bolted together without any dead
material on the full radial extent. There are 17 active plastic scintillator tiles
interspersed between the stainless steel and brass absorber plates, segmented in
pseudorapidity to provides an overall granularity of $\Delta\phi\times\Delta\eta=0.087\times0.087$. The
same segmentation is maintained in end-cap calorimeters, HE, up to $|\eta|<1.6$,
while it becomes two times larger in the complementary region. The maximum
material amount in both HB and HE corresponds to approximately 10 interaction
lengths $\lambda_I$. The energy resolution on single electron and hadron jets is shown in
figure \ref{cal_reso}.
\begin{figure}
\centering
\begin{tabular}{@{}p{0.52\columnwidth}@{} p{0.52\columnwidth}@{}}
\includegraphics[width=0.52\columnwidth]{Images/cal_reso1.pdf}&
\includegraphics[width=0.52\columnwidth]{Images/cal_reso2.pdf}
\end{tabular}
\caption{Left: ECAL energy resolution as a function of the electron energy
as measured from a beam test. The energy was measured in a $3\times3$ crystals array
with the electron impacting the central one. The stochastic, noise and constant
terms are given. Right: the jet transverse energy resolution as a function of the
transverse energy for barrel jets, end-cap jets and very forward jets reconstructed
with an iterative cone algorithm with cone radius $R=0.5$. Figures from \cite{CMS}.}
\label{cal_reso}
\end{figure}

\subsection{Trigger and Data Acquisition} \label{TRG}
High bunch crossing rates and design luminosity at LHC correspond to approximately 
20--25 superimposed events every 25 ns, for a total of $10^9$ events per second.
The large amount of data associated to them is impossible to store and process,
therefore a dramatic rate reduction has to be achieved. This is obtained with two
steps: the Level 1 Trigger \cite{L1} and the High Level Trigger, HLT \cite{HLT}.

The Level 1 Trigger is based on custom and programmable electronics, while
HLT is a software system implemented on a $\sim1000$ commercial processors farm.
The maximum allowed output rate for Level 1 Trigger is 100 kHz, which should
be even kept lower, about 30 kHz, for safe operation. Level 1 Trigger uses rough
information from coarse segmentation of calorimeters and Muon Detectors and
holds the high-resolution data in a pipeline until acceptance/rejection decision
is made. HLT exploits the full amount of collected data for each bunch crossing 
accepted by Level 1 Trigger and is capable of complex calculations such as
the off-line ones. HLT algorithms are those expected to undergo major changes
in time, particularly with increasing luminosity. Configuration and operation of
the trigger components are handled by a software system called Trigger Supervisor.

The Level 1 Trigger relies on local, regional and global components. The Global
Calorimeter and Global Muon Triggers determine the highest-rank calorimeter and
muon objects across the entire experiment and transfer them to the Global Trigger,
the top entity of the Level 1 hierarchy. The latter takes the decision to reject an
event or to accept it for further evaluation by the HLT. The total allowed latency
time for the Level 1 Trigger is $3.2\,\mu\mathrm{s}$. A schematic representation of the Level 1
Trigger data flow is presented in figure \ref{trg_sch}.
\begin{figure}
\centering
\includegraphics[width=0.8\columnwidth]{Images/trg_sch.pdf}
\caption{Schematic representation of the Level 1 Trigger data flow.}
\label{trg_sch}
\end{figure}

\subsubsection{Muon Trigger}
All Muon Detectors -- DT, CSC and RPC -- contribute to the Trigger. Barrel
DTs provide Local Trigger in the form of track segments in $\phi$ and hit patterns in
$\eta$. End-cap CSCs provide 3-dimensional track segments. Both CSCs and DTs
provide also timing information to identify the bunch crossing corresponding to
candidate muons. The Local DT Trigger is implemented in custom electronics.
BTIs, Bunch and Track Identifiers, search for coincidences of aligned hits in the
four equidistant planes of staggered drift tubes in each chamber superlayer. From
the associated hits, track segments defined by position and angular direction are
determined. TRACOs, Track Correlators, attempt to correlate track segments
measured in the two $\phi$ superlayers of each DT chamber, enhancing the angular resolution and producing a
quality hierarchy.

The requirement of robustness implies redundancy, which introduces, however,
a certain amount of noise or duplicate tracks giving rise to false Triggers. Therefore 
the BTIs, the TRACOs and the different parts of the Local Trigger contain
complex noise and ghost reduction mechanisms. The position, transverse momentum 
and quality of tracks are coded and transmitted to the DT regional Trigger,
called the Drift Tube Track Finder (DTTF), through high-speed optical
links.

The Global Muon Trigger (GMT) combines the information from DTs, CSCs and
RPCs, achieving an improved momentum resolution and efficiency compared to
the stand-alone systems. It also reduces the Trigger rate and suppresses backgrounds 
by making use of the complementarity and redundancy of the three Muon
Systems. The Global Muon Trigger also exploits MIP/ISO bits\footnote{The MIP bit is set if the calorimeter energy is consistent with the passage og a minimum ionizing particle, the isolation bit is set if a certain energy threshold in the trigger towers surrounding the muon is not exceeded.} from the Regional
Calorimeter Trigger. A muon is considered isolated if its energy deposit in the
calorimeter region from which it emerged is below a defined threshold. DT and
CSC candidates are first matched with barrel and forward RPC candidates based
on their spatial coordinates. If a match is possible, the kinematic parameters are
merged. Several merging options are possible and can be selected individually for
all track parameters, taking into account the strengths of the individual Muon Systems. 
Muons are back-extrapolated through the calorimeter regions to the vertex,
in order to retrieve the corresponding MIP and ISO bits, which are then added to
the GMT output and can be taken into account by the Global Trigger (GT). Finally,
the muons are sorted by transverse momentum and quality to deliver four final
candidates to the GT. The Muon Trigger is designed to cover up to $|\eta|<2.4$.

\subsubsection{Global Trigger}
The Global Trigger takes the decision to accept or reject an event at Level 1, based
on candidate $e/\gamma$, muons, jets, as well as global quantities such as the sums of
transverse energies (defined as $E_T=E\sin\theta$), the missing transverse energy and its direction, the scalar transverse energy sum of all jets above a chosen threshold (usually
identified by the symbol $H_T$), and several threshold-dependent jet multiplicities.
Objects representing particles and jets are ranked and sorted. Up to four objects
are available and characterized by their $p_T$ or $E_T$, direction and quality. Charge,
MIP and ISO bits are also available for muons. The Global Trigger has five basic
stages implemented in Field-Programmable Gate-Arrays (FPGAs): input, logic, decision, distribution and read-out.
If the Level 1 Accept decision is positive, the event is sent to the Data Acquisition
stage.

\subsubsection{High Level Trigger and Data Acquisition}
The CMS Trigger and DAQ system is designed to collect and analyse the detector
information at the LHC bunch crossing frequency of 40 MHz. The DAQ system 
must sustain a maximum input rate of 100 kHz,
and must provide enough computing power for a software filter system, the High
Level Trigger (HLT), to reduce the rate of stored events by a factor of 1000. In CMS
all events that pass the Level 1 Trigger are sent to a computer farm (Event Filter)
that performs physics selections, using faster versions of the offline reconstruction
software, to filter events and achieve the required output rate. The various subdetector 
front-end systems store data continuously in 40 MHz pipelined buffers.
Upon arrival of a synchronous Level 1 Trigger Accept via the Timing, Trigger and
Control System (TTCS) the corresponding data are extracted from the front-end
buffers and pushed into the DAQ system by the Front-End Drivers (FEDs). The
event builder assembles the event fragments belonging to the same Level 1 Trigger
from all FEDs into a complete event, and transmits it to one Filter Unit (FU) in
the Event Filter for further processing. The DAQ system includes back-pressure
from the filter farm through the event builder to the FEDs. During operation,
Trigger thresholds and pre-scales will be optimized in order to fully utilize the
available DAQ and HLT throughput capacity.

\section{Monte Carlo Event Generator}
Monte Carlo (MC) event generators provide an event-by-event prediction of complete 
hadronic final states based on QCD calculation. They allow to study the
topology of events generated in hadronic interactions and are used as input for
detector simulation programs to investigate detector effects. The event simulation
is divided into different stages as illustrated in figure \ref{MC_sch}. First, the partonic
cross section is evaluated by calculating the matrix element in fixed order pQCD.
The event generators presently available for the simulation of proton-proton collisions 
provide perturbative calculations for beauty production up to NLO. Higher
order corrections due to initial and final state radiation are approximated by running 
a parton shower algorithm. The parton shower generates a set of secondary
partons originating from subsequent gluon emission of the initial partons. It is
followed by the hadronization algorithm which clusters the individual partons into
colour-singlet hadrons. In a final step, the short lived hadrons are decayed. In the
framework of the analysis presented here, the MC event generator PYTHIA 6.4
\cite{PHY} is used to compute efficiencies, kinematic distributions, and for comparisons
with the experimental results. This programs were run with its default parameter
settings, except when mentioned otherwise.
\begin{figure}
\centering
\includegraphics[width=0.7\columnwidth]{Images/MC_sch.pdf}
\caption{Schematic view of the subsequent steps of a MC event generator:
matrix element (ME), parton shower (PS), hadronization and decay.}
\label{MC_sch}
\end{figure}

\subsubsection{PYTHIA}
In the PYTHIA program, the matrix elements are calculated in LO pQCD and
convoluted with the proton PDF, chosen herein to be CTEQ6L1. The mass of
the b-quark is set to $m_b=4.8$ GeV. The underlying event is simulated with the
D6T tune% \cite{D6T}
. Pile-up events were not included in the simulation. The parton
shower algorithm is based on a leading-logarithmic approximation for QCD radiation 
and a string fragmentation model (implemented in JETSET% \cite{JET}
) is applied.
The longitudinal fragmentation is described by the Lund symmetric fragmentation 
function % \cite{Lun}
for light quarks and by the Peterson fragmentation function for
charm and beauty quarks, that is
\begin{displaymath}\quad
f(z)\propto\frac{1}{z\Big[1-\frac{1}{z}-\frac{\varepsilon_Q}{(1-z)}\Big]^2} \quad,
\end{displaymath}
where $z$ is defined as
\begin{displaymath}\quad
z=\frac{(E+p_\parallel)_\mathrm{hadron}}{(E+p)_\mathrm{quark}} \quad,
\end{displaymath}
$(E+p_\parallel)_\mathrm{hadron}$ is the sum of the energy and momentum component parallel to the fragmentation direction carried by the primary hadron, $(E+p)_\mathrm{quark}$ is the energy-momentum of the quark after accounting for initial state radiation, gluon bremsstrahlung and photon radiation in the final state. The parameters of the Peterson fragmentation function
are set to $\epsilon_c=0.05$ and $\epsilon_b=0.005$. In order to estimate the systematic uncertainty 
introduced by the choice of the fragmentation function, samples generated
with different values of $\epsilon_b$ are studied. The hadronic decay chain used in PYTHIA
is also implemented by the JETSET program. For comparison, additional event
samples are generated where the EvtGen program is used to decay the b-hadrons. 
EvtGen is an event generator designed for the simulation of the physics
of b-hadron decays, and in particular provides a framework to handle complex
sequential decays and CP violating decays.


\clearpage

\chapter{Data collection and event selection}\label{sec:selection}

The data used for this analysis have been collected by CMS detector during 2012 $pp$ run, at a centre-of-mass energy $\sqrt{s}=8\TeV$.
The integrated luminosity collected and certified is \SI{20.5}{\per\femto\barn}.

The events have been selected by two-level criteria: firstly an online trigger selection is used during data taking, and then an offline selection and candidate identification is performed after the events are fully reconstructed.

\section{Online event selection}
\label{sec:onsel}
All the events used in this analysis, both for signal regions and control regions, are selected by a single trigger.
This requires the presence of at least one pair of reconstructed muons in the event with opposite charge, each of them with a transverse momentum greater than $3.5\GeV$, a pseudo-rapidity smaller, in module, than $2.2$, and a distance of closest approach with respect to the beam axis smaller than \SI{2}{\centi\metre}.
The dimuon system is required to have a transverse momentum greater than $6.9\GeV$, an invariant mass between $1\GeV$ and $4.8\GeV$, and a distance of closest approach (DCA) between the muons smaller than \SI{0.5}{\centi\metre}.
In addition, the two muons are required to form a common vertex, with fit $\chi^2$ probability greater than $10\%$, a flight distance significance with respect to the beamspot, measured in the plane transverse to the beam axis, greater than 3, and $\cos{\alpha}>0.9$, where $\alpha$ is the angle in the transverse plane between the dimuon momentum vector and the vector from the beamspot to the dimuon vertex.

No requirements on the hadronic particles in the final state are present at trigger level.

\section{Offline candidate identification}
\label{sec:offsel}

In the offline selection the full final state, composed by two muons and two hadrons, is reconstructed.
A set of four reconstructed objects compatible with those four particles is considered as a candidate.
The offline cuts are applied independently to each candidate in an event.

\subsubsection{Candidate pre-selection cuts}

The two reconstructed muons are required to have opposite charge and match those that triggered the event.
This is done by requiring $\Delta R = \sqrt{(\Delta\eta)^2+(\Delta\phi)^2}<0.1$, where $\Delta\eta$ and $\Delta\phi$ are the pseudorapidity and azimuthal angle differences between the directions of the muons reconstructed at trigger level and in the offline analysis.
In addition, they have to satisfy general muon identification requirements: the muon track candidate from the silicon  tracker must match a track segment from the muon detector, the $\chi^2$ per degree of freedom in a global fit to the silicon tracker and muon detector hits must be less than 1.9, there must be at least 6 silicon tracker hits, including at least 2 from the pixel detector, and the transverse and longitudinal impact parameters with respect to the beamspot must be less than \SI{3}{\centi\metre} and \SI{30}{\centi\metre}, respectively.
The same requirements applied at trigger level and described in Section~\ref{sec:onsel} are also applied to the offline reconstructed dimuon system.

The two charged hadron candidates are required to have opposite charge and each of them must fail the muon identification criteria.

The \PBz\ candidates are obtained by fitting a second time the four charged tracks after applying a common vertex constraint.
This operation is used to improve the resolution of the track parameters.
The \PBz\ candidates is required to have a transverse momentum greater than 8\GeV, and a pseudorapidity smaller, in module, than 2.2.

Since the CMS detector does not have particle identification capability, in each candidate is still present an ambiguity: the mass of the kaon can be assigned to the positive charged hadron track and the mass of the pion to the negative one to reconstruct a \cPKstz candidate, or viceversa to reconstruct a \cPAKstz candidate.
The invariant mass $m$ of the \PBz\ candidate is required to be within 280\MeV of the nominal \PBz\ mass $(m_\PBz)$~\cite{PDG}, for at least one of the mass assignment hypothesis, $\PKm\Pgpp\Pgmp\Pgmm$ or $\PKp\Pgpm\Pgmp\Pgmm$.

The mass sideband is defined as the set of \PBz\ candidates with $3\sigma_m < \abs{m-m_{\PBz}} < 280\MeV$, where $\sigma_m$ is the average mass resolution (${\approx}45$\MeV) as obtained from fitting the $m$ distribution of simulated signal events with a sum of two Gaussian functions with a common mean.



\subsubsection{Candidate optimised selection cuts}

The two charged hadrons of the candidate are required to have a transverse momentum greater than 0.8\GeV, and a significance of the extrapolated distance $d$ of closest approach to the beamspot in the transverse plane greater 2.
The uncertainty associated to $d$ is defined as the sum in quadrature of the uncertainty of the track position and the beamspot transverse size.

For at least one of these two identity assignment hypotheses, the hadron pair invariant mass is requested to be within 90\MeV of the nominal \cPKstz\ mass~\cite{PDG}.

The \PBz vertex fit $\chi^2$ probability must be larger than 10\%, while the distance from the beamspot in the transverse plane, $L$, must be greater than 12 times the sum in quadrature of the uncertainty on $L$ and the beamspot transverse size.
Then, $\cos{\alpha_{xy}}$, where $\alpha_{xy}$ is the angle in the transverse plane between the \PBz\ momentum vector and the line-of-flight between the beamspot and the \PBz\ vertex, is required to be larger than 0.9994.

The values of the cuts on the hadronic track transverse momentum and $d$ significance, the \cPKstz\ mass window, and the \PBz\ candidate vertex fit probability, displacement significance and pointing angle are optimised by maximising the signal significance in the region $\abs{m-m_{\PBz}} < 2.5\sigma_m$, using signal event samples from simulation and background event samples from sideband data in $m$.

After applying these selection criteria, about 5\% of the events have more than one candidate.
For these events, a single candidate is chosen based on the best \PBz\ vertex $\chi^2$ probability.

\subsubsection{Additional selection cuts}

For each candidate, the dimuon invariant mass $q$ and its uncertainty $\sigma_{q}$ are calculated.
Two control samples, corresponding to the \BtoKstJpsi and \BtoKstpsip decay channels, are defined by the requirements $\abs{q - m_{\cPJgy}} < 3\sigma_{q}$ and $\abs{q - m_{\psi'}} < 3\sigma_{q}$, respectively, where $m_{\cPJgy}$ and $m_{\psi'}$ are the nominal masses~\cite{PDG} of the indicated meson.
On average, the value of $\sigma_{q}$ is about 26\MeV.

The distribution of the \PBz invariant mass for the events in signal $q^2$ regions is shown in Figure~\ref{fig:sigInvMass}, while the same distributions for events in the control regions are shown in Figure~\ref{fig:normInvMass}.
The peaks corresponding to the \PBz decays are clearly visible in all three distributions, and can be distinguished from the exponential background shape.

\begin{figure}[hbtp]
  \begin{center}
    \includegraphics[width=0.6\columnwidth]{SigInvMass.pdf}
    \caption{\PBz invariant mass from data, computed from the whole $q^2$ spectrum excluding the $\JPsi$ and $\psi'$ ranges as described in the text.
      Just to guide the eye the plot is fitted with a double Gaussian function with unique mean to measure the signal yield
      ($1232 \pm 44$ events) and with two Gaussian functions and a double exponential to distinguish the background.}
    \label{fig:sigInvMass}
  \end{center}
\end{figure}

\begin{figure}[hbtp]
  \begin{center}
    \includegraphics[width=0.45\columnwidth]{NormInvMass.pdf}
    \includegraphics[width=0.45\columnwidth]{PsiPInvMass.pdf}
    \caption{\PBz invariant mass for both control channels, \BtoKstJpsimumu (left) and \BtoKstpsipmumu (right), from data.
      Just to guide the eye the plot is fitted with a double Gaussian function with unique mean and an exponential to describe the signal and the background respectively.}
    \label{fig:normInvMass}
  \end{center}
\end{figure}

A strong contamination from \BtoKstJpsi and \BtoKstpsip decays is still present in the sample of events passing the selections, mainly because of unreconstructed soft photons in the charmonium decay, i.e., $\cPJgy\ {\rm or}\ \psi' \to \Pgmp \Pgmm \Pgg$.
These events have $q<m_{\cPJgy}$ and $q<m_{\psi'}$, respectively, and are not included in the control sample described above.
They also have $m$ value lower than $m_{\PBz}$, and they can be efficiently removed by a combined requirement on $q$ and $m$.
For $q<m_{\cPJgy}$ $(q>m_{\cPJgy})$, it is required that $\abs{(m-m_{\PBz})-(q-m_{\cPJgy})}>160\: (60)\MeV$.
For $q<m_{\psi'}$ $(q>m_{\psi'})$, it is required that $\abs{(m-m_{\PBz})-(q-m_{\psi'})}>60\: (30)\MeV$.
These cuts are tuned using MC simulations, in such a way that less than 10\% of the background events with $q^2$ values close to the control regions originate from the control channels.

%% The selection criteria are such that they do not depend upon the choice of the primary vertex, and their optimization procedure makes use of both MC simulated signal events generated with the same pileup distribution as in data, and sideband data.

To avoid the contamination from $\Pgf\to\PKp\PKm$ decays, we additionally require that the invariant mass of the hadron pair, in the hypothesis that both tracks have the kaon mass, $m(\PKp\PKm)$, is larger than 1.035\GeV.
This cut has been tuned using the data/MC comparison of the $m(\PKp\PKm)$ distribution in the \BtoKstJpsimumu control channel, as shown in Figure~\ref{fig:kkmass}.
%% As shown in Fig.~\ref{fig:phicut}, where $m(\PKp\PKm)$ distribution is plotted for both  MC events and 

\begin{figure}[hbtp]
  \begin{center}
    \includegraphics[width=0.6\columnwidth]{KKMass.pdf}
    \caption{Invariant mass of the two hadron tracks when the kaon mass is assigned to both hadrons.
      The plot is obtained after applying all selections but the one on the invariant mass of the two hadron tracks with kaon mass assigned.
      The two superimposed plots are obtained from simulation and data, the former with the control channel \BtoKstJpsimumu, the latter is background subtracted (the plot of the sidebands is subtracted from the plot of the signal region), and no $\JPsi$ nor $\psi'$ rejections are applied, therefore the spectrum is dominated by events from the decay \BtoKstJpsimumu.
      The first peak on the left corresponds precisely to the $\phi$ particle (m($\phi$) = 1020\MeV).
      Superimposed to the $\phi$ peak there is shown the Gaussian fit ($\sigma = 4.7\pm0.1$\MeV).
      The vertical dashed line corresponds to the 1.035\GeV selection cut.}
    \label{fig:kkmass}
  \end{center}
\end{figure}

After applying the full set of requirements described here 3191 events remain in the data sample, including the sideband region.

\subsubsection{CP-state assignment}

The selected four-track candidate is identified as a \PBz\ or $\PaBz$, and the corresponding masses are assigned to the hadronic tracks, depending on whether the $\PKp\Pgpm$ or $\PKm\Pgpp$ invariant mass hypothesis is closest to the nominal \cPKstz\ mass.
The candidates assigned to the correct state will be called right-tagged, while we will refer to the candidates assigned to the incorrect state as mis-tagged.
The fraction of mis-tagged events is estimated from simulation to be in the range 12--14\%, depending on $q^2$.

\section{Dimuon mass square binning}
\label{sec:q2}

%% The $q^2$ bins to be used in the analysis are defined in the Table~\ref{tab:q2 bins}.
%% They are chosen in such way to match the measurements performed in previous experiments.

%% Nine bins of $q^2$ are used in the analysis, including two which are dominated
%% by the control samples.
The $q^2$ range used in this analysis extends from 1\GeV$^2$ to 19\GeV$^2$ and it is divided in nine bins, defined in Table~\ref{tab:q2bins}.
The bin definition is the same used in the previous CMS angular analysis on the same dataset and it is the result of a compromise between being coherent with the definition used in the previous measurements and having an expected signal yield homogeneously distributed over the $q^2$ bins.

\begin{table}[!htb]
  \begin{center}
    \begin{small}
      \caption{Range definition of the dimuon invariant mass bins.
        %% Both $J/\psi$ and $\psi'$ regions, namely $q^2$ bin~4 and bin~6, are exclusively used for the control channels.
        \label{tab:q2bins}}
      \begin{tabular}{c|l}
        Bin index & $q^2$ range ($GeV^2/c^4$) \\
        \hline
        0 & 1-2  \\
        1 & 2-4.3  \\
        2 & 4.3-6  \\
        3 & 6-8.68   \\
        4 & 8.68-10.09 ($J/\Psi$ region) \\
        5 & 10.09-12.86\\
        6 & 12.86-14.18 ($\Psi'$ region)\\
        7 & 14.18-16\\
        8 & 16-19\\
      \end{tabular}
    \end{small}
  \end{center}
\end{table}

The selection criteria and the analysis techniques are identical for any $q^2$ bin and in each of them the analysis is performed independently.
The two $q^2$ bins containing the control channel regions are not used to fit the signal events.


  

\clearpage

\chapter{Efficiency}\label{sec:eff}

The efficiency for signal and for control channels is defined as the ratio of number of events passing the selection and whose selected candidate is matched with the generated final state, over the total number of events generated.
It includes the effects of detector geometric acceptance, the trigger selection efficiency and the offline selection efficiency, and is entirely computed from MC simulation.
The efficiency is built as a three-dimensional function of the angular observables \TL, \TK, and \PHI, and computed independently for each bin of $q^2$, both for signal and for the control regions.
The use of such a function allows to account for any possible correlation among the variables introduced by the selection cuts.

In the CMS official MC samples, not all the generated events are reconstructed, to save computing resources. Before the simulation of the detector response, some basic cuts on the generated kinematic variables, $p_T$ and $\eta$, of the signal final state muon pair are applied to remove the majority of events for which the final state is not completely in the geometric detector acceptance. In this thesis I will refer to this cuts as GEN-filter.

Only the events passing the GEN-filter are then reconstructed and compose the MC samples. To correctly take this into account, the efficiency is split into two different terms: 

\begin{equation}\label{eq:eff}
    \epsilon^{R/M}(q^2,\theta_L,\theta_K,\phi)=\mathcal{A}(q^2,\theta_L,\theta_K,\phi)\times\epsilon_{reco}(q^2,\theta_L,\theta_K,\phi)= \frac{N_{gen}}{D_{gen}}\times\frac{N_{reco}}{D_{reco}}
\end{equation}

where $\mathcal{A}$ is called acceptance and is the fraction of generated events that pass the GEN-filter, and $\epsilon_{reco}$ is the selection efficiency, namely the fraction of events which pass the selection with respect to those which have been reconstructed.

More precisely, the acceptance terms are defined as:
\begin{itemize}
    \item[$D_{gen}$] is the number of generated events which pass a selection $\pt(\PBz)>8~\GeV$ and $|\eta(\PBz)|<2.2$, in a given bin of $q^2$;
    \item[$N_{gen}$] is the number of generated events with pass the above selection and, in addition, requires both muons to have $\pt(\mu_{GEN})>3.3\GeV$ and $|\eta(\mu_{GEN})|<2.3$;
\end{itemize}
This quantity is computed using only generator level quantities, as a function of generator-level angular observables. %%, and make use of the large \texttt{GEN} sample described in Sec.~\ref{sec:dataset}.

The selection efficiency instead, is defined as:
\begin{equation}\label{eq:effReco}
    \epsilon_{reco}(q^2,\theta_L,\theta_K,\phi)= \frac{N_{reco}}{D_{reco}}
\end{equation}
where
\begin{itemize}
    \item[$D_{reco}$] is the number of events in the MC samples that pass the GEN-filter and on which the reconstruction is run;
    \item[$N_{reco}$] is the number of reconstructed events passing all selection cuts defined in Sec.~\ref{sec:selection};
\end{itemize}

The $D_{reco}$ is computed as a function of the generator level observables, while $N_{reco}$ is computed as a function of the reconstructed quantities.
The rationale for this choice is that the effect of the detector resolution, passing from the generated observables to the reconstructed one, is taken into account directly into the efficiency, without the need to add an additional term in the fitting function.

Since we cannot assume that the efficiency function is the same for events where the correct flavour has been identified and events where it is not, it is computed separately for candidates with correct tag ($\epsilon^R$) and wrong tag ($\epsilon^M$).
The classification of an event as wrong or correct tagged is defined by comparing the MC truth with the result of the tagging algorithm described in Sec.~\ref{sec:selection}.

%%%%%%%%%%%%%%%%%%%%%%%%%55
%% An important point is the definition of the observable against which the efficiency should be parametrized. 
%% This choice is driven by the angular analysis, namely by the form of the probability density function (\pdf) which describes the angular shape of the decay $\frac{d^4\Gamma}{dq^2d\Omega}$ as well as by kinematical consideration on how the observables are defined and measured.

%% The complete form of the \pdf can be written as a function of the angles \TL, \TK, and \PHI via a combination of $\cos$ and $\sin$ functions.
%% However, the two $\theta$ angles are defined as the angle between two vector, $\mu^+$ and dimuon system, and $\PK^+$ and \PKst, respectively.
%% As such, they are defined only in the interval $[0,\pi)$, or, in other word, only the $\cos\theta_x$ is defined, via the inner product.
%% So, there is no loss of information by defining $\sin\theta_x=+\sqrt{1-\cos^2\theta_x}$, namely to choose the positive solution, and disregarding the negative one.
%% So the efficiency is parametrized as a function of \cTL, \cTK, and \PHI.

%% It is important to stress that the absolute value of the efficiency is not important for the \pdf fit, but only the shape as a function of the angular observables.

Since the \TK and \TL variables are defined in the range $[0,\pi]$, there is no loss in information in building the efficiency as a function of \TK, \TL and \PHI, or as a function of \cTK, \cTL and \PHI. Since this second choice leads to slightly more smooth functions, it has been chosen for the construction of the efficiency. 

\section{Parameterisation}\label{sec:eff_param}

%% The efficiency is defined as the product of two ratios as described in eq.~\ref{eq:eff} for each bin of $q^2$.
%% In the following, we will take as implicit the dependence on $q^2$ in order to simplify the formula: $\epsilon(q^2,\theta_L,\theta_K,\phi)=\epsilon(\theta_L,\theta_K,\phi)$.
In the first angular analysis based on this dataset, the efficiency was built as a two-dimensional function against the angular variables \TK and \TL.
The method used to parameterise it was composed by two steps.
Firstly, a two-dimensional function is constructed by performing the bin-by-bin ratios of the binned distributions of numerators and denominators.
Then, this binned efficiency is fitted with a polynomial function, which is used as parameterisation of the efficiency.
Some complex procedures were needed in order to have this fit converging and to grant that the final efficiency functions were positive in the whole range of definition.

Most of the problems in this technique were due to both the low MC event statistics in some $q^2$ bins, especially for the mis-tagged event efficiency, and the large number of free parameters in the two-dimensional polynomial function.
Extending this procedure to a three dimensional efficiency would imply the usage of three-dimensional binned distributions, with a global decrease of the bin statistics, and the usage of a three-dimensional polynomial function, with a larger number of free parameters to fit.
For this reason, the usage of this method was not considered for this analysis, but new parameterisation techniques were tested.

Two independent approaches have been tested: a two dimensional reduction of the binned method and a parameterisation based on Kernel Density Estimator (KDE) distributions.
In this section both of them are described and some example functions are showed.
In the next section a closure test developed to determine their accuracy is described and the better performances of the KDE-based method will be highlighted.

%% The first extend the efficiency description from two to three dimensions by considering two variables at a time, while integrating over the third, and then iterating on the variables choice.
\subsection{Two-dimensional binned method}\label{sec:eff_2Dprod}
The first method is based on binned efficiency functions.
Instead of building directly a three-dimensional binned efficiency, facing the low per-bin statistics, we use two-dimensional functions to compose the final efficiency.
To model the efficiency correlation between each couple of parameters, three functions are created by integrating out, recursively, one angular variable and building the efficiency as a function of the remaining two.
The final three dimensional efficiency is then defined as the product of the three two dimensional ones.
In formula:
\begin{equation}\label{eq:eff2d}
    \epsilon(\cTL,\cTK,\phi)=F\times\epsilon(\cTL,\cTK)\times\epsilon(\cTL,\phi)\times\epsilon(\cTK,\phi)
\end{equation}
Since the product of the three functions results not correctly normalised, it is rescaled by a factor $F$, in such a way that the global efficiency, averaged over all the angular variables, matches the simple ratio of the number of events in the numerator and denominator distributions.

This approach has the advantage to reduce the three-dimensional problem to three two-dimensional ones, easier to deal with and less prone to lack of statistics.
The correlation among the three angular variables are taken into account by the final product.
A disadvantage is that the efficiency is computed from binned distributions, where the bin width is determined by the statistics of events available in the MC samples used.
Any structure or behaviour inside a given bin is not resolved, and some kind of smoothing is needed.

The solution used to mitigate this binning problem is to use an interpolation between the bins of each two-dimensional efficiencies of Equation~\ref{eq:eff2d}.
Both a linear and parabolic interpolations have been tested.

An example of the two dimension binned efficiencies after a linear interpolation is applied, for the $q^2$ bin~1, are shown in Fig.~\ref{fig:eff2D}.

\begin{figure}[hbt]
    \includegraphics[width=1.\textwidth]{Plot/cEff2d3_2D_q2bin1.pdf}
    \caption{Distribution of the two-dimensional binned efficiency of equation~\ref{eq:eff2d} after the application of a linear interpolation between the bins, for $q^2$ bin~1: $\epsilon(\cTL,\cTK)$ (left), $\epsilon(\cTL,\phi)$ (centre), and $\epsilon(\cTK,\phi)$ (right).}
    \label{fig:eff2D}
\end{figure}

\subsection{Kernel Density Estimator method}\label{sec:eff_kde}
The second approach uses a non-parametric description of the efficiency based on a Kernel Density Estimator (KDE)~\cite{opac-b1089297,Cranmer:2000du}.

The general idea behind this method, in its simplest form for a uni-dimensional problem, is to start from an unbinned distribution of events as a function of a given variable $x$, and describe its true distribution \pdftrue by building a kernel, namely a function $K(x)$ with unitary integral over the range of definition of $x$, on top of every event, and use the sum of all kernels, with proper normalisation, as a non-parametric description of \pdftrue:
\begin{equation}\label{eq:KDE}
    \pdfKDEx=\frac{1}{N}\sum_{i=1}^{N} K(x-x_i)
\end{equation}
where $i$ is an index running over the set of events and $N$ is the total number of events.
The kernel $K(x)$ can be any non-negative function with unitary integral, and can be of different type; some common examples are uniform, triangular, or Gaussian kernels.
A remarkable property of the KDE technique is that the quality of the PDF description depends rather weakly on the kernel used.
The resulting \pdfKDEx in the limit $N\to\infty$ is a convolution of \pdftrue with the kernel $K(x)$.
In this analysis, we decided to use a Gaussian kernel, because of its ability to produce a smooth \pdf even in regions where a low event statistics is available.

The KDE method can be extended to multi-dimensional problems, building a \pdf as a function of a set of $n$ variables $x_1,\ldots,x_n)$.
In this case, the kernel used should be $n$-dimensional as well.

An implementation of KDE method using a Gaussian kernel is available, for both the uni-dimensional and the multi-dimensional application, within the {\sc RooFit} package~\cite{RooFit}.

The advantage of the KDE over a parametric description is that no prior assumption on the actual \pdftrue is needed, and the structure of the original \pdf is described as accurately as the statistics allow.
The last point is particularly relevant in the case under study, since in spite of the large statistics of events in the MC samples used to compute the efficiency, the phase space is very large, given the number of variables considered for the efficiency description.
A disadvantage is that an unbinned distribution of events is needed, so it is not possible to use the KDE method directly on the efficiency, which is binned by construction, but it needs to be used on the numerator and denominator distributions, and then the efficiency will be defined through the ratios of these \pdfs.

The efficiency at the observable boundaries is not expected to be null, especially for \PHI, given its periodic nature. 
In such cases, the kernel estimator would introduce a significant decreasing of the \pdf, since it would behave, close to the border, like a convolution between a step function and a Gaussian function.
In order to improve the behaviour of the modelling close the observable boundaries, data are mirrored across the boundaries.
It means that, for any Gaussian function added for a data point next to a boundary, the tail of this function that exceeds it is reflected inside the boundary.

Even if the KDE with Gaussian kernel is a non-parametric method, the values of the widths and correlation terms of the multi-dimensional kernel need to be determined.
For simplicity, the correlation terms are kept equal to zero; the usage of these terms could be useful when the distribution to model shows a strong correlation between two or more of its variables, but it is not the case for these numerator and denominator distributions.
The {\sc RooFit} class used to build the \pdfKDE uses a set of standard values for the width parameters, determined as a function of the variables range of definition.
An allowed tuning is the definition of an global scale factor, applied to the whole set of width parameters.
On one hand, a wide kernel is less sensitive to the limited statistic of the data sample and allows to produce smoother \pdfs.
On the other hand, a narrow kernel reproduces in a more accurate way the fine structures of the original distribution, especially when this distribution is steep.
Several width scale factors (1.0, 0.5, 0.3) have been tested and the choice of the value used for the efficiency is based on the goodness of the results of the closure test described in Section~\ref{sec:closure}.

%% A even more optimal solution would have been to use an adaptive approach, where
%% a non-constant width of the kernel is used, depending on the density of the
%% data in the around the kernel centroid. This method is known to behave better
%% than a fixed width one, but at the cost of a important overhead in the actual
%% modeling computing time ($\gtrsim10x$).
%% Given the size of the datasets used in this analysis, and since the closure
%% test of the fixed width approach are good~\ref{sec:closure}, the adaptive
%% approach has been disregarded.

The procedure used to obtain the efficiency function is to use the KDE method on each of the four distributions ($N_{gen}, D_{gen}, N_{reco}, D_{reco}$) used in Equation~\ref{eq:eff}, normalise each of the resulting \pdfs to the number of events in the original distributions, and then combine the four functions into final efficiency.
For technical reason, it is not possible, in ROOT, to save the output of a multidimensional KDE \pdf into an output file and the time consumption to create the efficiency function, starting from the MC event distributions, any time the fit procedure is run is too large.
For this reason, it was decided to save a binned distribution obtained from the sampling of the KDE \pdfs: numerator and denominator functions, obtained via the KDE algorithm, have been sampled into a three dimensional histogram, with $40\times40\times40$ bins, then the four histograms were combined with a bin-per-bin application of Equation~\ref{eq:eff} in an efficiency function, which has been finally saved in the output file.
%% The sampling to a 3D histogram is a workaround for this deficiency.
The granularity of the bins have been chosen as a compromise between the fine structure of the efficiency distribution and computing time needed to perform the \pdf sampling.

An example of the event distribution of the four terms of Equation~\ref{eq:eff}, $D_{gen}$, $N_{gen}$, $D_{reco}$, and $N_{reco}$, for right-tagged events in $q^2$ bin~1, as functions of each of the angular variables, together with the projections of the correctly-normalised \pdfs, as obtained by the KDE method with width scale factor equal to 0.5, are shown in Fig.~\ref{fig:NDKde1}.

%% \begin{figure}[hbt]
%%     \subfigure[$N_{gen}$]{\includegraphics[width=1.\textwidth]{Plot/can1D_h3genNum__CosThetaK_AbsCosThetaMu_AbsPhiKstMuMuPlane_Bin1.pdf}}

%%     \subfigure[$D_{gen}$]{\includegraphics[width=1.\textwidth]{Plot/can1D_h3genDen__CosThetaK_AbsCosThetaMu_AbsPhiKstMuMuPlane_Bin1.pdf}}

%%     \subfigure[$N_{reco}$]{\includegraphics[width=1.\textwidth]{Plot/can1D_h3recoNum__CosThetaK_AbsCosThetaMu_AbsPhiKstMuMuPlane_Bin1.pdf}}

%%     \subfigure[$D_{reco}$]{\includegraphics[width=1.\textwidth]{Plot/can1D_h3recoDen__CosThetaK_AbsCosThetaMu_AbsPhiKstMuMuPlane_Bin1.pdf}}

%%     \caption{Distribution of the four terms of eq.~\ref{eq:eff}, $D_{gen}$,
%%         $N_{gen}$, $D_{reco}$, and $N_{reco}$ (black dot), for $q^2$ bin 0
%%         togheter with the kernel estimator {\texttt pdf} (red line)}
%%     \label{fig:NDKde0}
%% \end{figure}

\begin{figure}[hbt]
    \subfigure[$N_{gen}$]{\includegraphics[width=1.\textwidth]{Plot/can1D_h3genNum__CosThetaK_AbsCosThetaMu_AbsPhiKstMuMuPlane_Bin2.pdf}}
    \subfigure[$D_{gen}$]{\includegraphics[width=1.\textwidth]{Plot/can1D_h3genDen__CosThetaK_AbsCosThetaMu_AbsPhiKstMuMuPlane_Bin2.pdf}}
    \subfigure[$N_{reco}$]{\includegraphics[width=1.\textwidth]{Plot/can1D_h3recoNum__CosThetaK_AbsCosThetaMu_AbsPhiKstMuMuPlane_Bin2.pdf}}
    \subfigure[$D_{reco}$]{\includegraphics[width=1.\textwidth]{Plot/can1D_h3recoDen__CosThetaK_AbsCosThetaMu_AbsPhiKstMuMuPlane_Bin2.pdf}}
    \caption{Event distributions of the four efficiency terms of Equation~\ref{eq:eff}, $D_{gen}$, $N_{gen}$, $D_{reco}$, and $N_{reco}$ (black dot), for right-tagged events in $q^2$ bin~1, together with the projections of the correctly-normalised \pdfs, as obtained by the KDE method with width scale factor equal to 0.5, (red line)}
    \label{fig:NDKde1}
\end{figure}

%% \begin{figure}[hbt]
%%     \subfigure[$N_{gen}$]{\includegraphics[width=1.\textwidth]{Plot/can1D_h3genNum__CosThetaK_AbsCosThetaMu_AbsPhiKstMuMuPlane_Bin3.pdf}}

%%     \subfigure[$D_{gen}$]{\includegraphics[width=1.\textwidth]{Plot/can1D_h3genDen__CosThetaK_AbsCosThetaMu_AbsPhiKstMuMuPlane_Bin3.pdf}}

%%     \subfigure[$N_{reco}$]{\includegraphics[width=1.\textwidth]{Plot/can1D_h3recoNum__CosThetaK_AbsCosThetaMu_AbsPhiKstMuMuPlane_Bin3.pdf}}

%%     \subfigure[$D_{reco}$]{\includegraphics[width=1.\textwidth]{Plot/can1D_h3recoDen__CosThetaK_AbsCosThetaMu_AbsPhiKstMuMuPlane_Bin3.pdf}}

%%     \caption{Distribution of the four terms of eq.~\ref{eq:eff}, $D_{gen}$,
%%         $N_{gen}$, $D_{reco}$, and $N_{reco}$ (black dot), for $q^2$ bin 2
%%         togheter with the kernel estimator {\texttt pdf} (red line)}
%%     \label{fig:NDKde2}
%% \end{figure}

%% \begin{figure}[hbt]
%%     \subfigure[$N_{gen}$]{\includegraphics[width=1.\textwidth]{Plot/can1D_h3genNum__CosThetaK_AbsCosThetaMu_AbsPhiKstMuMuPlane_Bin4.pdf}}

%%     \subfigure[$D_{gen}$]{\includegraphics[width=1.\textwidth]{Plot/can1D_h3genDen__CosThetaK_AbsCosThetaMu_AbsPhiKstMuMuPlane_Bin4.pdf}}

%%     \subfigure[$N_{reco}$]{\includegraphics[width=1.\textwidth]{Plot/can1D_h3recoNum__CosThetaK_AbsCosThetaMu_AbsPhiKstMuMuPlane_Bin4.pdf}}

%%     \subfigure[$D_{reco}$]{\includegraphics[width=1.\textwidth]{Plot/can1D_h3recoDen__CosThetaK_AbsCosThetaMu_AbsPhiKstMuMuPlane_Bin4.pdf}}

%%     \caption{Distribution of the four terms of eq.~\ref{eq:eff}, $D_{gen}$,
%%         $N_{gen}$, $D_{reco}$, and $N_{reco}$ (black dot), for $q^2$ bin 3
%%         togheter with the kernel estimator {\texttt pdf} (red line)}
%%     \label{fig:NDKde3}
%% \end{figure}

%% \begin{figure}[hbt]
%%     \subfigure[$N_{gen}$]{\includegraphics[width=1.\textwidth]{Plot/can1D_h3genNum__CosThetaK_AbsCosThetaMu_AbsPhiKstMuMuPlane_Bin5.pdf}}

%%     \subfigure[$D_{gen}$]{\includegraphics[width=1.\textwidth]{Plot/can1D_h3genDen__CosThetaK_AbsCosThetaMu_AbsPhiKstMuMuPlane_Bin5.pdf}}

%%     \subfigure[$N_{reco}$]{\includegraphics[width=1.\textwidth]{Plot/can1D_h3recoNum__CosThetaK_AbsCosThetaMu_AbsPhiKstMuMuPlane_Bin5.pdf}}

%%     \subfigure[$D_{reco}$]{\includegraphics[width=1.\textwidth]{Plot/can1D_h3recoDen__CosThetaK_AbsCosThetaMu_AbsPhiKstMuMuPlane_Bin5.pdf}}

%%     \caption{Distribution of the four terms of eq.~\ref{eq:eff}, $D_{gen}$,
%%         $N_{gen}$, $D_{reco}$, and $N_{reco}$ (black dot), for $q^2$ bin 4
%%         togheter with the kernel estimator {\texttt pdf} (red line)}
%%     \label{fig:NDKde4}
%% \end{figure}

%% \begin{figure}[hbt]
%%     \subfigure[$N_{gen}$]{\includegraphics[width=1.\textwidth]{Plot/can1D_h3genNum__CosThetaK_AbsCosThetaMu_AbsPhiKstMuMuPlane_Bin6.pdf}}

%%     \subfigure[$D_{gen}$]{\includegraphics[width=1.\textwidth]{Plot/can1D_h3genDen__CosThetaK_AbsCosThetaMu_AbsPhiKstMuMuPlane_Bin6.pdf}}

%%     \subfigure[$N_{reco}$]{\includegraphics[width=1.\textwidth]{Plot/can1D_h3recoNum__CosThetaK_AbsCosThetaMu_AbsPhiKstMuMuPlane_Bin6.pdf}}

%%     \subfigure[$D_{reco}$]{\includegraphics[width=1.\textwidth]{Plot/can1D_h3recoDen__CosThetaK_AbsCosThetaMu_AbsPhiKstMuMuPlane_Bin6.pdf}}

%%     \caption{Distribution of the four terms of eq.~\ref{eq:eff}, $D_{gen}$,
%%         $N_{gen}$, $D_{reco}$, and $N_{reco}$ (black dot), for $q^2$ bin 5
%%         togheter with the kernel estimator {\texttt pdf} (red line)}
%%     \label{fig:NDKde5}
%% \end{figure}

%% \begin{figure}[hbt]
%%     \subfigure[$N_{gen}$]{\includegraphics[width=1.\textwidth]{Plot/can1D_h3genNum__CosThetaK_AbsCosThetaMu_AbsPhiKstMuMuPlane_Bin7.pdf}}

%%     \subfigure[$D_{gen}$]{\includegraphics[width=1.\textwidth]{Plot/can1D_h3genDen__CosThetaK_AbsCosThetaMu_AbsPhiKstMuMuPlane_Bin7.pdf}}

%%     \subfigure[$N_{reco}$]{\includegraphics[width=1.\textwidth]{Plot/can1D_h3recoNum__CosThetaK_AbsCosThetaMu_AbsPhiKstMuMuPlane_Bin7.pdf}}

%%     \subfigure[$D_{reco}$]{\includegraphics[width=1.\textwidth]{Plot/can1D_h3recoDen__CosThetaK_AbsCosThetaMu_AbsPhiKstMuMuPlane_Bin7.pdf}}

%%     \caption{Distribution of the four terms of eq.~\ref{eq:eff}, $D_{gen}$,
%%         $N_{gen}$, $D_{reco}$, and $N_{reco}$ (black dot), for $q^2$ bin 6
%%         togheter with the kernel estimator {\texttt pdf} (red line)}
%%     \label{fig:NDKde6}
%% \end{figure}

%% \begin{figure}[hbt]
%%     \subfigure[$N_{gen}$]{\includegraphics[width=1.\textwidth]{Plot/can1D_h3genNum__CosThetaK_AbsCosThetaMu_AbsPhiKstMuMuPlane_Bin8.pdf}}

%%     \subfigure[$D_{gen}$]{\includegraphics[width=1.\textwidth]{Plot/can1D_h3genDen__CosThetaK_AbsCosThetaMu_AbsPhiKstMuMuPlane_Bin8.pdf}}

%%     \subfigure[$N_{reco}$]{\includegraphics[width=1.\textwidth]{Plot/can1D_h3recoNum__CosThetaK_AbsCosThetaMu_AbsPhiKstMuMuPlane_Bin8.pdf}}

%%     \subfigure[$D_{reco}$]{\includegraphics[width=1.\textwidth]{Plot/can1D_h3recoDen__CosThetaK_AbsCosThetaMu_AbsPhiKstMuMuPlane_Bin8.pdf}}

%%     \caption{Distribution of the four terms of eq.~\ref{eq:eff}, $D_{gen}$,
%%         $N_{gen}$, $D_{reco}$, and $N_{reco}$ (black dot), for $q^2$ bin 7
%%         togheter with the kernel estimator {\texttt pdf} (red line)}
%%     \label{fig:NDKde7}
%% \end{figure}

%% \begin{figure}[hbt]
%%     \subfigure[$N_{gen}$]{\includegraphics[width=1.\textwidth]{Plot/can1D_h3genNum__CosThetaK_AbsCosThetaMu_AbsPhiKstMuMuPlane_Bin9.pdf}}

%%     \subfigure[$D_{gen}$]{\includegraphics[width=1.\textwidth]{Plot/can1D_h3genDen__CosThetaK_AbsCosThetaMu_AbsPhiKstMuMuPlane_Bin9.pdf}}

%%     \subfigure[$N_{reco}$]{\includegraphics[width=1.\textwidth]{Plot/can1D_h3recoNum__CosThetaK_AbsCosThetaMu_AbsPhiKstMuMuPlane_Bin9.pdf}}

%%     \subfigure[$D_{reco}$]{\includegraphics[width=1.\textwidth]{Plot/can1D_h3recoDen__CosThetaK_AbsCosThetaMu_AbsPhiKstMuMuPlane_Bin9.pdf}}

%%     \caption{Distribution of the four terms of eq.~\ref{eq:eff}, $D_{gen}$,
%%         $N_{gen}$, $D_{reco}$, and $N_{reco}$ (black dot), for $q^2$ bin 8
%%         togheter with the kernel estimator {\texttt pdf} (red line)}
%%     \label{fig:NDKde8}
%% \end{figure}


An example of the projections of the efficiency, as obtained by the KDE method with width scale factor equal to 0.5, on each single and on each couple of angular variables, for right-tagged events in $q^2$ bin~1, are shown in Fig.~\ref{fig:effPro1}.

%% \begin{figure}[hbt]
%%     \includegraphics[width=1.\textwidth]{Plot/canProjection1.pdf}
%%     \caption{Projection of the efficiency on two (upper row) and one variable (lower row), integrating over the other variable for $q^2$ bin 0}
%%     \label{fig:effPro0}
%% \end{figure}

\begin{figure}[hbt]
    \includegraphics[width=1.\textwidth]{Plot/canProjection2.pdf}
    \caption{Projections of the efficiency, as obtained by the KDE method with width scale factor equal to 0.5, on two (upper row) and one angular variable (lower row), for right-tagged events in $q^2$ bin~1}
    \label{fig:effPro1}
\end{figure}

%% \begin{figure}[hbt]
%%     \includegraphics[width=1.\textwidth]{Plot/canProjection3.pdf}
%%     \caption{Projection of the efficiency on two (upper row) and one variable (lower row), integrating over the other variable for $q^2$ bin 2}
%%     \label{fig:effPro2}
%% \end{figure}

%% \begin{figure}[hbt]
%%     \includegraphics[width=1.\textwidth]{Plot/canProjection4.pdf}
%%     \caption{Projection of the efficiency on two (upper row) and one variable (lower row), integrating over the other variable for $q^2$ bin 3}
%%     \label{fig:effPro3}
%% \end{figure}

%% \begin{figure}[hbt]
%%     \includegraphics[width=1.\textwidth]{Plot/canProjection5.pdf}
%%     \caption{Projection of the efficiency on two (upper row) and one variable (lower row), integrating over the other variable for $q^2$ bin 4}
%%     \label{fig:effPro4}
%% \end{figure}

%% \begin{figure}[hbt]
%%     \includegraphics[width=1.\textwidth]{Plot/canProjection6.pdf}
%%     \caption{Projection of the efficiency on two (upper row) and one variable (lower row), integrating over the other variable for $q^2$ bin 5}
%%     \label{fig:effPro5}
%% \end{figure}

%% \begin{figure}[hbt]
%%     \includegraphics[width=1.\textwidth]{Plot/canProjection7.pdf}
%%     \caption{Projection of the efficiency on two (upper row) and one variable (lower row), integrating over the other variable for $q^2$ bin 6}
%%     \label{fig:effPro6}
%% \end{figure}

%% \begin{figure}[hbt]
%%     \includegraphics[width=1.\textwidth]{Plot/canProjection8.pdf}
%%     \caption{Projection of the efficiency on two (upper row) and one variable (lower row), integrating over the other variable for $q^2$ bin 7}
%%     \label{fig:effPro7}
%% \end{figure}

%% \begin{figure}[hbt]
%%     \includegraphics[width=1.\textwidth]{Plot/canProjection9.pdf}
%%     \caption{Projection of the efficiency on two (upper row) and one variable (lower row), integrating over the other variable for $q^2$ bin 8}
%%     \label{fig:effPro8}
%% \end{figure}

\clearpage

\section{Closure test}\label{sec:closure}

A validation of the efficiency parameterisation is performed via a closure test based on MC samples.
We compare the event distribution of the three angular observables, \cTK, \cTL, and \PHI, as reconstructed after the full simulation and after the application of the whole set of selections described in Section~\ref{sec:selection}, with the full set of generated event distribution, before the application of the GEN-filter, weighted, event-by-event, with the tested efficiency function.
In order to remove any statistical correlation in the closure test, half of each sample have been used to estimated the efficiency, and the closure has been performed on the other half.

The equality to test is the following:
\begin{equation}\label{eq:effClosure}
    \mathtt{pdf}(q^2,\cTK,\cTL,\phi)_{reco}^{R/M}=\mathtt{pdf}(q^2,\cTK,\cTL,\phi)_{gen}\otimes\epsilon^{R/M}(q^2,\cTK,\cTL,\phi)
\end{equation}

The closure test has been performed for any efficiency function produced: for each $q^2$ bin, separately for right (R) and mis-tag (M), and for signal and control samples.

An example of the closure test results of the first method efficiencies, as described in Section~\ref{eff_2Dprod}, is shown, for right-tagged events in $q^2$ bin~1, for the linear interpolation in Fig.~\ref{fig:clEff2Dlin}, and for the parabolic interpolation in Fig.~\ref{fig:clEff2Dpar}.
The result quality for the linearly-interpolated efficiency does not follow very well the original distribution where it is very steep and it introduces a \PHI dependence where none is present.
The quality improvement due to the second-order interpolation is not enough to have an accurate description of the shape in the \cTK projection and the modulation against \PHI is still present.
An improvement to this technique could come from an increase in the number of bins used to build the two-dimensional components of the efficiency, but the available statistics of the MC samples, especially for the mis-tagged events, prevent to go further with this.

\begin{figure}[hbt]
    \includegraphics[width=.66\textwidth,trim={0 280 0 0},clip]{Plot/Closure2D_lin.pdf}
    \includegraphics[width=.33\textwidth,trim={0 0 280 280},clip]{Plot/Closure2D_lin.pdf}
    \caption{Comparison of the reconstructed and selected MC event distributions (histogram) with generated event distributions (dots) weighted with the efficiency, parameterised as in Equation~\ref{eq:eff2d} with linear interpolation of the two-dimension efficiency functions. Both the distributions and the efficiency refer to right-tagged events in $q^2$ bin~1.}
    \label{fig:clEff2Dlin}
\end{figure}

\begin{figure}[hbt]
    \includegraphics[width=.66\textwidth,trim={0 280 0 0},clip]{Plot/Closure2D_par.pdf}
    \includegraphics[width=.33\textwidth,trim={0 0 280 280},clip]{Plot/Closure2D_par.pdf}
    \caption{Comparison of the reconstructed and selected MC event distributions (histogram) with generated event distributions (dots) weighted with the efficiency, parameterised as in Equation~\ref{eq:eff2d} with parabolic interpolation of the two-dimension efficiency functions. Both the distributions and the efficiency refer to right-tagged events in $q^2$ bin~1.}
    \label{fig:clEff2Dpar}
\end{figure}

An example of the closure test results for the KDE method is shown in Fig.~\ref{fig:clKDEwidthBin1} for right-tagged events in $q^2$ bin~0, and in Fig.~\ref{fig:clKDEwidthBin8} for right-tagged events in $q^2$ bin~7, comparing three different width scale factors: 1.0, 0.5, and 0.3.
It is possible to notice that using smaller widths improves the result quality where some steep features are present, in particular near the boundary of \cTK for $q^2$ bin~0, and of \PHI for $q^2$ bin~7.
On the other hand, a small width tends to create fine structures due to statistic fluctuations, in particular near the boundary of \PHI for $q^2$ bin~0.

\begin{figure}[hbt]
    \includegraphics[width=1.\textwidth]{Plot/closureKDEBin1width10.pdf}
    \includegraphics[width=1.\textwidth]{Plot/closureKDEBin1width05.pdf}
    \includegraphics[width=1.\textwidth]{Plot/closureKDEBin1width03.pdf}
    \caption{Comparison of the reconstructed and selected MC event distributions (histogram) with generated event distributions (dots) weighted with the efficiency, built through the KDE method, with width scale factor set to 1.0 (top), 0.5 (middle), and 0.3 (bottom). Both the distributions and the efficiency refer to right-tagged events in $q^2$ bin~0.}
    \label{fig:clKDEwidthBin1}
\end{figure}

\begin{figure}[hbt]
    \includegraphics[width=1.\textwidth]{Plot/closureKDEBin8width10.pdf}
    \includegraphics[width=1.\textwidth]{Plot/closureKDEBin8width05.pdf}
    \includegraphics[width=1.\textwidth]{Plot/closureKDEBin8width03.pdf}
    \caption{Comparison of the reconstructed and selected MC event distributions (histogram) with generated event distributions (dots) weighted with the efficiency, built through the KDE method, with width scale factor set to 1.0 (top), 0.5 (middle), and 0.3 (bottom). Both the distributions and the efficiency refer to right-tagged events in $q^2$ bin~7.}
    \label{fig:clKDEwidthBin8}
\end{figure}

From the overall results of the closure test, the optimal parameterisation for this analysis in the KDE method. The width scale factor is chosen to be 0.5 for the efficiency for right-tagged events, while the scale factor of 1.0 was found to be optimal for the efficiency of mis-tagged events, for which less statistic is available in the MC samples.
An example of the closure test results for mis-tagged events is shown in Fig.~\ref{fig:clKDEmistag}, for $q^2$ bin~0 and $q^2$ bin~7.

\begin{figure}[hbt]
    \includegraphics[width=1.\textwidth]{Plot/closureTest8_WT_sgn_bin1.pdf}
    \includegraphics[width=1.\textwidth]{Plot/closureTest8_WT_sgn_bin8.pdf}
    \caption{Comparison of the reconstructed and selected MC event distributions (histogram) with generated event distributions (dots) weighted with the efficiency, built through the KDE method. Both the distributions and the efficiency refer to mis-tagged events in $q^2$ bin~0 (top) and in $q^2$ bin~0 (bottom).}
    \label{fig:clKDEmistag}
\end{figure}

%%%%%%%%%%%%%%%%%%%

%% Figures~\ref{fig:clKDE1} through~\ref{fig:clKDE9} show the results of
%% closure test for $q^2$ bin 1 to 9, respectively, for correctly tagged
%% signal events.  The same distributions for wrongly tagged signal
%% events are shown in Fig~\ref{fig:clKDE1wt},~\ref{fig:clKDE9wt}.

%% \begin{figure}[hbt]
%%     \includegraphics[width=1.\textwidth]{Plot/closureTest8_RT_sgn_bin1.pdf}

%%     \caption{Closure test for the reconstructed angles using KDE, for $q^2$ bin 1}
%%     \label{fig:clKDE1}
%% \end{figure}

%% \begin{figure}[hbt]
%%     \includegraphics[width=1.\textwidth]{Plot/closureTest8_RT_sgn_bin2.pdf}

%%     \caption{Closure test for the reconstructed angles using KDE, for $q^2$ bin 2}
%%     \label{fig:clKDE2}
%% \end{figure}

%% \begin{figure}[hbt]
%%     \includegraphics[width=1.\textwidth]{Plot/closureTest8_RT_sgn_bin3.pdf}

%%     \caption{Closure test for the reconstructed angles using KDE, for $q^2$ bin 3}
%%     \label{fig:clKDE3}
%% \end{figure}

%% \begin{figure}[hbt]
%%     \includegraphics[width=1.\textwidth]{Plot/closureTest8_RT_sgn_bin4.pdf}

%%     \caption{Closure test for the reconstructed angles using KDE, for $q^2$ bin 4}
%%     \label{fig:clKDE4}
%% \end{figure}

%% \begin{figure}[hbt]
%%     \includegraphics[width=1.\textwidth]{Plot/closureTest8_RT_sgn_bin5.pdf}

%%     \caption{Closure test for the reconstructed angles using KDE, for $q^2$ bin 5}
%%     \label{fig:clKDE5}
%% \end{figure}

%% \begin{figure}[hbt]
%%     \includegraphics[width=1.\textwidth]{Plot/closureTest8_RT_sgn_bin6.pdf}

%%     \caption{Closure test for the reconstructed angles using KDE, for $q^2$ bin 6}
%%     \label{fig:clKDE6}
%% \end{figure}

%% \begin{figure}[hbt]
%%     \includegraphics[width=1.\textwidth]{Plot/closureTest8_RT_sgn_bin7.pdf}

%%     \caption{Closure test for the reconstructed angles using KDE, for $q^2$ bin 7}
%%     \label{fig:clKDE7}
%% \end{figure}

%% \begin{figure}[hbt]
%%     \includegraphics[width=1.\textwidth]{Plot/closureTest8_RT_sgn_bin8.pdf}

%%     \caption{Closure test for the reconstructed angles using KDE, for $q^2$ bin 8}
%%     \label{fig:clKDE8}
%% \end{figure}

%% \begin{figure}[hbt]
%%     \includegraphics[width=1.\textwidth]{Plot/closureTest8_RT_sgn_bin9.pdf}

%%     \caption{Closure test for the reconstructed angles using KDE, for $q^2$ bin 9}
%%     \label{fig:clKDE9}
%% \end{figure}

%% \clearpage

%% \begin{figure}[hbt]
%%     \includegraphics[width=1.\textwidth]{Plot/closureTest8_WT_sgn_bin1.pdf}

%%     \caption{Closure test using KDE, for $q^2$ bin 1, for wrongly tagged events}
%%     \label{fig:clKDE1wt}
%% \end{figure}

%% \begin{figure}[hbt]
%%     \includegraphics[width=1.\textwidth]{Plot/closureTest8_WT_sgn_bin2.pdf}

%%     \caption{Closure test using KDE, for $q^2$ bin 2, for wrongly tagged events}
%%     \label{fig:clKDE2wt}
%% \end{figure}

%% \begin{figure}[hbt]
%%     \includegraphics[width=1.\textwidth]{Plot/closureTest8_WT_sgn_bin3.pdf}

%%     \caption{Closure test using KDE, for $q^2$ bin 3, for wrongly tagged events}
%%     \label{fig:clKDE3wt}
%% \end{figure}

%% \begin{figure}[hbt]
%%     \includegraphics[width=1.\textwidth]{Plot/closureTest8_WT_sgn_bin4.pdf}

%%     \caption{Closure test using KDE, for $q^2$ bin 4, for wrongly tagged events}
%%     \label{fig:clKDE4wt}
%% \end{figure}

%% \begin{figure}[hbt]
%%     \includegraphics[width=1.\textwidth]{Plot/closureTest8_WT_sgn_bin5.pdf}

%%     \caption{Closure test using KDE, for $q^2$ bin 5, for wrongly tagged events}
%%     \label{fig:clKDE5wt}
%% \end{figure}

%% \begin{figure}[hbt]
%%     \includegraphics[width=1.\textwidth]{Plot/closureTest8_WT_sgn_bin6.pdf}

%%     \caption{Closure test using KDE, for $q^2$ bin 6, for wrongly tagged events}
%%     \label{fig:clKDE6wt}
%% \end{figure}


%% \begin{figure}[hbt]
%%     \includegraphics[width=1.\textwidth]{Plot/closureTest8_WT_sgn_bin7.pdf}

%%     \caption{Closure test using KDE, for $q^2$ bin 7, for wrongly tagged events}
%%     \label{fig:clKDE7wt}
%% \end{figure}

%% \begin{figure}[hbt]
%%     \includegraphics[width=1.\textwidth]{Plot/closureTest8_WT_sgn_bin8.pdf}

%%     \caption{Closure test using KDE, for $q^2$ bin 8, for wrongly tagged events}
%%     \label{fig:clKDE8wt}
%% \end{figure}

%% \begin{figure}[hbt]
%%     \includegraphics[width=1.\textwidth]{Plot/closureTest8_WT_sgn_bin9.pdf}

%%     \caption{Closure test using KDE, for $q^2$ bin 9, for wrongly tagged events}
%%     \label{fig:clKDE9wt}
%% \end{figure}

%% Finally, the closure test for two control samples, \BKsJ and \BKsPsip are shown
%% in Fig.~\ref{fig:clKDEJpsi} and~\ref{fig:clKDEpsip} for the $q^2$ bins where
%% such control samples are present.

%% \begin{figure}[hbt]
%%     \includegraphics[width=1.\textwidth]{Plot/closureTest8_RT_jp_bin5.pdf}

%%     \includegraphics[width=1.\textwidth]{Plot/closureTest8_WT_jp_bin5.pdf}

%%     \caption{Closure test for efficiency for the reconstructed angles
%%         using the KDE approach, for $q^2$ for bin 5, correct tag (top) and wrong tag (bottom), for \BKsJ control sample.}
%%     \label{fig:clKDEJpsi}
%% \end{figure}

%% \begin{figure}[hbt]
%%     \includegraphics[width=1.\textwidth]{Plot/closureTest8_RT_psip_bin6.pdf}

%%     \includegraphics[width=1.\textwidth]{Plot/closureTest8_RT_psip_bin7.pdf}

%%     \includegraphics[width=1.\textwidth]{Plot/closureTest8_WT_psip_bin6.pdf}

%%     \includegraphics[width=1.\textwidth]{Plot/closureTest8_WT_psip_bin7.pdf}

%%     \caption{Closure test for efficiency for the reconstructed angles
%%         using the KDE approach, for correctly tagged events (top two rows), for
%%         $q^2$ bin 6 (first row) and 7 (second row), for \BKsPsip control
%%         sample.
%%         The two bottom row has the same plots for wrongly tagged events.
%%         The $\psi'(2s)$ invariant mass falls inside bin 7, but the tail of its
%%         radiative decay populate also the previous one, bin 6.}
%%     \label{fig:clKDEpsip}
%% \end{figure}

\clearpage

\chapter{Fit strategy} \label{sec:fit}

The angular parameters are extracted through an unbinned fit to four variables, the $\PKp\Pgpm\Pgmp\Pgmm$ invariant mass and the three angular variables, using an extended maximum likelihood estimator.

In the following sections I will give a detailed description of the probability density function used in the fit, and of the methods used to estimate its parameters.

\section{Probability density function}
\label{sec:TotalPDF}

The probability density function (\pdf) used in the fit has the following expression:
\begin{equation} \label{eq:angALL}
  \begin{split}
    \mathrm{pdf}(m,\TK,\TL,\PHI) & = Y^{C}_{S} \biggl[ S^{C}(m)  \, S^a(\TK,\TL,\PHI) \, \epsilon^{C}(\TK,\TL,\PHI) \biggr. \\
      & \biggl. + \frac{f^{M}}{1-f^{M}}~S^{M}(m) \, S^a(-\TK,-\TL,\PHI) \, \epsilon^{M}(\TK,\TL,\PHI) \biggr] \\
    & + Y_{B}\,B^m(m) \, B^{\TK}(\cTK) \, B^{\TL}(\cTL) \, B^{\PHI}(\PHI), \\
  \end{split}
\end{equation}
where its three terms correspond to the \pdfs for right-tagged signal, mis-tagged signal, and background events, respectively.

The parameters $Y^{C}_{S}$ and $Y_{B}$ are the yields of right-tagged signal events and background events, respectively, while the parameter $f^{M}$ is the fraction of signal events that are mis-tagged.

The functions $S^{C}(m)$ and $S^{M}(m)$ are the signal \PBz-mass \pdfs, for right-tagged and mis-tagged signal events, respectively.
Each of them is composed as the sum of two Gaussian functions, with a common mean for all four Gaussian functions.
%% In the fit, the mean, the four Gaussian function's width parameters, and the two fractions specifying the relative contribution of the two Gaussian functions in $S^{C}(m)$ and $S^{M}(m)$ are determined from simulation.
%% The function $S^a(\TK,\TL,\PHI)$ describes the signal in the three-dimensional (3D) space of the angular variables and corresponds to Eq.~(\ref{eq:PDF}).

The four \pdfs $B^m(m)$, $B^{\TK}(\cTK)$, $B^{\TL}(\cTL)$, $B^{\PHI}(\PHI)$ describe the background in the space of the \PBz candidate invariant mass and the angular variables.
The mass \pdf $B^m(m)$ is an exponential function, the angular \pdfs $B^{\TK}(\cTK)$ and $B^{\TL}(\cTL)$ are polynomials functions, ranging from second to fourth degree depending on the $q^2$ bin, and $B^{\PHI}(\PHI)$ is a first-order polynomial function.
The factorisation assumption of the background \pdf in Equation~(\ref{eq:angALL}) is discussed in Section~\ref{sec:fact}.

The three-dimensional functions $\epsilon^{C}(\TK,\TL,\PHI)$ and $\epsilon^{M}(\TK,\TL,\PHI)$ are the efficiencies for right-tagged and mis-tagged signal events, respectively.
The construction of these functions has been described in Section~\ref{sec:eff}.

\subsection{Fraction of mis-tagged signal events}
\label{sec:mistag}

%% The signal and control channels are self-tagging decays.
%% This means that in principle one can distinguish whether the mother particle is a $B^0$ or $\overline{B}^0$ by simply measuring the charges of the daughter hadrons, but this in turn requires the capability of disentangling kaons and pions.
%% Unfortunately the CMS experiment does not possess such a capability for charged particles above $\mathcal{O}(1~\GeV)$.
%% To overcome this deficit an algorithm, based on the analysis of the invariant mass of the two hadrons has been used.
%% Both the $\pi$ and $K$ mass hypothesis is considered for each of the two hadrons of the decay: the hypothesis with an invariant mass for the two-hadrons closer to the nominal $\PKst$ one is retained.

%% As described in Section~\ref{sec:selection}, the CP-state of the candidate \PBz is affigned on the base of 
The algorithm used to tag the CP-state of the \PBz candidate, described in Section~\ref{sec:selection}, has an intrinsic percentage of failure which is referred to as mistag fraction, $f^M$, defined as the ratio of mistagged signal events divided by the total number of signal events.
The mistag fraction is determined from simulated events, comparing the results of the tag method to the MC truth, and by counting the number of correctly and wrongly tagged events.
The resulting mistag fractions, for each $q^2$ bin, are shown in Table~\ref{tab:mistag}.
%%%%%%%%%%%%%%%%%%%%%%%%

\begin{table}[!htb]
  \begin{center}
    %% \begin{small}
      \caption{Mistag fraction as determined from simulated MC samples, for each $q^2$ bin.
        The values in $q^2$ bins 4 and 6 are evaluated on \BtoKstJpsimumu and \BtoKstpsipmumu MC samples, respectively. For the other $q^2$ bins the values are evaluated on signal MC sample.
        \label{tab:mistag}}
      \begin{tabular}{c|l|l}
        $q^2$ bin & Mistag fraction & Statistical \\
        index     & $f^M$           & uncertainty \\
        \hline
        $0$ & $0.124$  & $0.002$  \\
        $1$ & $0.129$  & $0.001$  \\
        $2$ & $0.134$  & $0.001$  \\
        $3$ & $0.132$  & $0.001$  \\
        $4$ & $0.1373$ & $0.0005$ \\
        $5$ & $0.132$  & $0.001$  \\
        $6$ & $0.140$  & $0.002$  \\
        $7$ & $0.132$  & $0.001$  \\
        $8$ & $0.137$  & $0.001$  \\
      \end{tabular}
    %% \end{small}
  \end{center}
\end{table}

The mistag fraction parameter is fixed in the fit to the values evaluated on MC simulation.
Any differences between simulated and real events could lead to a bias in the mistag fraction values used in the \pdf, that could propagate to a bias in the analysis results.
The contribution of this effect to the systematic uncertainty is discussed in Section~\ref{sec:sys-mistag}.

\subsection{Background parameterisation}
\label{sec:backg}

Several kind of background events can contaminate the dataset used for the fit.
In this section I will present the studies performed to describe and evaluate the many sources of this contamination.

The main contribution derives from the combinatorial background, i.e. events in which the four particles of the final state do not come from the same decay vertex.
Since the four-body invariant mass distribution of these events does not show any structure, their contribution can be evaluated from the mass sidebands, and extrapolated to the full mass range.

In addition, no correlation is expected between the four-body mass distribution and the angular variable distributions.
This allows to estimate the shapes of the angular distributions on the sidebands, and assume them valid also to describe the combinatorial background contamination in the \PBz mass region.

The \pdf used to model the angular shape of this background is the product of three uncorrelated polynomial functions, $B^{\TK}(\cTK) \, B^{\TL}(\cTL) \, B^{\PHI}(\PHI)$.
The orders of the polynomial functions are chosen individually for each $q^2$ bin, in order to have them successfully describe any shape feature, while keeping the degrees of freedom as low as possible, to have a converging fit even with the low sideband statistics.
The polynomial degrees that we use are listed in the Table~\ref{tab:back-degree}.

\begin{table}[!htb]
  \begin{center}
    %% \begin{small}
    \caption{Degree of the polynomial functions used to described the angular shape of the combinatorial background distributions, for each $q^2$ bin.
      \label{tab:back-degree}}
    \begin{tabular}{c|c|c|c}
      $q^2$ bin & $B^{\TK}$ & $B^{\TL}$ & $B^{\PHI}$ \\
      index & degree & degree & degree \\
      \hline
      0 & 3 & 2 & 1 \\
      1 & 4 & 2 & 1 \\
      2 & 4 & 3 & 1 \\
      3 & 2 & 4 & 1 \\
      5 & 4 & 2 & 1 \\
      7 & 2 & 3 & 1 \\
      8 & 2 & 2 & 1 \\
    \end{tabular}
    %% \end{small}
  \end{center}
\end{table}

An example of the sideband event distributions, for $q^2$ bin~0, are plotted in Figure~\ref{fig:bin0-bkg-k}-\ref{fig:bin0-bkg-phi}, together with the projections of the combinatorial background \pdf on the angular variables, after the fit to the sideband distributions described in Section~\ref{sec:fitseq}.

\begin{figure}[!hbt]
  \centering
  \includegraphics[width=0.7\textwidth]{Figures/background/bin0-k.pdf}
  \caption{The sideband event distribution and the projection of the background \pdf as a function of $\cTK$, for $q^2$ bin~0.}
  \label{fig:bin0-bkg-k}
\end{figure}

\begin{figure}[!hbt]
  \centering
  \includegraphics[width=0.7\textwidth]{Figures/background/bin0-l.pdf}
  \caption{The sideband event distribution and the projection of the background \pdf as a function of $\cTL$, for $q^2$ bin~0.}
  \label{fig:bin0-bkg-l}
\end{figure}

\begin{figure}[!hbt]
  \centering
  \includegraphics[width=0.7\textwidth]{Figures/background/bin0-phi.pdf}
  \caption{The sideband event distribution and the projection of the background \pdf as a function of $\PHI$, for $q^2$ bin~0.}
  \label{fig:bin0-bkg-phi}
\end{figure}

%%%%%%%%%%%%%%555
%% Using simulation, we search for possible backgrounds that might peak in the \PBz\ mass region.
%% The event selection is applied to inclusive MC samples of \PBz, \PBs, \PBp, and $\Lambda_{\rm b}$ decays to \cPJgy\ X and $\psi'$ X, where X denotes all possible SM particles required to complete the known exclusive decay channels, and with the \cPJgy\ and $\psi'$ decaying to \Pgmp \Pgmm.
%% No evidence for a peaking structure near the \PBz\ mass is found.
%% The distributions of the few events that satisfy the selection criteria are similar to the shape of the combinatorial background.
%% As an additional check, we generate events with $\PBs \to \cPKstz({\rm K}^+ \pi^-) \Pgmp \Pgmm$ decays, using the same branching fraction as for $\PBz \to \cPKstz({\rm K}^+ \pi^-)$ $\Pgmp \Pgmm$.
%% About 70 such events, integrated over $q^2$, are found to cluster near the \PBs\ mass.
%% This background is considered negligible since it should be rescaled by the ratio of branching fractions $\mathcal{B}(\PBs \to \cPJgy \cPKstz) / \mathcal{B}(\BtoKstJpsi) \approx 10^{-2}$~\cite{PDG}.
%% Possible backgrounds from events with hadrons misidentified as muons or with muons misidentified as hadrons, \eg, from random \PD\ mesons associated with random stable charged hadrons or from $\PBz \to \PD {\rm X}$ or $\PBz \to {\rm J}/\psi {\rm K}^*$ decays, are considered negligible because of the good muon identification capabilities of the CMS detector~\cite{Chatrchyan:2012xi}.
%% We also investigated possible background from events in which a $\PBp \to \PKp \Pgmp \Pgmm$ decay is combined with a random pion, and from events with a $\Lambda_b \to \Pp \PK \Pgmp \Pgmm$ decay in which the proton is assigned the pion mass.
%% Both these potential sources of background are found to be negligible.
%% The low-mass sideband might be affected by background events that have a different origin with respect to the combinatorial background that characterizes the signal region, \eg, partially reconstructed multibody \PB\ decays.
%% We address this possible background contamination in Section~\ref{sec:Systematics}
.
%%%%%%%%%%%%%%%%%%55

\subsection{Test of factorisable background hypothesis}
\label{sec:fact}

The result of a test on the correlation of the variable distributions for the combinatorial background events is shown in this section.

For each angular variable, the mass-sideband sample has been divided in two sub-samples, cutting in the middle of the variable range.
Then, the two sub-sample distributions as functions of the other variables are compared.
Furthermore, the angular variable distributions of the lower and higher sidebands are compared.

These comparisons are performed for each signal bin, for each control sample, and for two ``special bins'' that merge events in bin ranges [0-3] and [8-9], respectively.
As example, the distributions for the special $q^2$ bin [0-3] are showed in Figure~\ref{fig:side9}, where four plots are shown, one for each variable, and in each plot six distributions are compared, corresponding to the three pair of sub-samples obtained cutting on the not-plotted variables.
%% The colour code identifies the sub-sample:
%% \begin{center}
%% \begin{tabular}{c|c}
%% dim green & $\cos\theta_K$ lower half\\
%% grey & $\cos\theta_K$ higher half\\
%% black & $\cos\theta_L$ lower half\\
%% red & $\cos\theta_L$ higher half\\
%% bright green & $\phi$ lower half\\
%% blue & $\phi$ higher half\\
%% magenta & low mass sideband\\
%% light blue & high mass sideband\\
%% \end{tabular}
%% \end{center}

For some bins the low statistics don't allow to compare the distributions, but in the other bins and in the merged ones the good compatibility is probed.

%% \begin{figure}[!hbt]
%%   \centering
%%   \includegraphics[width=0.85\textwidth]{Figures/bkg_fact/sidebands0.pdf}
%%   \caption{Distributions of $\cos\theta_K$ (top left), $\cos\theta_L$ (top right), $\phi$ (bottom left), $m_{K\pi\mu\mu}$ (bottom right) for subsamples of the data sideband events, in the $q^2$ bin 0. The procedure to obtained these subsamples is described in App.~\ref{sec:app-bkg.fact}. }
%%   \label{fig:side0}
%% \end{figure}

%% \begin{figure}[!hbt]
%%   \centering
%%   \includegraphics[width=0.85\textwidth]{Figures/bkg_fact/sidebands1.pdf}
%%   \caption{Distributions of $\cos\theta_K$ (top left), $\cos\theta_L$ (top right), $\phi$ (bottom left), $m_{K\pi\mu\mu}$ (bottom right) for subsamples of the data sideband events, in the $q^2$ bin 1. The procedure to obtained these subsamples is described in App.~\ref{sec:app-bkg.fact}. }
%%   \label{fig:side1}
%% \end{figure}

%% \begin{figure}[!hbt]
%%   \centering
%%   \includegraphics[width=0.85\textwidth]{Figures/bkg_fact/sidebands2.pdf}
%%   \caption{Distributions of $\cos\theta_K$ (top left), $\cos\theta_L$ (top right), $\phi$ (bottom left), $m_{K\pi\mu\mu}$ (bottom right) for subsamples of the data sideband events, in the $q^2$ bin 2. The procedure to obtained these subsamples is described in App.~\ref{sec:app-bkg.fact}. }
%%   \label{fig:side2}
%% \end{figure}

%% \begin{figure}[!hbt]
%%   \centering
%%   \includegraphics[width=0.85\textwidth]{Figures/bkg_fact/sidebands3.pdf}
%%   \caption{Distributions of $\cos\theta_K$ (top left), $\cos\theta_L$ (top right), $\phi$ (bottom left), $m_{K\pi\mu\mu}$ (bottom right) for subsamples of the data sideband events, in the $q^2$ bin 3. The procedure to obtained these subsamples is described in App.~\ref{sec:app-bkg.fact}. }
%%   \label{fig:side3}
%% \end{figure}

%% \begin{figure}[!hbt]
%%   \centering
%%   \includegraphics[width=0.85\textwidth]{Figures/bkg_fact/sidebands4.pdf}
%%   \caption{Distributions of $\cos\theta_K$ (top left), $\cos\theta_L$ (top right), $\phi$ (bottom left), $m_{K\pi\mu\mu}$ (bottom right) for subsamples of the data sideband events, in the $J/\psi$ control region. The procedure to obtained these subsamples is described in App.~\ref{sec:app-bkg.fact}. }
%%   \label{fig:side4}
%% \end{figure}

%% \begin{figure}[!hbt]
%%   \centering
%%   \includegraphics[width=0.85\textwidth]{Figures/bkg_fact/sidebands5.pdf}
%%   \caption{Distributions of $\cos\theta_K$ (top left), $\cos\theta_L$ (top right), $\phi$ (bottom left), $m_{K\pi\mu\mu}$ (bottom right) for subsamples of the data sideband events, in the $q^2$ bin 5. The procedure to obtained these subsamples is described in App.~\ref{sec:app-bkg.fact}. }
%%   \label{fig:side5}
%% \end{figure}

%% \begin{figure}[!hbt]
%%   \centering
%%   \includegraphics[width=0.85\textwidth]{Figures/bkg_fact/sidebands6.pdf}
%%   \caption{Distributions of $\cos\theta_K$ (top left), $\cos\theta_L$ (top right), $\phi$ (bottom left), $m_{K\pi\mu\mu}$ (bottom right) for subsamples of the data sideband events, in the $\psi(2S)$ control region. The procedure to obtained these subsamples is described in App.~\ref{sec:app-bkg.fact}. }
%%   \label{fig:side6}
%% \end{figure}

%% \begin{figure}[!hbt]
%%   \centering
%%   \includegraphics[width=0.85\textwidth]{Figures/bkg_fact/sidebands7.pdf}
%%   \caption{Distributions of $\cos\theta_K$ (top left), $\cos\theta_L$ (top right), $\phi$ (bottom left), $m_{K\pi\mu\mu}$ (bottom right) for subsamples of the data sideband events, in the $q^2$ bin 7. The procedure to obtained these subsamples is described in App.~\ref{sec:app-bkg.fact}. }
%%   \label{fig:side7}
%% \end{figure}

%% \begin{figure}[!hbt]
%%   \centering
%%   \includegraphics[width=0.85\textwidth]{Figures/bkg_fact/sidebands8.pdf}
%%   \caption{Distributions of $\cos\theta_K$ (top left), $\cos\theta_L$ (top right), $\phi$ (bottom left), $m_{K\pi\mu\mu}$ (bottom right) for subsamples of the data sideband events, in the $q^2$ bin 8. The procedure to obtained these subsamples is described in App.~\ref{sec:app-bkg.fact}. }
%%   \label{fig:side8}
%% \end{figure}

\begin{figure}[!hbt]
  \centering
  \includegraphics[width=0.85\textwidth]{Figures/bkg_fact/sidebands9.pdf}
  \caption{Distributions of the sideband events as a function of $\cos\theta_K$ (top left), $\cos\theta_L$ (top right), $\phi$ (bottom left), $m_{K\pi\mu\mu}$ (bottom right), in the $q^2$ bin range [0,3]. As described in Section~\ref{sec:fact}, each distribution corresponds to a different sub-sample of the sideband events: $\cos\theta_K$ lower half (dim green) and upper half (grey), $\cos\theta_L$ lower half (black) and upper half (red), $\phi$ lower half (bright green) and upper half (blue), and low and high mass sidebands (magenta and light blue, respectively).}
  \label{fig:side9}
\end{figure}

%% \begin{figure}[!hbt]
%%   \centering
%%   \includegraphics[width=0.85\textwidth]{Figures/bkg_fact/sidebands10.pdf}
%%   \caption{Distributions of $\cos\theta_K$ (top left), $\cos\theta_L$ (top right), $\phi$ (bottom left), $m_{K\pi\mu\mu}$ (bottom right) for subsamples of the data sideband events, in the $q^2$ bin range [7,8]. The procedure to obtained these subsamples is described in App.~\ref{sec:app-bkg.fact}. }
%%   \label{fig:side10}
%% \end{figure}

\section{The fitting sequence, components and strategy}
\label{sec:fitseq}

Before applying the fit procedure on data, some parameters of the \pdf in Equation~\ref{eq:angALL} are estimated, for each $q^2$ bin, on the simulated signal MC sample and kept fixed for the full fitting process.
These parameters are the mistag fraction $f^{M}$, as described in Section~\ref{sec:mistag}, and the seven parameters of the signal mass component of the \pdf: the two widths and the relative abundance for the double Gaussian describing the right-tagged events, the same for the double Gaussian describing the mis-tagged events, and the common mean.
To take into account the effect of any difference between simulated and real events, a specific systematic uncertainty have been computed and will be described in Section~\ref{sec:syst}.
%% The PDF components whose parameters are determined with a MC simulaion are: $f^{M}$, $S^{C}(m)$, $S^{M}(m)$, $\epsilon^{C}(\TK,\theta_l,\phi)$ and $\epsilon^{M}(\TK,\theta_l,\phi)$, while all the remaining parameters are determined on data.

The angular component of the signal \pdf, as described by Equation~(\ref{eq:PDF-f2}), depends on six  parameters, $F_\mathrm{L}$, $F_\mathrm{S}$, $A_\mathrm{S}$, $P_1$, $P_5'$, and $A^5_\mathrm{S}$.
In order to facilitate the convergence of the fit process, and avoid problems related to the limited number of events and the presence of the physical boundary in the parameter phase space, the angular parameters that have already been measured by the previous CMS analysis on the same dataset, $F_\mathrm{L}$, $F_\mathrm{S}$, and $A_\mathrm{S}$, have been fixed to the results of that measurement.
To take into account the effect of fixing these parameters, a systematic uncertainty has been computed and will be described in Section~\ref{sec:syst}.

%% In order to avoid fit convergence problems due to the limited number of signal candidate events the angular parameters $F_\mathrm{L}$, $F_\mathrm{S}$, and $A_\mathrm{S}$ are fixed to previous CMS measurements performed on the same dataset with the same event selection criteria~\cite{CMS:2012}.
%% For each $q^2$ bin, the observables of interest are extracted from an unbinned extended maximum-likelihood fit to four variables: the $\PKp\Pgpm\Pgmp\Pgmm$ invariant mass $m$ and the three angular variables ${\TK}$, ${\theta_l}$, and $\phi$.

%%%%%%%%%%%%%%%%%%%%%%%55555

The fit is performed in two steps.
In the first one, the sidebands events are fitted, using only the background component of the \pdf, to obtain the parameters of the $B^m(m)$, $B^{\TK}(\TK)$, $B^{\theta_l}(\theta_l)$, and $B^{\phi}(\phi)$ distributions.
These parameters are then kept fixed in the second step of the fit.
To correctly propagate the uncertainties on the background parameters to the analysis results, a specific systematic uncertainty has been computed and will be described in Section~\ref{sec:syst}.

%% The sideband regions are $3\sigma_m < \abs{m-m_{\PBz}} < 5.5\sigma_m$, where $\sigma_m$ is the average mass resolution ($\approx$45\MeV), obtained from fitting the MC simulation signal to a sum of two Gaussians with a common mean.
In the second step, the full set of events is fitted using the whole \pdf.
The free parameters in this fit are the angular parameters $P_1$, $P_5'$, and $A^5_\mathrm{S}$, and the yields $Y^{C}_{S}$ and $Y_{B}$.
%% The expression describing the angular distribution of $\mathrm{B}^0\to{\mathrm{K}^*\mu\mu}$, Eq.~(\ref{eq:PDF-f2}) and also its more general form in Ref.~\cite{Descotes-Genon:2013vna}, can become negative for certain values of the angular parameters.
%% In particular the PDF in Eq.~(\ref{eq:PDF}) is only guaranteed to be non-negative for a particular subset of the parameter space $P_1$, $P_5'$, and $A^5_\mathrm{S}$, whose mathematical expression is non trivial.
This last fit cannot be run in a single step, since the presence of a physical boundary for the validity of the fitted parameters complicates the numerical maximisation process of the likelihood by \textsc{minuit}~\cite{Minuit}.
Especially in the $q^2$ bins in which the likelihood maximum is close to this boundary, the maximisation results tend to be unstable and strongly dependent on the values of the parameters at the begin of the fit.
A strategy was then developed to avoid the effect of the physical boundary: the bi-dimensional space $P_1$ -- $P_5'$ is discretised by building a $90\times90$ rectangular grid, and for each of its points the values of $P_1$, $P_5'$ are fixed in the fit, and the likelihood is maximised as a function of the nuisance parameters $Y^{C}_{S}$, $Y_{B}$, and $A^5_\mathrm{S}$.
Once the likelihood has been minimised for each point of the grid, it is fit with a bi-variate Normal distribution.
The position of the maximum of this function, limited to the physical region, corresponds to the best estimate of the angular parameters $P_1$, $P_5'$.
To avoid that any eventual non-Gaussian behaviour of the likelihood distribution in regions far from the maximum could introduce a bias in the results, the fit with the bi-variate Normal distribution is limited to the grid points ($P_1^i$, $P_5'^i$) for which is valid the following request: $$\log\mathcal{L}(P_1^i,P_5'^i) > \log\mathcal{L}(P_1^{\mathrm{max}},P_5'^{\mathrm{max}}) - 0.5$$ where ($P_1^{\mathrm{max}}$, $P_5'^{\mathrm{max}}$) is the grid point for which the likelihood is maximum.
In this way the grid points fitted are limited to a region around the maximum position.
The dependence of the fit result as a function of the region width, has been tested and found to be negligible.

%% The presence of such a physical region greatly complicates the numerical maximisation process of the likelihood by \textsc{minuit}~\cite{Minuit} and especially the error determination by \textsc{minos}~\cite{Minuit}, in particular next to the boundary between physical and unphysical regions.
%% Therefore the second fit step is performed by discretizing the bidimensional space $P_1$ -- $P_5'$, and by maximising the likelihood as a function of the nuisance parameters $Y^{C}_{S}$, $Y_{B}$, and $A^5_\mathrm{S}$ at fixed values of $P_1$, $P_5'$.
%% Finally the distribution of the likelihood values is fit with a bivariate Normal distribution whose position of the maximum inside the physical region corresponds to the best estimate of the angular parameters $P_1$, $P_5'$.

%% When we perform the final fit we need to make sure that we are discretizing the subset of the physical region $P_1$ -- $P_5'$ containing the absolute maximum of the likelihood.
%% To this extent we fit the data 200 times, each time with starting values of the parameters $P_1$ and $P_5'$ chosen randomly according to a uniform distribution defined over their physical region.

%%%%%%%%%%%%%%%%%%%%%%%55555


%% In this note we require that all the fitts have the status ``GOOD'', with which are labelled the plots.
%% The ``GOOD'' label depends upon two conditions: the convergence is verified, and the positive definiteness of the covariance matrix is confirmed.

%% The interference terms $A_\mathrm{S}$ and $A^5_\mathrm{S}$ must vanish if either of the two interfering components vanish.
%% From Ref.~\cite{Descotes-Genon:2013vna}, these constraints are implemented as $\abs{A_\mathrm{S}} < \sqrt{12 F_\mathrm{S}(1-F_\mathrm{S})F_\mathrm{L}}R$ and as $\abs{A^5_\mathrm{S}} < \sqrt{3 F_\mathrm{S} (1-F_\mathrm{S}) (1-F_\mathrm{L}) (1+P_1)}R$, where $R$ is a ratio related to the S-wave and P-wave line shapes, estimated to be 0.89 near the $\cPKstz$ mass.
%% The constraint on $A_\mathrm{S}$ is naturally satisfied since the measurement of the parameters $F_\mathrm{S}$, $F_\mathrm{L}$, and $A_\mathrm{S}$ are inherited from the previous CMS analysis~\cite{CMS:2012}.

%% To ensure correct coverage for the uncertainties of the angular parameters, the Feldman-Cousins method~\cite{FC} is used with nuisance parameters.
%% Two main sets of pseudo-experimental samples are generated to compute the coverage for the two angular observables $P_1$ and $P_5'$, respectively.
%% The first (second) set, used to compute the coverage for $P_1$ ($P_5'$), is generated by assigning values to the other parameters as obtained by profiling the likelihood on data at fixed $P_1$ ($P_5'$) values.
%% When fitting the pseudo-experimental samples the same fit procedure as in data is applied (more details can be found in Sec.~\ref{sec:statUncert}).

%% The fit formalism and results are validated through fits to pseudo-experimental samples, MC simulation samples, and control channels.

\clearpage

\chapter{Validation of the fit algorithm}
\label{sec:validation}

Since the whole fit procedure is very complex, many validation checks are performed to verify the robustness of the final result.

In this section I will present the studies performed on the fit procedure, both on simulated MC events and on data control channels.
Since all the signal MC samples contain only events with resonant P-wave \PKpi state, it will be implicit in this section that the angular decay rate used when fitting MC events does not have the S-wave and interference terms, while the full decay rate is used when fitting on data control channels.

\section{Generator level fit to signal MC events}
\label{sec:fitval-gen}

The first validation step, presenting the minimum level of complexity, is performed by fitting the generator-level angular distributions of the signal MC sample, using the pure decay rate as \pdf: $S^a(\TK,\TL,\PHI)$.

The good status of the fit is considered a successful probe of the the minimisation process and of the correct description of the generated angular distribution through the decay rate.
The angular folding operations are applied to distributions and \pdfs, but this has no effect on the result, since the terms of the \pdf are either odd or even with respect to them.

Since the ``true'' values of the angular parameters used to simulate the events of the MC samples are not defined, the results of this fit will be used as a reference value for any comparison of the following results.
This fit is clean from any effect due to finite experimental resolution, so I expect that the fit results do not have any bias.

An example of the generator-level angular distributions and the projections of the fitted decay rate, for $q^2$ bin~3, are shown in Fig.~\ref{fig:gen-bin3}.

%% \begin{figure}[!hbt]
%%   \centering
%%   \includegraphics[width=0.9\textwidth]{Figures/GENFit/bin0.pdf}
%%   \caption{Fit results of the $q^2$ bin No.0 on the GEN
%%     sample. The plots show the projections of the fit results on
%%     three different angular variables: $cos\theta_l$, $cos\theta_k$
%%     and $\PHI$. The curve is the fit and the points with error
%%     bars are the GEN  sample. } 
%%   \label{fig:gen-bin0}
%% \end{figure}

%% \begin{figure}[!hbt]
%%   \centering
%%   \includegraphics[width=0.9\textwidth]{Figures/GENFit/bin1.pdf}
%%   \caption{Fit results of the $q^2$ bin No.1 on the GEN
%%     sample. The plots show the projections of the fit results on
%%     three different angular variables: $cos\theta_l$, $cos\theta_k$
%%     and $\PHI$. The curve is the fit and the points with error
%%     bars are the GEN  sample. } 
%%   \label{fig:gen-bin1}
%% \end{figure}

%% \begin{figure}[!hbt]
%%   \centering
%%   \includegraphics[width=0.9\textwidth]{Figures/GENFit/bin2.pdf}
%%   \caption{Fit results of the $q^2$ bin No.2 on the GEN
%%     sample. The plots show the projections of the fit results on
%%     three different angular variables: $cos\theta_l$, $cos\theta_k$
%%     and $\PHI$. The curve is the fit and the points with error
%%     bars are the GEN  sample. } 
%%   \label{fig:gen-bin2}
%% \end{figure}

\begin{figure}[!hbt]
  \centering
  \includegraphics[width=0.9\textwidth]{Figures/GENFit/bin3.pdf}
  \caption{Generator-level variable distributions of the MC sample before the GEN-filter, together with the projections of the fitted \pdf, as described in Section~\ref{sec:fitval-gen}, for $q^2$ bin~3.}
  \label{fig:gen-bin3}
\end{figure}

%% \begin{figure}[!hbt]
%%   \centering
%%   \includegraphics[width=0.9\textwidth]{Figures/GENFit/bin5.pdf}
%%   \caption{Fit results of the $q^2$ bin No.5 on the GEN
%%     sample. The plots show the projections of the fit results on
%%     three different angular variables: $cos\theta_l$, $cos\theta_k$
%%     and $\PHI$. The curve is the fit and the points with error
%%     bars are the GEN  sample. } 
%%   \label{fig:gen-bin5}
%% \end{figure}

%% \begin{figure}[!hbt]
%%   \centering
%%   \includegraphics[width=0.9\textwidth]{Figures/GENFit/bin7.pdf}
%%   \caption{Fit results of the $q^2$ bin No.7 on the GEN
%%     sample. The plots show the projections of the fit results on
%%     three different angular variables: $cos\theta_l$, $cos\theta_k$
%%     and $\PHI$. The curve is the fit and the points with error
%%     bars are the GEN  sample. } 
%%   \label{fig:gen-bin7}
%% \end{figure}


%% \begin{figure}[!hbt]
%%   \centering
%%   \includegraphics[width=0.9\textwidth]{Figures/GENFit/bin8.pdf}
%%   \caption{Fit results of the $q^2$ bin No.8 on the GEN
%%     sample. The plots show the projections of the fit results on
%%     three different angular variables: $cos\theta_l$, $cos\theta_k$
%%     and $\PHI$. The curve is the fit and the points with error
%%     bars are the GEN  sample. } 
%%   \label{fig:gen-bin8}
%% \end{figure}



\section{Reconstruction-level fit to signal MC samples}
\label{sec:fitval-reco}

The second validation step is performed by fitting the reconstruction-level angular distributions of the signal MC sample, after applying the criteria of candidate selection and CP-state tag described in Section~\ref{sec:selection}.
This fit is performed using only the angular terms of the signal components in the \pdf described in Equation~\ref{eq:angALL}.

The simulated sample contains both right-tagged and mis-tagged events, which can be distinguished by using the MC truth information.
As a first step, these two categories are fitted individually, using for each of them the corresponding angular component of the \pdf, which is the decay rate times the efficiency function.
The results of the fit show that the inclusion of the efficiency function in the fit procedure is correctly implemented.

An example of the reconstruction-level angular distributions of the signal MC sample and the projections of the angular component of the \pdf, for $q^2$ bin~3, is shown in Figure~\ref{fig:rtag-bin3} for right-tagged events and in Figure~\ref{fig:wtag-bin3} for mis-tagged events.

%% \begin{figure}[!hbt]
%%   \centering
%%   \includegraphics[width=0.9\textwidth]{Figures/RECOFit/correct/AngleS_Canv0.pdf}
%%   \caption{The fitting results of the $q^2$ bin No.0 on the correctly
%%     tagged RECO  sample. The plots show the projections of the fitting results on
%%     three different angular variables: $cos\theta_l$, $cos\theta_k$
%%     and $\PHI$. The curve is the fitting and the points with error
%%     bars are from RECO sample. } 
%%   \label{fig:rtag-bin0}
%% \end{figure}


%% \begin{figure}[!hbt]
%%   \centering
%%   \includegraphics[width=0.9\textwidth]{Figures/RECOFit/correct/AngleS_Canv1.pdf}
%%   \caption{The fitting results of the $q^2$ bin No.1 on the correctly
%%     tagged RECO  sample. The plots show the projections of the fitting results on
%%     three different angular variables: $cos\theta_l$, $cos\theta_k$
%%     and $\PHI$. The curve is the fitting and the points with error
%%     bars are from RECO sample. } 
%%   \label{fig:rtag-bin1}
%% \end{figure}

%% \begin{figure}[!hbt]
%%   \centering
%%   \includegraphics[width=0.9\textwidth]{Figures/RECOFit/correct/AngleS_Canv2.pdf}
%%   \caption{The fitting results of the $q^2$ bin No.2 on the correctly
%%     tagged RECO  sample. The plots show the projections of the fitting results on
%%     three different angular variables: $cos\theta_l$, $cos\theta_k$
%%     and $\PHI$. The curve is the fitting and the points with error
%%     bars are from RECO sample. } 
%%   \label{fig:rtag-bin2}
%% \end{figure}

\begin{figure}[!hbt]
  \centering
  \includegraphics[width=0.9\textwidth]{Figures/RECOFit/correct/AngleS_Canv3.pdf}
  \caption{Angular variable distributions of the right-tagged MC events after the selection criteria application, together with the projections of the fitted \pdf, as described in Section~\ref{sec:fitval-reco}, for $q^2$ bin~3.}
  \label{fig:rtag-bin3}
\end{figure}

%% \begin{figure}[!hbt]
%%   \centering
%%   \includegraphics[width=0.9\textwidth]{Figures/RECOFit/correct/AngleS_Canv5.pdf}
%%   \caption{The fitting results of the $q^2$ bin No.5 on the correctly
%%     tagged RECO  sample. The plots show the projections of the fitting results on
%%     three different angular variables: $cos\theta_l$, $cos\theta_k$
%%     and $\PHI$. The curve is the fitting and the points with error
%%     bars are from RECO sample. } 
%%   \label{fig:rtag-bin5}
%% \end{figure}

%% \begin{figure}[!hbt]
%%   \centering
%%   \includegraphics[width=0.9\textwidth]{Figures/RECOFit/correct/AngleS_Canv7.pdf}
%%   \caption{The fitting results of the $q^2$ bin No.7 on the correctly
%%     tagged RECO  sample. The plots show the projections of the fitting results on
%%     three different angular variables: $cos\theta_l$, $cos\theta_k$
%%     and $\PHI$. The curve is the fitting and the points with error
%%     bars are from RECO sample. } 
%%   \label{fig:rtag-bin7}
%% \end{figure}

%% \begin{figure}[!hbt]
%%   \centering
%%   \includegraphics[width=0.9\textwidth]{Figures/RECOFit/correct/AngleS_Canv8.pdf}
%%   \caption{The fitting results of the $q^2$ bin No.8 on the correctly
%%     tagged RECO  sample. The plots show the projections of the fitting results on
%%     three different angular variables: $cos\theta_l$, $cos\theta_k$
%%     and $\PHI$. The curve is the fitting and the points with error
%%     bars are from RECO sample. } 
%%   \label{fig:rtag-bin8}
%% \end{figure}

%% \begin{figure}[!hbt]
%%   \centering
%%   \includegraphics[width=0.9\textwidth]{Figures/RECOFit/wrong/SignalRECO_Canv0.pdf}
%%   \caption{The fitting results of the $q^2$ bin No.0 on the wrongly
%%     tagged RECO sample. The plots show the projections of the fitting
%%     results on three different angular variables: $cos\theta_l$,
%%     $cos\theta_k$ and $\PHI$. The curve is the fitting and the points
%%     with error bars are from RECO sample. }
%%   \label{fig:wtag-bin0}
%% \end{figure}


%% \begin{figure}[!hbt]
%%   \centering
%%   \includegraphics[width=0.9\textwidth]{Figures/RECOFit/wrong/SignalRECO_Canv1.pdf}
%%   \caption{The fitting results of the $q^2$ bin No.1 on the wrongly
%%     tagged RECO  sample. The plots show the projections of the fitting results on
%%     three different angular variables: $cos\theta_l$, $cos\theta_k$
%%     and $\PHI$. The curve is the fitting and the points with error
%%     bars are from RECO sample. } 
%%   \label{fig:wtag-bin1}
%% \end{figure}

%% \begin{figure}[!hbt]
%%   \centering
%%   \includegraphics[width=0.9\textwidth]{Figures/RECOFit/wrong/SignalRECO_Canv2.pdf}
%%   \caption{The fitting results of the $q^2$ bin No.2 on the wrongly
%%     tagged RECO  sample. The plots show the projections of the fitting results on
%%     three different angular variables: $cos\theta_l$, $cos\theta_k$
%%     and $\PHI$. The curve is the fitting and the points with error
%%     bars are from RECO sample. } 
%%   \label{fig:wtag-bin2}
%% \end{figure}

\begin{figure}[!hbt]
  \centering
  \includegraphics[width=0.9\textwidth]{Figures/RECOFit/wrong/SignalRECO_Canv3.pdf}
  \caption{Angular variable distributions of the mis-tagged MC events after the selection criteria application, together with the projections of the fitted \pdf, as described in Section~\ref{sec:fitval-reco}, for $q^2$ bin~3.}
  \label{fig:wtag-bin3}
\end{figure}

%% \begin{figure}[!hbt]
%%   \centering
%%   \includegraphics[width=0.9\textwidth]{Figures/RECOFit/wrong/SignalRECO_Canv5.pdf}
%%   \caption{The fitting results of the $q^2$ bin No.5 on the wrongly
%%     tagged RECO  sample. The plots show the projections of the fitting results on
%%     three different angular variables: $cos\theta_l$, $cos\theta_k$
%%     and $\PHI$. The curve is the fitting and the points with error
%%     bars are from RECO sample. } 
%%   \label{fig:wtag-bin5}
%% \end{figure}

%% \begin{figure}[!hbt]
%%   \centering
%%   \includegraphics[width=0.9\textwidth]{Figures/RECOFit/wrong/SignalRECO_Canv7.pdf}
%%   \caption{The fitting results of the $q^2$ bin No.7 on the wrongly
%%     tagged RECO  sample. The plots show the projections of the fitting results on
%%     three different angular variables: $cos\theta_l$, $cos\theta_k$
%%     and $\PHI$. The curve is the fitting and the points with error
%%     bars are from RECO sample. } 
%%   \label{fig:wtag-bin7}
%% \end{figure}

%% \begin{figure}[!hbt]
%%   \centering
%%   \includegraphics[width=0.9\textwidth]{Figures/RECOFit/wrong/SignalRECO_Canv8.pdf}
%%   \caption{The fitting results of the $q^2$ bin No.8 on the wrongly
%%     tagged RECO  sample. The plots show the projections of the fitting results on
%%     three different angular variables: $cos\theta_l$, $cos\theta_k$
%%     and $\PHI$. The curve is the fitting and the points with error
%%     bars are from RECO sample. } 
%%   \label{fig:wtag-bin8}
%% \end{figure}

As a second step, the full MC sample, containing both right-tagged and mis-tagged events, is used in the fit, and both the signal terms of the \pdf are used.
The mistag fraction parameter in the \pdf is fixed to the values reported in Section~\ref{sec:mistag}.
Since the mistag fraction values are computed on the same MC sample used for this fit, there is a statistical correlation between it and the fitted distributions; anyway its effects on the fit results are negligible because of the extremely small statistical uncertainty on this parameter.

The fit is still limited to the three angular variables, because all the parameters of the mass shapes and the relative abundances are fixed in the fit and there is no information to extract from the mass distributions.

An example of the reconstruction-level angular distributions of the signal MC sample and the projections of the angular component of the \pdf, for both right-tagged and mis-tagged events in $q^2$ bin~3, is shown in Figure~\ref{fig:fullreco-bin3}.

%% \begin{figure}[!hbt]
%%   \centering
%%   \includegraphics[width=0.9\textwidth]{Figures/RECOFit/full-reco/SignalRECO_Canv0.pdf}
%%   \caption{The fitting results of the $q^2$ bin No.0 on the full RECO
%%     sample. The plots show the projections of the fitting results on
%%     three different angular variables: $cos\theta_l$, $cos\theta_k$
%%     and $\PHI$. The curve is the fitting and the points with error
%%     bars are from RECO sample. }
%%   \label{fig:fullreco-bin0}
%% \end{figure}


%% \begin{figure}[!hbt]
%%   \centering
%%   \includegraphics[width=0.9\textwidth]{Figures/RECOFit/full-reco/SignalRECO_Canv1.pdf}
%%   \caption{The fitting results of the $q^2$ bin No.1 on the full RECO
%%     sample. The plots show the projections of the fitting results on
%%     three different angular variables: $cos\theta_l$, $cos\theta_k$
%%     and $\PHI$. The curve is the fitting and the points with error
%%     bars are from RECO sample. }
%%   \label{fig:fullreco-bin1}
%% \end{figure}

%% \begin{figure}[!hbt]
%%   \centering
%%   \includegraphics[width=0.9\textwidth]{Figures/RECOFit/full-reco/SignalRECO_Canv2.pdf}
%%   \caption{The fitting results of the $q^2$ bin No.2 on the full RECO
%%     sample. The plots show the projections of the fitting results on
%%     three different angular variables: $cos\theta_l$, $cos\theta_k$
%%     and $\PHI$. The curve is the fitting and the points with error
%%     bars are from RECO sample. }
%%   \label{fig:fullreco-bin2}
%% \end{figure}



\begin{figure}[!hbt]
  \centering
  \includegraphics[width=0.9\textwidth]{Figures/RECOFit/full-reco/SignalRECO_Canv3.pdf}
  \caption{Angular variable distributions of all the MC events after the selection criteria application, together with the projections of the fitted \pdf, as described in Section~\ref{sec:fitval-reco}, for $q^2$ bin~3.}
  \label{fig:fullreco-bin3}
\end{figure}



%% \begin{figure}[!hbt]
%%   \centering
%%   \includegraphics[width=0.9\textwidth]{Figures/RECOFit/full-reco/SignalRECO_Canv5.pdf}
%%   \caption{The fitting results of the $q^2$ bin No.5 on the full RECO
%%     sample. The plots show the projections of the fitting results on
%%     three different angular variables: $cos\theta_l$, $cos\theta_k$
%%     and $\PHI$. The curve is the fitting and the points with error
%%     bars are from RECO sample. }
%%   \label{fig:fullreco-bin5}
%% \end{figure}

%% \begin{figure}[!hbt]
%%   \centering
%%   \includegraphics[width=0.9\textwidth]{Figures/RECOFit/full-reco/SignalRECO_Canv7.pdf}
%%   \caption{The fitting results of the $q^2$ bin No.7 on the full RECO
%%     sample. The plots show the projections of the fitting results on
%%     three different angular variables: $cos\theta_l$, $cos\theta_k$
%%     and $\PHI$. The curve is the fitting and the points with error
%%     bars are from RECO sample. }
%%   \label{fig:fullreco-bin7}
%% \end{figure}


%% \begin{figure}[!hbt]
%%   \centering
%%   \includegraphics[width=0.9\textwidth]{Figures/RECOFit/full-reco/SignalRECO_Canv8.pdf}
%%   \caption{The fitting results of the $q^2$ bin No.8 on the full RECO
%%     sample. The plots show the projections of the fitting results on
%%     three different angular variables: $cos\theta_l$, $cos\theta_k$
%%     and $\PHI$. The curve is the fitting and the points with error
%%     bars are from RECO sample. }
%%   \label{fig:fullreco-bin8}
%% \end{figure}

The results of the fits performed in this section are compared with the results of the generator-level fits.

The results of right-tagged event fit and of the generator-level fit, are shown in Figure~\ref{fig:correct-closure-fl} for the $F_L$ parameter, in Figure~\ref{fig:correct-closure-p5p} for the $P_5'$ parameter, and in Figure~\ref{fig:correct-closure-p1} for the $P_1$ parameter.
The results of mis-tagged event fit and of the generator-level fit, are shown in Figure~\ref{fig:wrong-closure-fl} for the $F_L$ parameter, in Figure~\ref{fig:wrong-closure-p5p} for the $P_5'$ parameter, and in Figure~\ref{fig:wrong-closure-p1} for the $P_1$ parameter.
The results of right-tagged event fit and of the generator-level fit, are shown in Figure~\ref{fig:fullreco-closure-fl} for the $F_L$ parameter, in Figure~\ref{fig:fullreco-closure-p5p} for the $P_5'$ parameter, and in Figure~\ref{fig:fullreco-closure-p1} for the $P_1$ parameter.

\begin{figure}[!hbt]
  \centering
  \includegraphics[width=0.7\textwidth]{Figures/RECOFit/correct/Fl.pdf}
  \caption{Results for the $F_L$ parameter from the reconstruction-level fit to right-tagged event distributions (black) and from the generator-level fit (red), for each $q^2$ bin.
    The vertical shaded regions correspond to the $q^2$ bins dedicated to the $J/\psi$ and $\psi'$ control channels.}
  \label{fig:correct-closure-fl}
\end{figure}


\begin{figure}[!hbt]
  \centering
  \includegraphics[width=0.7\textwidth]{Figures/RECOFit/correct/P5p.pdf}
  \caption{Results for the $P_5'$ parameter from the reconstruction-level fit to right-tagged event distributions (black) and from the generator-level fit (red), for each $q^2$ bin.
    The vertical shaded regions correspond to the $q^2$ bins dedicated to the $J/\psi$ and $\psi'$ control channels.}
  \label{fig:correct-closure-p5p}
\end{figure}

\begin{figure}[!hbt]
  \centering
  \includegraphics[width=0.7\textwidth]{Figures/RECOFit/correct/P1.pdf}
  \caption{Results for the $P_1$ parameter from the reconstruction-level fit to right-tagged event distributions (black) and from the generator-level fit (red), for each $q^2$ bin.
    The vertical shaded regions correspond to the $q^2$ bins dedicated to the $J/\psi$ and $\psi'$ control channels.}
  \label{fig:correct-closure-p1}
\end{figure}

\begin{figure}[!hbt]
  \centering
  \includegraphics[width=0.7\textwidth]{Figures/RECOFit/wrong/Fl.pdf}
  \caption{Results for the $F_L$ parameter from the reconstruction-level fit to mis-tagged event distributions (black) and from the generator-level fit (red), for each $q^2$ bin.
    The vertical shaded regions correspond to the $q^2$ bins dedicated to the $J/\psi$ and $\psi'$ control channels.}
  \label{fig:wrong-closure-fl}
\end{figure}


\begin{figure}[!hbt]
  \centering
  \includegraphics[width=0.7\textwidth]{Figures/RECOFit/wrong/P5p.pdf}
  \caption{Results for the $P_5'$ parameter from the reconstruction-level fit to mis-tagged event distributions (black) and from the generator-level fit (red), for each $q^2$ bin.
    The vertical shaded regions correspond to the $q^2$ bins dedicated to the $J/\psi$ and $\psi'$ control channels.}
  \label{fig:wrong-closure-p5p}
\end{figure}

\begin{figure}[!hbt]
  \centering
  \includegraphics[width=0.7\textwidth]{Figures/RECOFit/wrong/P1.pdf}
  \caption{Results for the $P_1$ parameter from the reconstruction-level fit to mis-tagged event distributions (black) and from the generator-level fit (red), for each $q^2$ bin.
    The vertical shaded regions correspond to the $q^2$ bins dedicated to the $J/\psi$ and $\psi'$ control channels.}
  \label{fig:wrong-closure-p1}
\end{figure}

\begin{figure}[!hbt]
  \centering
  \includegraphics[width=0.7\textwidth]{Figures/RECOFit/full-reco/Fl.pdf}
  \caption{Results for the $F_L$ parameter from the reconstruction-level fit (black) and from the generator-level fit (red), for each $q^2$ bin.
    The vertical shaded regions correspond to the $q^2$ bins dedicated to the $J/\psi$ and $\psi'$ control channels.}
  \label{fig:fullreco-closure-fl}
\end{figure}


\begin{figure}[!hbt]
  \centering
  \includegraphics[width=0.7\textwidth]{Figures/RECOFit/full-reco/P5p.pdf}
  \caption{Results for the $P_5'$ parameter from the reconstruction-level fit (black) and from the generator-level fit (red), for each $q^2$ bin.
    The vertical shaded regions correspond to the $q^2$ bins dedicated to the $J/\psi$ and $\psi'$ control channels.}
  \label{fig:fullreco-closure-p5p}
\end{figure}


\begin{figure}[!hbt]
  \centering
  \includegraphics[width=0.7\textwidth]{Figures/RECOFit/full-reco/P1.pdf}
  \caption{Results for the $P_1$ parameter from the reconstruction-level fit (black) and from the generator-level fit (red), for each $q^2$ bin.
    The vertical shaded regions correspond to the $q^2$ bins dedicated to the $J/\psi$ and $\psi'$ control channels.}
  \label{fig:fullreco-closure-p1}
\end{figure}


%% \subsection{Validation of independent MC samples}
%% \label{sec:fitval-half}

%% Since the efficiency derivation procedures use the same full
%% statistics of RECO level signal simulation sample as we fit
%% described in \ref{sec:fitval-reco}, we need to check whether the
%% efficiency works well with an independent sample. In order to check
%% this, we re-derive the efficiency with half of the RECO sample and
%% perform the fitting on the other half of the sample to check the
%% results.

%% In this section, we show the fitting results thus obtained for the
%% RECO sample and the closure test for the correctly tagged
%% events. Basically these results agree well with those reported in
%% Sections \ref{sec:fitval-reco-full} and \ref{sec:fitval-closure-rtag}
%% respectively. These agreements indicate that there is no bias caused
%% by the possible correlation in derivation of efficiency.

%% \subsubsection{Fitting the independents MC samples}
%% \label{sec:fitval-fitres-half}

%% The fitting results of correctly tagged events from half the RECO
%% samples for all $q^2$ bins are shown in Fig.~\ref{fig:halfrtag-bin0}
%% to Fig.~\ref{fig:halfrtag-bin8}.

%% \begin{figure}[!hbt]
%%   \centering
%%   \includegraphics[width=0.9\textwidth]{Figures/HalfFit/SignalRECO_Canv0.pdf}
%%   \caption{The fitting results of the $q^2$ bin No.0 on the half of the correctly
%%     tagged RECO  sample. The plots show the projections of the fitting results on
%%     three different angular variables: $cos\theta_l$, $cos\theta_k$
%%     and $\PHI$. The curve is the fitting and the points with error
%%     bars are from RECO sample. }
%%   \label{fig:halfrtag-bin0}
%% \end{figure}


%% \begin{figure}[!hbt]
%%   \centering
%%   \includegraphics[width=0.9\textwidth]{Figures/HalfFit/SignalRECO_Canv1.pdf}
%%   \caption{The fitting results of the $q^2$ bin No.1 on the half of the correctly
%%     tagged RECO  sample. The plots show the projections of the fitting results on
%%     three different angular variables: $cos\theta_l$, $cos\theta_k$
%%     and $\PHI$. The curve is the fitting and the points with error
%%     bars are from RECO sample. }
%%   \label{fig:halfrtag-bin1}
%% \end{figure}

%% \begin{figure}[!hbt]
%%   \centering
%%   \includegraphics[width=0.9\textwidth]{Figures/HalfFit/SignalRECO_Canv2.pdf}
%%   \caption{The fitting results of the $q^2$ bin No.2 on the half of the correctly
%%     tagged RECO  sample. The plots show the projections of the fitting results on
%%     three different angular variables: $cos\theta_l$, $cos\theta_k$
%%     and $\PHI$. The curve is the fitting and the points with error
%%     bars are from RECO sample. }
%%   \label{fig:halfrtag-bin2}
%% \end{figure}

%% \begin{figure}[!hbt]
%%   \centering
%%   \includegraphics[width=0.9\textwidth]{Figures/HalfFit/SignalRECO_Canv3.pdf}
%%   \caption{The fitting results of the $q^2$ bin No.3 on the half of the correctly
%%     tagged RECO  sample. The plots show the projections of the fitting results on
%%     three different angular variables: $cos\theta_l$, $cos\theta_k$
%%     and $\PHI$. The curve is the fitting and the points with error
%%     bars are from RECO sample. }
%%   \label{fig:halfrtag-bin3}
%% \end{figure}

%% \begin{figure}[!hbt]
%%   \centering
%%   \includegraphics[width=0.9\textwidth]{Figures/HalfFit/SignalRECO_Canv5.pdf}
%%   \caption{The fitting results of the $q^2$ bin No.5 on the half of the correctly
%%     tagged RECO  sample. The plots show the projections of the fitting results on
%%     three different angular variables: $cos\theta_l$, $cos\theta_k$
%%     and $\PHI$. The curve is the fitting and the points with error
%%     bars are from RECO sample. }
%%   \label{fig:halfrtag-bin5}
%% \end{figure}

%% \begin{figure}[!hbt]
%%   \centering
%%   \includegraphics[width=0.9\textwidth]{Figures/HalfFit/SignalRECO_Canv7.pdf}
%%   \caption{The fitting results of the $q^2$ bin No.7 on the half of the correctly
%%     tagged RECO  sample. The plots show the projections of the fitting results on
%%     three different angular variables: $cos\theta_l$, $cos\theta_k$
%%     and $\PHI$. The curve is the fitting and the points with error
%%     bars are from RECO sample. }
%%   \label{fig:halfrtag-bin7}
%% \end{figure}

%% \begin{figure}[!hbt]
%%   \centering
%%   \includegraphics[width=0.9\textwidth]{Figures/HalfFit/SignalRECO_Canv8.pdf}
%%   \caption{The fitting results of the $q^2$ bin No.8 on the half of the correctly
%%     tagged RECO  sample. The plots show the projections of the fitting results on
%%     three different angular variables: $cos\theta_l$, $cos\theta_k$
%%     and $\PHI$. The curve is the fitting and the points with error
%%     bars are from RECO sample. }
%%   \label{fig:halfrtag-bin8}
%% \end{figure}

%% 

%% \subsubsection{Closure test with the correctly tagged events}
%% \label{sec:fitval-closure-half}


%% The closure test results of the correctly tagged events with
%% independent RECO sample for all $q^2$ bins are shown in
%% Fig.~\ref{fig:hc-closure-fl} to
%% Fig.~\ref{fig:hc-closure-p1}.

%% \begin{figure}[!hbt]
%%   \centering
%%   \includegraphics[width=0.7\textwidth]{Figures/HalfFit/Fl.pdf}
%%   \caption{Closure test of the fitting results of $F_L$ for each $q^2$
%%     bins from half the correctly tagged RECO sample with GEN sample.
%%     The red points are from the GEN fitting and the black points are from
%%     the RECO fitting. The vertical shaded regions correspond to the $J/\psi$ and $\psi'$ resonances. }
%%   \label{fig:hc-closure-fl}
%% \end{figure}


%% \begin{figure}[!hbt]
%%   \centering
%%   \includegraphics[width=0.7\textwidth]{Figures/HalfFit/P5p.pdf}
%%   \caption{Closure test of the fitting results of $P_5'$ for each
%%     $q^2$ bins from half the correctly tagged RECO sample with GEN
%%     sample.  The red points are from the GEN fitting and the black points
%%     are from the RECO fitting. The vertical shaded regions correspond to the $J/\psi$ and $\psi'$ resonances. }
%%   \label{fig:hc-closure-p5p}
%% \end{figure}

%% \begin{figure}[!hbt]
%%   \centering
%%   \includegraphics[width=0.7\textwidth]{Figures/HalfFit/P1.pdf}
%%   \caption{Closure test of the fitting results of $P_1$ for each $q^2$
%%     bins from half the correctly tagged RECO sample with GEN sample.
%%     The red points are from the GEN fitting and the black points are from
%%     the RECO fitting. The vertical shaded regions correspond to the $J/\psi$ and $\psi'$ resonances. }
%%   \label{fig:hc-closure-p1}
%% \end{figure}

\section{Reconstructed level fit to low statistics simulated samples}
\label{sec:datalike-MC}

The fit algorithm is also validated with simulated samples having the same statistics of the real data sample, and containing both the signal and the background components.
The goal is to verify whether the analysis is able to measure the interesting observables, in conditions as close as possible to the real data sample.

The data events are obtained by dividing the MC samples in sub-samples with a number of events exactly equal to the signal yield, as obtained from fitting on data the \PBz mass distribution.
The number of sub-samples that we can produced is limited by the statistics available in the MC sample, and for simplicity has been rounded down to 200 sub-samples, for each $q^2$ bin.

The background distributions are generated with pseudo-experiments, using the \pdf described in Section~\ref{sec:backg}, and parameter values measured with data sidebands.
The number of events generated for each sub-sample are equal to the background yield, as obtained from fitting on data the \PBz mass distribution.
To match the signal MC sub-samples, also for background a total of 200 sets of events have been generated, for each $q^2$ bin.

These samples have been used to validate the fitting procedure, first using only the signal, then merging each signal sub-sample with a background one.
I will refer to a merged sub-sample as ``cocktail'' MC sample.

\subsection{Data-like samples of signal MC events}
\label{sec:Cocktail-MC-pure}

As a first step, the angular distribution of the signal sub-samples is fitted, using only the signal components of the \pdf.
As for the fit to the full MC sample, including the mass in the fitted distributions would not change the result.

Each of the 200 MC sub-samples are fitted.
An example of the distributions of the resulting parameters, for $q^2$ bin~3, is shown in Figure~\ref{fig:closure-signal-cocktail-bin3}.

\begin{figure}[!hbt]
  \centering
  \includegraphics[width=1.0\textwidth]{Figures/cocktail-Fit/bin3-value-s.pdf}
  \caption{Distributions of the results of the fits to the 200 signal MC sub-samples, for $q^2$ bin~3.}
  \label{fig:closure-signal-cocktail-bin3}
\end{figure}

The distributions of these results are compared with the results of the fit to the full MC sample.
Any significant difference can be an hint of biases on the results introduced by the fit procedure.

The mean value of the result distributions are shown in Figure~\ref{fig:sub-samp-FL}, in Figure~\ref{fig:sub-samp-P1}, and in Figure~\ref{fig:sub-samp-P5p}, for the $F_L$, $P_1$, and $P_5'$ parameters, respectively.
Only the results from converging fits are included in this distributions, so in general their number is lower than 200.
The error bars assigned to the mean values are the standard deviations of the result distributions, divided by the square root of the number of results in them. 


\begin{figure}[!hbt]
  \centering
  \includegraphics[width=0.7\textwidth]{Figures/cocktail-Fit/com-reco-Fl.pdf}
  \caption{Average values of the $F_L$ result distribution from the fit to 200 signal MC sub-samples (blue), together with the $F_L$ results of the fit to the full MC sample (red). The error bars associated to the sub-sample fit results represent the statistical uncertainty associated to the arithmetic average of the results, as descibed in Section~\ref{sec:Cocktail-MC-pure}.}
  \label{fig:sub-samp-FL}
\end{figure}


\begin{figure}[!hbt]
  \centering
  \includegraphics[width=0.7\textwidth]{Figures/cocktail-Fit/com-reco-P1.pdf}
  \caption{Average values of the $P_1$ result distribution from the fit to 200 signal MC sub-samples (blue), together with the $P_1$ results of the fit to the full MC sample (red). The error bars associated to the sub-sample fit results represent the statistical uncertainty associated to the arithmetic average of the results, as descibed in Section~\ref{sec:Cocktail-MC-pure}.}
  \label{fig:sub-samp-P1}
\end{figure}

\begin{figure}[!hbt]
  \centering
  \includegraphics[width=0.7\textwidth]{Figures/cocktail-Fit/com-reco-P5p.pdf}
  \caption{Average values of the $P_5'$ result distribution from the fit to 200 signal MC sub-samples (blue), together with the $P_5'$ results of the fit to the full MC sample (red). The error bars associated to the sub-sample fit results represent the statistical uncertainty associated to the arithmetic average of the results, as descibed in Section~\ref{sec:Cocktail-MC-pure}.}
  \label{fig:sub-samp-P5p}
\end{figure}



%% Pull distributions from fitting on the data-like simulation samples
%% are reported in App.\ref{sec:cock-signal-pull} for each $q^2$ bin.
%% Ideally the pull distributions should follow
%% a N(0,1) normal distribution if the fitting procedures are unbiased
%% and evaluating the error correctly.

%% Since we use the average values of the fitting as the true values in
%% the construction of the pull, the mean of the pull distributions are
%% not interesting here and the focus is the width of the pull.  The unit
%% width of these distributions indicate the fitting is good and the
%% error estimation from the fitting is reasonable.


\subsection{Data-like ``cocktail'' MC samples}
\label{sec:Cocktail-MC-full}

As a second step, the mass and angular distributions of the 200 ``cocktail'' MC samples are fitted using the full \pdf function.
This fit is very similar to the one performed on data, even if there is no S-wave component in the signal events.
For this reason, this validation check is one of the most important steps of the procedure.
As for the signal MC sub-samples, also here the mean and the standard deviation, divided by the square root of fits, are used to represent the ``cocktail'' fit results.

The comparison between the average results of the 200 ``cocktails'' fit and the fit results of Reco MC, is shown in Figure ~\ref{fig:closure-full-cocktail-p1} and Figure~\ref{fig:closure-full-cocktail-P5'}, for the $P_1$ and $P_5'$  parameters, respectively.

%% In some bins the measurements are biased, which is caused by fitting programs and procedures.
%% We account the discrepancies between average values and the fitting result of Reco-MC as a systematic uncertainties called fitting bias.
%% The discrepancies of every bin are shoen in the Figure ~\ref{fig:closure-full-cocktail-p1} and Figure ~\ref{fig:closure-full-cocktail-P5'}.
%% The detail of fitting bias are in Section~\ref{sec:fitbias-syst}.

\begin{figure}[!hbt]
  \centering
  \includegraphics[width=0.7\textwidth]{Figures/cocktail-Fit/avg-full-P1.pdf}
  \caption{Average values of the $P_1$ result distribution from the fit to 200 ``cocktail'' samples (blue), together with the $P_1$ results of the fit to the full MC sample (red). The error bars associated to the cocktail-MC fit results represent the statistical uncertainty associated to the arithmetic average of the results, as descibed in Section~\ref{sec:Cocktail-MC-full}.}
  \label{fig:closure-full-cocktail-p1}
\end{figure}

\begin{figure}[!hbt]
  \centering
  \includegraphics[width=0.7\textwidth]{Figures/cocktail-Fit/avg-full-P5.pdf}
  \caption{Average values of the $P_5'$ result distribution from the fit to 200 ``cocktail'' samples (blue), together with the $P_5'$ results of the fit to the full MC sample (red).The error bars associated to the cocktail-MC fit results represent the statistical uncertainty associated to the arithmetic average of the results, as descibed in Section~\ref{sec:Cocktail-MC-full}.}
  \label{fig:closure-full-cocktail-P5'}
\end{figure}



%% \subsubsection{Comparison of data-like statistics fit with and without background}
%% \label{sec:Cocktail-MC-com}

%% Then we compared the average fitting results with full simulation described in
%% Section~\ref{sec:Cocktail-MC-full}. The comparison are in the Figure~\ref{fig:comparison-cocktail-P1}
%% and Figure~\ref{fig:comparison-cocktail-P5}.
%% When fitting with the background, we fixed the parameter $F_L$ to have a better convergence
%% as we will do for data fit (details in Section~\ref{sec:formula}), so we can
%% compare the value of $P_1$ and $P_5'$.

%% \begin{figure}[!hbt]
%%   \centering
%%   \includegraphics[width=0.7\textwidth]{Figures/cocktail-Fit/com-toy-P1.pdf}
%%   \caption{Comparison between the average results of 200 full cocktail
%%           samples and 200 pure signal cocktail samples of $P1$ for each $q^2$
%%     bins. The red points are average results of full cocktail MC and the blue are
%%      pure signal cocktail MC.}
%%   \label{fig:comparison-cocktail-P1}
%% \end{figure}


%% \begin{figure}[!hbt]
%%   \centering
%%   \includegraphics[width=0.7\textwidth]{Figures/cocktail-Fit/com-toy-P5p.pdf}
%%   \caption{Comparison between the average results of 200 full cocktail
%%           samples and 200 pure signal cocktail samples of $P_5'$ for each $q^2$
%%     bins. The red points are average results of full cocktail MC and the blue are
%%      pure signal cocktail MC.}
%%   \label{fig:comparison-cocktail-P5}
%% \end{figure}

%% \subsection{Error comparison between the blinded data and data-like simulation}
%% \label{sec:error-comparison}

%% This section describes the check on the statistical errors we perform
%% on data in a blinded way. We have fit the data in the signal region,
%% but "blinding" the central values for $P_1$ and $P_5'$. That means we
%% check the size of the statistical errors from the fitting only,
%% without looking at the
%% central values.  The errors are from the preliminary
%% fitting results obtained with "HESSE".  If using "MINOS" and more
%% detailed scanning of the initial values, probably there will be some
%% improvements, but we can only fully tune those options after the green
%% light for unblinding.

%% We compare the errors with what are expected from data-like statistics
%% full simulations which are described in
%% Section~\ref{sec:Cocktail-MC-full}, to verify the error estimation in
%% our analysis. The Figure~\ref{fig:error-com-0} to
%% Figure~\ref{fig:error-com-8} are the comparison of all $q^2$ bins'
%% results. In the figures, the red lines are the errors from the data,
%% and the histograms are the distributions from the simulation. The
%% first subplots are for $P_1$, the second ones are for $P_5'$, the
%% third are for the fraction of signal.


%% \begin{figure}[!hbt]
%%   \centering
%%   \includegraphics[width=1.0\textwidth]{Figures/errorbar/error-0.pdf}
%%   \caption{Validation between errors of 200 full cocktail samples and
%%     data channel for $q^2$ bin0. The red line is the error of data
%%     channel, the histogram is the distribution of simulations. And the
%%     first is $P_1$, the second is $P_5'$, the third is fraction of
%%     signal.}
%%   \label{fig:error-com-0}
%% \end{figure}

%% \begin{figure}[!hbt]
%%   \centering
%%   \includegraphics[width=1.0\textwidth]{Figures/errorbar/error-1.pdf}
%%   \caption{Validation between errors of 200 full cocktail samples and
%%     data channel for $q^2$ bin1. The red line is the error of data
%%     channel, the histogram is the distribution of simulations. And the
%%     first is $P_1$, the second is $P_5'$, the third is fraction of
%%     signal.}
%%   \label{fig:error-com-1}
%% \end{figure}

%% \begin{figure}[!hbt]
%%   \centering
%%   \includegraphics[width=1.0\textwidth]{Figures/errorbar/error-2.pdf}
%%   \caption{Validation between errors of 200 full cocktail samples and
%%     data channel for $q^2$ bin2. The red line is the error of data
%%     channel, the histogram is the distribution of simulations. And the
%%     first is $P_1$, the second is $P_5'$, the third is fraction of
%%     signal.}
%%   \label{fig:error-com-2}
%% \end{figure}

%% \begin{figure}[!hbt]
%%   \centering
%%   \includegraphics[width=1.0\textwidth]{Figures/errorbar/error-3.pdf}
%%   \caption{Validation between errors of 200 full cocktail samples and
%%     data channel for $q^2$ bin3. The red line is the error of data
%%     channel, the histogram is the distribution of simulations. And the
%%     first is $P_1$, the second is $P_5'$, the third is fraction of
%%     signal.}
%%   \label{fig:error-com-3}
%% \end{figure}

%% \begin{figure}[!hbt]
%%   \centering
%%   \includegraphics[width=1.0\textwidth]{Figures/errorbar/error-5.pdf}
%%   \caption{Validation between errors of 200 full cocktail samples and
%%     data channel for $q^2$ bin5. The red line is the error of data
%%     channel, the histogram is the distribution of simulations. And the
%%     first is $P_1$, the second is $P_5'$, the third is fraction of
%%     signal.}
%%   \label{fig:error-com-5}
%% \end{figure}

%% \begin{figure}[!hbt]
%%   \centering
%%   \includegraphics[width=1.0\textwidth]{Figures/errorbar/error-7.pdf}
%%   \caption{Validation between errors of 200 full cocktail samples and
%%     data channel for $q^2$ bin7. The red line is the error of data
%%     channel, the histogram is the distribution of simulations. And the
%%     first is $P_1$, the second is $P_5'$, the third is fraction of
%%     signal.}
%%   \label{fig:error-com-7}
%% \end{figure}

%% \begin{figure}[!hbt]
%%   \centering
%%   \includegraphics[width=1.0\textwidth]{Figures/errorbar/error-8.pdf}
%%   \caption{Validation between errors of 200 full cocktail samples and
%%     data channel for $q^2$ bin8. The red line is the error of data
%%     channel, the histogram is the distribution of simulations. And the
%%     first is $P_1$, the second is $P_5'$, the third is fraction of
%%     signal.}
%%   \label{fig:error-com-8}
%% \end{figure}

\section{Validation with data control channels}
\label{sec:controlchannel}

The analysis technique is validated with the data by means of the control channels.
In this way, the S-wave component of the PDF is tested, and we have a check of the efficiency behaviour on real data.

\subsection{Sideband fit in control regions}
\label{sec:bkgforcc}

The background shape is determined as for the other bins, by using the data sidebands as a function of the angular observables.
Each of the three angular observables is fit with a polynomial with different degrees, specified in Table~\ref{tab:psi-bkg}.
The background distributions and \pdf projections are plotted in Figure~\ref{fig:back-l-bin4} to Figure~\ref{fig:back-phi-bin4}, for the \BtoKstJpsi control channel, and in Figure~\ref{fig:back-l-bin6} to Figure~\ref{fig:back-phi-bin6}, for the \BtoKstpsip control channel.

\begin{table*}[!htb]
  \begin {center}
    %% \begin{small}
      \caption{Degrees of the polynomial functions used for control channel \pdfs.
        \label{tab:psi-bkg}}
      \begin{tabular}{c|c|c|c}
        $q^2$ bin & $B^{\cos\theta_\mathrm{K}}$ & $B^{\cos\theta_l}$ & $B^{\PHI}$ \\
        index & degree & degree & degree \\
        \hline
        \BtoKstJpsi & 4 & 4 & 5 \\
        \BtoKstpsip & 4 & 3 & 5 \\
      \end{tabular}
    %% \end{small}
  \end{center}
\end{table*}

\begin{figure}[!hbt]
  \centering
  \includegraphics[width=0.7\textwidth]{Figures/J/bkg-l.pdf}
  \caption{Distribution of the \cTL variable in mass sidebands of the \BtoKstJpsi control channel and the background \pdf projection.}
  \label{fig:back-l-bin4}
\end{figure}

\begin{figure}[!hbt]
  \centering
  \includegraphics[width=0.7\textwidth]{Figures/J/bkg-k.pdf}
  \caption{Distribution of the \cTK variable in mass sidebands of the \BtoKstJpsi control channel and the background \pdf projection.}
  \label{fig:back-k-bin4}
\end{figure}

\begin{figure}[!hbt]
  \centering
  \includegraphics[width=0.7\textwidth]{Figures/J/bkg-phi.pdf}
  \caption{Distribution of the \PHI variable in mass sidebands of the \BtoKstJpsi control channel and the background \pdf projection.}
  \label{fig:back-phi-bin4}
\end{figure}

\begin{figure}[!hbt]
  \centering
  \includegraphics[width=0.7\textwidth]{Figures/Psi/bkg-l.pdf}
  \caption{Distribution of the \cTL variable in mass sidebands of the \BtoKstpsip control channel and the background \pdf projection.}
  \label{fig:back-l-bin6}
\end{figure}

\begin{figure}[!hbt]
  \centering
  \includegraphics[width=0.7\textwidth]{Figures/Psi/bkg-k.pdf}
  \caption{Distribution of the \cTK variable in mass sidebands of the \BtoKstpsip control channel and the background \pdf projection.}
  \label{fig:back-k-bin6}
\end{figure}

\begin{figure}[!hbt]
  \centering
  \includegraphics[width=0.7\textwidth]{Figures/Psi/bkg-phi.pdf}
  \caption{Distribution of the \PHI variable in mass sidebands of the \BtoKstpsip control channel and the background \pdf projection.}
  \label{fig:back-phi-bin6}
\end{figure}


\subsection{Control channel fit}
\label{sec:fitcc}

The two control channels are fitted, with  $F_s$ and $A_s$ fixed, as described in Section~\ref{sec:fitseq}, while $F_L$ is kept floating.
The projection plots of the fit results are shown Figure~\ref{fig:result-bin4} and Figure~\ref{fig:result-bin6}, for \BtoKstJpsi and \BtoKstpsip channels respectively.

\begin{figure}[!hbt]
  \centering
  \includegraphics[width=1.0\textwidth]{Figures/J/TotalPDF-J.pdf}
  \caption{The fitting results of the control channel \cPJgy on data.
    The plots show the projections of the fitting results on three different angular variables: \PBz mass, \cTL, \cTK and \PHI.}
  \label{fig:result-bin4}
\end{figure}

\begin{figure}[!hbt]
  \centering
  \includegraphics[width=1.0\textwidth]{Figures/Psi/TotalPDF-Psi.pdf}
  \caption{The fitting results of the control channel $\Psi'$ on data.
    The plots show the projections of the fitting results on three different angular variables: \PBz mass, \cTL, \cTK and \PHI.}
  \label{fig:result-bin6}
\end{figure}

The results of the measurements for the control channels \BtoKstJpsi and \BtoKstpsip are summarised in Table~\ref{tab:res-control-channel}.

\begin{table*}[!htb]
  \begin {center}
    \begin{small}
      \caption{Results of the fit to the data control channels, as described in Section~\ref{sec:fitcc}. The reported uncertainty is fully statistical.
        \label{tab:res-control-channel}}
      \begin{tabular}{l|c|c|c|c}
        control channel & $F_L$ & $P_1$ & $P_5'$ & $A_s^5$ \\
        \hline
        \BtoKstJpsi & $0.537 \pm 0.002$ & $-0.081 \pm 0.011$ & $-0.024 \pm 0.007$ & $-0.002 \pm 0.002$ \\
        \BtoKstpsip & $0.538 \pm 0.008$ & $-0.031 \pm 0.001$ & $-0.039 \pm 0.001$ & $0.005 \pm 0.001$  \\
      \end{tabular}
    \end{small}
  \end{center}
\end{table*}


For both the control channels, the $F_L$ parameter has been measured in the previous CMS analysis and also by other experiments.
The results from this work, from the previous CMS analysis and from other experiments are in good agreement, as shown in Table~\ref{tab:com.control channel}.

\begin{table*}[!htb]
  \begin {center}
    \begin{small}
      \caption{Measurements from CMS (both in this and in the previous analysis), PDG, and BaBar\cite{BaBar2} of $F_L$ in the control channels.
        The first uncertainty is statistical and the second is systematic.
        \label{tab:com.control channel}}
      \begin{tabular}{l|c|c|c|c|c|c|c}
        control channel & \multicolumn{3}{|c|}{$B^0 \rightarrow K^{*0}(K^+\pi^-) J/\psi(\mu^+ \mu^-)$} & \multicolumn{3}{|c|}{$B^0 \rightarrow K^{*0}(K^+\pi^-) \psi'(\mu^+ \mu^-)$}\\
        \hline
        Experiment  & $F_L$  &  Err(stat) & Err(syst) & $F_L$  &  Err(stat) & Err(syst)\\
        \hline
        This work & $0.537$ & $\pm0.002$ &  $-$  & $0.538$ & $\pm0.008$ &  $-$\\
        \hline
        CMS   &  $0.537$ & $\pm0.002$ &  $-$    &  $0.538$ & $\pm0.008$ &  $-$  \\
        \hline
        PDG   &  $0.571$ & $\pm0.007$ & $-$      &  $0.463$ & $^{+0.028}_{-0.040} $ & $-$ \\
        \hline
        BaBar &  $0.556$ & $\pm0.009$ & $\pm0.010 $   &  $0.48$ & $\pm0.005$ & $\pm0.002 $\\
      \end{tabular}
    \end{small}
  \end{center}
\end{table*}

As further test, the fit on the two control channels has been repeated with the $F_L$ parameter fixed, coherently with the procedure used on data and described in Section~\ref{sec:fitseq}.
The results of the two kind of fits, with $F_L$ either free to float or fixed, are compared in Table~\ref{tab:flfixed-control-channel}.

\begin{table*}[!htb]
  \begin {center}
    \begin{small}
      \caption{Results of the fit to the data control channels, as described in Section~\ref{sec:fitcc}. The reported uncertainty is fully statistical.
        \label{tab:flfixed-control-channel}}
      \begin{tabular}{l|l|c|c|c}
        control channel & $F_L$ state & $P_1$ & $P_5'$ & $A_s^5$ \\
        \hline
        \BtoKstJpsi & floating & $-0.081 \pm 0.011$ & $-0.024 \pm 0.007$ & $-0.002 \pm 0.002$ \\
                    & fixed    & $-0.082 \pm 0.003$ & $-0.024 \pm 0.002$ & $-0.001 \pm 0.002$ \\
        \hline
        \BtoKstpsip & floating & $-0.031 \pm 0.001$ & $-0.039 \pm 0.001$ & $0.005 \pm 0.001$  \\
                    & fixed    & $-0.033 \pm 0.025$ & $-0.040 \pm 0.031$ & $0.005 \pm 0.011$  \\
      \end{tabular}
    \end{small}
  \end{center}
\end{table*}

The difference between the fit results of the $P_1$ and $P_5'$ parameters are reported in the Table~\ref{tab:difference Fl free or fix}.
Since these differences are very small compared to the statistical errors of the results, this test shows that the bias in the results introduced by the choice of fixing some parameters in the \pdf is negligible in the final result.

\begin{table*}[!htb]
  \begin {center}
    \begin{small}
      \caption{Difference between the $P_1$ and $P_5'$ results obtained from a fit sequence with the $F_L$ parameter fixed or free to float.
        \label{tab:difference Fl free or fix}}
      \begin{tabular}{l|c|c}
        control channel & $P1$ & $P_5'$ \\
        \hline
        \BtoKstJpsi & $0.001$ & $<0.001$ \\
        \BtoKstpsip & $0.002$ & $0.001$  \\
      \end{tabular}
    \end{small}
  \end{center}
\end{table*}

\clearpage

\chapter{Systematic uncertainties}\label{sec:syst}

In this section the systematic uncertainties considered for this analysis are discussed.
Some of the systematics are handled in similar methods as those discussed in the previous CMS analysis~\cite{Khachatryan:2015isa,AN-14-129}.

The sources of systematic uncertainties investigated are:
\begin{description}
\item[limited amount of simulated events:] the propagation of the statistical uncertainty of the MC sample used to compute the efficiency;
  %% the efficiency is computed using a finite set of simulated events; the size of this sample affects the accuracy of the determination of the efficiency;

\item[simulation mismodelling:] the effect of eventual mis-modelling in the simulated angular shape;
  %% the results from the fitting on the generated pure signal events are used to estimate simulation mis-modelling;

\item[efficiency shape:] the effect of mis-modelling in the efficiency functions, computed using the control channels;

\item[fitting bias:] the possible biases from the fitting procedures, evaluated on data-like ``cocktail'' MC samples;

\item[wrong CP assignment:] the effect of wrong CP assignment on fit results;

\item[background distributions:] the effects of the background \pdf uncertainties, due to finite sideband statistics, on the fit results;

\item[uncertainty from fixed \pdf parameters:] the propagation of the uncertainties on the angular parameters $F_L$, $A_S$, $F_S$;

\item[angular resolution:] the effect of the finite reconstruction resolution on the fit results;
  
\item[feed-through background:] the effect of the contamination with $B^0 \to {\rm J}\psi K^{*0}$ and $B^0 \to \psi' K^{*0}$ feed-through events, in $q^2$ bins just below and above the resonance regions;

\item[bivariate gauss fit range:] the dependency of the results on the range of the bivariate fit to the likelihood distribution in the $P_1$, $P'_5$ plane, when estimating the best-fit value.

\end{description}

In the following sections, the systematic uncertainties sources are discussed and estimated.

\subsection{Limited amount of simulated events}
\label{sec:sys-lim.MCstat}

The use of kernel density estimator to determine the numerator and denominator of the efficiency is based on a sample of simulated events, and a systematic uncertainty is expected from the limited size of the sample used.
The unbinned approach prevents the use of a simple binomial error estimation to be propagated to the parameterisation itself.
%% Instead, two different method have been used to estimate this uncertainty.

%% In the first one, the original sample is subdivided in four subsamples, and
%% each one is used independently to estimate the efficiency using the KDE method,
%% obtaining four efficiencies, which are used to perform the \pdf fit. The spread
%% of the fitting parameters resulting from these four fittings, divided by $\sqrt{4}$,
%% are used as systematic uncertainty related to the limited size of simulated sample.

The method used to evaluate this systematic contribution makes use of a set of 101 efficiency function based on pseudo-experiments.
For each efficiency, we generate pseudo-experiment datasets for numerator and denominator terms of the efficiency, with the same number of events as the original samples, based on the \pdf returned by the KDE description of numerator and denominator, respectively.
These additional datasets are then used to compute efficiency using the same KDE approach used in the original sample.
Finally, 101 fits are performed using each of the new efficiency functions.
The spread of the fit result distributions obtained by these fits is used as systematic uncertainty.

This method is first tested on a cocktail-MC sample, as the ones used for the validation of the fit described in Section~\ref{sec:Cocktail-MC-pure}.
The results of the fits with the toy-efficiencies are compared to the result of the fit with the original efficiency, computed both on the same cocktail-MC sample and on the full MC sample.
This comparison validates the procedure of generating toy-efficiencies, showing that the results obtained with them are compatible with the result obtained with the original efficiency.
Then, the method is applied to the data, after the unbinding of the signal region .%% , and these results are used to compute the systematic uncertainty.

An example of the fit result distributions, for $q^2$ bin~3, is shown in Figure~\ref{fig:BKG-bin3-toy}, for the fits on a cocktail-MC sample, and in Figure~\ref{fig:BKG-bin3}, for the fits on the data sample.

\begin{figure}[!hbt]
  \centering
  \includegraphics[width=0.9\textwidth]{Figures/MCstat/bin3_toy.pdf}
  \caption{Results of the fits on a cocktail-MC sample with toy-efficiency functions, for $q^2$ bin~3.
    On the left, the distributions of the $F_L$ (top), $P_1$ (middle), and $P_5'$ (bottom) parameter results are compared with the results from the fits with the original efficiency: on the same cocktail-MC sample (green line) and on the full MC sample (red line).
    On the right: the distributions of the {\tt HESSE} errors on the $F_L$(top), $P_1$ (middle), and $P_5'$ (bottom) parameters are shown.}
  \label{fig:BKG-bin3-toy}
\end{figure}

\begin{figure}[!hbt]
  \centering
  \includegraphics[width=0.9\textwidth]{Figures/MCstat/bin3.pdf}
  \caption{Results of the fits on the data sample with toy-efficiency functions, for $q^2$ bin~3.
    On the left, the distributions of the $F_L$ (top), $P_1$ (middle), and $P_5'$ (bottom) parameter results are shown.
    On the right: the distributions of the {\tt HESSE} errors on the $F_L$(top), $P_1$ (middle), and $P_5'$ (bottom) parameters are shown.}
  %% \caption{The average results with different toy-efficiency functions using one of the cocktail sample, in $q^2$ bin~3.
  %%   Top left: $F_L$ value, top right: $F_L$ fit error, middle right: $P_1$ value, middle right: $P_1$ fit error, bottom right: $P_5'$ value, middle right: $P_5'$ fit error.
  %%   The red lines indicate the fitting results on the full RECO MC.
  %%   The green lines indicate the fit results on the used toy sample using the original efficiency.}
  \label{fig:BKG-bin3}
\end{figure}

The spread of the fit results are summarised in Table~\ref{tab:lim.MCstat.toy}, for the fit on the cocktail-MC sample, and in Table~\ref{tab:lim.MCstat} for the fit to the data sample.
The latter are used as systematic uncertainty.

\begin{table*}[!htb]
  \begin {center}
    \begin{small}
      \caption{Spread values of the toy-efficiency result distributions, computed fitting a data-like statistics cocktail-MC sample.
        \label{tab:lim.MCstat.toy}}
      \begin{tabular}{l|c|c|c|c}
        $q^2$ bin index  & $F_L$ & $P_1$ & $P_5'$ \\
        \hline
        $ 0 $ & $\pm0.0157$ & $\pm0.0720$ & $\pm0.0859$\\
        $ 1 $ & $\pm0.0110$ & $\pm0.1208$ & $\pm0.1070$\\
        $ 2 $ & $\pm0.0140$ & $\pm0.0726$ & $\pm0.0627$\\
        $ 3 $ & $\pm0.0047$ & $\pm0.0319$ & $\pm0.0264$\\
        $ 5 $ & $\pm0.0051$ & $\pm0.0170$ & $\pm0.0099$\\
        $ 7 $ & $\pm0.0043$ & $\pm0.0440$ & $\pm0.0384$\\
        $ 8 $ & $\pm0.0070$ & $\pm0.0790$ & $\pm0.0622$\\
      \end{tabular}
    \end{small}
  \end{center}
\end{table*}

\begin{table*}[!htb]
  \begin {center}
    \begin{small}
      \caption{Spread values of the toy-efficiency result distributions, computed fitting real data.
        \label{tab:lim.MCstat}}
      \begin{tabular}{l|c|c|c|c}
        $q^2$ bin index  & $P_1$ & $P_5'$ \\
        \hline
        $ 0 $ & $\pm0.050$ & $\pm0.046$\\
        $ 1 $ & $\pm0.062$ & $\pm0.066$\\
        $ 2 $ & $\pm0.057$ & $\pm0.031$\\
        $ 3 $ & $\pm0.036$ & $\pm0.032$\\
        $ 5 $ & $\pm0.068$ & $\pm0.049$\\
        $ 7 $ & $\pm0.073$ & $\pm0.112$\\
        $ 8 $ & $\pm0.029$ & $\pm0.036$\\
      \end{tabular}
    \end{small}
  \end{center}
\end{table*}

% \subsection{Kernel width}

% The kernel estimator was used with a constant width kernel width, chosen as a
% compromise to optimize the modelling of the efficiency shape without including
% spurious effect. The systematic associated with this choice is estimated by
% using, in the \pdf fit, a wider and a narrower kernel width, and comparing the results.


\subsection{Simulation mismodelling}
\label{sec:sys-mismodel}

The effects of the simulation mismodelling is measured through the capability of the analysis to retrieve the interesting observables in extremely favourable conditions, using pure-signal simulation with high statistics.

The fit results on generator-level MC sample are compared with the one on reconstruction-level MC sample, as described in Section~\ref{sec:fitval-reco}.
The discrepancies between them are considered as symmetric systematic uncertainties.

This systematic uncertainty also evaluates the impact of the non perfect symmetry of the efficiency with respect to the angular folding applied.
The folding procedure cancels some of the angular parameters only if applied at \pdf level.
The actual fit is performed at reconstruction level, taking into account the efficiency as a function of the angular variables.
The efficiency does not have the same exact symmetries as the \pdf, and this might cause the cancellation to be incomplete.
By comparing the results of the fit at generator level, when the cancellation is exact, with that at reconstruction level, when it is not, we evaluate the uncertainty related to the non-perfect cancellation.

The results are summarised for each $q^2$ bin in Table~\ref{tab:mis.modelling}.

\begin{table*}[!htb]
  \begin {center}
    \begin{small}
      \caption{Systematic uncertainties: simulation mismodelling.
        \label{tab:mis.modelling}}
      \begin{tabular}{l|c|c|c|c}
        $q^2$ bin index  & $F_L$ & $P_1$ & $P_5'$ \\
        \hline
        $ 0 $ & $\pm0.011$ & $\pm0.005$ & $\pm0.023$\\
        $ 1 $ & $\pm0.001$ & $\pm0.005$ & $\pm0.013$\\
        $ 2 $ & $\pm0.009$ & $\pm0.001$ & $\pm0.015$\\
        $ 3 $ & $\pm0.012$ & $\pm0.006$ & $\pm0.012$\\
        $ 5 $ & $\pm0.006$ & $\pm0.001$ & $\pm0.021$\\
        $ 7 $ & $\pm0.008$ & $\pm0.033$ & $\pm0.010$\\
        $ 8 $ & $\pm0.004$ & $\pm0.006$ & $\pm0.014$\\
      \end{tabular}
    \end{small}
  \end{center}
\end{table*}

\subsection{Efficiency shape}
\label{sec:effshape-syst}

The main validation of the correctness of the efficiency is performed by comparing the efficiency-corrected results for the control channels with the corresponding world-average values.
The efficiency as a function of the angular variables is checked by comparing the $F_\mathrm{L}$ measurements from the \BtoKstJpsi sample, composed of 165\,000 signal events.
The value of $F_\mathrm{L}$ obtained in this analysis is $0.537 \pm 0.002\stat$, compared with the world-average value of $0.571 \pm 0.007\,\text{(stat+syst)}$, indicating a discrepancy of $0.034$, which is used in the other $q^2$ bins and propagated to the $P_1$ and $P_5'$ parameters.
In each $q^2$ bin, a total of 200 values for the $F_L$ parameter is randomly generated from a Gaussian distribution, with mean the value of $F_L$ used in the fit to the data and with width the discrepancy to propagate, $0.034$.
Finally, the data sample is fitted fixing the $F_L$ parameter to each of these 200 values and the RMS of the fit results is taken as systematic uncertainty.
The results are summarised in Table~\ref{tab:eff.shape}.

For completeness the $F_\mathrm{L}$ variable is also measured with the \BtoKstpsip sample, obtaining a value $0.538 \pm 0.008\stat$ to be compared with the world-average value of $0.463^{+0.028}_{-0.040}\,\text{(stat+syst)}$.

\begin{table*}[!htb]
  \begin {center}
    \begin{small}
      \caption{Systematic uncertainties: efficiency shape.
        \label{tab:eff.shape}}
      \begin{tabular}{l|c|c|c}
        $q^2$ bin index  & $P_1$ & $P_5'$  \\
        \hline
        $ 0 $ & $\pm0.017$ & $\pm0.005$ \\
        $ 1 $ & $\pm0.048$ & $\pm0.060$ \\
        $ 2 $ & $\pm0.093$ & $\pm0.065$ \\
        $ 3 $ & $\pm0.094$ & $\pm0.045$ \\
        $ 5 $ & $\pm0.083$ & $\pm0.059$ \\
        $ 7 $ & $\pm0.100$ & $\pm0.060$ \\
        $ 8 $ & $\pm0.068$ & $\pm0.041$ \\
      \end{tabular}
    \end{small}
  \end{center}
\end{table*}

\paragraph{Cross check: branching fraction of $\PBz\to\PKst\Ppsi'$ and $\PBz\to\PKst\PJpsi$ }

A further test to validate the efficiency shape obtained from MC is to compare the branching fraction of the two control samples: $\PBz\to\PKst\Ppsi'$ and $\PBz\to\PKst\PJpsi$.

The ratio of the two BR can be computed as follow:
\begin{equation}\label{RBRatio}
  \frac{\mathcal{B}(\PBz\to\PKst\psi')}{\mathcal{B}(\PBz\to\PKst J/\psi)}
  = \frac{Y_{\Ppsi'}}{\epsilon_{\psi'}}\frac{\epsilon_{\PJpsi}}{Y_{J/\psi}}
  \frac{\mathcal{B}(J/\psi \to \Pgmp \Pgmm)}{\mathcal{B}(\psi' \to \Pgmp \Pgmm)}
\end{equation}
where $Y_{\psi'}$ and $\epsilon_{\psi'}$ are the yield and the efficiency for the $\psi'$ channel, and likewise for $Y_{\PJpsi}$ and $\epsilon_{J/\psi}$ for $J/\psi$ one.

The ratio of $\mathcal{B}$ for $J/\psi(\psi') \to \Pgmp \Pgmm$ is $7.54\pm0.86$ (PDG).
The value of the ratio is (PDG): $\frac{\mathcal{B}(\PBz\to\PKst\psi')}{\mathcal{B}(\PBz\to\PKst J/\psi)}=0.484 \pm 0.018 \textrm{\tiny{(stat)}} \pm 0.011 \textrm{\tiny{(syst)}} \pm 0.012 \textrm{\tiny{($R^{ee}_\psi$)}}$.

A first way to compute the ratio is to use directly the absolute efficiency for the two channels as obtained from MC.
The ratio we obtain with this method is $0.476\pm 0.008({\rm stat}) \pm 0.055({\rm R}_{\psi}^{\mu\mu})$, in very good agreement with PDG.

However, the computation above assumes that the angular shape of the control channel is correctly simulated in MC.
To take properly into account the real signal and efficiency shape, we repeated the same computation using as efficiency $\varepsilon_{J/\psi/\Ppsi'}=\int_{phase~space}S(\vec{x};\vec{p})\times\varepsilon(\vec{x})d\vec{x}$, where $S(\vec{x};\vec{p})$ is the signal \pdf, $\vec{p}$ is the set of angular parameter we got from the fit on the data on each control region, and $\varepsilon(\vec{x})$ is the efficiency for each region as a function of angular variable.
The result of this more detailed computation is: $0.480\pm 0.008({\rm stat}) \pm 0.055({\rm R}_{\psi}^{\mu\mu})$, again in very good agreement with PDG.

The same ratio can be computed using only right tag events, only wrong tag events, or both: in all the cases the agreement is very good.

\subsection{Fitting Bias}
\label{sec:fitbias-syst}

The fitting procedure itself could generate biases in the results, in addition to the uncertainties from the fitting components (efficiency, background shape, \pdf, etc) described above.
The fitting procedure uncertainties thus arise from the possible biases from the fitting methods and procedures.

We estimate this contribution from the fit results of the data-like simulated samples, which are described in Section~\ref{sec:Cocktail-MC}.

To evaluate the fitting bias, we compare the average result of the fits to cocktail MC samples and the fitting result of full statistics MC sample.
These differences are used as an estimation of this systematic uncertainty.
The results are summarised in Table~\ref{tab:fit bias}.

\begin{table*}[!htb]
  \begin {center}
    \begin{small}
      \caption{Systematic uncertainties: fitting bias.
        \label{tab:fit bias}}
      \begin{tabular}{l|c|c|c}
        $q^2$ bin index  & $P_1$ & $P_5'$  \\
        \hline
        $0$ & $\pm0.005$ & $\pm0.040$ \\
        $1$ & $\pm0.007$ & $\pm0.010$ \\
        $2$ & $\pm0.019$ & $\pm0.119$ \\
        $3$ & $\pm0.019$ & $\pm0.106$ \\
        $5$ & $\pm0.041$ & $\pm0.052$ \\
        $7$ & $\pm0.071$ & $\pm0.048$ \\
        $8$ & $\pm0.078$ & $\pm0.031$ \\
      \end{tabular}
    \end{small}
  \end{center}
\end{table*}


\subsection{Wrong CP assignment}
\label{sec:sys-mistag}

% This source of uncertainties will be evaluated from the data after the
% unblinding of the analysis, following what has been done in
% BPH-13-010. The method is described in the following steps:

% \begin{itemize}
%     \item measure $B^0$ width with $K^{*0}(K\pi)J/\psi(\mu\mu)$
%       control sample and compare the data/MC difference, as a cross-check.
%     \item measure mistag ratio with $K^{*0}(K\pi)J/\psi(\mu\mu)$
%       control sample and compare the data/MC difference, as a cross-check.
%     \item Finally propagate the uncertainty of the mistag fraction to
%       the results. This will be done by fitting data N times with
%       mis-tag ratio randomly generated according to gaussian centered
%       at nominal value and with $\sigma$ from the previous two
%       checks.
% \end{itemize}

% \subsubsection{Mistag fraction}
% \label{sec:mistag}
The error on mistag fraction has been estimated using the control channel \BtoKstJpsi, in $q^2$ bin~4.
A full fit on this channel has been performed, leaving also the mistag fraction free to float.
This fit is possible on the control channel thanks to its large statistics, and the result has been compared with the mistag fraction estimated from MC sample.
The difference between these values is 0.008 and, compared with the statistical uncertainty of the MC estimate in this bin, $\sigma\lesssim0.002$, it results to be the dominant uncertainty related to the mistag fraction.

To assess the propagation of this systematic effect to the physical parameter of the final fit on data, a set of fits has been performed, using as free parameters $A_s^5$, $P_1$, $P_5'$, the signal and background yields, and using a fixed mistag fraction.
A set of 10 different values of mistag fractions has been used, randomly generated with Gaussian distribution around the mistag fraction estimated from MC for each $q^2$ bin, and with $\sigma$ equal to the difference of 0.008 computed on the control channel.
For each $q^2$ bin, the RMS of the results of the 10 fits is used as systematic uncertainty for the fitted angular parameters, as summarised in Table~\ref{tab:Mistagfraction}.

\begin{table*}[!htb]
  \begin {center}
    \caption{Systematic uncertainties: mistag fraction.\label{tab:Mistagfraction}}
    \begin{tabular}{l|c|c|c}
      Bin & $A_s^5$  & $P_1$ & $P_5'$  \\
      \hline
      \hline
      0  & 0.000035 & 0.014  &  0.013 \\
      1  & 0.000014 & 0.022  &  0.015 \\
      2  & 0.0050   & 0.016  &  0.014 \\ 
      3  & 0.0025   & 0.0084 &  0.0058\\ 
      5  & 0.0037   & 0.043  &  0.032  \\ 
      7  & 0.015    & 0.11   &  0.066 \\
      8  & 0.0050   & 0.025  &  0.016 \\
    \end{tabular}

  \end{center}
\end{table*}

\subsection{Background distributions}
\label{sec:sys-bkg}

The parameters of the background component in the \pdf are estimated on the mass sidebands and kept fixed in the fit to the full mass range, as described in Section~\ref{sec:backg}.
Due to the limited statistics of the mass sideband samples, the background parameters have a not-negligible uncertainty, that should be propagated to the analysis results as systematic uncertainty.

The errors of the background \pdf parameters are evaluated by the {\tt HESSE} algorithm running during the fit to the mass sidebands.
A set of 200 background toy-functions are then generated by using the same polynomial expression described in Section~\ref{sec:backg}, and randomly generating the value of each parameter from a Gaussian distribution, with mean and width equal to the {\tt HESSE} results of the sideband fit.

To propagate this uncertainty to the signal parameters, the fit to the data is run 200 times, by using the new background toy-functions. The widths of the result distributions are used as systematic uncertainties.

An example of the distributions of the toy-background fit results, for $q^2$ bin~2, is shown in Figure~\ref{fig:BKG-bin2}.

\begin{figure}[!hbt]
  \centering
  \includegraphics[width=0.8\textwidth]{Figures/BKG/bin2.pdf}
  \caption{The results of the fits on data performed with background toy-functions, for the $q^2$ bin~2.
    The distributions of $P_1$ (top left), $P_5'$ (top right), signal yield (bottom left), and background yield (bottom right) are shown.}
  \label{fig:BKG-bin2}
\end{figure}

The resulting uncertainties are summarised in the Table~\ref{tab:bkg. shape}.

\begin{table*}[!htb]
  \begin {center}
    \begin{small}
      \caption{Systematic uncertainties: background shape.
        \label{tab:bkg. shape}}
      \begin{tabular}{l|c|c|c}
        $q^2$ bin index & $P_1$ & $P_5'$ \\
        \hline
        $0$ & $\pm0.013$ & $\pm0.010$ \\
        $1$ & $\pm0.031$ & $\pm0.024$ \\
        $2$ & $\pm0.046$ & $\pm0.030$ \\
        $3$ & $\pm0.022$ & $\pm0.013$ \\
        $5$ & $\pm0.070$ & $\pm0.049$ \\
        $7$ & $\pm0.069$ & $\pm0.051$ \\
        $8$ & $\pm0.012$ & $\pm0.015$ \\
      \end{tabular}
    \end{small}
  \end{center}
\end{table*}

\subsection{Mass Distribution}
\label{sec:sys-mass distribution}

% Some components of the p.d.f are derived from simulations. The
% simulation might not describe the behaviour of the real detector with
% enough accuracy which may affect the fitting results.  

%% The mass distributions for the correctly tagged and mistagged events are each described by the sum of two Gaussian functions, with a common mean for all four Gaussian functions.
%% The mean value is obtained from the fit to the data, while the other parameters (four $\sigma$ and two ratios) are obtained from fits to MC-simulated events.

The parameters of the mass shape in the signal component of the \pdf are estimated through a fit to the MC mass distribution and kept fixed in the data fit, as described in Section~\ref{sec:fitseq}.
To evaluate the bias due to mis-modelling of the MC description of the mass shapes, and propagate it to the analysis results as a systematics, the control channels are used.
Thanks to the high-statistics of these samples, it is possible to perform a fit with some mass parameters free to float.
The angular parameters are measured on both control channels, at first with signal mass shape for right-tagged events free to float, and then with signal mass shape for mis-tagged events free to float.
The difference of the angular parameter values obtained from these fits, with respect to the results with the standard fit sequence, are calculated.
For each parameter, the largest of these differences is used as systematic uncertainty for all the $q^2$ bins.

The maximum difference of $P1$ is 0.012, of $P5'$ is 0.019.

\subsection{Uncertainty from fixed \pdf parameters}
\label{sec:sys-fixedparm}

Fixing three of the six angular parameters in the fit could have an effect of modifying the statistical uncertainties for the fitted parameters.
This is true, in general, whenever there is a correlation between any fixed parameter and a free one.
Since this effect could lead to an underestimation of the statistical error, a specific systematic uncertainty has been created to compensate this effect.
For each measured parameter we define a \textit{scale factor} as the correction factor, greater than or equal to one, that should be applied to the underestimated statistical error to compensate this effect. 

%% Firstly, this scale factor is computed.

Since it is not possible to fit the data with the $F_L$, $F_s$, $A_s$ parameters free to float, this scale factor is computed on pseudo-experiments with one hundred times the statistics of the data sample.
Ten pseudo-experiments are used, for each $q^2$ bin, and the average of the ten resulting scale factors is used to compute the uncertainty.
The same procedure described in Sec.~\ref{sec:statUncert} to fit a pseudo-experiment for the Feldman-Cousins (FC) method is used here to fit each high-statistics pseudo-experiment.
The only difference is that the number of points where the 2D-likelihood is probed here is around one hundred, in order to guarantee a more robust result.
This fitting procedure is applied twice, both fixing the $F_L$, $F_s$, $A_s$ parameters and leaving them free to float.
Then, two scale factors are defined, one for $P_1$ and one for $P'_5$, as the ratio between the confidence intervals obtained with the two fits (fit with floating parameters over fit with fixed parameters).
Note that here the {\tt custom MINOS} method, defined in Sec.~\ref{sec:statUncert}, is used to define the parameters' confidence interval.

Finally, each scale-factor average was used to compute the systematic uncertainty.
If it is smaller than one, it means that the estimated effect of freeing the parameters is to reduce the statistical error.
In this case, no correction is applied.
Otherwise, if it is larger than one, we define the uncertainty value as the one that, if added in quadrature with the statistical error, would increase it by this estimated factor.
Thus, it is
\begin{equation} \label{eq:fix.syst.def}
S = \sigma \sqrt{f^2-1}
\end{equation}
where $\sigma$ is the statistical error and $f$ is the scale-factor average.

A test to check whether the \textit{scale factors} depend on the statistics of the pseudo-experiments used is described in Section~\ref{sec:app-fixSyst.test}.

The scale-factor averages and the systematic uncertainty values are reported in table~\ref{tab:systFix}.

\begin{table*}[!htb]
  \caption{Scale-factor average values and systematic uncertainties computed to compensate the statistical error reduction introduced fixing some PDF parameters\label{tab:systFix}}
  \begin{center}
    \begin{tabular}{l|cc|cc}
      $q^2$ bin index & SF($P_1$)  & $P_1$  & SF($P'_5$) & $P'_5$ \\
      \hline
      0 & 1.014 & 0.077 & 1.003 & 0.025  \\
      1 & 1.116 & 0.211 & 1.099 & 0.148  \\
      2 & 1.113 & 0.139 & 1.385 & 0.206  \\
      3 & 1.082 & 0.103 & 1.028 & 0.041  \\
      5 & 1.048 & 0.053 & 1.143 & 0.069  \\
      7 & 0.982 & 0.000 & 1.090 & 0.072  \\
      8 & 1.091 & 0.083 & 0.989 & 0.000  \\
    \end{tabular}
  \end{center}
\end{table*}

To verify that the procedure of fixing those parameters in the fit is only affecting the statistical uncertainty, but it is not introducing any significant bias in the best-fit values, we compared the results of the fits to the sets of pseudo-experiments, both with floating and fixed parameters, with the input values used to generate them.
An example of the plots used for this comparison, for $q^2$ bin~2, is shown in Figure~\ref{fig:fixSyst2}.
No significant bias are present.

%% \begin{figure}[!hbt]
%%   \centering
%%   \includegraphics[width=0.7\textwidth]{Figures/FixSyst/fixSysPlot_b0.pdf}
%%   \caption{Distributions of the best-fit values of the fits performed, in $q^2$ bin 0, on the high-statistics pseudo-experiments used to compute the systematic uncertainty for the fixed parameters, as described in Section~\ref{sec:sys-fixedparm}. The blue (red) histogram is the distribution of the results of the fit with fixed (floating) $F_L$, $F_s$, $A_s$ parameters. The black line marks the input value used to generate the pseudo-experiments. Top: results for $P_1$. Bottom: results for $P'_5$. }
%%   \label{fig:fixSyst0}
%% \end{figure}

%% \begin{figure}[!hbt]
%%   \centering
%%   \includegraphics[width=0.7\textwidth]{Figures/FixSyst/fixSysPlot_b1.pdf}
%%   \caption{Distributions of the best-fit values of the fits performed, in $q^2$ bin 1, on the high-statistics pseudo-experiments used to compute the systematic uncertainty for the fixed parameters, as described in Section~\ref{sec:sys-fixedparm}. The blue (red) histogram is the distribution of the results of the fit with fixed (floating) $F_L$, $F_s$, $A_s$ parameters. The black line marks the input value used to generate the pseudo-experiments. Top: results for $P_1$. Bottom: results for $P'_5$. }
%%   \label{fig:fixSyst1}
%% \end{figure}

\begin{figure}[!hbt]
  \centering
  \includegraphics[width=0.7\textwidth]{Figures/FixSyst/fixSysPlot_b2.pdf}
  \caption{Distributions of the best-fit values of the fits performed, in $q^2$ bin 2, on the high-statistics pseudo-experiments used to compute the systematic uncertainty for the fixed parameters, as described in Section~\ref{sec:sys-fixedparm}. The blue (red) histogram is the distribution of the results of the fit with fixed (floating) $F_L$, $F_s$, $A_s$ parameters. The black line marks the input value used to generate the pseudo-experiments. Top: results for $P_1$. Bottom: results for $P'_5$. }
  \label{fig:fixSyst2}
\end{figure}

%% \begin{figure}[!hbt]
%%   \centering
%%   \includegraphics[width=0.7\textwidth]{Figures/FixSyst/fixSysPlot_b3.pdf}
%%   \caption{Distributions of the best-fit values of the fits performed, in $q^2$ bin 3, on the high-statistics pseudo-experiments used to compute the systematic uncertainty for the fixed parameters, as described in Section~\ref{sec:sys-fixedparm}. The blue (red) histogram is the distribution of the results of the fit with fixed (floating) $F_L$, $F_s$, $A_s$ parameters. The black line marks the input value used to generate the pseudo-experiments. Top: results for $P_1$. Bottom: results for $P'_5$. }
%%   \label{fig:fixSyst3}
%% \end{figure}

%% \begin{figure}[!hbt]
%%   \centering
%%   \includegraphics[width=0.7\textwidth]{Figures/FixSyst/fixSysPlot_b5.pdf}
%%   \caption{Distributions of the best-fit values of the fits performed, in $q^2$ bin 5, on the high-statistics pseudo-experiments used to compute the systematic uncertainty for the fixed parameters, as described in Section~\ref{sec:sys-fixedparm}. The blue (red) histogram is the distribution of the results of the fit with fixed (floating) $F_L$, $F_s$, $A_s$ parameters. The black line marks the input value used to generate the pseudo-experiments. Top: results for $P_1$. Bottom: results for $P'_5$. }
%%   \label{fig:fixSyst5}
%% \end{figure}

%% \begin{figure}[!hbt]
%%   \centering
%%   \includegraphics[width=0.7\textwidth]{Figures/FixSyst/fixSysPlot_b7.pdf}
%%   \caption{Distributions of the best-fit values of the fits performed, in $q^2$ bin 7, on the high-statistics pseudo-experiments used to compute the systematic uncertainty for the fixed parameters, as described in Section~\ref{sec:sys-fixedparm}. The blue (red) histogram is the distribution of the results of the fit with fixed (floating) $F_L$, $F_s$, $A_s$ parameters. The black line marks the input value used to generate the pseudo-experiments. Top: results for $P_1$. Bottom: results for $P'_5$. }
%%   \label{fig:fixSyst7}
%% \end{figure}

%% \begin{figure}[!hbt]
%%   \centering
%%   \includegraphics[width=0.7\textwidth]{Figures/FixSyst/fixSysPlot_b8.pdf}
%%   \caption{Distributions of the best-fit values of the fits performed, in $q^2$ bin 8, on the high-statistics pseudo-experiments used to compute the systematic uncertainty for the fixed parameters, as described in Section~\ref{sec:sys-fixedparm}. The blue (red) histogram is the distribution of the results of the fit with fixed (floating) $F_L$, $F_s$, $A_s$ parameters. The black line marks the input value used to generate the pseudo-experiments. Top: results for $P_1$. Bottom: results for $P'_5$. }
%%   \label{fig:fixSyst8}
%% \end{figure}


%%%%%%%%%%%%%%%%%%%%555

\clearpage


\subsection{Angular Resolution}
\label{sec:sys-angres}

This systematic uncertainty is due to the limited detector resolution on angular distributions.
To evaluate its impact, the likelihood fit has been performed on the full simulated dataset on reconstructed quantities after all selections, considering only the right-tagged candidates.
The same fit has been performed on the same sample, but using the generated angular quantities in place of the reconstructed ones.
%% Each fit has been repeated for a set of 100 initial values, choosen randomly in the allowed phase space, and only the fully converged fit has been considered to compare the results, both for gen and for reco-level angles.
The difference between the two results, shown in table~\ref{tab:systReso}, is used as a systematic uncertainty for each $q^2$ bin.

\begin{table*}[!htb]
  \caption{Difference on the target physics observables ($F_L$, $P_1$, and $P'_5$) when obtained via a fit on reconstructed and generated angular distributions after all selection, performed on full MC sample. $\Delta{x}=|x_{RECO} - x_{GEN}|$\label{tab:systReso}}
  \begin{center}
    \begin{tabular}{l|ccc}
      $q^2$ bin index & $\Delta{F_L}$  & $\Delta{P_1}$  & $\Delta{P'_5}$ \\
      \hline
      0  & -3.98\e{-4}  & -1.50\e{-3} &  1.05\e{-4}   \\
      1  & -7.83\e{-4}  & -3.30\e{-3} &  1.03\e{-3}  \\
      2  & -8.83\e{-3}  & -6.88\e{-3} &  1.33\e{-3}   \\
      3  & -4.94\e{-4}  & -7.38\e{-3} &  8.10\e{-4}   \\
      5  &  1.16\e{-3}  & -1.51\e{-2} &  2.36\e{-3}   \\
      7  &  2.90\e{-4}  & -7.26\e{-3} & -8.28\e{-3}   \\
      8  & -1.93\e{-3}  & -6.81\e{-2} &  1.23\e{-2}   \\
    \end{tabular}
  \end{center}
\end{table*}

\clearpage

\subsection{Feed-through background}
\label{sec:feedthr}

The $q^2$ bins just below and above the resonance regions may be contaminated by $B^0 \to {\rm J/}\psi K^{*0}$ and $B^0 \to \psi' K^{*0}$ feed-through events that are not removed by the selection criteria.
The potential uncertainties due to this contamination have been evaluated.

The distribution of residual feed-through events are described by a special \pdf.
The distributions in $q^2$ bins 3 and 5 are evaluated from $B^0 \to {\rm J/}\psi K^{*0}$ simulation, as shown in Figure~\ref{fig:feed-J-bin3} and Figure~\ref{fig:feed-J-bin5}.
Similarly, the distributions in $q^2$ bin~5 are evaluated from $B^0 \to \psi' K^{*0}$ simulated sample


\begin{figure}[!hbt]
  \centering
  \includegraphics[width=0.7\textwidth]{Figures/Syst/Feed-through/JPsi-bin3.pdf}
  \caption{Distributions of the feed-through events of the \BtoKstJpsi MC sample and the projections of the \pdf used to describe it, for $q^2$ bin~3.
  }
  \label{fig:feed-J-bin3}
\end{figure}

\begin{figure}[!hbt]
  \centering
  \includegraphics[width=0.7\textwidth]{Figures/Syst/Feed-through/JPsi-bin5.pdf}
  \caption{Distributions of the feed-through events of the \BtoKstJpsi MC sample and the projections of the \pdf used to describe it, for $q^2$ bin~5.
  }
  \label{fig:feed-J-bin5}
\end{figure} 

Data are then fitted with this additional component of feed through backgrounds, as shown in Fig~\ref{fig:feed-t-bin3} and Fig~\ref{fig:feed-t-bin5}.

\begin{figure}[!hbt]
  \centering
  \includegraphics[width=0.7\textwidth]{Figures/Syst/Feed-through/TotalPDFq2Bin_3_Canv0.pdf}
  \caption{Distributions of the data events, and the projections of the \pdf with the additional component for feed-through background, for $q^2$ bin~3.}
  \label{fig:feed-t-bin3}
\end{figure}

\begin{figure}[!hbt]
  \centering
  \includegraphics[width=0.7\textwidth]{Figures/Syst/Feed-through/TotalPDFq2Bin_5_Canv0.pdf}
  \caption{Distributions of the data events, and the projections of the \pdf with the additional component for feed-through background, for $q^2$ bin~5.}
  \label{fig:feed-t-bin5}
\end{figure} 

Discrepancies on the measured observables are then conservatively considered as symmetric systematic uncertainties.
There are two results for $q^2$ bin~5, one from $B^0 \to {\rm J/}\psi K^{*0}$ and the other from $B^0 \to \psi' K^{*0}$.
Then we choose the bigger one as the systematic uncertainty.
The result are in the Table~\ref{tab:feed-through}.

\begin{table*}[!htb]
  \begin {center}
    \begin{small}
      \caption{Systematic uncertainties from the feed through backgrounds.
        \label{tab:feed-through}}
      \begin{tabular}{l|c|c|c}
        $q^2$ bin index & $P_1$ & $P_5'$ \\
        \hline
        $3$ & $\pm 0.004$ & $\pm 0.012$ \\
        $5$ & $\pm 0.012$ & $\pm 0.020$ \\
        $7$ & $\pm 0.011$ & $\pm 0.024$ \\
      \end{tabular}
    \end{small}
  \end{center}
\end{table*}

\subsection{Bivariate Guassian fit range}\label{sec:bestFit}

As described in Section~\ref{sec:fitseq}, the procedure to get the best fit from the data is to discretise the $P_1,P'_5$ space, and maximise the likelihood as a function of nuisance parameters at fixed values of $P_1,P'_5$.
So we have a scan of the likelihood in the $P_1$, $P'_5$ plane.
Finally, the likelihood distribution is fitted with a bivariate normal distribution.

In order to check the stability of the fit, as well as any possible systematics due to the procedure, the angular parameters are evaluated varying the fit range.
The range of the fit is defined as $\pm1\sigma$ (as computed by the FC procedure described in Section~\ref{sec:statUncert}), multiplied by a scale factor (in the range $[0.1,2]$), around the centre defined as the value of $P_1$ or $P'_5$, among the points of the grid defined in Section~\ref{sec:fitseq}, where the likelihood has an absolute maximum.
So, for instance, a scale factor of $1.5$ means that the fit range will be $[P_1-1.5\sigma,P_1+1.5\sigma]$, and so on.

An example of the results of these scans is shown in Figure~\ref{fig:gausFitRange}, for $q^2$ bin~3.
To give an element of comparison, the $y$-axis range in the plots of the results is set equal to the confidence interval, as computed by the FC procedure described in Section~\ref{sec:statUncert}. 
%% the vertical axis range in these plots is chosen to be $\pm1\sigma$ for that specific bin.
The difference between the value of $P_1$ and $P'_5$ from the absolute maximum and the normal fit is small, when compared with statistical uncertainties and to the \textit{fit bias} uncertainty.
Also the trend with respect to the fit range is small.

A quantitative difference has been extracted by comparing the value of the position of the absolute maximum with that of a bivariate Gaussian fit, via linear fit in the scale factor range $[0.4,1.6]$, evaluated for scale factor equal to 1.
The differences are reported in table~\ref{tab:rangeGausFit}.

Since these differences have the same nature of the \textit{fit bias} uncertainty, but are smaller than it, no specific systematic uncertainty has been introduced for them. 

\begin{figure}
  \centering
  %% \includegraphics[width=0.45\textwidth]{Figures/FC/testGausFitRange_bin0.pdf}
  %% \includegraphics[width=0.45\textwidth]{Figures/FC/testGausFitRange_bin1.pdf}
  %% \includegraphics[width=0.45\textwidth]{Figures/FC/testGausFitRange_bin2.pdf}
  \includegraphics[width=0.8\textwidth]{Figures/FC/testGausFitRange_bin3.pdf}
  %% \includegraphics[width=0.45\textwidth]{Figures/FC/testGausFitRange_bin5.pdf}
  %% \includegraphics[width=0.45\textwidth]{Figures/FC/testGausFitRange_bin7.pdf}
  %% \includegraphics[width=0.45\textwidth]{Figures/FC/testGausFitRange_bin8.pdf}
  \caption{Comparison between the position of the absolute maximum of the likelihood (green line) and the results of the bivariate Gaussian fit, as a function of the fit range (black dots), for the $P_1$ (left) and $P_5'$ (right) parameters, in $q^2$ bin~3.
  }
  \label{fig:gausFitRange}
\end{figure}

\begin{table*}[!htb]
  \begin {center}
    \begin{small}
      \caption{Systematic uncertainties: bias from bivariate Gaussian fit to the likelihood.
        \label{tab:rangeGausFit}}
      \begin{tabular}{l|cc}
        \hline
        $q^2$ bin index   & P1 & P5p \\
        \hline
        $ 0 $    &   $\pm0.012$ & $\pm0.009 $   \\
        $ 1 $    &   $\pm0.005$ & $\pm0.005 $   \\
        $ 2 $    &   $\pm0.002$ & $\pm0.007 $   \\
        $ 3 $    &   $\pm0.016$ & $\pm0.0002$   \\
        $ 5 $    &   $\pm0.009$ & $\pm0.020 $   \\
        $ 7 $    &   $\pm0.010$ & $\pm0.012 $   \\
        $ 8 $    &   $\pm0.004$ & $\pm0.001 $   \\
        \hline
      \end{tabular}
    \end{small}
  \end{center}
\end{table*}

\clearpage
\subsection{Total Systematic Uncertainties}
\label{sec:sys-total}

The summary of all systematic uncertainties are shown in Table~\ref{tab:systematics}.

%% \begin{table*}[htbp!]
%%   \centering
%%   \caption{\label{tab:systematics} Systematic uncertainty contributions for the measurements of $P1$ and $P5'$.
%%     The total uncertainty in each $q^2$ bin is obtained by adding each contribution in quadrature.
%%     For each item, the range indicates the variation of the uncertainty in the signal $q^2$ bins.}
%%   \begin{tabular}{l|cccc}
%%     Systematic uncertainty & $P_1 (10^{-3})$ & $P_5' (10^{-3})$ \\[1pt]
%%     \hline \\[-2ex]
%%     Simulation mismodelling               &  1--33  &  10--23  \\[1pt]
%%     fitting bias                          &  5--78  &  10--119 \\[1pt]
%%     MC statistical uncertainty            & 29--73  &  31--112 \\[1pt]
%%     Efficiency                            & 17--100 &   5--65  \\[1pt]
%%     $\textrm{K}\pi$ mistagging                   &  8--110 &   6--66  \\[1pt]
%%     Background distribution               & 12--70  &  10--51  \\[1pt]
%%     Mass distribution                     &     12  &      19  \\[1pt]
%%     Feed-through background               &  4--12  &   3--24  \\[1pt]
%%     $F_L,F_S,A_S$ uncertainty propagation &  0--126 &   0--200 \\[1pt]
%%     Angular resolution                    &  2--68  & 0.2--12  \\[1pt]
%%     Bivariate fit range                   &  2--16  & 0.2--20  \\[1pt]
%%     \hline
%%     Total systematic uncertainty          & 70--220 &  70--230 \\[1pt]
%%   \end{tabular}
%% \end{table*}

\begin{table*}[htbp]
  \centering
  \caption{\label{tab:systematics} Systematic uncertainty contributions for the measurements of $P1$ and $P5'$.
    The total uncertainty in each $q^2$ bin is obtained by adding each contribution in quadrature.
    For each item, the range indicates the variation of the uncertainty in the signal $q^2$ bins.}
  %% \caption{\label{tab:systematics}
  %%   Systematic uncertainties in $P_1$ and $P_5'$. For each source, the range indicates the variation over the bins in $q^2$.}     
  \begin{tabular}{lcc}
    Source & $P_1 (\times 10^{-3})$ & $P_5' (\times 10^{-3})$ \\[1pt]
    \hline \\[-2ex]
    Simulation mismodeling       &   1--33   &  10--23  \\[1pt]
    Fit bias                     &   5--78   &  10--120 \\[1pt]
    Finite size of simulated samples  &  29--73   &  31--110 \\[1pt]
    Efficiency                   &  17--100  &   5--65  \\[1pt]
    $\PK\pi$ mistagging          &   8--110  &   6--66  \\[1pt]
    Background distribution      &  12--70   &  10--51  \\[1pt]
    %Partially reconstruced \PBz  &   ?--?    &   ?--?   \\[1pt]
    Mass distribution            &      12   &      19  \\[1pt]
    Feed-through background      &   4--12   &   3--24  \\[1pt]
    $F_\mathrm{L}$, $F_\mathrm{S}$, $A_\mathrm{S}$ uncertainty propagation & 0--210 & 0--210 \\[1pt]
    Angular resolution           &   2--68   & 0.1--12  \\[1pt]
    \hline
    Total                        & 100--230  &  70--250 \\[1pt]
  \end{tabular}
\end{table*}

\clearpage

\chapter{Fitting Results on data after unblinding}
\label{sec:result}

In this section I will present the results of the fit to the data sample, after removing the \textit{blind} status to the analysis.
Firstly, the methodology used to extract the statistical uncertainty is presented.
Then the best-fit values of the parameters are reported, together with the total statistical and systematical uncertainties.
Finally the distributions and the \pdf profiles for each $q^2$ bin are shown.

%% Here are results of the interesting variables from the fitting on the data after the unblinding.
%% The results serve as central values, with only statistical errors here.
%% The projection plots of each signal $q^2$ bin from the fitting are provided.
%% The fit procedure is describe in detail in sec.~\ref{sec:fitseq}.
%% The statistical uncertainties evaluation is described in sec.~\ref{sec:statUncert}.
%% The systematic uncertainties are described and evaluated, included those after blinding, are described in sec~\ref{sec:syst}.

If not otherwise specified, we are using the fitting procedures described in Sect.\ref{sec:fitseq} and other relevant sections.

\subsection{Statistical uncertainties determination}\label{sec:statUncert}

The determination of the statistical uncertainties for the measured parameters cannot just be delegated to the fit software, and in particular to {\tt MINOS}, due to the presence of physical boundaries on the parameters.
These boundaries are discussed in Section~\ref{sec:bound}, and comes from the requirement that the \pdf is positive defined everywhere.

Different approaches to the problem have been explored: in the following we describe three of them: {\tt custom MINOS}, {\tt hybrid frequentist-Bayesian}, and {\tt Feldman-Cousins}.
For the final determination of the statistical uncertainties we have used the latter, despite of its complexity and the huge CPU time required.

In this thesis, only the {\tt Feldman-Cousins} approach is described.

%% The former two methods are described in detail in appendix~\ref{sec:CM-Hyb}, and the corresponding coverage studies in appendix~\ref{sec:coverage}.
%% At the end of this section, a detailed comparison of the three methods will be presented, showing a good agreement among them.

%% , as was expected given the good results of the coverage studies show in appendix~\ref{sec:coverage}.

\subsubsection{Feldman-Cousins}
The approach used for the computation of statistical uncertainties, strongly suggested by the CMS Statistical Committee, is to apply the Feldman-Cousins method~\cite{FC} with nuisance parameters.
Given the enormous time that would be needed to build a full bi-dimensional confidence interval in the $P_1$-$P_5'$ parameter space, we decided to perform the F-C procedure only along the maximum of the profile along $P_1$ and $P'_5$, respectively, of the $\mathcal{L}$ distribution, and extract two mono-dimensional confidence intervals.

We consider the $\mathcal{L}$ distribution build fitting the data, as described in Section~\ref{sec:fitseq}, and look for the maximum of that distribution along one variable, while the other is fixed.
In general, an evaluation of the likelihood is not available for all the points in the scan of the $P_1,~P'_5$ plane, since, for some $P_1,~P'_5$ points, the fit is not converging.
Thus, building the profiles of the likelihood just looking for the maximum among the valid points leads often to unstable results.
So, we perform a bivariate Gaussian fit to the $\mathcal{L}$ points available, similar to the one used to estimate the best-fit values of the parameters, and take the two profiles of the fitted function, instead.
%% The maximum of the profiled function is always searched for inside the physical region: the results are shown in fig~\ref{fig:profileL}.
The profiles of the function are always restricted inside the physical region: the results are shown in fig~\ref{fig:profileL}.

\begin{figure}
  \centering
  \includegraphics[width=0.3\textwidth]{Figures/FC/scanFC_b0.png}
  \includegraphics[width=0.3\textwidth]{Figures/FC/scanFC_b1.png}
  \includegraphics[width=0.3\textwidth]{Figures/FC/scanFC_b2.png}

  \includegraphics[width=0.3\textwidth]{Figures/FC/scanFC_b3.png}
  \includegraphics[width=0.3\textwidth]{Figures/FC/scanFC_b5.png}
  \includegraphics[width=0.3\textwidth]{Figures/FC/scanFC_b7.png}

  \includegraphics[width=0.3\textwidth]{Figures/FC/scanFC_b8.png}
  \caption{Distribution of $\mathcal{L}$ in the $P_1,~P'_5$, with contour curves at $\Delta\log{\mathcal{L}}=0.5$ and 2.0, with the indication of the physical region.
    Superimposed is a bivariate Gaussian fit, and the position of the maximums of the profiled $\mathcal{L}$ (red) and that of the Gaussian fit (blue).}
  \label{fig:profileL}
\end{figure}

This procedure is more robust than just looking for the maximum of the $\mathcal{L}$ on a profile, since we do not have a determination of $\mathcal{L}$ for every bin.
This is typically due to the fact that the physical boundary in the  $P_1,~P'_5$ region depends on $A_s^5$, so the fit of a particular point might converge to a set of values outside the physical region, and so it fails.
Moreover, there is the chance that a determination of the $\log\mathcal{L}$ has a downward fluctuation which can be the minimum even though the shape of the $\log\mathcal{L}$ shows a parabolic behaviour, with a different vertex.
This kind of behaviour can be seen in Figure~\ref{fig:ProblematicPProf}.

\begin{figure}
  \centering
  \includegraphics[width=0.5\textwidth]{Figures/FC/ProblematicPProf.pdf}
  \caption{An example for profile of $\log\mathcal{L}$, where it is possible to see the two possible issues to find the minimum just scanning the results, instead of performing a fit.
    Some bins are empty since the fit for those points fails, and there is a downward fluctuation of $\log\mathcal{L}$ which creates a fake minimum.}
  \label{fig:ProblematicPProf}
\end{figure}

For each of the point in the scan of the $P_1,~P'_5$ plane defined above, thereafter referred to as GEN-points, we generate a set of 100 toy MC with data-like statistics, using the full \pdf, with $P_1$ and $P'_5$ parameters defined by the coordinates of the GEN-point, and with nuisance parameter $A_s^5$ which guarantees the wider physical region for $P_1,~P'_5$, namely $0.99$ or $-0.99$, depending on the bin.

Each toy is fitted with a procedure identical to that used for the data, the only difference is that instead of evaluating the likelihood on a $90\times90$ grid in the $P_1,~P'_5$ plane, only 20 points are scanned, in order to reduce the time needed to perform the whole procedure.
These 20 points are chosen randomly, according to a bivariate Gaussian, around a central value.
The width of the Gaussian is that returned by the fit on the data on the bin under consideration, while correlation is not considered.
The central value is the result of a fit on the toy with $P_1,~P'_5$ free to float, or, if the fit does not converge, the GEN-point itself.
If one of these points falls outside the physical boundary, it is discarded and an additional point is generated to replace it.
The $\mathcal{L}$ is evaluated in these 20 points and it is finally fitted with a bivariate Gaussian.
An example of such a fit is on fig.~\ref{fig:ExampleFC}.

Eventually, for each toy of any GEN-point, a determination of the $\mathcal{L}$ function is available.
In order to check if the GEN-point is inside or outside the $68\%$ CL region, we compare the $\Delta\log\mathcal{L}$ of the toys with the one of the data.
In particular, when computing the boundary of the parameter $P_1$ ($P'_5$), $\Delta\log\mathcal{L}_{toy_{i}}$ is defined, on the projection of $\log\mathcal{L}$ of the $toy_{i}$ on the $P_1$ ($P'_5$) axis, as the difference between the maximum of this projection and its value at $P_1=P_{1~\mathrm{GEN}}$ ($P'_5=P'_{5~\mathrm{GEN}}$), where ($P_{1~\mathrm{GEN}}$, $P'_{5~\mathrm{GEN}}$) are the coordinates of the GEN-point.
$\Delta\log\mathcal{L}_{Data}$ is defined as the difference between $\log\mathcal{L}$ of the best fit on data and that of the GEN-point.
%% In particular: $\Delta\log\mathcal{L}_{toy_{i}}$ is defined as the difference
%% between the $\log\mathcal{L}$ of the $toy_{i}$ and the $\log\mathcal{L}$ of the
%% GEN-point. $\Delta\log\mathcal{L}_{Data}$ is defined as the difference between
%% $\log\mathcal{L}$ of the best fit on data and that of the GEN-point.

The {\em ratio} value is computed as the fraction of toys with $\Delta\log\mathcal{L}$ lesser than that of the data.
A GEN point is considered inside the $1\sigma$ region, namely the $68.27\%$ CL region, if the ratio is lower than $68.27\%$.
An example of this comparison is shown in fig.~\ref{fig:ExampleFC}.

\begin{figure}[hbt]
  \centering
  \includegraphics[width=0.45\textwidth]{Figures/FC/FitGaus2D_Bin7_Reg0_GP3069_Toy7.png}
  \includegraphics[width=0.54\textwidth]{Figures/FC/histo_GP3084_7_0.pdf}
  \caption{An example of a bivariate fit on the 20 scan points for one GEN-point (left).
    Distribution of the $\Delta\log\mathcal{L}$ values for the 100 toys for one GEN-point and comparison with $\Delta\log\mathcal{L}$ for data, computed for that GEN-point (right).}
  \label{fig:ExampleFC}
\end{figure}

The final $68\%$ confidence interval for $P_1$ and $P'_5$ is found by looking at the distribution of the ratio as a function of $P_1$ and $P'_5$, respectively, and fitting this distribution with a linear function.
The crossing of the linear function with the $68.27\%$ line defines the $\pm1\sigma$ value.
In total four directions were scanned, corresponding to lower and upper bound for $P_1$ and $P'_5$.

In order to reduce the time consumption of this procedure, we started scanning GEN-points around the value of $\Delta\log\mathcal{L}=0.5$, and then extend the scanned region inside or outside depending on the results.
In case the slope of the linear fit was not very steep, more GEN-points were scanned to make the result more robust.

%% After a series of optimization, each individual fit to one toy took approximately 4 minutes, so, given that we perform the fit 20+1 times for each toy, the CPU time for each toy is about 1.6 h, including the toy generation, whose contribution is negligible.
%% We scanned in total about 1000 GEN-points, including many which were not used for the final computation of the FC intervals due to various reasons (debugging of the procedure, different iterations and refinement, etc), for a grand total of about $2\cdot{10^6}$ fits, corresponding to 1.6 year of CPU time.
%% The whole procedure, including setup, coding, debugging, learning and interpretation of results took about 2.5 months in total.

The results of the FC procedure described above are shown in fig~\ref{fig:FC0}~to~\ref{fig:FC8} for all the seven $q^2$ bins considered.
The intervals, as well as the central values, are summarised in table~\ref{tab:FC}.

\begin{figure}
  \centering
  \includegraphics[width=0.7\textwidth]{Figures/FC/scanFC_b0_v0.png}

  \includegraphics[width=0.4\textwidth]{Figures/FC/scanFC_1d_b0_reg1_v0.pdf}
  \includegraphics[width=0.4\textwidth]{Figures/FC/scanFC_1d_b0_reg0_v0.pdf}

  \includegraphics[width=0.4\textwidth]{Figures/FC/scanFC_1d_b0_reg2_v0.pdf}
  \includegraphics[width=0.4\textwidth]{Figures/FC/scanFC_1d_b0_reg3_v0.pdf}

  \caption{F-C results for bin 0.
    Scan of the GEN-points, superimposed to data $\mathcal{L}$ distribution: red points are outside the 68\% CL region, the blue one are inside.
    The blue lines defines the $\pm1\sigma$ region (top).
    Ratio distribution as a function of $P_1$ for lower and upper bounds, with linear fit and 68\% horizontal line (middle left, right).
    Likewise for $P'_5$ (bottom).}
  \label{fig:FC0}
\end{figure}

\begin{figure}
  \centering
  \includegraphics[width=0.7\textwidth]{Figures/FC/scanFC_b1_v0.png}

  \includegraphics[width=0.4\textwidth]{Figures/FC/scanFC_1d_b1_reg1_v0.pdf}
  \includegraphics[width=0.4\textwidth]{Figures/FC/scanFC_1d_b1_reg0_v0.pdf}

  \includegraphics[width=0.4\textwidth]{Figures/FC/scanFC_1d_b1_reg2_v0.pdf}
  \includegraphics[width=0.4\textwidth]{Figures/FC/scanFC_1d_b1_reg3_v0.pdf}

  \caption{F-C results for bin 1.
    Scan of the GEN-points, superimposed to data $\mathcal{L}$ distribution: red points are outside the 68\% CL region, the blue one are inside.
    The blue lines defines the $\pm1\sigma$ region (top).
    Ratio distribution as a function of $P_1$ for lower and upper bounds, with linear fit and 68\% horizontal line (middle left, right).
    Likewise for $P'_5$ (bottom).}
  \label{fig:FC1}
\end{figure}

\begin{figure}
  \centering
  \includegraphics[width=0.7\textwidth]{Figures/FC/scanFC_b2_v0.png}

  \includegraphics[width=0.4\textwidth]{Figures/FC/scanFC_1d_b2_reg1_v0.pdf}
  \includegraphics[width=0.4\textwidth]{Figures/FC/scanFC_1d_b2_reg0_v0.pdf}

  \includegraphics[width=0.4\textwidth]{Figures/FC/scanFC_1d_b2_reg2_v0.pdf}
  \includegraphics[width=0.4\textwidth]{Figures/FC/scanFC_1d_b2_reg3_v0.pdf}

  \caption{F-C results for bin 2.
    Scan of the GEN-points, superimposed to data $\mathcal{L}$ distribution: red points are outside the 68\% CL region, the blue one are inside.
    The blue lines defines the $\pm1\sigma$ region (top).
    Ratio distribution as a function of $P_1$ for lower and upper bounds, with linear fit and 68\% horizontal line (middle left, right).
    Likewise for $P'_5$ (bottom).}
  \label{fig:FC2}
\end{figure}

\begin{figure}
  \centering
  \includegraphics[width=0.7\textwidth]{Figures/FC/scanFC_b3_v0.png}

  \includegraphics[width=0.4\textwidth]{Figures/FC/scanFC_1d_b3_reg1_v0.pdf}
  \includegraphics[width=0.4\textwidth]{Figures/FC/scanFC_1d_b3_reg0_v0.pdf}

  \includegraphics[width=0.4\textwidth]{Figures/FC/scanFC_1d_b3_reg2_v0.pdf}
  \includegraphics[width=0.4\textwidth]{Figures/FC/scanFC_1d_b3_reg3_v0.pdf}

  \caption{F-C results for bin 3.
    Scan of the GEN-points, superimposed to data $\mathcal{L}$ distribution: red points are outside the 68\% CL region, the blue one are inside.
    The blue lines defines the $\pm1\sigma$ region (top).
    Ratio distribution as a function of $P_1$ for lower and upper bounds, with linear fit and 68\% horizontal line (middle left, right).
    Likewise for $P'_5$ (bottom).}
  \label{fig:FC3}
\end{figure}

\begin{figure}
  \centering
  \includegraphics[width=0.7\textwidth]{Figures/FC/scanFC_b5_v0.png}

  \includegraphics[width=0.4\textwidth]{Figures/FC/scanFC_1d_b5_reg1_v0.pdf}
  \includegraphics[width=0.4\textwidth]{Figures/FC/scanFC_1d_b5_reg0_v0.pdf}

  \includegraphics[width=0.4\textwidth]{Figures/FC/scanFC_1d_b5_reg2_v0.pdf}
  \includegraphics[width=0.4\textwidth]{Figures/FC/scanFC_1d_b5_reg3_v0.pdf}

  \caption{F-C results for bin 5.
    Scan of the GEN-points, superimposed to data $\mathcal{L}$ distribution: red points are outside the 68\% CL region, the blue one are inside.
    The blue lines defines the $\pm1\sigma$ region (top).
    Ratio distribution as a function of $P_1$ for lower and upper bounds, with linear fit and 68\% horizontal line (middle left, right).
    Likewise for $P'_5$ (bottom).}
  \label{fig:FC5}
\end{figure}

\begin{figure}
  \centering
  \includegraphics[width=0.7\textwidth]{Figures/FC/scanFC_b7_v0.png}

  \includegraphics[width=0.4\textwidth]{Figures/FC/scanFC_1d_b7_reg1_v0.pdf}
  \includegraphics[width=0.4\textwidth]{Figures/FC/scanFC_1d_b7_reg0_v0.pdf}

  \includegraphics[width=0.4\textwidth]{Figures/FC/scanFC_1d_b7_reg2_v0.pdf}
  \includegraphics[width=0.4\textwidth]{Figures/FC/scanFC_1d_b7_reg3_v0.pdf}

  \caption{F-C results for bin 7.
    Scan of the GEN-points, superimposed to data $\mathcal{L}$ distribution: red points are outside the 68\% CL region, the blue one are inside.
    The blue lines defines the $\pm1\sigma$ region (top).
    Ratio distribution as a function of $P_1$ for lower and upper bounds, with linear fit and 68\% horizontal line (middle left, right).
    Likewise for $P'_5$ (bottom).}
  \label{fig:FC7}
\end{figure}

\begin{figure}
  \centering
  \includegraphics[width=0.7\textwidth]{Figures/FC/scanFC_b8_v0.png}

  \includegraphics[width=0.4\textwidth]{Figures/FC/scanFC_1d_b8_reg1_v0.pdf}
  \includegraphics[width=0.4\textwidth]{Figures/FC/scanFC_1d_b8_reg0_v0.pdf}

  \includegraphics[width=0.4\textwidth]{Figures/FC/scanFC_1d_b8_reg2_v0.pdf}
  \includegraphics[width=0.4\textwidth]{Figures/FC/scanFC_1d_b8_reg3_v0.pdf}

  \caption{F-C results for bin 8.
    Scan of the GEN-points, superimposed to data $\mathcal{L}$ distribution: red points are outside the 68\% CL region, the blue one are inside.
    The blue lines defines the $\pm1\sigma$ region (top).
    Ratio distribution as a function of $P_1$ for lower and upper bounds, with linear fit and 68\% horizontal line (middle left, right).
    Likewise for $P'_5$ (bottom).}
  \label{fig:FC8}
\end{figure}

\begin{table*}[!htb]

  \begin{center}
    \caption{Summary of results of $P_1$ and $P'_5$ in different $q^2$ bins with the $\pm1\sigma$ statistical uncertainties as computed with the FC procedure.}\label{tab:FC}

    \begin{tabular}{l|rl|rl}
      & \multicolumn{2}{c|}{$P_1$} &  \multicolumn{2}{c}{$P'_5$} \\ 
      Bin  & Fit & & Fit &   \\ 
      \hline
      0 &  $0.119 $  &$^{+ 0.46}_{- 0.47}$ & $0.101$  & $^{+ 0.32}_{- 0.31}$ \\
      1 &  $-0.685$ &$^{+ 0.58}_{- 0.27}$  & $-0.567$ & $^{+ 0.34}_{- 0.31}$ \\
      2 &  $0.533 $  &$^{+ 0.24}_{- 0.33}$ & $-0.957$ & $^{+ 0.22}_{- 0.21}$ \\
      3 &  $-0.470$ &$^{+ 0.27}_{- 0.23}$  & $-0.643$ & $^{+ 0.15}_{- 0.19}$ \\
      5 &  $-0.531$ &$^{+ 0.2}_{- 0.14} $  & $-0.690$ & $^{+ 0.11}_{- 0.14}$ \\
      7 &  $-0.329$ &$^{+ 0.24}_{- 0.23}$  & $-0.664$ & $^{+ 0.13}_{- 0.2}$ \\
      8 &  $-0.533$ &$^{+ 0.19}_{- 0.19}$  & $-0.559$ & $^{+ 0.12}_{- 0.12}$ \\
      \hline                                                  

    \end{tabular}
  \end{center}
\end{table*}

\clearpage
\subsubsection{Correlation coefficient}

From the distribution of the likelihood, in Figure~\ref{fig:profileL}, it is also possible to get the correlation coefficient between the two measured parameters, $P_1,~P'_5$.
The $\mathcal{L}$ in the $P_1,~P'_5$ plane is fitted with a bivariate Gaussian, with the following expression:
\begin{equation}\label{eq:bivariateGauss}
  f(x,y)=\frac{\exp \left\{ -\frac 1{2(1-\rho ^2)}\left[ \left( \frac{x-\mu _x}{\sigma _x}\right) ^2-2\rho \left( \frac{x-\mu _x}{\sigma _x}\right) \left(\frac{y-\mu _y}{\sigma _y}\right) +\left( \frac{y-\mu _y}{\sigma _y}\right)^2\right] \right\} }{2\pi \sigma _x\sigma _y\sqrt{1-\rho ^2}} 
\end{equation}
where $x=P_1$, $y=P'_5$ etc. The $\rho$ parameter in Equation~\ref{eq:bivariateGauss} is the correlation coefficient among the two parameters and it is reported in Table~\ref{tab:correlation} for all the bins.

\begin{table*}[!htb]

  \begin{center}
    \caption{Correlation coefficient between $P_1$ and $P'_5$ in different $q^2$ bins.}\label{tab:correlation}

    \begin{tabular}{l|c}
      Bin  & $\rho$ \\ 
      \hline
      \hline
      0 &  $-0.0526$  \\
      1 &  $-0.0452$  \\
      2 &  $+0.4715$  \\
      3 &  $+0.0761$  \\
      5 &  $+0.6077$  \\
      7 &  $+0.4188$  \\
      8 &  $+0.4621$  \\

    \end{tabular}
  \end{center}
\end{table*}

%% \subsubsection{Comparison of FC with the two other methods ((CM and Hyb)}
%% In table~\ref{tab:FCcomparison} a detailed comparison of the statistical uncertainties found with the FC procedure are compared with two other approaches proposed.
%% From the table is possible to see that the two alternative method (CM and Hyb) give intervals which are in very good agreement with the one provided by the FC one, with very few exceptions, where the difference is anyhow limited.
%% This is in agreement with the coverage studies for the two alternative methods described in sec~\ref{sec:coverage}.
%% It is worth to stress the huge difference in time spent to compute the intervals with the first two method and the FC one.
%% The first two can be performed easily once the detailed discretisation of the likelihood is done, while the third (FC) required millions of fit, $>1.5$~year of CPU time and two months of work.

%% \begin{table*}
%%   \begin{center}
%%     \caption{Comparison of the results of the statistical uncertainties computed with the three methods described in the text: {\tt Feldman-Cousins (FC)}, {\tt custom MINOS (CM)}, {\tt hybrid frequentist-Bayesian (Hyb)}.
%%       The agreement of the three method is very good for all the bins, with very few exceptions.}
%%     \label{tab:FCcomparison}

%%     %\setlength\extrarowheight{3pt}
%%     \begin{tabular}{l|rccc|rccc}
%%       & \multicolumn{4}{c|}{$P_1$}       &  \multicolumn{4}{c}{$P'_5$} \\ 
%%       Bin  & & FC & CM & Hyb       & & FC & CM & Hyb    \\ 
%%       \hline
%%       0 & $ 0.12$ &  $^{+ 0.46}_{- 0.47}$  & $^{+ 0.44 }_{- 0.463 }$ & $^{+ 0.42 }_{- 0.447 }$  & $ 0.10$ & $^{+ 0.32}_{- 0.31}$ & $^{+ 0.313 }_{- 0.333 }$ & $^{+ 0.313 }_{- 0.333 }$ \\
%%       1 & $-0.69$ &  $^{+ 0.58}_{- 0.27}$  & $^{+ 0.59 }_{- 0.267 }$ & $^{+ 0.537 }_{- 0.25 }$  & $-0.57$ & $^{+ 0.34}_{- 0.31}$ & $^{+ 0.35 }_{- 0.29 }$ & $^{+ 0.35 }_{- 0.29 }$     \\
%%       2 & $ 0.53$ &  $^{{+ 0.24}}_{- 0.33}$  & $^{{+ 0.333 }}_{- 0.36 }$ & $^{{+ 0.297 }}_{- 0.32 }$  & $-0.96$ & $^{+ 0.22}_{{- 0.21}}$ & $^{+ 0.23 }_{{- 0.163 }}$ & $^{+ 0.23 }_{{- 0.163 }}$   \\
%%       3 & $-0.47$ &  $^{+ 0.27}_{- 0.23}$  & $^{+ 0.307 }_{- 0.25 }$ & $^{+ 0.283 }_{- 0.23 }$  & $-0.64$ & $^{+ 0.15}_{- 0.19}$ & $^{+ 0.18 }_{- 0.183 }$ & $^{+ 0.18 }_{- 0.183 }$   \\
%%       \hline                                                                                                                         
%%       5 & $-0.53$ &  $^{{+ 0.2}}_{- 0.14} $  & $^{{+ 0.153} }_{- 0.137 }$ & $^{{+ 0.16} }_{- 0.14 }$  & $-0.69$ & $^{+ 0.11}_{- 0.14}$ & $^{+ 0.097 }_{- 0.12 }$ & $^{+ 0.107 }_{- 0.123 }$ \\
%%       \hline                                                                                                                                           
%%       7 & $-0.33$ &  $^{+ 0.24}_{- 0.23}$  & $^{+ 0.257 }_{- 0.23 }$ & $^{+ 0.25 }_{- 0.227 }$  & $-0.66$ & $^{+ 0.13}_{- 0.2}$  & $^{+ 0.143 }_{- 0.17 }$ & $^{+ 0.143 }_{- 0.17 }$  \\
%%       8 & $-0.53$ &  $^{+ 0.19}_{- 0.19}$  & $^{+ 0.217 }_{- 0.21 }$ & $^{+ 0.207 }_{- 0.2 }$   & $-0.56$ & $^{+ 0.12}_{- 0.12}$ & $^{+ 0.137 }_{- 0.143 }$ & $^{+ 0.137 }_{- 0.143 }$\\


%%     \end{tabular}
%%   \end{center}
%% \end{table*}



%% \clearpage

\subsection{Results of central values}
\label{sec:res-centval}

The results of the fit on data are summarised in Table~\ref{tab:dataresult}.
%% The \pdf we use for the final fit, once the two folding have been applied, has 2 physical parameters, $P1$, $P_5'$, which are the results of this analysis.
%% Other nuisance parameter, left free to float in the fit, are $Y^{C}_{S}$, $Y_{B}$, and $A^5_\mathrm{S}$.

The table presents results for all five parameters that was floating in the fit, both the analysis targets, $P1$ and $P_5'$, and the nuisance, $Y^{C}_{S}$, $Y_{B}$, and $A^5_\mathrm{S}$.
The errors reported are those from the FC procedure for $P1$, $P_5'$, while are those returned by MINOS for the other three.


\begin{table*}[!htb]
  \begin{center}
    \begin{footnotesize}
      \caption{The measured values of signal yield $Y^{C}_{S}$, background yield $Y_{B}$, $A^5_\mathrm{S}$, $P1$, and $P5'$ for the decay \BKpimm in bins of $q^2$.
        The first uncertainty is statistical and the second (when present) is systematic. 
        \label{tab:dataresult}}

      \begin{tabular}{c|ccc|cc}
        \hline
        $q^2~(\GeV^2)$      & $Y^{C}_{S}$ & $Y_{B}$ & $A_s^5$ & $P_1$ & $P_5'$  \\
        \hline         
        1.00--2.00     & $80  \pm 12$ &    $95 \pm 11$    & $0.043 \pm 0.066$      & $0.119 ^{+ 0.46}_{- 0.47}\pm0.058$  & $0.101 ^{+ 0.32}_{- 0.31}\pm0.116$  \\
        2.00--4.30     & $145 \pm 16$ &    $290\pm 20$    & $0.039 \pm 0.063$      & $-0.685^{+ 0.58}_{- 0.27}\pm0.088$  & $-0.567^{+ 0.34}_{- 0.31}\pm0.153$  \\
        4.30--6.00     & $119 \pm 14$ &    $216\pm 17$    & $-0.052\pm 0.092$      & $0.533 ^{+ 0.24}_{- 0.33}\pm0.175$  & $-0.957^{+ 0.22}_{- 0.21}\pm0.161$  \\
        6.00--8.68     & $247 \pm 21$ &    $351\pm 23$    & $0.057 \pm 0.005$     & $-0.470^{+ 0.27}_{- 0.23}\pm0.131$  & $-0.643^{+ 0.15}_{- 0.19}\pm0.138$  \\
        10.09--12.86   & $354 \pm 23$ &    $575\pm  1$    & $-0.005\pm 0.008$    & $-0.531^{+ 0.2 }_{- 0.14}\pm0.215$  & $-0.690^{+ 0.11}_{- 0.14}\pm0.246$  \\
        14.18--16.00   & $213 \pm 17$ &    $185\pm 16$    & $0.015 \pm 0.063$     & $-0.329^{+ 0.24}_{- 0.23}\pm0.245$  & $-0.664^{+ 0.13}_{- 0.2 }\pm0.188$  \\
        16.00--19.00   & $239 \pm 19$ &    $ 82\pm 0 $    & $-0.004\pm 0.119$    & $-0.533^{+ 0.19}_{- 0.19}\pm0.131$  & $-0.559^{+ 0.12}_{- 0.12}\pm0.072$  \\

        \hline
      \end{tabular}
    \end{footnotesize}
  \end{center}
\end{table*}

The results for $P1$, $P_5'$ and $As5$ are shown in Fig~\ref{fig:fitresultAs5}, Fig~\ref{fig:fitresultP1}, Fig~\ref{fig:fitresultp5}.
On the plots, the predictions from two theoretical groups are also shown.
The orange band shows the predictions from Matias et al\cite{Genon:Swave}.
and the pink band shows the predictions from Paul et al\cite{Paul2015}.
Both predictions are adapted to our $q^2$ binning scheme.

We also put the latest LHCb results~\cite{LHCbP5p} on the plots for comparison, also with only statistical errors.
%% Note they are using different $q^2$ binnings than us.


\begin{figure}[!hbtp]
  \centering
  \includegraphics[width=0.7\textwidth]{Figures/DATAFit/As5.pdf}
  \caption{The fitting results of the target physics observables.
    The results for $AS5$ are shown.}
  \label{fig:fitresultAs5}
\end{figure}


\begin{figure}[htbp!]
  \begin{center}
    \includegraphics[width=0.9\textwidth]{P1.pdf}
    \caption{CMS measurements of the $P_1$ angular parameter versus $q^2$ for \BtoKstmumu decays, in comparison to results from the LHCb~\cite{LHCbP5p2} Collaboration.
      The statistical uncertainties are shown by the inner vertical bars, while the outer vertical bars give the total uncertainties.
      The horizontal bars show the bin widths. The vertical shaded regions correspond to the \cPJgy\ and $\psi'$ resonances.
      The hatched regions show the predictions from two SM calculations described in the text, averaged over each $q^2$ bin.}
    \label{fig:fitresultP1}
  \end{center}
\end{figure}

\begin{figure}[htbp!]
  \begin{center}
    \includegraphics[width=0.9\textwidth]{P5p.pdf}
    \caption{CMS measurements of the $P_5'$ angular parameter versus $q^2$ for \BtoKstmumu decays, in comparison to results from the LHCb~\cite{LHCbP5p2} and Belle~\cite{BelleP5p} Collaborations.
      The statistical uncertainties are shown by the inner vertical bars, while the outer vertical bars give the total uncertainties.
      The horizontal bars show the bin widths. The vertical shaded regions correspond to the \cPJgy\ and $\psi'$ resonances.
      The hatched regions show the predictions from two SM calculations described in the text, averaged over each $q^2$ bin.}
    \label{fig:fitresultp5}
  \end{center}
\end{figure}

\subsubsection{Validation of the yield results}\label{sec:yieldComp}

The yield of signal and background events per each bin is compared with the previous CMS analysis~\cite{AN-14-129}, which is base on the same dataset, on Table~\ref{tab:yieldComp}.

\begin{table*}[!htb]
  \begin{center}
    \begin{footnotesize}
      \caption{Comparison of values of signal yield $Y^{C}_{S}$ and background yield $Y_{B}$ with the same values found in previous CMS analysis.\label{tab:yieldComp}}
      %%  (including both right-tagged and mis-tagged events)
      \begin{tabular}{c|ccc|ccc}
        \hline
        & \multicolumn{2}{c}{$Y^{C}_{S}$} & $\Delta$  & \multicolumn{2}{c}{$Y_{B}$} & $\Delta$ \\
        $q^2~(\GeV^2)$      & this analysis & previous CMS & & this analysis & previous CMS &\\
        \hline         
        1.00--2.00     & $80  \pm 12$ & $ 84 \pm 11$ & $-4\pm16$  &   $ 95\pm 11$    & $ 91 \pm 12$ & $4  \pm 16$\\
        2.00--4.30     & $145 \pm 16$ & $145 \pm 16$ & $0\pm23$   &   $290\pm 20$    & $289 \pm 20$ & $1  \pm 28$\\
        4.30--6.00     & $119 \pm 14$ & $117 \pm 15$ & $2\pm21$   &   $216\pm 17$    & $218 \pm 18$ & $-2 \pm 25$\\
        6.00--8.68     & $247 \pm 21$ & $254 \pm 21$ & $-7\pm30$  &   $351\pm 23$    & $344 \pm 23$ & $7  \pm 33$\\
        10.09--12.86   & $354 \pm 23$ & $362 \pm 25$ & $-8\pm34$  &   $575\pm 28$    & $567 \pm 29$ & $8  \pm 40$\\
        14.18--16.00   & $213 \pm 17$ & $225 \pm 18$ & $-12\pm25$ &   $185\pm 16$    & $175 \pm 17$ & $10 \pm 23$\\
        16.00--19.00   & $239 \pm 19$ & $239 \pm 18$ & $0\pm26$   &   $ 82\pm 12$    & $ 82 \pm 12$ & $0  \pm 17$\\

        \hline
      \end{tabular}
    \end{footnotesize}
  \end{center}
\end{table*}



\clearpage

\subsection{Detailed distributions in each $q^2$ bin}
\label{sec:res-proj}

The results of detailed distributions in each $q^2$ bin, together with the projections of the \pdf, are shown in the following figures from Fig.~\ref{fig:res_bin0} to Fig.~\ref{fig:res_bin8}.

\begin{figure}
  \centering
  \includegraphics[width=0.8\textwidth]{Figures/DATAFit/TotalPDFq2Bin_0_Canv0.pdf}
  \caption{Data distributions (black points) and the projections of the fitted \pdf (black curves), of its signal right-tagged component (blue curves), of its signal mis-tagged component (green curves), and of its background component (red curves), for $q^2$ bin~0.
    The data distribution and \pdf projections are show as functions of \PBz invariant mass, \cTL, \cTK, and \PHI.}
  \label{fig:res_bin0}
\end{figure}

\begin{figure}
  \centering
  \includegraphics[width=0.8\textwidth]{Figures/DATAFit/TotalPDFq2Bin_1_Canv0.pdf}
  \caption{Data distributions (black points) and the projections of the fitted \pdf (black curves), of its signal right-tagged component (blue curves), of its signal mis-tagged component (green curves), and of its background component (red curves), for $q^2$ bin~1.
    The data distribution and \pdf projections are show as functions of \PBz invariant mass, \cTL, \cTK, and \PHI.}
  \label{fig:res_bin1}
\end{figure}


\begin{figure}
  \centering
  \includegraphics[width=0.8\textwidth]{Figures/DATAFit/TotalPDFq2Bin_2_Canv0.pdf}
  \caption{Data distributions (black points) and the projections of the fitted \pdf (black curves), of its signal right-tagged component (blue curves), of its signal mis-tagged component (green curves), and of its background component (red curves), for $q^2$ bin~2.
    The data distribution and \pdf projections are show as functions of \PBz invariant mass, \cTL, \cTK, and \PHI.}
  \label{fig:res_bin2}
\end{figure}

\begin{figure}
  \centering
  \includegraphics[width=0.8\textwidth]{Figures/DATAFit/TotalPDFq2Bin_3_Canv0.pdf}
  \caption{Data distributions (black points) and the projections of the fitted \pdf (black curves), of its signal right-tagged component (blue curves), of its signal mis-tagged component (green curves), and of its background component (red curves), for $q^2$ bin~3.
    The data distribution and \pdf projections are show as functions of \PBz invariant mass, \cTL, \cTK, and \PHI.}
  \label{fig:res_bin3}
\end{figure}

\begin{figure}[!hbtp]
  \centering
  \includegraphics[width=0.8\textwidth]{Figures/DATAFit/TotalPDFq2Bin_5_Canv0.pdf}
  \caption{Data distributions (black points) and the projections of the fitted \pdf (black curves), of its signal right-tagged component (blue curves), of its signal mis-tagged component (green curves), and of its background component (red curves), for $q^2$ bin~5.
    The data distribution and \pdf projections are show as functions of \PBz invariant mass, \cTL, \cTK, and \PHI.}
  \label{fig:res_bin5}
\end{figure}

\begin{figure}
  \centering
  \includegraphics[width=0.8\textwidth]{Figures/DATAFit/TotalPDFq2Bin_7_Canv0.pdf}
  \caption{Data distributions (black points) and the projections of the fitted \pdf (black curves), of its signal right-tagged component (blue curves), of its signal mis-tagged component (green curves), and of its background component (red curves), for $q^2$ bin~7.
    The data distribution and \pdf projections are show as functions of \PBz invariant mass, \cTL, \cTK, and \PHI.}
  \label{fig:res_bin7}
\end{figure}

\begin{figure}
  \centering
  \includegraphics[width=0.8\textwidth]{Figures/DATAFit/TotalPDFq2Bin_8_Canv0.pdf}
  \caption{Data distributions (black points) and the projections of the fitted \pdf (black curves), of its signal right-tagged component (blue curves), of its signal mis-tagged component (green curves), and of its background component (red curves), for $q^2$ bin~8.
    The data distribution and \pdf projections are show as functions of \PBz invariant mass, \cTL, \cTK, and \PHI.}
  \label{fig:res_bin8}
\end{figure}

\clearpage

\chapter{Future perspective and conclusions}
\label{sec:FutConcl}

\section{Perspective at LHC Run II}
\label{sec:Future}
Since the discrepancy originally observed by the LHCb collaboration in the $P_5'$ angular parameter is far to be resolved, either from the theoretical side, where a better understanding of the effects of the hadronic uncertainties is essential to have a reliable Standard Model prediction, or from the experimental side.

In particular regarding this latter aspect, in the last few years the analyses from many experiments added their contribution to the LHCb result, to validate it and, by combining the results, improve the precision of the experimental measurement.

Despite the large statistics collected during LHC Run I by CMS, LHCb and ATLAS experiments, the uncertainty of the results of these analyses are still statistically dominated.
This makes this angular analysis an hot topic for the near future.

Since 2015 LHC Run II is ongoing, and the machine is showing very good performance, delivering an instantaneous luminosity that is increasing year-by-year.
In particular for CMS, the integrated luminosity expected to be collected in the full Run II period, ranging from 2015 to 2018, is \SI{150}{\per\femto\barn}, which is a huge value if compared to the statistics collected in Run I, considering also that the $B^0$ cross section increases by a factor of about two, thanks to the higher centre-of-mass energy of the collisions: $\sqrt{s}=13\TeV$.

Performing the analysis on Run II data can give a large improvement to the experimental precision reached in the measurement of the angular parameters.
In this section I will describe some new issues that an analysis on CMS Run-II data should face, and some upgrades that could improve the results and make them more robust.

\subsection{Trigger developments}
\label{sec:TrigDeve}

The main negative aspect, at analysis level, of the increased LHC luminosity is the trigger selections.
Since the computing resources are limited, the event reconstruction can be run only on a limited amount of data.
Since it is pointless to collect data if they can never be reconstructed, the limited number of reconstructed events can be translated in a upper limit of the rate of events that pass the trigger system.
This limit, in the first three years of Run II, is set to \SI{1000}{\hertz}, and is calculated taking into account the machine down-times, when the detector is not collecting data but the reconstruction process continues.

When LHC increases the instantaneous luminosity of the collisions, the rate would increase as well, because the probability of having an proton-proton interaction with final state that fires the trigger is proportional to the luminosity.
It is an even worse situation when the average pileup is increased, because many HLT algorithms, and almost all the L1T ones, can sum up the contribution of final states originated in different proton-proton interactions.
The firing probability in this case increases more than linearly, as a function of the luminosity.
Due to the rate upper limit, when the delivered luminosity increases, we need to set tighter requirements in the trigger algorithms.

Focusing now on the trigger selection used by the \BtoKstmumu angular analysis, in the 2012 run there was a simple requirement on the presence of two muons with $p_T>3.5\GeV$ and $|\eta|<2.2$, and forming a common displaced vertex with some quality requirements, as described in details in Section~\ref{sec:onsel}.
Since a simple increase of the $p_T$ and $\eta$ threshold, to keep the rate stable, would have led to a large drop in the signal efficiency, many studies have been performed in the selection optimisation, to achieve a sufficient rate reduction without affecting largely the signal efficiency.
The main results of these studies are the introduction of the requirement of one hadronic track at HLT since 2015, and the use of an upper cut on the angular distance, $\Delta R_{\mu\mu}$, between the two muons at L1T, since 2017.
On the other hand, in 2015 and 2016, the L1T algorithm could not have cuts on multi-object quantities (like $\Delta R_{\mu\mu}$), and the only available option was to cut tighter on the muon $p_T$ and $\eta$ quantities.

A first test of the new HLT algorithm on 2016 dataset, after applying the same selection criteria optimised for the 2012 analysis plus the matching of the triggering hadronic track to at least one of the two hadrons in the candidate, led to an average efficiency in the range 30-70\% of the 2012 one.
The range is due to different detector conditions over the year, whose problems are unrelated with the HLT strategy used.

\subsection{Analysis upgrades}
\label{sec:AnalUpg}

On the analysis methods, there are many aspects that can be improved with respect to the version presented in this thesis.

The first upgrade will aim to extend the analysis to measure the full set of angular parameters present in the decay rate.
An increased number of events in the dataset and a better or equal signal-to-noise fraction will be crucial for this purpose.
Also a better handling of the physical boundary can help with improving the fit stability, needed to extend the measurement.

A second goal is to improve the fit performances, both in terms of time consumption, for which parallel GPU-based computing can be used, and in term of stability, for which a simpler efficiency parameterisation can be tested.

Furthermore, a great improvement to the signal-to-noise ratio can be achieved by using a selection based on multi-variate analysis, like a Neural Network. This techniques would allow to maximise the background rejection by a full exploitation of the information contained in the variables.


\section{Summary}
\label{sec:Summ}

In this thesis, I have presented an important result of the CMS Collaboration in the Flavour Physics sector.
The angular analysis of the \BtoKstmumu decay has been performed with the data collected by the CMS Experiment in the 2012 run of pp collisions at $\sqrt{s}=8\TeV$, corresponding to an integrated luminosity of \SI{20.5}{\per\femto\barn}.

After presenting a general status of the theoretical description of this analysis and describing the LHC machine and the CMS detector, the details of the analysis are reported.

Firstly, the selection criteria applied to the collected data, and the parameterisation of their efficiency and of the detector acceptance, evaluated on signal simulated samples, have been described.

A complex fitting algorithm has been set up, to extract in a stable and reliable way the $P_1$ and $P_5'$ parameters from the distributions of the \PKp\Pgpm\Pgmp\Pgmm invariant mass and of the three angular variables.
This fitting algorithm has been validated in many ways, by testing it on MC samples and on data control channels, \BtoKstJpsimumu and \BtoKstpsipmumu.

In order to make the results as robust as possible, many sources of possible systematic uncertainty has been studied.
The statistical uncertainties have been evaluated using an simplified form of a bi-dimensional Feldman-Cousins approach, to guarantee the correct coverage even when the result is close to an nonphysical region in the parameter phase-space.

Finally, the fit procedure has been applied on data, and the results extracted.
Currently, they are among the most precise measurements of these parameters, they are compatible with the results from the other experiment and they show no discrepancy to the Standard Model predictions.

\clearpage



%%%%%%%%%% acknowledgments
%% \begin{acknowledgments}

%% Thanks everybody!
%% \end{acknowledgments}

%%%%%%%%%% BIBLIOGRAPHY
%% \bibliography{auto_generated}   % will be created by the tdr script.
%% \clearpage

%%%%%%%%% APPENDICES
%% \appendix
%% \input{Appendix.tex}
%% \clearpage

\end{document}

