\chapter{Future perspective and conclusions}
\label{sec:FutConcl}

\section{Perspective at LHC Run II}
\label{sec:Future}
The puzzle originated by the discrepancy observed by the LHCb collaboration in the $P_5'$ angular parameter is far to be resolved, both from the theoretical side, where a better understanding of the effects of the hadronic uncertainties is essential to have a reliable Standard Model prediction, and from the experimental side.

In particular regarding this latter aspect, in the last few years the analyses from many experiments added their contribution to the LHCb result, to validate it and, by combining the results, improve the precision of the experimental measurement.

Despite the large statistics collected during LHC Run I by CMS, LHCb and ATLAS experiments, the uncertainty of the results of these analyses is still statistically dominated.
This makes this angular analysis an hot topic for the near future.

Since 2015 LHC Run II is ongoing, and the machine is showing very good performance, delivering an instantaneous luminosity that is increasing year-by-year.
In particular for CMS, the integrated luminosity expected to be collected in the full Run II period, ranging from 2015 to 2018, is \SI{150}{\per\femto\barn}, which is a huge value if compared to the statistics collected in Run I, considering also that the $B^0$ cross section increases by a factor of about two, thanks to the higher centre-of-mass energy of the collisions: $\sqrt{s}=13\TeV$.

Performing the analysis on Run II data can give a large improvement to the experimental precision reached in the measurement of the angular parameters.
In this section I will describe some new issues that an analysis on CMS Run-II data should face, and some upgrades that could improve the results and make them more robust.

\subsection{Trigger developments}
\label{sec:TrigDeve}

The main negative aspect, at analysis level, of the increased LHC luminosity is the trigger selections.
Due to the limited availability of computing resources, the event reconstruction can be run only on a certain amount of data.
Since it is pointless to collect data if they can never be reconstructed, the limited number of reconstructed events can be translated in an upper limit of the rate of events that pass the HLT system.
This limit, in the first three years of Run II, is about \SI{1000}{\hertz}, for the data stream dedicated to physics, and it is calculated taking into account the LHC down-times, when the detector is not collecting data but the reconstruction process continues.

In addition, the maximum frequency at which the silicon detectors can be read sets a limit in the L1T output rate to \SI{100}{\kilo\hertz}.

When LHC increases the instantaneous luminosity of the collisions, the rate would increase as well, because the probability of having a proton-proton interaction with final state that fires the trigger is proportional to the luminosity.
The situation is even worse when the average pileup is increased, because some HLT algorithms, and most of the L1T ones, can sum up the contribution of final states originated in different proton-proton interactions.
The firing probability in this case increases more than linearly, as a function of the luminosity.
Due to the rate upper limit, when the delivered luminosity increases, we need to set tighter requirements in the trigger algorithms.

Focusing now on the trigger selection used by the \BtoKstmumu angular analysis, in the 2012 run there was a simple requirement on the presence of two muons with $p_T>3.5\GeV$ and $|\eta|<2.2$, and forming a common displaced vertex with some quality requirements, as described in details in Section~\ref{sec:onsel}.
Since a simple increase of the $p_T$ and $\eta$ threshold, to keep the rate stable, would have led to a large drop in the signal efficiency, many studies have been performed in the selection optimisation, to achieve a sufficient rate reduction without affecting largely the signal efficiency.

\subsubsection{2015 and 2016 trigger}
\label{sec:2016}

The L1T seed used during 2015 and 2016 runs is not containing any specific improvement, but only tighter thresholds are applied on the two muons.
Two algorithms are contributing to the trigger:
\begin{itemize}
\item a seed with no cuts on the muon transverse momenta, and a cut on their pseudorapidity $|\eta_{L1\mu}|<1.4\,(1.6)$. The two thresholds were used for collisions with higher (1.4) and lower (1.6) values of instantaneous luminosity;
\item a seed with no cuts on the pseudorapidity, and a cut on muon transverse momenta. During collisions at higher (lower) instantaneous luminosity, the cut on the leading muon is $p_{T\,L1\mu}>12\,(11)\GeV$, and the cut on the second muon is $p_{T\,L1\mu}>5\,(4)\GeV$.
\end{itemize}

The HLT path used has a raised requirement on the transverse momenta of the two muons, with respect to the 2012 version, $p_{T\,L3\mu}>4\GeV$.
There are not cuts on the muon pseudorapidity, while the vertexing requirements are still present.
In addition, it constraints the requirement of an additional hadronic track, with $p_{T\,h}>0.8\GeV$ and forming a common vertex with the muon pair.
Thanks to the latter requirement, the trigger manages to deal with the higher luminosity values reached without needing to increase dramatically the cuts on $p_T$.

A preliminary and underestimated prevision on the number of signal events expected in 2016 dataset can be obtained by applying the same offline selection criteria used in Run~1 analysis to the new data sample, and adding the matching requirement between the hadronic track firing the trigger and one of the two tracks used to build the candidate at reconstructed-level.
The candidate-\PBz invariant mass distribution of the selected sample, obtained from the full 2016 dataset, is shown in Figure~\ref{fig:prosp-2016}, for the signal and \JPsi $q^2$ regions.
Note that the selection cuts used were optimised on the 2012 data and re-optimising them on the Run~2 data would improve the signal over background ratio.
In addition, the analysis upgrades described in Section~\ref{sec:AnalUpg} are expected to further improve it.

\begin{figure}[!hbt]
  \centering
  \begin{tabular}{cc}
    \includegraphics[width=0.45\textwidth]{Signal_2016.png} &
    \includegraphics[width=0.45\textwidth]{JPsi_2016.png}
  \end{tabular}
  \caption{Distribution of the candidate-\PBz invariant mass of the 2016 dataset, for the signal (left) and \JPsi (right) $q^2$ regions.
    The selection criteria applied to the dataset are described in Section~\ref{sec:2016}.
    The rough fits are performed with a \pdf obtained from the sum of a Gaussian and an exponential function.
  \label{fig:prosp-2016}}
\end{figure}


\subsubsection{2017 trigger}
\label{sec:2017}

The L1T firmware used during 2017 run allowed to exploit the correlations between pairs of objects, giving the possibility to cut on di-object quantities, like invariant mass and $\Delta R=\sqrt{\Delta\eta^2 + \Delta\phi^2}$.
The two L1T seeds used for the \BtoKstmumu channel has been developed to contain cuts on the distance between the two muons:
\begin{itemize}
\item a seed with no cut on the muon transverse momenta, used a stable cut on their pseudorapidity $|\eta_{L1\mu}|<1.5$ and a cut $\Delta R_{\mu\mu}<1.4$;
\item a seed with no cut on the pseudorapidity, and a cut on muon transverse momenta, $p_{T\,L1\mu}>4\GeV$ for both the leading and second muons, and a cut on the dimuon distance, $\Delta R_{\mu\mu}<1.2$.
\end{itemize}
In addition the quality requirement on the two muons has been increased, with respect to 2016 version, from an intermediate quality to the highest quality possible.
These new cuts on the dimuon distance and quality allows to keep looser thresholds on muon transverse momenta and pseudorapidity, even if the instantaneous luminosity has been increased with respect to 2016 run.

The HLT path used in 2017 is very similar to the 2016 version.
The only difference, introduced to keep the rate at affordable values, is that the cut on hadronic track transverse momentum is increased, $p_{T\,h}>1.2\GeV$, and a new requirement has been introduced on the significance of the hadronic track impact parameter on the transverse plane with respect to the beamspot, $d_{xy}/\sigma(d_{xy}) > 2$.
The latter cut is identical to the one applied at reconstructed level in the Run~1 analysis, as described in Section~\ref{sec:opt-selec}.

The same selections described in the previous section and applied to the 2016 dataset, are now applied to a partial 2017 dataset, containing the events corresponding to an integrated luminosity of \SI{28}{\per\femto\barn}.
The candidate-\PBz invariant mass distribution of the selected sample is shown in Figure~\ref{fig:prosp-2017}, for the signal and \JPsi $q^2$ regions.

\begin{figure}[!hbt]
  \centering
  \begin{tabular}{cc}
    \includegraphics[width=0.45\textwidth]{Signal_2017.png} &
    \includegraphics[width=0.45\textwidth]{JPsi_2017.png}
  \end{tabular}
  \caption{Distribution of the candidate-\PBz invariant mass of a partial 2017 dataset, corresponding to \SI{28}{\per\femto\barn}, for the signal (left) and \JPsi (right) $q^2$ regions.
    The selection criteria applied to the dataset are described in Section~\ref{sec:2016}.
    The rough fits are performed with a \pdf obtained from the sum of a Gaussian and an exponential function.
  \label{fig:prosp-2017}}
\end{figure}

%% A first test of the new HLT algorithm on 2016 dataset, after applying the same selection criteria optimised for the 2012 analysis plus the matching of the triggering hadronic track to at least one of the two hadrons in the candidate, led to an average efficiency in the range 30-70\% of the 2012 one.
%% The range is due to different detector conditions over the year, whose problems are unrelated with the HLT strategy used.

\subsection{Analysis upgrades}
\label{sec:AnalUpg}

On the analysis methods, there are many aspects that can be improved with respect to the version presented in this thesis.

The first upgrade will aim to extend the analysis to measure the full set of angular parameters present in the decay rate.
An increased number of events in the dataset and a better or equal signal-to-noise fraction will be crucial for this purpose.
Also a better handling of the physical boundary can help with improving the fit stability, needed to extend the measurement.

A second goal is to improve the fit performances, both in terms of time consumption, for which parallel GPU-based computing can be used, and in term of stability, for which a simpler efficiency parameterisation can be tested.

Furthermore, a great improvement to the signal-to-noise ratio can be achieved by using a selection based on multi-variate analysis, like a Neural Network. This techniques would allow to maximise the background rejection by a full exploitation of the information contained in the variables.


\section{Summary}
\label{sec:Summ}

In this thesis, I have presented an important result of the CMS Collaboration in the Flavour Physics sector.
The angular analysis of the \BtoKstmumu decay has been performed with the data collected by the CMS Experiment in the 2012 run of pp collisions at $\sqrt{s}=8\TeV$, corresponding to an integrated luminosity of \SI{20.5}{\per\femto\barn}.

After presenting a general status of the theoretical description of this analysis and describing the LHC machine and the CMS detector, the details of the analysis are reported.

Firstly, the selection criteria applied to the collected data, and the parameterisation of their efficiency and of the detector acceptance, evaluated on signal simulated samples, have been described.

A complex fitting algorithm has been set up, to extract in a stable and reliable way the $P_1$ and $P_5'$ parameters from the distributions of the \PKp\Pgpm\Pgmp\Pgmm invariant mass and of the three angular variables.
This fitting algorithm has been validated in many ways, by testing it on MC samples and on data control channels, \BtoKstJpsimumu and \BtoKstpsipmumu.

In order to make the results as robust as possible, many sources of possible systematic uncertainty has been studied.
The statistical uncertainties have been evaluated using a simplified form of a bi-dimensional Feldman-Cousins approach, to guarantee the correct coverage even when the result is close to an nonphysical region in the parameter phase-space.

Finally, the fit procedure has been applied on data, and the results extracted.
Currently, they are among the most precise measurements of these parameters, they are compatible with the results from the other experiments and they show no discrepancies to the Standard Model predictions.
