\section{Fit strategy} \label{sec:fit}

The angular parameters are extracted through an unbinned fit, using an extended maximum likelihood estimator.

In the following sections I will give a detailed description of the probability density function used in the fit, and of the methods used to estimate its parameters.

\subsection{Probability density function}
\label{sec:TotalPDF}

The probability density function (\pdf) used in the fit has the following expression:
\begin{equation} \label{eq:angALL}
  \begin{split}
    \mathrm{pdf}(m,\theta_\PK,\theta_\ell,\varphi) & = Y^{C}_{S} \biggl[ S^{C}(m)  \, S^a(\theta_\PK,\theta_\ell,\varphi) \, \epsilon^{C}(\theta_\PK,\theta_\ell,\varphi) \biggr. \\
      & \biggl. + \frac{f^{M}}{1-f^{M}}~S^{M}(m) \, S^a(-\theta_\PK,-\theta_\ell,\varphi) \, \epsilon^{M}(\theta_\PK,\theta_\ell,\varphi) \biggr] \\
    & + Y_{B}\,B^m(m) \, B^{\theta_\PK}(\theta_\PK) \, B^{\theta_\ell}(\theta_\ell) \, B^{\varphi}(\varphi), \\
  \end{split}
\end{equation}
%%%%%%%%%%%%%% FROM AN
where:
\begin{description}
    \item[$Y_{S}^C$] is the yields of signal events;
    \item[$Y_{B}$] is the yields of background events;
    \item[$f_i^M$] is the CP-mistag fraction (i.e. the number of mis-tagged signal events divided by the number of total signal events);
    \item[$S_i^R(m)$] describes the shape of the right-tagged signal events as a function of the $K\pi\mu\mu$ invariant mass;
    \item[$S_i^M(m)$] describes the shape of the mis-tagged signal events as a function of the $K\pi\mu\mu$ invariant mass;
    \item[$S_i^a(\cos\theta_l,\cos\theta_\mathrm{K},\phi)$] describes the shape of the signal events as a function of the three angular observables;
    \item[$B_i^m(m) \cdot B_i^{\cos\theta_\mathrm{K}}(\cos\theta_\mathrm{K}) \cdot B_i^{\cos\theta_l}(\cos\theta_l) \cdot B_i^{\phi}(\phi) $] are four functions describeing the shapes of the combinatorial background events, as functions of the $K\pi\mu\mu$ invariant mass and the three angular observables; 
    \item[$\epsilon_i^R(\cos\theta_l,\cos\theta_\mathrm{K},\phi)$] describes the efficiency for right-tagged signal events in the 3D-space of the angular observables;
    \item[$\epsilon_i^M(\cos\theta_l,\cos\theta_\mathrm{K},\phi)$] describes the efficiency for mis-tagged signal events in the 3D-space of the angular observables;
\end{description}

All the subscript $i$ runs over the $q^2$ bins listed in the Table~\ref{tab:q2 bins} if not otherwise specified.

%%%%%%%%%%%%% FROM PAPER
where its three terms correspond to the \pdfs for right-tagged signal, mis-tagged signal, and background events, respectively.

The parameters $Y^{C}_{S}$ and $Y_{B}$ are the yields of right-tagged signal events and background events, respectively, while the parameter $f^{M}$ is the fraction of signal events that are mis-tagged.

The functions $S^{C}(m)$ and $S^{M}(m)$ are the signal \PBz-mass \pdfs, for right-tagged and mis-tagged signal events, respectively.
Each of them is composed as the sum of two Gaussian functions, with a common mean for all four Gaussian functions.

%% In the fit, the mean, the four Gaussian function's width parameters, and the two fractions specifying the relative contribution of the two Gaussian functions in $S^{C}(m)$ and $S^{M}(m)$ are determined from simulation.
%% The function $S^a(\theta_\PK,\theta_\ell,\varphi)$ describes the signal in the three-dimensional (3D) space of the angular variables and corresponds to Eq.~(\ref{eq:PDF}).
The four \pdfs $B^m(m) \, B^{\theta_\PK}(\theta_\PK) \, B^{\theta_\ell}(\theta_\ell) \, B^{\varphi}(\varphi)$ describe the background in the space of the \PBz candidate invariant mass and the angular variables.
$B^m(m)$ is an exponential function, $B^{\theta_\PK}(\theta_\PK)$ and $B^{\theta_\ell}(\theta_\ell)$ are second- to fourth-order polynomials, depending on the $q^2$ bin, and $B^{\varphi}(\varphi)$ is a first-order polynomial.
The factorization assumption of the background pdf in Eq.~(\ref{eq:angALL}) is validated by dividing the range of an angular variable into two at its center point and comparing the distributions of events from the two halves in the other angular variables.
%%%%%%%%%%%%%%%%%%%%%%%%

The three-dimensional functions $\epsilon^{C}(\theta_\PK,\theta_\ell,\varphi)$ and $\epsilon^{M}(\theta_\PK,\theta_\ell,\varphi)$ are the efficiencies for right-tagged and mis-tagged signal events, respectively.
The construction of these functions is described in Section~\ref{sec:eff}.
%%%%%%%%%%%%%%%%%%%%%%%%


\subsubsection{Correctly and wrongly CP-tagged signal events}
\label{sec:fullform}

The signal and control channels are self-tagging decays.
This means that in principle one can distinguish whether the mother particle is a $B^0$ or $\overline{B}^0$ by simply measuring the charges of the daughter hadrons, but this in turn requires the capability of disentangling kaons and pions.
Unfortunately the CMS experiment does not possess such a capability for charged particles above $\mathcal{O}(1~\GeV)$.
To overcome this deficit an algorithm, based on the analysis of the invariant mass of the two hadrons has been used.
Both the $\pi$ and $K$ mass hypothesis is considered for each of the two hadrons of the decay: the hypothesis with an invariant mass for the two-hadrons closer to the nominal $\PKst$ one is retained.
The algorithm has an intrinsic percentage of failure which is referred to as mistag fraction, $f^M_i$, defined as the ratio of mistagged signal events divided by the total signal events.
The mistag fraction is determined from simulated events, comparing the results of the algorithm to the MC truth.
The mistag fraction is determined by counting the number of correctly and wrongly tagged events, where only truth-matched events are considered: results for each bin are shown in Table~\ref{tab:Mis tag fraction}.

\begin{table*}[!htb]
    \begin{center}
        \begin{small}
            \caption{Number of correctly and wrongly CP-tagged events determined with signal and con-
                trol channel simulation samples for each $q^2$ bin in Table~\ref{tab:q2 bins}.
                \label{tab:Mis tag fraction}}
            \begin{tabular}{|l|c|c|c|c|}
                \hline
                $q^2$ bin index  & Correctly tagged &  Wrongly tagged   & Mistag fraction ($f^M_i$)  &   Error  \\
                \hline
                $ 0$    &  $ 30518 $      &  $ 4312 $  & $ 0.124 $ & $ 0.002 $    \\
                $ 1 $   &  $ 64438$       &  $9542$    & $ 0.129 $ & $ 0.001 $\\
                $ 2$    &  $ 51149 $      &  $7892$    & $ 0.134 $ & $ 0.001 $ \\
                $3 $    &  $ 91065$       &  $13845$   & $ 0.132 $ & $ 0.001 $   \\
                $4  $   &   $ 472326$     &  $75157$   & $0.1373 $ & $ 0.0005$ \\
                $ 5 $   &  $119644 $      &  $18256$   & $0.132  $ & $ 0.001 $\\
                $ 6$    &  $30808$        &  $5013 $   & $0.140 $  & $ 0.002$\\
                $ 7 $   &  $69773$        &  $10623$   & $0.132 $  & $ 0.001$\\
                $ 8 $   &  $72769$        &  $11523$   & $0.137$   & $0.001$\\
                \hline
            \end{tabular}
        \end{small}
    \end{center}
\end{table*}

The mis-tag fraction is fixed in the final fit considering it is small.
The MC estimation of mis-tag fraction may have some differences with respect to the real data case.
The contribution of this effect to the systematic uncertainty, evaluated from the data, is discussed in Sect.~\ref{sec:sys-mistag}.

%% We consider this as one source of
%% uncertainties, which will be evaluated from the data after the
%% unblinding of the analysis, in Sect. \ref{sec:sys-mistag}.



\subsubsection{Background shape parameterisation}
\label{sec:backg}

After the transformation described in section~\ref{sec:transform} on
data, we can get the background shape from the data sidebands fitting,
which is described in Eq.~\ref{eq:PDF} as:
$$B_i^m \cdot B_i^{\cos\theta_\mathrm{K}}(\cos\theta_\mathrm{K}) \cdot
B_i^{\cos\theta_l}(\cos\theta_l) \cdot B_i^{\phi}(\phi) $$

The different components are described as follow: $B_i^m$ is described as a single exponential with translation term tied to the $S^R_i(m)$ mean, and $B_i^{\cos\theta_\mathrm{K}}(\cos\theta_\mathrm{K}) \cdot B_i^{\cos\theta_l}(\cos\theta_l) \cdot B_i^{\phi}(\phi)$ are different degree polynomial depending on $q^2$ bin (in Table~\ref{tab:q2 bins}).
These functional forms are chosen to be good enough to describe the
data and to allow the fitting to converge easily. The details are
listed in the Table~\ref{tab:background shape}. The detail of background 
of $q^2$ bin 0 are from Figure~\ref{fig:bin0-bkg-l} to Figure~\ref{fig:bin0-bkg-phi}.
Other $q^2$ bins are in App.~\ref{sec:background}.

The hypothesis that the angular description of the background is factorisable for each variable has been tested. The results of this test are presented of App.~\ref{sec:app-bkg.fact}.

\begin{table*}[!htb]
    \begin{center}
        \begin{small}
            \caption{Mathematical description of the background parameterization for 
                data as a function for each $q^2$ bin in Table~\ref{tab:q2 bins}.
                \label{tab:background shape}}
            \begin{tabular}{|l|c|c|c|c|}
                \hline
                $q^2$ bin index   & $B_i^m$         & $B_i^{\cos\theta_\mathrm{K}}(\cos\theta_\mathrm{K})$    & $B_i^{\cos\theta_l}(\cos\theta_l)$  &  $B_i^{\phi}(\phi)$   \\
                \hline
                $ 0$    &  $ 1 exp $      &  $ 3^{rd} degree $  & $ 2^{nd} degree $ & $ 1^{st} degree $    \\
                $ 1 $   &  $ 1 exp$       &  $ 4^{th} degree$    & $ 2^{nd} degree $ & $ 1^{st} degree $\\
                $ 2$    &  $ 1 exp $      &  $ 4^{th} degree$    & $ 3^{rd} degree $ & $ 1^{st} degree $ \\
                $3 $    &  $ 1 exp$       &  $ 2^{nd} degree$   & $ 4^{th} degree $ & $ 1^{st} degree $   \\
                $ 5 $   &  $1 exp $      &   $ 4^{th} degree$   & $ 2^{nd} degree  $ & $ 1^{st} degree $\\
                $ 7 $   &  $1 exp$        &  $ 2^{nd} degree$   & $3^{rd} degree $  & $ 1^{st} degree$\\
                $ 8 $   &  $1 exp$        &  $ 2^{nd} degree$   & $2^{nd} degree$   & $1^{st} degree$\\
                \hline
            \end{tabular}
        \end{small}
    \end{center}
\end{table*}

\begin{figure}[!hbt]
  \centering
  \includegraphics[width=0.7\textwidth]{Figures/background/bin0-l.pdf}
  \caption{The data sidebands result of $q^2$ bins in the Table~\ref{tab:q2 bins} bin 0. The figure
    show the projections on $cos\theta_l$.}
  \label{fig:bin0-bkg-l}
\end{figure}


\begin{figure}[!hbt]
  \centering
  \includegraphics[width=0.7\textwidth]{Figures/background/bin0-k.pdf}
  \caption{The data sidebands result of $q^2$ bins in the Table~\ref{tab:q2 bins} bin 0. The figure
    show the projections on $cos\theta_\mathrm{K}$.}
  \label{fig:bin0-bkg-k}
\end{figure}

\begin{figure}[!hbt]
  \centering
  \includegraphics[width=0.7\textwidth]{Figures/background/bin0-phi.pdf}
  \caption{The data sidebands result of $q^2$ bins in the Table~\ref{tab:q2 bins} bin 0. The figure
    show the projections on $\phi$.}
  \label{fig:bin0-bkg-phi}
\end{figure}

\subsection{The fitting sequence, components and strategy}
\label{sec:fitseq}

The decay rate, as described by Eq.~(\ref{eq:PDF-f2}), depends upon six parameters,
$F_\mathrm{L}$, $F_\mathrm{S}$, $A_\mathrm{S}$, $P_1$, $P_5'$, and $A^5_\mathrm{S}$.
In order to avoid fit convergence problems due to the limited number of
signal candidate events the angular parameters $F_\mathrm{L}$, $F_\mathrm{S}$,
and $A_\mathrm{S}$ are fixed to previous CMS measurements performed
on the same dataset with the same event selection criteria~\cite{CMS:2012}.
For each $q^2$ bin, the observables of interest are extracted from an unbinned
extended maximum-likelihood fit to four variables: the $\PKp\Pgpm\Pgmp\Pgmm$
invariant mass $m$ and the three angular variables ${\theta_\PK}$, ${\theta_l}$,
and $\phi$.

The PDF components whose parameters are determined with a MC simulaion are:
$f^{M}$, $S^{C}(m)$, $S^{M}(m)$, $\epsilon^{C}(\theta_\PK,\theta_l,\phi)$ and $\epsilon^{M}(\theta_\PK,\theta_l,\phi)$,
while all the remaining parameters are determined on data.

The fit is performed in two steps. The initial fit uses the data from the sidebands of the $\PBz$
mass to obtain the $B^m(m)$, $B^{\theta_\PK}(\theta_\PK)$, $B^{\theta_l}(\theta_l)$, and $B^{\phi}(\phi)$
distributions (the signal component is absent from this fit). The sideband regions are
$3\sigma_m < \abs{m-m_{\PBz}} < 5.5\sigma_m$, where $\sigma_m$ is the average mass resolution ($\approx$45\MeV),
obtained from fitting the MC simulation signal to a sum of two Gaussians with a common mean. The distributions
obtained in this step are then fixed for the second step, which is a fit to the data over the full mass range.
The free parameters in this fit are the angular parameters $P_1$, $P_5'$, and $A^5_\mathrm{S}$,
and the yields $Y^{C}_{S}$ and $Y_{B}$.

The expression describing the angular distribution of $\mathrm{B}^0\to{\mathrm{K}^*\mu\mu}$, Eq.~(\ref{eq:PDF-f2}) and also its more
general form in Ref.~\cite{Descotes-Genon:2013vna}, can become negative for certain values of the angular parameters.
In particular the PDF in Eq.~(\ref{eq:PDF}) is only guaranteed to be non-negative for a particular subset of the parameter
space $P_1$, $P_5'$, and $A^5_\mathrm{S}$, whose mathematical expression is non trivial. The presence of such a physical
region greatly complicates the numerical maximisation process of the likelihood by \textsc{minuit}~\cite{Minuit}
and especially the error determination by \textsc{minos}~\cite{Minuit},
in particular next to the boundary between physical and unphysical regions.
Therefore the second fit step is performed by discretizing the bidimensional space $P_1$ -- $P_5'$, and by maximising the
likelihood as a function of the nuisance parameters $Y^{C}_{S}$, $Y_{B}$, and $A^5_\mathrm{S}$ at fixed values of $P_1$,
$P_5'$. Finally the distribution of the likelihood values is fit with a bivariate Normal distribution whose position
of the maximum inside the physical region corresponds to the best estimate of the angular parameters $P_1$, $P_5'$.

When we perform the final fit we need to make sure that we are discretizing the subset of the physical region
$P_1$ -- $P_5'$ containing the absolute maximum of the likelihood.
To this extent we fit the data 200 times, each time with starting values of the parameters $P_1$ and
$P_5'$ chosen randomly according to a uniform distribution defined over their physical region.

In this note we require that all the fitts have the status ``GOOD'', with which are labelled the plots.
The ``GOOD'' label depends upon two conditions: the convergence is verified, and the positive definiteness
of the covariance matrix is confirmed.

The interference terms $A_\mathrm{S}$ and $A^5_\mathrm{S}$ must vanish if either of the two interfering components
vanish. From Ref.~\cite{Descotes-Genon:2013vna}, these constraints are implemented as
$\abs{A_\mathrm{S}} < \sqrt{12 F_\mathrm{S}(1-F_\mathrm{S})F_\mathrm{L}}R$ and as
$\abs{A^5_\mathrm{S}} < \sqrt{3 F_\mathrm{S} (1-F_\mathrm{S}) (1-F_\mathrm{L}) (1+P_1)}R$, where $R$ is a ratio related
to the S-wave and P-wave line shapes, estimated to be 0.89 near the $\cPKstz$ mass.
The constraint on $A_\mathrm{S}$ is naturally satisfied since the measurement of the parameters $F_\mathrm{S}$,
$F_\mathrm{L}$, and $A_\mathrm{S}$ are inherited from the previous CMS analysis~\cite{CMS:2012}.

To ensure correct coverage for the uncertainties of the angular parameters, the Feldman-Cousins method~\cite{FC}
is used with nuisance parameters.
Two main sets of pseudo-experimental samples are generated to compute the coverage for the two angular observables
$P_1$ and $P_5'$, respectively. The first (second) set, used to compute the coverage for $P_1$ ($P_5'$), is generated
by assigning values to the other parameters as obtained by profiling the likelihood on data at fixed $P_1$ ($P_5'$)
values. When fitting the pseudo-experimental samples the same fit procedure as in data is applied (more details can be
found in Sec.~\ref{sec:statUncert}).

The fit formalism and results are validated through fits to pseudo-experimental samples,
MC simulation samples, and control channels.
