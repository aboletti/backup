\chapter{Efficiency}\label{sec:eff}

The efficiency for signal and for control channels is defined as the ratio of number of events passing the selection and whose selected candidate is matched with the generated final state, over the total number of events generated.
It includes the effects of detector geometric acceptance, the trigger selection efficiency and the offline selection efficiency, and is entirely computed from MC simulation.
The efficiency is built as a three-dimensional function of the angular observables \TL, \TK, and \PHI, and computed independently for each bin of $q^2$, both for signal and for the control regions.
The use of such a function allows to account for any possible correlation among the variables introduced by the selection cuts.

In the CMS official MC samples, not all the generated events are reconstructed, to save computing resources. Before the simulation of the detector response, some basic cuts on the generated kinematic variables, $p_T$ and $\eta$, of the signal final state muon pair are applied to remove the majority of events for which the final state is not completely in the geometric detector acceptance. In this thesis I will refer to this cuts as GEN-filter.

Only the events passing the GEN-filter are then reconstructed and compose the MC samples. To correctly take this into account, the efficiency is split into two different terms: 

\begin{equation}\label{eq:eff}
    \epsilon^{R/M}(q^2,\theta_L,\theta_K,\phi)=\mathcal{A}(q^2,\theta_L,\theta_K,\phi)\times\epsilon_{reco}(q^2,\theta_L,\theta_K,\phi)= \frac{N_{gen}}{D_{gen}}\times\frac{N_{reco}}{D_{reco}}
\end{equation}

where $\mathcal{A}$ is called acceptance and is the fraction of generated events that pass the GEN-filter, and $\epsilon_{reco}$ is the selection efficiency, namely the fraction of events which pass the selection with respect to those which have been reconstructed.

More precisely, the acceptance terms are defined as:
\begin{itemize}
    \item[$D_{gen}$] is the number of generated events which pass a selection $\pt(\PBz)>8~\GeV$ and $|\eta(\PBz)|<2.2$, in a given bin of $q^2$;
    \item[$N_{gen}$] is the number of generated events with pass the above selection and, in addition, requires both muons to have $\pt(\mu_{GEN})>3.3\GeV$ and $|\eta(\mu_{GEN})|<2.3$;
\end{itemize}
This quantity is computed using only generator level quantities, as a function of generator-level angular observables. %%, and make use of the large \texttt{GEN} sample described in Sec.~\ref{sec:dataset}.

The selection efficiency instead, is defined as:
\begin{equation}\label{eq:effReco}
    \epsilon_{reco}(q^2,\theta_L,\theta_K,\phi)= \frac{N_{reco}}{D_{reco}}
\end{equation}
where
\begin{itemize}
    \item[$D_{reco}$] is the number of events in the MC samples that pass the GEN-filter and on which the reconstruction is run;
    \item[$N_{reco}$] is the number of reconstructed events passing all selection cuts defined in Sec.~\ref{sec:selection};
\end{itemize}

The $D_{reco}$ is computed as a function of the generator level observables, while $N_{reco}$ is computed as a function of the reconstructed quantities.
The rationale for this choice is that the effect of the detector resolution, passing from the generated observables to the reconstructed one, is taken into account directly into the efficiency, without the need to add an additional term in the fitting function.

Since we cannot assume that the efficiency function is the same for events where the correct flavour has been identified and events where it is not, it is computed separately for candidates with correct tag ($\epsilon^R$) and wrong tag ($\epsilon^M$).
The classification of an event as wrong or correct tagged is defined by comparing the MC truth with the result of the tagging algorithm described in Sec.~\ref{sec:selection}.

%%%%%%%%%%%%%%%%%%%%%%%%%55
%% An important point is the definition of the observable against which the efficiency should be parametrized. 
%% This choice is driven by the angular analysis, namely by the form of the probability density function (\pdf) which describes the angular shape of the decay $\frac{d^4\Gamma}{dq^2d\Omega}$ as well as by kinematical consideration on how the observables are defined and measured.

%% The complete form of the \pdf can be written as a function of the angles \TL, \TK, and \PHI via a combination of $\cos$ and $\sin$ functions.
%% However, the two $\theta$ angles are defined as the angle between two vector, $\mu^+$ and dimuon system, and $\PK^+$ and \PKst, respectively.
%% As such, they are defined only in the interval $[0,\pi)$, or, in other word, only the $\cos\theta_x$ is defined, via the inner product.
%% So, there is no loss of information by defining $\sin\theta_x=+\sqrt{1-\cos^2\theta_x}$, namely to choose the positive solution, and disregarding the negative one.
%% So the efficiency is parametrized as a function of \cTL, \cTK, and \PHI.

%% It is important to stress that the absolute value of the efficiency is not important for the \pdf fit, but only the shape as a function of the angular observables.

Since the \TK and \TL variables are defined in the range $[0,\pi]$, there is no loss in information in building the efficiency as a function of \TK, \TL and \PHI, or as a function of \cTK, \cTL and \PHI. Since this second choice leads to slightly more smooth functions, it has been chosen for the construction of the efficiency. 

\section{Parameterisation}\label{sec:eff_param}

%% The efficiency is defined as the product of two ratios as described in eq.~\ref{eq:eff} for each bin of $q^2$.
%% In the following, we will take as implicit the dependence on $q^2$ in order to simplify the formula: $\epsilon(q^2,\theta_L,\theta_K,\phi)=\epsilon(\theta_L,\theta_K,\phi)$.
In the first angular analysis based on this dataset, the efficiency was built as a two-dimensional function against the angular variables \TK and \TL.
The method used to parameterise it was composed by two steps.
Firstly, a two-dimensional function is constructed by performing the bin-by-bin ratios of the binned distributions of numerators and denominators.
Then, this binned efficiency is fitted with a polynomial function, which is used as parameterisation of the efficiency.
Some complex procedures were needed in order to have this fit converging and to grant that the final efficiency functions were positive in the whole range of definition.

Most of the problems in this technique were due to both the low MC event statistics in some $q^2$ bins, especially for the mis-tagged event efficiency, and the large number of free parameters in the two-dimensional polynomial function.
Extending this procedure to a three dimensional efficiency would imply the usage of three-dimensional binned distributions, with a global decrease of the bin statistics, and the usage of a three-dimensional polynomial function, with a larger number of free parameters to fit.
For this reason, the usage of this method was not considered for this analysis, but new parameterisation techniques were tested.

Two independent approaches have been tested: a two dimensional reduction of the binned method and a parameterisation based on Kernel Density Estimator (KDE) distributions.
In this section both of them are described and some example functions are showed.
In the next section a closure test developed to determine their accuracy is described and the better performances of the KDE-based method will be highlighted.

%% The first extend the efficiency description from two to three dimensions by considering two variables at a time, while integrating over the third, and then iterating on the variables choice.
\subsection{Two-dimensional binned method}\label{sec:eff_2Dprod}
The first method is based on binned efficiency functions.
Instead of building directly a three-dimensional binned efficiency, facing the low per-bin statistics, we use two-dimensional functions to compose the final efficiency.
To model the efficiency correlation between each couple of parameters, three functions are created by integrating out, recursively, one angular variable and building the efficiency as a function of the remaining two.
The final three dimensional efficiency is then defined as the product of the three two dimensional ones.
In formula:
\begin{equation}\label{eq:eff2d}
    \epsilon(\cTL,\cTK,\phi)=F\times\epsilon(\cTL,\cTK)\times\epsilon(\cTL,\phi)\times\epsilon(\cTK,\phi)
\end{equation}
Since the product of the three functions results not correctly normalised, it is rescaled by a factor $F$, in such a way that the global efficiency, averaged over all the angular variables, matches the simple ratio of the number of events in the numerator and denominator distributions.

This approach has the advantage to reduce the three-dimensional problem to three two-dimensional ones, easier to deal with and less prone to lack of statistics.
The correlation among the three angular variables are taken into account by the final product.
A disadvantage is that the efficiency is computed from binned distributions, where the bin width is determined by the statistics of events available in the MC samples used.
Any structure or behaviour inside a given bin is not resolved, and some kind of smoothing is needed.

The solution used to mitigate this binning problem is to use an interpolation between the bins of each two-dimensional efficiencies of Equation~\ref{eq:eff2d}.
Both a linear and parabolic interpolations have been tested.

An example of the two dimension binned efficiencies after a linear interpolation is applied, for the $q^2$ bin~1, are shown in Fig.~\ref{fig:eff2D}.

\begin{figure}[hbt]
    \includegraphics[width=1.\textwidth]{Plot/cEff2d3_2D_q2bin1.pdf}
    \caption{Distribution of the two-dimensional binned efficiency of equation~\ref{eq:eff2d} after the application of a linear interpolation between the bins, for $q^2$ bin~1: $\epsilon(\cTL,\cTK)$ (left), $\epsilon(\cTL,\phi)$ (centre), and $\epsilon(\cTK,\phi)$ (right).}
    \label{fig:eff2D}
\end{figure}

\subsection{Kernel Density Estimator method}\label{sec:eff_kde}
The second approach uses a non-parametric description of the efficiency based on a Kernel Density Estimator (KDE)~\cite{opac-b1089297,Cranmer:2000du}.

The general idea behind this method, in its simplest form for a uni-dimensional problem, is to start from an unbinned distribution of events as a function of a given variable $x$, and describe its true distribution {\texttt{pdf}$_{\mathrm{TRUE}}$} by building a kernel, namely a function $K(x)$ with unitary integral over the range of definition of $x$, on top of every event, and use the sum of all kernels, with proper normalisation, as a non-parametric description of {\texttt{pdf}$_{\mathrm{TRUE}}$}:
\begin{equation}\label{eq:KDE}
    \pdfKDEx=\frac{1}{N}\sum_{i=1}^{N} K(x-x_i)
\end{equation}
where $i$ is an index running over the set of events and $N$ is the total number of events.
The kernel $K(x)$ can be any non-negative function with unitary integral, and can be of different type; some common examples are uniform, triangular, or Gaussian kernels.
A remarkable property of the KDE technique is that the quality of the PDF description depends rather weakly on the kernel used.
The resulting \pdfKDEx in the limit $N\to\infty$ is a convolution of {\texttt{pdf}$_{\mathrm{TRUE}}$} with the kernel $K(x)$.
In this analysis, we decided to use a Gaussian kernel, because of its ability to produce a smooth \pdf even in regions where a low event statistics is available.

The KDE method can be extended to multi-dimensional problems, building a \pdf as a function of a set of $n$ variables $x_1,\ldots,x_n)$.
In this case, the kernel used should be $n$-dimensional as well.

An implementation of KDE method using a Gaussian kernel is available, for both the uni-dimensional and the multi-dimensional application, within the {\sc RooFit} package~\cite{RooFit}.

The advantage of the KDE over a parametric description is that no prior assumption on the actual {\texttt{pdf}$_{\mathrm{TRUE}}$} is needed, and the structure of the original \pdf is described as accurately as the statistics allow.
The last point is particularly relevant in the case under study, since in spite of the large statistics of events in the MC samples used to compute the efficiency, the phase space is very large, given the number of variables considered for the efficiency description.
A disadvantage is that an unbinned distribution of events is needed, so it is not possible to use the KDE method directly on the efficiency, which is binned by construction, but it needs to be used on the numerator and denominator distributions, and then the efficiency will be defined through the ratios of these \pdfs.

The efficiency at the observable boundaries is not expected to be null, especially for \PHI, given its periodic nature. 
In such cases, the kernel estimator would introduce a significant decreasing of the \pdf, since it would behave, close to the border, like a convolution between a step function and a Gaussian function.
In order to improve the behaviour of the modelling close the observable boundaries, data are mirrored across the boundaries.
It means that, for any Gaussian function added for a data point next to a boundary, the tail of this function that exceeds it is reflected inside the boundary.

Even if the KDE with Gaussian kernel is a non-parametric method, the values of the widths and correlation terms of the multi-dimensional kernel need to be determined.
For simplicity, the correlation terms are kept equal to zero; the usage of these terms could be useful when the distribution to model shows a strong correlation between two or more of its variables, but it is not the case for these numerator and denominator distributions.
The {\sc RooFit} class used to build the \pdfKDE uses a set of standard values for the width parameters, determined as a function of the variables range of definition.
An allowed tuning is the definition of an global scale factor, applied to the whole set of width parameters.
On one hand, a wide kernel is less sensitive to the limited statistic of the data sample and allows to produce smoother \pdfs.
On the other hand, a narrow kernel reproduces in a more accurate way the fine structures of the original distribution, especially when this distribution is steep.
Several width scale factors (1.0, 0.5, 0.3) have been tested and the choice of the value used for the efficiency is based on the goodness of the results of the closure test described in Section~\ref{sec:closure}.

%% A even more optimal solution would have been to use an adaptive approach, where
%% a non-constant width of the kernel is used, depending on the density of the
%% data in the around the kernel centroid. This method is known to behave better
%% than a fixed width one, but at the cost of a important overhead in the actual
%% modeling computing time ($\gtrsim10x$).
%% Given the size of the datasets used in this analysis, and since the closure
%% test of the fixed width approach are good~\ref{sec:closure}, the adaptive
%% approach has been disregarded.

The procedure used to obtain the efficiency function is to use the KDE method on each of the four distributions ($N_{gen}, D_{gen}, N_{reco}, D_{reco}$) used in Equation~\ref{eq:eff}, normalise each of the resulting \pdfs to the number of events in the original distributions, and then combine the four functions into final efficiency.
For technical reason, it is not possible, in ROOT, to save the output of a multidimensional KDE \pdf into an output file and the time consumption to create the efficiency function, starting from the MC event distributions, any time the fit procedure is run is too large.
For this reason, it was decided to save a binned distribution obtained from the sampling of the KDE \pdfs: numerator and denominator functions, obtained via the KDE algorithm, have been sampled into a three dimensional histogram, with $40\times40\times40$ bins, then the four histograms were combined with a bin-per-bin application of Equation~\ref{eq:eff} in an efficiency function, which has been finally saved in the output file.
%% The sampling to a 3D histogram is a workaround for this deficiency.
The granularity of the bins have been chosen as a compromise between the fine structure of the efficiency distribution and computing time needed to perform the \pdf sampling.

An example of the event distribution of the four terms of Equation~\ref{eq:eff}, $D_{gen}$, $N_{gen}$, $D_{reco}$, and $N_{reco}$, for right-tagged events in $q^2$ bin~1, as functions of each of the angular variables, together with the projections of the correctly-normalised \pdfs, as obtained by the KDE method with width scale factor equal to 0.5, are shown in Fig.~\ref{fig:NDKde1}.

%% \begin{figure}[hbt]
%%     \subfigure[$N_{gen}$]{\includegraphics[width=1.\textwidth]{Plot/can1D_h3genNum__CosThetaK_AbsCosThetaMu_AbsPhiKstMuMuPlane_Bin1.pdf}}

%%     \subfigure[$D_{gen}$]{\includegraphics[width=1.\textwidth]{Plot/can1D_h3genDen__CosThetaK_AbsCosThetaMu_AbsPhiKstMuMuPlane_Bin1.pdf}}

%%     \subfigure[$N_{reco}$]{\includegraphics[width=1.\textwidth]{Plot/can1D_h3recoNum__CosThetaK_AbsCosThetaMu_AbsPhiKstMuMuPlane_Bin1.pdf}}

%%     \subfigure[$D_{reco}$]{\includegraphics[width=1.\textwidth]{Plot/can1D_h3recoDen__CosThetaK_AbsCosThetaMu_AbsPhiKstMuMuPlane_Bin1.pdf}}

%%     \caption{Distribution of the four terms of eq.~\ref{eq:eff}, $D_{gen}$,
%%         $N_{gen}$, $D_{reco}$, and $N_{reco}$ (black dot), for $q^2$ bin 0
%%         togheter with the kernel estimator {\texttt pdf} (red line)}
%%     \label{fig:NDKde0}
%% \end{figure}

\begin{figure}[hbt]
    \subfigure[$N_{gen}$]{\includegraphics[width=1.\textwidth]{Plot/can1D_h3genNum__CosThetaK_AbsCosThetaMu_AbsPhiKstMuMuPlane_Bin2.pdf}}
    \subfigure[$D_{gen}$]{\includegraphics[width=1.\textwidth]{Plot/can1D_h3genDen__CosThetaK_AbsCosThetaMu_AbsPhiKstMuMuPlane_Bin2.pdf}}
    \subfigure[$N_{reco}$]{\includegraphics[width=1.\textwidth]{Plot/can1D_h3recoNum__CosThetaK_AbsCosThetaMu_AbsPhiKstMuMuPlane_Bin2.pdf}}
    \subfigure[$D_{reco}$]{\includegraphics[width=1.\textwidth]{Plot/can1D_h3recoDen__CosThetaK_AbsCosThetaMu_AbsPhiKstMuMuPlane_Bin2.pdf}}
    \caption{Event distributions of the four efficiency terms of Equation~\ref{eq:eff}, $D_{gen}$, $N_{gen}$, $D_{reco}$, and $N_{reco}$ (black dots), for right-tagged events in $q^2$ bin~1, together with the projections of the correctly-normalised \pdfs (red line), as obtained by the KDE method with width scale factor equal to $0.5$.}
    \label{fig:NDKde1}
\end{figure}

%% \begin{figure}[hbt]
%%     \subfigure[$N_{gen}$]{\includegraphics[width=1.\textwidth]{Plot/can1D_h3genNum__CosThetaK_AbsCosThetaMu_AbsPhiKstMuMuPlane_Bin3.pdf}}

%%     \subfigure[$D_{gen}$]{\includegraphics[width=1.\textwidth]{Plot/can1D_h3genDen__CosThetaK_AbsCosThetaMu_AbsPhiKstMuMuPlane_Bin3.pdf}}

%%     \subfigure[$N_{reco}$]{\includegraphics[width=1.\textwidth]{Plot/can1D_h3recoNum__CosThetaK_AbsCosThetaMu_AbsPhiKstMuMuPlane_Bin3.pdf}}

%%     \subfigure[$D_{reco}$]{\includegraphics[width=1.\textwidth]{Plot/can1D_h3recoDen__CosThetaK_AbsCosThetaMu_AbsPhiKstMuMuPlane_Bin3.pdf}}

%%     \caption{Distribution of the four terms of eq.~\ref{eq:eff}, $D_{gen}$,
%%         $N_{gen}$, $D_{reco}$, and $N_{reco}$ (black dot), for $q^2$ bin 2
%%         togheter with the kernel estimator {\texttt pdf} (red line)}
%%     \label{fig:NDKde2}
%% \end{figure}

%% \begin{figure}[hbt]
%%     \subfigure[$N_{gen}$]{\includegraphics[width=1.\textwidth]{Plot/can1D_h3genNum__CosThetaK_AbsCosThetaMu_AbsPhiKstMuMuPlane_Bin4.pdf}}

%%     \subfigure[$D_{gen}$]{\includegraphics[width=1.\textwidth]{Plot/can1D_h3genDen__CosThetaK_AbsCosThetaMu_AbsPhiKstMuMuPlane_Bin4.pdf}}

%%     \subfigure[$N_{reco}$]{\includegraphics[width=1.\textwidth]{Plot/can1D_h3recoNum__CosThetaK_AbsCosThetaMu_AbsPhiKstMuMuPlane_Bin4.pdf}}

%%     \subfigure[$D_{reco}$]{\includegraphics[width=1.\textwidth]{Plot/can1D_h3recoDen__CosThetaK_AbsCosThetaMu_AbsPhiKstMuMuPlane_Bin4.pdf}}

%%     \caption{Distribution of the four terms of eq.~\ref{eq:eff}, $D_{gen}$,
%%         $N_{gen}$, $D_{reco}$, and $N_{reco}$ (black dot), for $q^2$ bin 3
%%         togheter with the kernel estimator {\texttt pdf} (red line)}
%%     \label{fig:NDKde3}
%% \end{figure}

%% \begin{figure}[hbt]
%%     \subfigure[$N_{gen}$]{\includegraphics[width=1.\textwidth]{Plot/can1D_h3genNum__CosThetaK_AbsCosThetaMu_AbsPhiKstMuMuPlane_Bin5.pdf}}

%%     \subfigure[$D_{gen}$]{\includegraphics[width=1.\textwidth]{Plot/can1D_h3genDen__CosThetaK_AbsCosThetaMu_AbsPhiKstMuMuPlane_Bin5.pdf}}

%%     \subfigure[$N_{reco}$]{\includegraphics[width=1.\textwidth]{Plot/can1D_h3recoNum__CosThetaK_AbsCosThetaMu_AbsPhiKstMuMuPlane_Bin5.pdf}}

%%     \subfigure[$D_{reco}$]{\includegraphics[width=1.\textwidth]{Plot/can1D_h3recoDen__CosThetaK_AbsCosThetaMu_AbsPhiKstMuMuPlane_Bin5.pdf}}

%%     \caption{Distribution of the four terms of eq.~\ref{eq:eff}, $D_{gen}$,
%%         $N_{gen}$, $D_{reco}$, and $N_{reco}$ (black dot), for $q^2$ bin 4
%%         togheter with the kernel estimator {\texttt pdf} (red line)}
%%     \label{fig:NDKde4}
%% \end{figure}

%% \begin{figure}[hbt]
%%     \subfigure[$N_{gen}$]{\includegraphics[width=1.\textwidth]{Plot/can1D_h3genNum__CosThetaK_AbsCosThetaMu_AbsPhiKstMuMuPlane_Bin6.pdf}}

%%     \subfigure[$D_{gen}$]{\includegraphics[width=1.\textwidth]{Plot/can1D_h3genDen__CosThetaK_AbsCosThetaMu_AbsPhiKstMuMuPlane_Bin6.pdf}}

%%     \subfigure[$N_{reco}$]{\includegraphics[width=1.\textwidth]{Plot/can1D_h3recoNum__CosThetaK_AbsCosThetaMu_AbsPhiKstMuMuPlane_Bin6.pdf}}

%%     \subfigure[$D_{reco}$]{\includegraphics[width=1.\textwidth]{Plot/can1D_h3recoDen__CosThetaK_AbsCosThetaMu_AbsPhiKstMuMuPlane_Bin6.pdf}}

%%     \caption{Distribution of the four terms of eq.~\ref{eq:eff}, $D_{gen}$,
%%         $N_{gen}$, $D_{reco}$, and $N_{reco}$ (black dot), for $q^2$ bin 5
%%         togheter with the kernel estimator {\texttt pdf} (red line)}
%%     \label{fig:NDKde5}
%% \end{figure}

%% \begin{figure}[hbt]
%%     \subfigure[$N_{gen}$]{\includegraphics[width=1.\textwidth]{Plot/can1D_h3genNum__CosThetaK_AbsCosThetaMu_AbsPhiKstMuMuPlane_Bin7.pdf}}

%%     \subfigure[$D_{gen}$]{\includegraphics[width=1.\textwidth]{Plot/can1D_h3genDen__CosThetaK_AbsCosThetaMu_AbsPhiKstMuMuPlane_Bin7.pdf}}

%%     \subfigure[$N_{reco}$]{\includegraphics[width=1.\textwidth]{Plot/can1D_h3recoNum__CosThetaK_AbsCosThetaMu_AbsPhiKstMuMuPlane_Bin7.pdf}}

%%     \subfigure[$D_{reco}$]{\includegraphics[width=1.\textwidth]{Plot/can1D_h3recoDen__CosThetaK_AbsCosThetaMu_AbsPhiKstMuMuPlane_Bin7.pdf}}

%%     \caption{Distribution of the four terms of eq.~\ref{eq:eff}, $D_{gen}$,
%%         $N_{gen}$, $D_{reco}$, and $N_{reco}$ (black dot), for $q^2$ bin 6
%%         togheter with the kernel estimator {\texttt pdf} (red line)}
%%     \label{fig:NDKde6}
%% \end{figure}

%% \begin{figure}[hbt]
%%     \subfigure[$N_{gen}$]{\includegraphics[width=1.\textwidth]{Plot/can1D_h3genNum__CosThetaK_AbsCosThetaMu_AbsPhiKstMuMuPlane_Bin8.pdf}}

%%     \subfigure[$D_{gen}$]{\includegraphics[width=1.\textwidth]{Plot/can1D_h3genDen__CosThetaK_AbsCosThetaMu_AbsPhiKstMuMuPlane_Bin8.pdf}}

%%     \subfigure[$N_{reco}$]{\includegraphics[width=1.\textwidth]{Plot/can1D_h3recoNum__CosThetaK_AbsCosThetaMu_AbsPhiKstMuMuPlane_Bin8.pdf}}

%%     \subfigure[$D_{reco}$]{\includegraphics[width=1.\textwidth]{Plot/can1D_h3recoDen__CosThetaK_AbsCosThetaMu_AbsPhiKstMuMuPlane_Bin8.pdf}}

%%     \caption{Distribution of the four terms of eq.~\ref{eq:eff}, $D_{gen}$,
%%         $N_{gen}$, $D_{reco}$, and $N_{reco}$ (black dot), for $q^2$ bin 7
%%         togheter with the kernel estimator {\texttt pdf} (red line)}
%%     \label{fig:NDKde7}
%% \end{figure}

%% \begin{figure}[hbt]
%%     \subfigure[$N_{gen}$]{\includegraphics[width=1.\textwidth]{Plot/can1D_h3genNum__CosThetaK_AbsCosThetaMu_AbsPhiKstMuMuPlane_Bin9.pdf}}

%%     \subfigure[$D_{gen}$]{\includegraphics[width=1.\textwidth]{Plot/can1D_h3genDen__CosThetaK_AbsCosThetaMu_AbsPhiKstMuMuPlane_Bin9.pdf}}

%%     \subfigure[$N_{reco}$]{\includegraphics[width=1.\textwidth]{Plot/can1D_h3recoNum__CosThetaK_AbsCosThetaMu_AbsPhiKstMuMuPlane_Bin9.pdf}}

%%     \subfigure[$D_{reco}$]{\includegraphics[width=1.\textwidth]{Plot/can1D_h3recoDen__CosThetaK_AbsCosThetaMu_AbsPhiKstMuMuPlane_Bin9.pdf}}

%%     \caption{Distribution of the four terms of eq.~\ref{eq:eff}, $D_{gen}$,
%%         $N_{gen}$, $D_{reco}$, and $N_{reco}$ (black dot), for $q^2$ bin 8
%%         togheter with the kernel estimator {\texttt pdf} (red line)}
%%     \label{fig:NDKde8}
%% \end{figure}


An example of the projections of the efficiency, as obtained by the KDE method with width scale factor equal to 0.5, on each single and on each couple of angular variables, for right-tagged events in $q^2$ bin~1, are shown in Fig.~\ref{fig:effPro1}.

%% \begin{figure}[hbt]
%%     \includegraphics[width=1.\textwidth]{Plot/canProjection1.pdf}
%%     \caption{Projection of the efficiency on two (upper row) and one variable (lower row), integrating over the other variable for $q^2$ bin 0}
%%     \label{fig:effPro0}
%% \end{figure}

\begin{figure}[hbt]
    \includegraphics[width=1.\textwidth]{Plot/canProjection2.pdf}
    \caption{Projections of the efficiency, as obtained by the KDE method with width scale factor equal to 0.5, on two (upper row) and one angular variable (lower row), for right-tagged events in $q^2$ bin~1}
    \label{fig:effPro1}
\end{figure}

%% \begin{figure}[hbt]
%%     \includegraphics[width=1.\textwidth]{Plot/canProjection3.pdf}
%%     \caption{Projection of the efficiency on two (upper row) and one variable (lower row), integrating over the other variable for $q^2$ bin 2}
%%     \label{fig:effPro2}
%% \end{figure}

%% \begin{figure}[hbt]
%%     \includegraphics[width=1.\textwidth]{Plot/canProjection4.pdf}
%%     \caption{Projection of the efficiency on two (upper row) and one variable (lower row), integrating over the other variable for $q^2$ bin 3}
%%     \label{fig:effPro3}
%% \end{figure}

%% \begin{figure}[hbt]
%%     \includegraphics[width=1.\textwidth]{Plot/canProjection5.pdf}
%%     \caption{Projection of the efficiency on two (upper row) and one variable (lower row), integrating over the other variable for $q^2$ bin 4}
%%     \label{fig:effPro4}
%% \end{figure}

%% \begin{figure}[hbt]
%%     \includegraphics[width=1.\textwidth]{Plot/canProjection6.pdf}
%%     \caption{Projection of the efficiency on two (upper row) and one variable (lower row), integrating over the other variable for $q^2$ bin 5}
%%     \label{fig:effPro5}
%% \end{figure}

%% \begin{figure}[hbt]
%%     \includegraphics[width=1.\textwidth]{Plot/canProjection7.pdf}
%%     \caption{Projection of the efficiency on two (upper row) and one variable (lower row), integrating over the other variable for $q^2$ bin 6}
%%     \label{fig:effPro6}
%% \end{figure}

%% \begin{figure}[hbt]
%%     \includegraphics[width=1.\textwidth]{Plot/canProjection8.pdf}
%%     \caption{Projection of the efficiency on two (upper row) and one variable (lower row), integrating over the other variable for $q^2$ bin 7}
%%     \label{fig:effPro7}
%% \end{figure}

%% \begin{figure}[hbt]
%%     \includegraphics[width=1.\textwidth]{Plot/canProjection9.pdf}
%%     \caption{Projection of the efficiency on two (upper row) and one variable (lower row), integrating over the other variable for $q^2$ bin 8}
%%     \label{fig:effPro8}
%% \end{figure}

\clearpage

\section{Closure test}\label{sec:closure}

A validation of the efficiency parameterisation is performed via a closure test based on MC samples.
We compare the event distribution of the three angular observables, \cTK, \cTL, and \PHI, as reconstructed after the full simulation and after the application of the whole set of selections described in Section~\ref{sec:selection}, with the full set of generated event distribution, before the application of the GEN-filter, weighted, event-by-event, with the tested efficiency function.
In order to remove any statistical correlation in the closure test, half of each sample have been used to estimated the efficiency, and the closure has been performed on the other half.

The equality to test is the following:
\begin{equation}\label{eq:effClosure}
    \mathtt{pdf}(q^2,\cTK,\cTL,\phi)_{reco}^{R/M}=\mathtt{pdf}(q^2,\cTK,\cTL,\phi)_{gen}\otimes\epsilon^{R/M}(q^2,\cTK,\cTL,\phi)
\end{equation}

The closure test has been performed for any efficiency function produced: for each $q^2$ bin, separately for right (R) and mis-tag (M), and for signal and control samples.

An example of the closure test results of the first method efficiencies, as described in Section~\ref{eff_2Dprod}, is shown, for right-tagged events in $q^2$ bin~1, for the linear interpolation in Fig.~\ref{fig:clEff2Dlin}, and for the parabolic interpolation in Fig.~\ref{fig:clEff2Dpar}.
The result quality for the linearly-interpolated efficiency does not follow very well the original distribution where it is very steep and it introduces a \PHI dependence where none is present.
The quality improvement due to the second-order interpolation is not enough to have an accurate description of the shape in the \cTK projection and the modulation against \PHI is still present.
An improvement to this technique could come from an increase in the number of bins used to build the two-dimensional components of the efficiency, but the available statistics of the MC samples, especially for the mis-tagged events, prevent to go further with this.

\begin{figure}[hbt]
    \includegraphics[width=.66\textwidth,trim={0 280 0 0},clip]{Plot/Closure2D_lin.pdf}
    \includegraphics[width=.33\textwidth,trim={0 0 280 280},clip]{Plot/Closure2D_lin.pdf}
    \caption{Comparison of the reconstructed and selected MC event distributions (histogram) with generated event distributions (dots) weighted with the efficiency, parameterised as in Equation~\ref{eq:eff2d} with linear interpolation of the two-dimension efficiency functions. Both the distributions and the efficiency refer to right-tagged events in $q^2$ bin~1.}
    \label{fig:clEff2Dlin}
\end{figure}

\begin{figure}[hbt]
    \includegraphics[width=.66\textwidth,trim={0 280 0 0},clip]{Plot/Closure2D_par.pdf}
    \includegraphics[width=.33\textwidth,trim={0 0 280 280},clip]{Plot/Closure2D_par.pdf}
    \caption{Comparison of the reconstructed and selected MC event distributions (histogram) with generated event distributions (dots) weighted with the efficiency, parameterised as in Equation~\ref{eq:eff2d} with parabolic interpolation of the two-dimension efficiency functions. Both the distributions and the efficiency refer to right-tagged events in $q^2$ bin~1.}
    \label{fig:clEff2Dpar}
\end{figure}

An example of the closure test results for the KDE method is shown in Fig.~\ref{fig:clKDEwidthBin1} for right-tagged events in $q^2$ bin~0, and in Fig.~\ref{fig:clKDEwidthBin8} for right-tagged events in $q^2$ bin~7, comparing three different width scale factors: 1.0, 0.5, and 0.3.
It is possible to notice that using smaller widths improves the result quality where some steep features are present, in particular near the boundary of \cTK for $q^2$ bin~0, and of \PHI for $q^2$ bin~7.
On the other hand, a small width tends to create fine structures due to statistic fluctuations, in particular near the boundary of \PHI for $q^2$ bin~0.

\begin{figure}[hbt]
    \includegraphics[width=1.\textwidth]{Plot/closureKDEBin1width10.pdf}
    \includegraphics[width=1.\textwidth]{Plot/closureKDEBin1width05.pdf}
    \includegraphics[width=1.\textwidth]{Plot/closureKDEBin1width03.pdf}
    \caption{Comparison of the reconstructed and selected MC event distributions (histogram) with generated event distributions (dots) weighted with the efficiency, built through the KDE method, with width scale factor set to 1.0 (top), 0.5 (middle), and 0.3 (bottom). Both the distributions and the efficiency refer to right-tagged events in $q^2$ bin~0.}
    \label{fig:clKDEwidthBin1}
\end{figure}

\begin{figure}[hbt]
    \includegraphics[width=1.\textwidth]{Plot/closureKDEBin8width10.pdf}
    \includegraphics[width=1.\textwidth]{Plot/closureKDEBin8width05.pdf}
    \includegraphics[width=1.\textwidth]{Plot/closureKDEBin8width03.pdf}
    \caption{Comparison of the reconstructed and selected MC event distributions (histogram) with generated event distributions (dots) weighted with the efficiency, built through the KDE method, with width scale factor set to 1.0 (top), 0.5 (middle), and 0.3 (bottom). Both the distributions and the efficiency refer to right-tagged events in $q^2$ bin~7.}
    \label{fig:clKDEwidthBin8}
\end{figure}

From the overall results of the closure test, the optimal parameterisation for this analysis in the KDE method. The width scale factor is chosen to be 0.5 for the efficiency for right-tagged events, while the scale factor of 1.0 was found to be optimal for the efficiency of mis-tagged events, for which less statistic is available in the MC samples.
An example of the closure test results for mis-tagged events is shown in Fig.~\ref{fig:clKDEmistag}, for $q^2$ bin~0 and $q^2$ bin~7.

\begin{figure}[hbt]
    \includegraphics[width=1.\textwidth]{Plot/closureTest8_WT_sgn_bin1.pdf}
    \includegraphics[width=1.\textwidth]{Plot/closureTest8_WT_sgn_bin8.pdf}
    \caption{Comparison of the reconstructed and selected MC event distributions (histogram) with generated event distributions (dots) weighted with the efficiency, built through the KDE method. Both the distributions and the efficiency refer to mis-tagged events in $q^2$ bin~0 (top) and in $q^2$ bin~0 (bottom).}
    \label{fig:clKDEmistag}
\end{figure}

%%%%%%%%%%%%%%%%%%%

%% Figures~\ref{fig:clKDE1} through~\ref{fig:clKDE9} show the results of
%% closure test for $q^2$ bin 1 to 9, respectively, for correctly tagged
%% signal events.  The same distributions for wrongly tagged signal
%% events are shown in Fig~\ref{fig:clKDE1wt},~\ref{fig:clKDE9wt}.

%% \begin{figure}[hbt]
%%     \includegraphics[width=1.\textwidth]{Plot/closureTest8_RT_sgn_bin1.pdf}

%%     \caption{Closure test for the reconstructed angles using KDE, for $q^2$ bin 1}
%%     \label{fig:clKDE1}
%% \end{figure}

%% \begin{figure}[hbt]
%%     \includegraphics[width=1.\textwidth]{Plot/closureTest8_RT_sgn_bin2.pdf}

%%     \caption{Closure test for the reconstructed angles using KDE, for $q^2$ bin 2}
%%     \label{fig:clKDE2}
%% \end{figure}

%% \begin{figure}[hbt]
%%     \includegraphics[width=1.\textwidth]{Plot/closureTest8_RT_sgn_bin3.pdf}

%%     \caption{Closure test for the reconstructed angles using KDE, for $q^2$ bin 3}
%%     \label{fig:clKDE3}
%% \end{figure}

%% \begin{figure}[hbt]
%%     \includegraphics[width=1.\textwidth]{Plot/closureTest8_RT_sgn_bin4.pdf}

%%     \caption{Closure test for the reconstructed angles using KDE, for $q^2$ bin 4}
%%     \label{fig:clKDE4}
%% \end{figure}

%% \begin{figure}[hbt]
%%     \includegraphics[width=1.\textwidth]{Plot/closureTest8_RT_sgn_bin5.pdf}

%%     \caption{Closure test for the reconstructed angles using KDE, for $q^2$ bin 5}
%%     \label{fig:clKDE5}
%% \end{figure}

%% \begin{figure}[hbt]
%%     \includegraphics[width=1.\textwidth]{Plot/closureTest8_RT_sgn_bin6.pdf}

%%     \caption{Closure test for the reconstructed angles using KDE, for $q^2$ bin 6}
%%     \label{fig:clKDE6}
%% \end{figure}

%% \begin{figure}[hbt]
%%     \includegraphics[width=1.\textwidth]{Plot/closureTest8_RT_sgn_bin7.pdf}

%%     \caption{Closure test for the reconstructed angles using KDE, for $q^2$ bin 7}
%%     \label{fig:clKDE7}
%% \end{figure}

%% \begin{figure}[hbt]
%%     \includegraphics[width=1.\textwidth]{Plot/closureTest8_RT_sgn_bin8.pdf}

%%     \caption{Closure test for the reconstructed angles using KDE, for $q^2$ bin 8}
%%     \label{fig:clKDE8}
%% \end{figure}

%% \begin{figure}[hbt]
%%     \includegraphics[width=1.\textwidth]{Plot/closureTest8_RT_sgn_bin9.pdf}

%%     \caption{Closure test for the reconstructed angles using KDE, for $q^2$ bin 9}
%%     \label{fig:clKDE9}
%% \end{figure}

%% \clearpage

%% \begin{figure}[hbt]
%%     \includegraphics[width=1.\textwidth]{Plot/closureTest8_WT_sgn_bin1.pdf}

%%     \caption{Closure test using KDE, for $q^2$ bin 1, for wrongly tagged events}
%%     \label{fig:clKDE1wt}
%% \end{figure}

%% \begin{figure}[hbt]
%%     \includegraphics[width=1.\textwidth]{Plot/closureTest8_WT_sgn_bin2.pdf}

%%     \caption{Closure test using KDE, for $q^2$ bin 2, for wrongly tagged events}
%%     \label{fig:clKDE2wt}
%% \end{figure}

%% \begin{figure}[hbt]
%%     \includegraphics[width=1.\textwidth]{Plot/closureTest8_WT_sgn_bin3.pdf}

%%     \caption{Closure test using KDE, for $q^2$ bin 3, for wrongly tagged events}
%%     \label{fig:clKDE3wt}
%% \end{figure}

%% \begin{figure}[hbt]
%%     \includegraphics[width=1.\textwidth]{Plot/closureTest8_WT_sgn_bin4.pdf}

%%     \caption{Closure test using KDE, for $q^2$ bin 4, for wrongly tagged events}
%%     \label{fig:clKDE4wt}
%% \end{figure}

%% \begin{figure}[hbt]
%%     \includegraphics[width=1.\textwidth]{Plot/closureTest8_WT_sgn_bin5.pdf}

%%     \caption{Closure test using KDE, for $q^2$ bin 5, for wrongly tagged events}
%%     \label{fig:clKDE5wt}
%% \end{figure}

%% \begin{figure}[hbt]
%%     \includegraphics[width=1.\textwidth]{Plot/closureTest8_WT_sgn_bin6.pdf}

%%     \caption{Closure test using KDE, for $q^2$ bin 6, for wrongly tagged events}
%%     \label{fig:clKDE6wt}
%% \end{figure}


%% \begin{figure}[hbt]
%%     \includegraphics[width=1.\textwidth]{Plot/closureTest8_WT_sgn_bin7.pdf}

%%     \caption{Closure test using KDE, for $q^2$ bin 7, for wrongly tagged events}
%%     \label{fig:clKDE7wt}
%% \end{figure}

%% \begin{figure}[hbt]
%%     \includegraphics[width=1.\textwidth]{Plot/closureTest8_WT_sgn_bin8.pdf}

%%     \caption{Closure test using KDE, for $q^2$ bin 8, for wrongly tagged events}
%%     \label{fig:clKDE8wt}
%% \end{figure}

%% \begin{figure}[hbt]
%%     \includegraphics[width=1.\textwidth]{Plot/closureTest8_WT_sgn_bin9.pdf}

%%     \caption{Closure test using KDE, for $q^2$ bin 9, for wrongly tagged events}
%%     \label{fig:clKDE9wt}
%% \end{figure}

%% Finally, the closure test for two control samples, \BKsJ and \BKsPsip are shown
%% in Fig.~\ref{fig:clKDEJpsi} and~\ref{fig:clKDEpsip} for the $q^2$ bins where
%% such control samples are present.

%% \begin{figure}[hbt]
%%     \includegraphics[width=1.\textwidth]{Plot/closureTest8_RT_jp_bin5.pdf}

%%     \includegraphics[width=1.\textwidth]{Plot/closureTest8_WT_jp_bin5.pdf}

%%     \caption{Closure test for efficiency for the reconstructed angles
%%         using the KDE approach, for $q^2$ for bin 5, correct tag (top) and wrong tag (bottom), for \BKsJ control sample.}
%%     \label{fig:clKDEJpsi}
%% \end{figure}

%% \begin{figure}[hbt]
%%     \includegraphics[width=1.\textwidth]{Plot/closureTest8_RT_psip_bin6.pdf}

%%     \includegraphics[width=1.\textwidth]{Plot/closureTest8_RT_psip_bin7.pdf}

%%     \includegraphics[width=1.\textwidth]{Plot/closureTest8_WT_psip_bin6.pdf}

%%     \includegraphics[width=1.\textwidth]{Plot/closureTest8_WT_psip_bin7.pdf}

%%     \caption{Closure test for efficiency for the reconstructed angles
%%         using the KDE approach, for correctly tagged events (top two rows), for
%%         $q^2$ bin 6 (first row) and 7 (second row), for \BKsPsip control
%%         sample.
%%         The two bottom row has the same plots for wrongly tagged events.
%%         The $\psi'(2s)$ invariant mass falls inside bin 7, but the tail of its
%%         radiative decay populate also the previous one, bin 6.}
%%     \label{fig:clKDEpsip}
%% \end{figure}
