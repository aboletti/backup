\section{Validation with MC samples and data sidebands}
\label{sec:validation}

To apply the angular fitter to the data reliably, we have
made many checks and validation of the whole procedures. In addition
to check and validate the fitting procedures, some systematic
uncertainties are also derived from some of these studies.


%%%%%%%%%%from previous note section
\subsection{The fitting with GEN level signal MC samples}
\label{sec:fitval-gen}

In this section, we describe the studies made on GEN level signal MC samples.
We fit the GEN sample with Eq.~(\ref{eq:PDF-f2}) in each $q^2$ bin.
The results of the fits will serve as a basis for comparison with the fitts
to the RECO-level MC samples.

The fitting results of GEN level signal sample for all $q^2$ bins are
shown in Fig.~\ref{fig:gen-bin0} to Fig.~\ref{fig:gen-bin8}.

From these results, it can be shown that the fitting mechanisms
works well on GEN level. Also the fitting results here are used to do
the closure tests with RECO level fitting.Details of this will be
described in Section~\ref{sec:fitval-closure}.

\begin{figure}[!hbt]
  \centering
  \includegraphics[width=0.9\textwidth]{Figures/GENFit/bin0.pdf}
  \caption{The fitting results of the $q^2$ bin No.0 on the GEN
    sample. The plots show the projections of the fitting results on
    three different angular variables: $cos\theta_l$, $cos\theta_k$
    and $\phi$. The curve is the fitting and the points with error
    bars are the GEN  sample. } 
  \label{fig:gen-bin0}
\end{figure}

\begin{figure}[!hbt]
  \centering
  \includegraphics[width=0.9\textwidth]{Figures/GENFit/bin1.pdf}
  \caption{The fitting results of the $q^2$ bin No.1 on the GEN
    sample. The plots show the projections of the fitting results on
    three different angular variables: $cos\theta_l$, $cos\theta_k$
    and $\phi$. The curve is the fitting and the points with error
    bars are the GEN  sample. } 
  \label{fig:gen-bin1}
\end{figure}

\begin{figure}[!hbt]
  \centering
  \includegraphics[width=0.9\textwidth]{Figures/GENFit/bin2.pdf}
  \caption{The fitting results of the $q^2$ bin No.2 on the GEN
    sample. The plots show the projections of the fitting results on
    three different angular variables: $cos\theta_l$, $cos\theta_k$
    and $\phi$. The curve is the fitting and the points with error
    bars are the GEN  sample. } 
  \label{fig:gen-bin2}
\end{figure}

\begin{figure}[!hbt]
  \centering
  \includegraphics[width=0.9\textwidth]{Figures/GENFit/bin3.pdf}
  \caption{The fitting results of the $q^2$ bin No.3 on the GEN
    sample. The plots show the projections of the fitting results on
    three different angular variables: $cos\theta_l$, $cos\theta_k$
    and $\phi$. The curve is the fitting and the points with error
    bars are the GEN  sample. } 
  \label{fig:gen-bin3}
\end{figure}

\begin{figure}[!hbt]
  \centering
  \includegraphics[width=0.9\textwidth]{Figures/GENFit/bin5.pdf}
  \caption{The fitting results of the $q^2$ bin No.5 on the GEN
    sample. The plots show the projections of the fitting results on
    three different angular variables: $cos\theta_l$, $cos\theta_k$
    and $\phi$. The curve is the fitting and the points with error
    bars are the GEN  sample. } 
  \label{fig:gen-bin5}
\end{figure}

\begin{figure}[!hbt]
  \centering
  \includegraphics[width=0.9\textwidth]{Figures/GENFit/bin7.pdf}
  \caption{The fitting results of the $q^2$ bin No.7 on the GEN
    sample. The plots show the projections of the fitting results on
    three different angular variables: $cos\theta_l$, $cos\theta_k$
    and $\phi$. The curve is the fitting and the points with error
    bars are the GEN  sample. } 
  \label{fig:gen-bin7}
\end{figure}


\begin{figure}[!hbt]
  \centering
  \includegraphics[width=0.9\textwidth]{Figures/GENFit/bin8.pdf}
  \caption{The fitting results of the $q^2$ bin No.8 on the GEN
    sample. The plots show the projections of the fitting results on
    three different angular variables: $cos\theta_l$, $cos\theta_k$
    and $\phi$. The curve is the fitting and the points with error
    bars are the GEN  sample. } 
  \label{fig:gen-bin8}
\end{figure}

\clearpage

\subsection{The fitting with RECO level signal MC samples}
\label{sec:fitval-reco}

In this section we describe the studies made on RECO-level signal MC
samples. We fit the RECO sample with the same PDF as in data but without
the S-wave term since MC is lacking of this particular component and without
the background description.

For RECO level fitting, we check the procedures step by step. First,
we only fit the correctly tagged events with the corresponding PDF,
then do the similar standalone fitting on the mis-tagged events. After
these are validated, we fit the full RECO sample together, taking into
account both components.


\subsubsection{Fitting the correctly tagged events}
\label{sec:fitval-reco-rtag}

The fitting results of correctly tagged events for all $q^2$ bins are
shown in Fig.~\ref{fig:rtag-bin0} to Fig.~\ref{fig:rtag-bin8}.

\begin{figure}[!hbt]
  \centering
  \includegraphics[width=0.9\textwidth]{Figures/RECOFit/correct/AngleS_Canv0.pdf}
  \caption{The fitting results of the $q^2$ bin No.0 on the correctly
    tagged RECO  sample. The plots show the projections of the fitting results on
    three different angular variables: $cos\theta_l$, $cos\theta_k$
    and $\phi$. The curve is the fitting and the points with error
    bars are from RECO sample. } 
  \label{fig:rtag-bin0}
\end{figure}


\begin{figure}[!hbt]
  \centering
  \includegraphics[width=0.9\textwidth]{Figures/RECOFit/correct/AngleS_Canv1.pdf}
  \caption{The fitting results of the $q^2$ bin No.1 on the correctly
    tagged RECO  sample. The plots show the projections of the fitting results on
    three different angular variables: $cos\theta_l$, $cos\theta_k$
    and $\phi$. The curve is the fitting and the points with error
    bars are from RECO sample. } 
  \label{fig:rtag-bin1}
\end{figure}

\begin{figure}[!hbt]
  \centering
  \includegraphics[width=0.9\textwidth]{Figures/RECOFit/correct/AngleS_Canv2.pdf}
  \caption{The fitting results of the $q^2$ bin No.2 on the correctly
    tagged RECO  sample. The plots show the projections of the fitting results on
    three different angular variables: $cos\theta_l$, $cos\theta_k$
    and $\phi$. The curve is the fitting and the points with error
    bars are from RECO sample. } 
  \label{fig:rtag-bin2}
\end{figure}

\begin{figure}[!hbt]
  \centering
  \includegraphics[width=0.9\textwidth]{Figures/RECOFit/correct/AngleS_Canv3.pdf}
  \caption{The fitting results of the $q^2$ bin No.3 on the correctly
    tagged RECO  sample. The plots show the projections of the fitting results on
    three different angular variables: $cos\theta_l$, $cos\theta_k$
    and $\phi$. The curve is the fitting and the points with error
    bars are from RECO sample. } 
  \label{fig:rtag-bin3}
\end{figure}

\begin{figure}[!hbt]
  \centering
  \includegraphics[width=0.9\textwidth]{Figures/RECOFit/correct/AngleS_Canv5.pdf}
  \caption{The fitting results of the $q^2$ bin No.5 on the correctly
    tagged RECO  sample. The plots show the projections of the fitting results on
    three different angular variables: $cos\theta_l$, $cos\theta_k$
    and $\phi$. The curve is the fitting and the points with error
    bars are from RECO sample. } 
  \label{fig:rtag-bin5}
\end{figure}

\begin{figure}[!hbt]
  \centering
  \includegraphics[width=0.9\textwidth]{Figures/RECOFit/correct/AngleS_Canv7.pdf}
  \caption{The fitting results of the $q^2$ bin No.7 on the correctly
    tagged RECO  sample. The plots show the projections of the fitting results on
    three different angular variables: $cos\theta_l$, $cos\theta_k$
    and $\phi$. The curve is the fitting and the points with error
    bars are from RECO sample. } 
  \label{fig:rtag-bin7}
\end{figure}

\begin{figure}[!hbt]
  \centering
  \includegraphics[width=0.9\textwidth]{Figures/RECOFit/correct/AngleS_Canv8.pdf}
  \caption{The fitting results of the $q^2$ bin No.8 on the correctly
    tagged RECO  sample. The plots show the projections of the fitting results on
    three different angular variables: $cos\theta_l$, $cos\theta_k$
    and $\phi$. The curve is the fitting and the points with error
    bars are from RECO sample. } 
  \label{fig:rtag-bin8}
\end{figure}

\clearpage

\subsubsection{Fitting the wrongly tagged events}
\label{sec:fitval-reco-wtag}

The fitting results of wrongly tagged events for all $q^2$ bins are
shown in Fig.~\ref{fig:wtag-bin0} to Fig.~\ref{fig:wtag-bin8}.

\begin{figure}[!hbt]
  \centering
  \includegraphics[width=0.9\textwidth]{Figures/RECOFit/wrong/SignalRECO_Canv0.pdf}
  \caption{The fitting results of the $q^2$ bin No.0 on the wrongly
    tagged RECO sample. The plots show the projections of the fitting
    results on three different angular variables: $cos\theta_l$,
    $cos\theta_k$ and $\phi$. The curve is the fitting and the points
    with error bars are from RECO sample. }
  \label{fig:wtag-bin0}
\end{figure}


\begin{figure}[!hbt]
  \centering
  \includegraphics[width=0.9\textwidth]{Figures/RECOFit/wrong/SignalRECO_Canv1.pdf}
  \caption{The fitting results of the $q^2$ bin No.1 on the wrongly
    tagged RECO  sample. The plots show the projections of the fitting results on
    three different angular variables: $cos\theta_l$, $cos\theta_k$
    and $\phi$. The curve is the fitting and the points with error
    bars are from RECO sample. } 
  \label{fig:wtag-bin1}
\end{figure}

\begin{figure}[!hbt]
  \centering
  \includegraphics[width=0.9\textwidth]{Figures/RECOFit/wrong/SignalRECO_Canv2.pdf}
  \caption{The fitting results of the $q^2$ bin No.2 on the wrongly
    tagged RECO  sample. The plots show the projections of the fitting results on
    three different angular variables: $cos\theta_l$, $cos\theta_k$
    and $\phi$. The curve is the fitting and the points with error
    bars are from RECO sample. } 
  \label{fig:wtag-bin2}
\end{figure}

\begin{figure}[!hbt]
  \centering
  \includegraphics[width=0.9\textwidth]{Figures/RECOFit/wrong/SignalRECO_Canv3.pdf}
  \caption{The fitting results of the $q^2$ bin No.3 on the wrongly
    tagged RECO  sample. The plots show the projections of the fitting results on
    three different angular variables: $cos\theta_l$, $cos\theta_k$
    and $\phi$. The curve is the fitting and the points with error
    bars are from RECO sample. } 
  \label{fig:wtag-bin3}
\end{figure}

\begin{figure}[!hbt]
  \centering
  \includegraphics[width=0.9\textwidth]{Figures/RECOFit/wrong/SignalRECO_Canv5.pdf}
  \caption{The fitting results of the $q^2$ bin No.5 on the wrongly
    tagged RECO  sample. The plots show the projections of the fitting results on
    three different angular variables: $cos\theta_l$, $cos\theta_k$
    and $\phi$. The curve is the fitting and the points with error
    bars are from RECO sample. } 
  \label{fig:wtag-bin5}
\end{figure}

\begin{figure}[!hbt]
  \centering
  \includegraphics[width=0.9\textwidth]{Figures/RECOFit/wrong/SignalRECO_Canv7.pdf}
  \caption{The fitting results of the $q^2$ bin No.7 on the wrongly
    tagged RECO  sample. The plots show the projections of the fitting results on
    three different angular variables: $cos\theta_l$, $cos\theta_k$
    and $\phi$. The curve is the fitting and the points with error
    bars are from RECO sample. } 
  \label{fig:wtag-bin7}
\end{figure}

\begin{figure}[!hbt]
  \centering
  \includegraphics[width=0.9\textwidth]{Figures/RECOFit/wrong/SignalRECO_Canv8.pdf}
  \caption{The fitting results of the $q^2$ bin No.8 on the wrongly
    tagged RECO  sample. The plots show the projections of the fitting results on
    three different angular variables: $cos\theta_l$, $cos\theta_k$
    and $\phi$. The curve is the fitting and the points with error
    bars are from RECO sample. } 
  \label{fig:wtag-bin8}
\end{figure}

\clearpage

\subsubsection{Fitting the full RECO samples}
\label{sec:fitval-reco-full}

Based on the previous two steps of fitting on correctly and wrongly tagged events,
we now show the results of the fits to a a full RECO sample containing both components.

The fitting results of full RECO events for all $q^2$ bins are
shown in Fig.~\ref{fig:fullreco-bin0} to Fig.~\ref{fig:fullreco-bin8}.


\begin{figure}[!hbt]
  \centering
  \includegraphics[width=0.9\textwidth]{Figures/RECOFit/full-reco/SignalRECO_Canv0.pdf}
  \caption{The fitting results of the $q^2$ bin No.0 on the full RECO
    sample. The plots show the projections of the fitting results on
    three different angular variables: $cos\theta_l$, $cos\theta_k$
    and $\phi$. The curve is the fitting and the points with error
    bars are from RECO sample. }
  \label{fig:fullreco-bin0}
\end{figure}


\begin{figure}[!hbt]
  \centering
  \includegraphics[width=0.9\textwidth]{Figures/RECOFit/full-reco/SignalRECO_Canv1.pdf}
  \caption{The fitting results of the $q^2$ bin No.1 on the full RECO
    sample. The plots show the projections of the fitting results on
    three different angular variables: $cos\theta_l$, $cos\theta_k$
    and $\phi$. The curve is the fitting and the points with error
    bars are from RECO sample. }
  \label{fig:fullreco-bin1}
\end{figure}

\begin{figure}[!hbt]
  \centering
  \includegraphics[width=0.9\textwidth]{Figures/RECOFit/full-reco/SignalRECO_Canv2.pdf}
  \caption{The fitting results of the $q^2$ bin No.2 on the full RECO
    sample. The plots show the projections of the fitting results on
    three different angular variables: $cos\theta_l$, $cos\theta_k$
    and $\phi$. The curve is the fitting and the points with error
    bars are from RECO sample. }
  \label{fig:fullreco-bin2}
\end{figure}



\begin{figure}[!hbt]
  \centering
  \includegraphics[width=0.9\textwidth]{Figures/RECOFit/full-reco/SignalRECO_Canv3.pdf}
  \caption{The fitting results of the $q^2$ bin No.3 on the full RECO
    sample. The plots show the projections of the fitting results on
    three different angular variables: $cos\theta_l$, $cos\theta_k$
    and $\phi$. The curve is the fitting and the points with error
    bars are from RECO sample. }
  \label{fig:fullreco-bin3}
\end{figure}



\begin{figure}[!hbt]
  \centering
  \includegraphics[width=0.9\textwidth]{Figures/RECOFit/full-reco/SignalRECO_Canv5.pdf}
  \caption{The fitting results of the $q^2$ bin No.5 on the full RECO
    sample. The plots show the projections of the fitting results on
    three different angular variables: $cos\theta_l$, $cos\theta_k$
    and $\phi$. The curve is the fitting and the points with error
    bars are from RECO sample. }
  \label{fig:fullreco-bin5}
\end{figure}

\begin{figure}[!hbt]
  \centering
  \includegraphics[width=0.9\textwidth]{Figures/RECOFit/full-reco/SignalRECO_Canv7.pdf}
  \caption{The fitting results of the $q^2$ bin No.7 on the full RECO
    sample. The plots show the projections of the fitting results on
    three different angular variables: $cos\theta_l$, $cos\theta_k$
    and $\phi$. The curve is the fitting and the points with error
    bars are from RECO sample. }
  \label{fig:fullreco-bin7}
\end{figure}


\begin{figure}[!hbt]
  \centering
  \includegraphics[width=0.9\textwidth]{Figures/RECOFit/full-reco/SignalRECO_Canv8.pdf}
  \caption{The fitting results of the $q^2$ bin No.8 on the full RECO
    sample. The plots show the projections of the fitting results on
    three different angular variables: $cos\theta_l$, $cos\theta_k$
    and $\phi$. The curve is the fitting and the points with error
    bars are from RECO sample. }
  \label{fig:fullreco-bin8}
\end{figure}

\clearpage


%%%%%%%% back from validation note section
\subsection{Validation with GEN/RECO samples}
\label{sec:fitval-closure}

After verifying the fitting procedures and checking results with GEN
and RECO MC samples as described in Sections \ref{sec:fitval-gen} and
\ref{sec:fitval-reco}, we could compare the fitting values obtained at
different RECO fitting steps with the corresponding GEN fitting
results. This provides the closure tests and are important indication
of the reliability of our fitting results.


\subsubsection{Closure test with the correctly tagged events}
\label{sec:fitval-closure-rtag}


The closure test results of the correctly tagged RECO events for all
$q^2$ bins are shown in Fig.~\ref{fig:correct-closure-fl} to
Fig.~\ref{fig:correct-closure-p1}.

\begin{figure}[!hbt]
  \centering
  \includegraphics[width=0.7\textwidth]{Figures/RECOFit/correct/Fl.pdf}
  \caption{Closure test of the fitting results of $F_L$ for each $q^2$
    bins from the correctly tagged RECO sample with GEN sample.  The red points are
    from the GEN fitting and the black points are from the RECO
    fitting. The vertical shaded regions correspond to the $J/\psi$ and $\psi'$ resonances.}
  \label{fig:correct-closure-fl}
\end{figure}


\begin{figure}[!hbt]
  \centering
  \includegraphics[width=0.7\textwidth]{Figures/RECOFit/correct/P5p.pdf}
  \caption{Closure test of the fitting results of $P_5'$ for each $q^2$
    bins from the correctly tagged  RECO sample with GEN sample. The red points are
    from the GEN fitting and the black points are from the RECO
    fitting. The vertical shaded regions correspond to the $J/\psi$ and $\psi'$ resonances.}
  \label{fig:correct-closure-p5p}
\end{figure}

\begin{figure}[!hbt]
  \centering
  \includegraphics[width=0.7\textwidth]{Figures/RECOFit/correct/P1.pdf}
  \caption{Closure test of the fitting results of $P_1$ for each $q^2$
    bins from the correctly tagged  RECO sample with GEN sample.  The red points are
    from the GEN fitting and the black points are from the RECO
    fitting. The vertical shaded regions correspond to the $J/\psi$ and $\psi'$ resonances.}
  \label{fig:correct-closure-p1}
\end{figure}

\clearpage

\subsubsection{Closure test with the wrongly tagged events}
\label{sec:fitval-closure-wtag}

The closure test results of the wrongly tagged RECO events for all
$q^2$ bins are shown in Fig.~\ref{fig:wrong-closure-fl} to
Fig.~\ref{fig:wrong-closure-p1}.

\begin{figure}[!hbt]
  \centering
  \includegraphics[width=0.7\textwidth]{Figures/RECOFit/wrong/Fl.pdf}
  \caption{Closure test of the fitting results of $F_L$ for each $q^2$
    bins from the wrongly tagged RECO sample with GEN sample.  The red points are
    from the GEN fitting and the black points are from the RECO
    fitting. The vertical shaded regions correspond to the $J/\psi$ and $\psi'$ resonances. }
  \label{fig:wrong-closure-fl}
\end{figure}


\begin{figure}[!hbt]
  \centering
  \includegraphics[width=0.7\textwidth]{Figures/RECOFit/wrong/P5p.pdf}
  \caption{Closure test of the fitting results of $P_5'$ for each $q^2$
    bins from the wrongly tagged  RECO sample with GEN sample.  The red points are
    from the GEN fitting and the black points are from the RECO
    fitting. The vertical shaded regions correspond to the $J/\psi$ and $\psi'$ resonances. }
  \label{fig:wrong-closure-p5p}
\end{figure}

\begin{figure}[!hbt]
  \centering
  \includegraphics[width=0.7\textwidth]{Figures/RECOFit/wrong/P1.pdf}
  \caption{Closure test of the fitting results of $P_1$ for each $q^2$
    bins from the wrongly tagged  RECO sample with GEN sample.  The red points are
    from the GEN fitting and the black points are from the RECO
    fitting. The vertical shaded regions correspond to the $J/\psi$ and $\psi'$ resonances. }
  \label{fig:wrong-closure-p1}
\end{figure}

\clearpage

\subsubsection{Closure test  the full RECO samples}
\label{sec:fitval-closure-full}

The closure test results of the full  RECO events for all
$q^2$ bins are shown in Fig.~\ref{fig:fullreco-closure-fl} to
Fig.~\ref{fig:fullreco-closure-p1}.

\begin{figure}[!hbt]
  \centering
  \includegraphics[width=0.7\textwidth]{Figures/RECOFit/full-reco/Fl.pdf}
  \caption{Closure test of the fitting results of $F_L$ for each $q^2$
    bins from the full RECO sample with GEN sample.  The red points are
    from the GEN fitting and the black points are from the RECO
    fitting. The vertical shaded regions correspond to the $J/\psi$ and $\psi'$ resonances. }
  \label{fig:fullreco-closure-fl}
\end{figure}


\begin{figure}[!hbt]
  \centering
  \includegraphics[width=0.7\textwidth]{Figures/RECOFit/full-reco/P5p.pdf}
  \caption{Closure test of the fitting results of $P_5'$ for each $q^2$
    bins from the full RECO sample with GEN sample.  The red points are
    from the GEN fitting and the black points are from the RECO
    fitting. The vertical shaded regions correspond to the $J/\psi$ and $\psi'$ resonances. }
  \label{fig:fullreco-closure-p5p}
\end{figure}


\begin{figure}[!hbt]
  \centering
  \includegraphics[width=0.7\textwidth]{Figures/RECOFit/full-reco/P1.pdf}
  \caption{Closure test of the fitting results of $P_1$ for each $q^2$
    bins from the full RECO sample with GEN sample.  The red points are
    from the GEN fitting and the black points are from the RECO
    fitting. The vertical shaded regions correspond to the $J/\psi$ and $\psi'$ resonances. }
  \label{fig:fullreco-closure-p1}
\end{figure}

\clearpage
\subsection{Validation of independent MC samples}
\label{sec:fitval-half}

Since the efficiency derivation procedures use the same full
statistics of RECO level signal simulation sample as we fit
described in \ref{sec:fitval-reco}, we need to check whether the
efficiency works well with an independent sample. In order to check
this, we re-derive the efficiency with half of the RECO sample and
perform the fitting on the other half of the sample to check the
results.

In this section, we show the fitting results thus obtained for the
RECO sample and the closure test for the correctly tagged
events. Basically these results agree well with those reported in
Sections \ref{sec:fitval-reco-full} and \ref{sec:fitval-closure-rtag}
respectively. These agreements indicate that there is no bias caused
by the possible correlation in derivation of efficiency.

\subsubsection{Fitting the independents MC samples}
\label{sec:fitval-fitres-half}

The fitting results of correctly tagged events from half the RECO
samples for all $q^2$ bins are shown in Fig.~\ref{fig:halfrtag-bin0}
to Fig.~\ref{fig:halfrtag-bin8}.

\begin{figure}[!hbt]
  \centering
  \includegraphics[width=0.9\textwidth]{Figures/HalfFit/SignalRECO_Canv0.pdf}
  \caption{The fitting results of the $q^2$ bin No.0 on the half of the correctly
    tagged RECO  sample. The plots show the projections of the fitting results on
    three different angular variables: $cos\theta_l$, $cos\theta_k$
    and $\phi$. The curve is the fitting and the points with error
    bars are from RECO sample. }
  \label{fig:halfrtag-bin0}
\end{figure}


\begin{figure}[!hbt]
  \centering
  \includegraphics[width=0.9\textwidth]{Figures/HalfFit/SignalRECO_Canv1.pdf}
  \caption{The fitting results of the $q^2$ bin No.1 on the half of the correctly
    tagged RECO  sample. The plots show the projections of the fitting results on
    three different angular variables: $cos\theta_l$, $cos\theta_k$
    and $\phi$. The curve is the fitting and the points with error
    bars are from RECO sample. }
  \label{fig:halfrtag-bin1}
\end{figure}

\begin{figure}[!hbt]
  \centering
  \includegraphics[width=0.9\textwidth]{Figures/HalfFit/SignalRECO_Canv2.pdf}
  \caption{The fitting results of the $q^2$ bin No.2 on the half of the correctly
    tagged RECO  sample. The plots show the projections of the fitting results on
    three different angular variables: $cos\theta_l$, $cos\theta_k$
    and $\phi$. The curve is the fitting and the points with error
    bars are from RECO sample. }
  \label{fig:halfrtag-bin2}
\end{figure}

\begin{figure}[!hbt]
  \centering
  \includegraphics[width=0.9\textwidth]{Figures/HalfFit/SignalRECO_Canv3.pdf}
  \caption{The fitting results of the $q^2$ bin No.3 on the half of the correctly
    tagged RECO  sample. The plots show the projections of the fitting results on
    three different angular variables: $cos\theta_l$, $cos\theta_k$
    and $\phi$. The curve is the fitting and the points with error
    bars are from RECO sample. }
  \label{fig:halfrtag-bin3}
\end{figure}

\begin{figure}[!hbt]
  \centering
  \includegraphics[width=0.9\textwidth]{Figures/HalfFit/SignalRECO_Canv5.pdf}
  \caption{The fitting results of the $q^2$ bin No.5 on the half of the correctly
    tagged RECO  sample. The plots show the projections of the fitting results on
    three different angular variables: $cos\theta_l$, $cos\theta_k$
    and $\phi$. The curve is the fitting and the points with error
    bars are from RECO sample. }
  \label{fig:halfrtag-bin5}
\end{figure}

\begin{figure}[!hbt]
  \centering
  \includegraphics[width=0.9\textwidth]{Figures/HalfFit/SignalRECO_Canv7.pdf}
  \caption{The fitting results of the $q^2$ bin No.7 on the half of the correctly
    tagged RECO  sample. The plots show the projections of the fitting results on
    three different angular variables: $cos\theta_l$, $cos\theta_k$
    and $\phi$. The curve is the fitting and the points with error
    bars are from RECO sample. }
  \label{fig:halfrtag-bin7}
\end{figure}

\begin{figure}[!hbt]
  \centering
  \includegraphics[width=0.9\textwidth]{Figures/HalfFit/SignalRECO_Canv8.pdf}
  \caption{The fitting results of the $q^2$ bin No.8 on the half of the correctly
    tagged RECO  sample. The plots show the projections of the fitting results on
    three different angular variables: $cos\theta_l$, $cos\theta_k$
    and $\phi$. The curve is the fitting and the points with error
    bars are from RECO sample. }
  \label{fig:halfrtag-bin8}
\end{figure}

\clearpage

\subsubsection{Closure test with the correctly tagged events}
\label{sec:fitval-closure-half}


The closure test results of the correctly tagged events with
independent RECO sample for all $q^2$ bins are shown in
Fig.~\ref{fig:hc-closure-fl} to
Fig.~\ref{fig:hc-closure-p1}.

\begin{figure}[!hbt]
  \centering
  \includegraphics[width=0.7\textwidth]{Figures/HalfFit/Fl.pdf}
  \caption{Closure test of the fitting results of $F_L$ for each $q^2$
    bins from half the correctly tagged RECO sample with GEN sample.
    The red points are from the GEN fitting and the black points are from
    the RECO fitting. The vertical shaded regions correspond to the $J/\psi$ and $\psi'$ resonances. }
  \label{fig:hc-closure-fl}
\end{figure}


\begin{figure}[!hbt]
  \centering
  \includegraphics[width=0.7\textwidth]{Figures/HalfFit/P5p.pdf}
  \caption{Closure test of the fitting results of $P_5'$ for each
    $q^2$ bins from half the correctly tagged RECO sample with GEN
    sample.  The red points are from the GEN fitting and the black points
    are from the RECO fitting. The vertical shaded regions correspond to the $J/\psi$ and $\psi'$ resonances. }
  \label{fig:hc-closure-p5p}
\end{figure}

\begin{figure}[!hbt]
  \centering
  \includegraphics[width=0.7\textwidth]{Figures/HalfFit/P1.pdf}
  \caption{Closure test of the fitting results of $P_1$ for each $q^2$
    bins from half the correctly tagged RECO sample with GEN sample.
    The red points are from the GEN fitting and the black points are from
    the RECO fitting. The vertical shaded regions correspond to the $J/\psi$ and $\psi'$ resonances. }
  \label{fig:hc-closure-p1}
\end{figure}


\subsection{Validation with data-like statistics simulation}
\label{sec:Cocktail-MC}

The analysis technique is also validated with the simulation
containing both the signal and the background components. The goal is
to verify whether the analysis is able to measure the interesting
observables, in data-like conditions, and with the same number of
events as in data.

We produced many ``cocktail MC'' samples, each with the same
statistics as the data. Signal distributions are obtained from a subsample of signal
MC. The background distribution is generated with pseudo-experiments with p.d.f
parameters measured with data sidebands.
A total of 200 independent toy MC sample has been generated.
These samples have been used to validate the fitting procedure, first using only
the signal, then considering signal plus background.

\subsubsection{Validation with data-like statistics pure signal simulation}
\label{sec:Cocktail-MC-pure}


In this section, the simulation samples used are the pure signal
components from the cocktail MC. The parameter $F_L$ is free to float
in these fittings.

The App.~\ref{sec:pure-signal-result}  and App.~\ref{sec:pure-signal-value} summarize the $F_L$, $P_1$
and $P_5'$ average results of the 200 fittings. The error bars are from the
$RMS$ of the measured distributions divided by the square root of the
number of fittings ( only good fittings are considered).

\begin{figure}[!hbt]
  \centering
  \includegraphics[width=0.7\textwidth]{Figures/cocktail-Fit/com-reco-Fl.pdf}
  \caption{Closure test of the fitting results between 200 pure signal cocktail
 samples and truth-matched Reco-MC of $Fl$ for each $q^2$ bin. The blue points
are average value of the 200 fitting values, its error just the RMS. The red
points are the fitting results of truth-matched Reco-MC. }
  \label{fig:Fit bias of Fl}
\end{figure}


\begin{figure}[!hbt]
  \centering
  \includegraphics[width=0.7\textwidth]{Figures/cocktail-Fit/com-reco-P1.pdf}
  \caption{Closure test of the fitting results between 200 pure signal cocktail
 samples and truth-matched Reco-MC of $P_1$ for each $q^2$ bin. The blue points
are average value of the 200 fitting values, its error just the $RMS$. The red
points are the fitting results of truth-matched Reco-MC. }
  \label{fig:Fit bias of P1}
\end{figure}

\begin{figure}[!hbt]
  \centering
  \includegraphics[width=0.7\textwidth]{Figures/cocktail-Fit/com-reco-P5p.pdf}
  \caption{Closure test of the fitting results between 200 pure signal cocktail
 samples and truth-matched Reco-MC of $P_5'$ for each $q^2$ bin. The blue points
are average value of the 200 fitting values, its error just the $RMS$. The red
points are the fitting results of truth-matched Reco-MC. }
  \label{fig:Fit bias of P5'}
\end{figure}

\clearpage

Pull distributions from fitting on the data-like simulation samples
are reported in App.\ref{sec:cock-signal-pull} for each $q^2$ bin.
Ideally the pull distributions should follow
a N(0,1) normal distribution if the fitting procedures are unbiased
and evaluating the error correctly.

Since we use the average values of the fitting as the true values in
the construction of the pull, the mean of the pull distributions are
not interesting here and the focus is the width of the pull.  The unit
width of these distributions indicate the fitting is good and the
error estimation from the fitting is reasonable.


\subsubsection{Validation with data-like statistics with background}
\label{sec:Cocktail-MC-full}

In this section, the simulation samples used are mixtures of two
components: signal and background, as described above.

Figure ~\ref{fig:closure-full-cocktail-p1} and Figure
~\ref{fig:closure-full-cocktail-P5'} compared the $P_1$ and $P_5'$
average results of the 200 fittings with the fitting results of Reco MC.
The error bars of average value are from  the $RMS$ of
the measured distributions divided by the square root of the number of
fittings ( only good fittings are considered).

In some bins the measurements are biased, which is caused by fitting
programs and procedures. We account
the discrepancies between average values and the fitting result of Reco-MC
as a systematic uncertainties called fitting bias. The discrepancies
of every bin are shoen in the Figure ~\ref{fig:closure-full-cocktail-p1}
and Figure ~\ref{fig:closure-full-cocktail-P5'}. The detail of fitting bias are in Section~\ref{sec:fitbias-syst}.


\begin{figure}[!hbt]
  \centering
  \includegraphics[width=0.7\textwidth]{Figures/cocktail-Fit/avg-full-P1.pdf}
  \caption{Closure test of the fitting results between 200 full cocktail
 samples and truth-matched Reco-MC of $P_1$ for each $q^2$ bin. The blue points
are average value of the 200 fitting values, its error just the RMS. The red
points are the fitting results of truth-matched Reco-MC.}
  \label{fig:closure-full-cocktail-p1}
\end{figure}

\begin{figure}[!hbt]
  \centering
  \includegraphics[width=0.7\textwidth]{Figures/cocktail-Fit/avg-full-P5.pdf}
  \caption{Closure test of the fitting results between 200 full cocktail
 samples and truth-matched Reco-MC of $P_5'$ for each $q^2$ bin. The blue points
are average value of the 200 fitting values, its error just the RMS. The red
points are the fitting results of truth-matched Reco-MC. }
  \label{fig:closure-full-cocktail-P5'}
\end{figure}

\clearpage

\subsubsection{Comparison of data-like statistics fit with and without background}
\label{sec:Cocktail-MC-com}

Then we compared the average fitting results with full simulation described in
Section~\ref{sec:Cocktail-MC-full}. The comparison are in the Figure~\ref{fig:comparison-cocktail-P1}
and Figure~\ref{fig:comparison-cocktail-P5}.
When fitting with the background, we fixed the parameter $F_L$ to have a better convergence
as we will do for data fit (details in Section~\ref{sec:formula}), so we can
compare the value of $P_1$ and $P_5'$.

\begin{figure}[!hbt]
  \centering
  \includegraphics[width=0.7\textwidth]{Figures/cocktail-Fit/com-toy-P1.pdf}
  \caption{Comparison between the average results of 200 full cocktail
          samples and 200 pure signal cocktail samples of $P1$ for each $q^2$
    bins. The red points are average results of full cocktail MC and the blue are
     pure signal cocktail MC.}
  \label{fig:comparison-cocktail-P1}
\end{figure}


\begin{figure}[!hbt]
  \centering
  \includegraphics[width=0.7\textwidth]{Figures/cocktail-Fit/com-toy-P5p.pdf}
  \caption{Comparison between the average results of 200 full cocktail
          samples and 200 pure signal cocktail samples of $P_5'$ for each $q^2$
    bins. The red points are average results of full cocktail MC and the blue are
     pure signal cocktail MC.}
  \label{fig:comparison-cocktail-P5}
\end{figure}


\clearpage
\input{ControlChannel.tex}
\clearpage

\subsection{Error comparison between the blinded data and data-like simulation}
\label{sec:error-comparison}

This section describes the check on the statistical errors we perform
on data in a blinded way. We have fit the data in the signal region,
but "blinding" the central values for $P_1$ and $P_5'$. That means we
check the size of the statistical errors from the fitting only,
without looking at the
central values.  The errors are from the preliminary
fitting results obtained with "HESSE".  If using "MINOS" and more
detailed scanning of the initial values, probably there will be some
improvements, but we can only fully tune those options after the green
light for unblinding.

We compare the errors with what are expected from data-like statistics
full simulations which are described in
Section~\ref{sec:Cocktail-MC-full}, to verify the error estimation in
our analysis. The Figure~\ref{fig:error-com-0} to
Figure~\ref{fig:error-com-8} are the comparison of all $q^2$ bins'
results. In the figures, the red lines are the errors from the data,
and the histograms are the distributions from the simulation. The
first subplots are for $P_1$, the second ones are for $P_5'$, the
third are for the fraction of signal.


\begin{figure}[!hbt]
  \centering
  \includegraphics[width=1.0\textwidth]{Figures/errorbar/error-0.pdf}
  \caption{Validation between errors of 200 full cocktail samples and
    data channel for $q^2$ bin0. The red line is the error of data
    channel, the histogram is the distribution of simulations. And the
    first is $P_1$, the second is $P_5'$, the third is fraction of
    signal.}
  \label{fig:error-com-0}
\end{figure}

\begin{figure}[!hbt]
  \centering
  \includegraphics[width=1.0\textwidth]{Figures/errorbar/error-1.pdf}
  \caption{Validation between errors of 200 full cocktail samples and
    data channel for $q^2$ bin1. The red line is the error of data
    channel, the histogram is the distribution of simulations. And the
    first is $P_1$, the second is $P_5'$, the third is fraction of
    signal.}
  \label{fig:error-com-1}
\end{figure}

\begin{figure}[!hbt]
  \centering
  \includegraphics[width=1.0\textwidth]{Figures/errorbar/error-2.pdf}
  \caption{Validation between errors of 200 full cocktail samples and
    data channel for $q^2$ bin2. The red line is the error of data
    channel, the histogram is the distribution of simulations. And the
    first is $P_1$, the second is $P_5'$, the third is fraction of
    signal.}
  \label{fig:error-com-2}
\end{figure}

\begin{figure}[!hbt]
  \centering
  \includegraphics[width=1.0\textwidth]{Figures/errorbar/error-3.pdf}
  \caption{Validation between errors of 200 full cocktail samples and
    data channel for $q^2$ bin3. The red line is the error of data
    channel, the histogram is the distribution of simulations. And the
    first is $P_1$, the second is $P_5'$, the third is fraction of
    signal.}
  \label{fig:error-com-3}
\end{figure}

\begin{figure}[!hbt]
  \centering
  \includegraphics[width=1.0\textwidth]{Figures/errorbar/error-5.pdf}
  \caption{Validation between errors of 200 full cocktail samples and
    data channel for $q^2$ bin5. The red line is the error of data
    channel, the histogram is the distribution of simulations. And the
    first is $P_1$, the second is $P_5'$, the third is fraction of
    signal.}
  \label{fig:error-com-5}
\end{figure}

\begin{figure}[!hbt]
  \centering
  \includegraphics[width=1.0\textwidth]{Figures/errorbar/error-7.pdf}
  \caption{Validation between errors of 200 full cocktail samples and
    data channel for $q^2$ bin7. The red line is the error of data
    channel, the histogram is the distribution of simulations. And the
    first is $P_1$, the second is $P_5'$, the third is fraction of
    signal.}
  \label{fig:error-com-7}
\end{figure}

\begin{figure}[!hbt]
  \centering
  \includegraphics[width=1.0\textwidth]{Figures/errorbar/error-8.pdf}
  \caption{Validation between errors of 200 full cocktail samples and
    data channel for $q^2$ bin8. The red line is the error of data
    channel, the histogram is the distribution of simulations. And the
    first is $P_1$, the second is $P_5'$, the third is fraction of
    signal.}
  \label{fig:error-com-8}
\end{figure}
