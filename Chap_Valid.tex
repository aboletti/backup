\chapter{Validation of the fit algorithm}
\label{sec:validation}

Since the whole fit procedure is very complex, many validation checks are performed to verify the robustness of the final result.

In this section I will present the studies performed on the fit procedure, both on simulated MC events and on data control channels.
Since all the signal MC samples contain only events with resonant P-wave \PKpi state, it will be implicit in this section that the angular decay rate used when fitting MC events does not have the S-wave and interference terms, while the full decay rate is used when fitting on data control channels.

\section{Generator level fit to signal MC events}
\label{sec:fitval-gen}

The first validation step, presenting the minimum level of complexity, is performed by fitting the generator-level angular distributions of the signal MC sample, using the pure decay rate as \pdf: $S^a(\TK,\TL,\PHI)$.

The good status of the fit is considered a successful probe of the the minimisation process and of the correct description of the generated angular distribution through the decay rate.
The angular folding operations are applied to distributions and \pdfs, but this has no effect on the result, since the terms of the \pdf are either odd or even with respect to them.

Since the ``true'' values of the angular parameters used to simulate the events of the MC samples are not defined, the results of this fit will be used as a reference value for any comparison of the following results.
This fit is clean from any effect due to finite experimental resolution, so I expect that the fit results do not have any bias.

An example of the generator-level angular distributions and the projections of the fitted decay rate, for $q^2$ bin~3, are shown in Fig.~\ref{fig:gen-bin3}.

%% \begin{figure}[!hbt]
%%   \centering
%%   \includegraphics[width=0.9\textwidth]{Figures/GENFit/bin0.pdf}
%%   \caption{Fit results of the $q^2$ bin No.0 on the GEN
%%     sample. The plots show the projections of the fit results on
%%     three different angular variables: $cos\theta_l$, $cos\theta_k$
%%     and $\PHI$. The curve is the fit and the points with error
%%     bars are the GEN  sample. } 
%%   \label{fig:gen-bin0}
%% \end{figure}

%% \begin{figure}[!hbt]
%%   \centering
%%   \includegraphics[width=0.9\textwidth]{Figures/GENFit/bin1.pdf}
%%   \caption{Fit results of the $q^2$ bin No.1 on the GEN
%%     sample. The plots show the projections of the fit results on
%%     three different angular variables: $cos\theta_l$, $cos\theta_k$
%%     and $\PHI$. The curve is the fit and the points with error
%%     bars are the GEN  sample. } 
%%   \label{fig:gen-bin1}
%% \end{figure}

%% \begin{figure}[!hbt]
%%   \centering
%%   \includegraphics[width=0.9\textwidth]{Figures/GENFit/bin2.pdf}
%%   \caption{Fit results of the $q^2$ bin No.2 on the GEN
%%     sample. The plots show the projections of the fit results on
%%     three different angular variables: $cos\theta_l$, $cos\theta_k$
%%     and $\PHI$. The curve is the fit and the points with error
%%     bars are the GEN  sample. } 
%%   \label{fig:gen-bin2}
%% \end{figure}

\begin{figure}[!hbt]
  \centering
  \includegraphics[width=0.9\textwidth]{Figures/GENFit/bin3.pdf}
  \caption{Generator-level variable distributions of the MC sample before the GEN-filter, together with the projections of the fitted \pdf, as described in Section~\ref{sec:fitval-gen}, for $q^2$ bin~3.}
  \label{fig:gen-bin3}
\end{figure}

%% \begin{figure}[!hbt]
%%   \centering
%%   \includegraphics[width=0.9\textwidth]{Figures/GENFit/bin5.pdf}
%%   \caption{Fit results of the $q^2$ bin No.5 on the GEN
%%     sample. The plots show the projections of the fit results on
%%     three different angular variables: $cos\theta_l$, $cos\theta_k$
%%     and $\PHI$. The curve is the fit and the points with error
%%     bars are the GEN  sample. } 
%%   \label{fig:gen-bin5}
%% \end{figure}

%% \begin{figure}[!hbt]
%%   \centering
%%   \includegraphics[width=0.9\textwidth]{Figures/GENFit/bin7.pdf}
%%   \caption{Fit results of the $q^2$ bin No.7 on the GEN
%%     sample. The plots show the projections of the fit results on
%%     three different angular variables: $cos\theta_l$, $cos\theta_k$
%%     and $\PHI$. The curve is the fit and the points with error
%%     bars are the GEN  sample. } 
%%   \label{fig:gen-bin7}
%% \end{figure}


%% \begin{figure}[!hbt]
%%   \centering
%%   \includegraphics[width=0.9\textwidth]{Figures/GENFit/bin8.pdf}
%%   \caption{Fit results of the $q^2$ bin No.8 on the GEN
%%     sample. The plots show the projections of the fit results on
%%     three different angular variables: $cos\theta_l$, $cos\theta_k$
%%     and $\PHI$. The curve is the fit and the points with error
%%     bars are the GEN  sample. } 
%%   \label{fig:gen-bin8}
%% \end{figure}



\section{Reconstruction-level fit to signal MC samples}
\label{sec:fitval-reco}

The second validation step is performed by fitting the reconstruction-level angular distributions of the signal MC sample, after applying the criteria of candidate selection and CP-state tag described in Section~\ref{sec:selection}.
This fit is performed using only the angular terms of the signal components in the \pdf described in Equation~\ref{eq:angALL}.

The simulated sample contains both right-tagged and mis-tagged events, which can be distinguished by using the MC truth information.
As a first step, these two categories are fitted individually, using for each of them the corresponding angular component of the \pdf, which is the decay rate times the efficiency function.
The results of the fit show that the inclusion of the efficiency function in the fit procedure is correctly implemented.

An example of the reconstruction-level angular distributions of the signal MC sample and the projections of the angular component of the \pdf, for $q^2$ bin~3, is shown in Figure~\ref{fig:rtag-bin3} for right-tagged events and in Figure~\ref{fig:wtag-bin3} for mis-tagged events.

%% \begin{figure}[!hbt]
%%   \centering
%%   \includegraphics[width=0.9\textwidth]{Figures/RECOFit/correct/AngleS_Canv0.pdf}
%%   \caption{The fitting results of the $q^2$ bin No.0 on the correctly
%%     tagged RECO  sample. The plots show the projections of the fitting results on
%%     three different angular variables: $cos\theta_l$, $cos\theta_k$
%%     and $\PHI$. The curve is the fitting and the points with error
%%     bars are from RECO sample. } 
%%   \label{fig:rtag-bin0}
%% \end{figure}


%% \begin{figure}[!hbt]
%%   \centering
%%   \includegraphics[width=0.9\textwidth]{Figures/RECOFit/correct/AngleS_Canv1.pdf}
%%   \caption{The fitting results of the $q^2$ bin No.1 on the correctly
%%     tagged RECO  sample. The plots show the projections of the fitting results on
%%     three different angular variables: $cos\theta_l$, $cos\theta_k$
%%     and $\PHI$. The curve is the fitting and the points with error
%%     bars are from RECO sample. } 
%%   \label{fig:rtag-bin1}
%% \end{figure}

%% \begin{figure}[!hbt]
%%   \centering
%%   \includegraphics[width=0.9\textwidth]{Figures/RECOFit/correct/AngleS_Canv2.pdf}
%%   \caption{The fitting results of the $q^2$ bin No.2 on the correctly
%%     tagged RECO  sample. The plots show the projections of the fitting results on
%%     three different angular variables: $cos\theta_l$, $cos\theta_k$
%%     and $\PHI$. The curve is the fitting and the points with error
%%     bars are from RECO sample. } 
%%   \label{fig:rtag-bin2}
%% \end{figure}

\begin{figure}[!hbt]
  \centering
  \includegraphics[width=0.9\textwidth]{Figures/RECOFit/correct/AngleS_Canv3.pdf}
  \caption{Angular variable distributions of the right-tagged MC events after the selection criteria application, together with the projections of the fitted \pdf, as described in Section~\ref{sec:fitval-reco}, for $q^2$ bin~3.}
  \label{fig:rtag-bin3}
\end{figure}

%% \begin{figure}[!hbt]
%%   \centering
%%   \includegraphics[width=0.9\textwidth]{Figures/RECOFit/correct/AngleS_Canv5.pdf}
%%   \caption{The fitting results of the $q^2$ bin No.5 on the correctly
%%     tagged RECO  sample. The plots show the projections of the fitting results on
%%     three different angular variables: $cos\theta_l$, $cos\theta_k$
%%     and $\PHI$. The curve is the fitting and the points with error
%%     bars are from RECO sample. } 
%%   \label{fig:rtag-bin5}
%% \end{figure}

%% \begin{figure}[!hbt]
%%   \centering
%%   \includegraphics[width=0.9\textwidth]{Figures/RECOFit/correct/AngleS_Canv7.pdf}
%%   \caption{The fitting results of the $q^2$ bin No.7 on the correctly
%%     tagged RECO  sample. The plots show the projections of the fitting results on
%%     three different angular variables: $cos\theta_l$, $cos\theta_k$
%%     and $\PHI$. The curve is the fitting and the points with error
%%     bars are from RECO sample. } 
%%   \label{fig:rtag-bin7}
%% \end{figure}

%% \begin{figure}[!hbt]
%%   \centering
%%   \includegraphics[width=0.9\textwidth]{Figures/RECOFit/correct/AngleS_Canv8.pdf}
%%   \caption{The fitting results of the $q^2$ bin No.8 on the correctly
%%     tagged RECO  sample. The plots show the projections of the fitting results on
%%     three different angular variables: $cos\theta_l$, $cos\theta_k$
%%     and $\PHI$. The curve is the fitting and the points with error
%%     bars are from RECO sample. } 
%%   \label{fig:rtag-bin8}
%% \end{figure}

%% \begin{figure}[!hbt]
%%   \centering
%%   \includegraphics[width=0.9\textwidth]{Figures/RECOFit/wrong/SignalRECO_Canv0.pdf}
%%   \caption{The fitting results of the $q^2$ bin No.0 on the wrongly
%%     tagged RECO sample. The plots show the projections of the fitting
%%     results on three different angular variables: $cos\theta_l$,
%%     $cos\theta_k$ and $\PHI$. The curve is the fitting and the points
%%     with error bars are from RECO sample. }
%%   \label{fig:wtag-bin0}
%% \end{figure}


%% \begin{figure}[!hbt]
%%   \centering
%%   \includegraphics[width=0.9\textwidth]{Figures/RECOFit/wrong/SignalRECO_Canv1.pdf}
%%   \caption{The fitting results of the $q^2$ bin No.1 on the wrongly
%%     tagged RECO  sample. The plots show the projections of the fitting results on
%%     three different angular variables: $cos\theta_l$, $cos\theta_k$
%%     and $\PHI$. The curve is the fitting and the points with error
%%     bars are from RECO sample. } 
%%   \label{fig:wtag-bin1}
%% \end{figure}

%% \begin{figure}[!hbt]
%%   \centering
%%   \includegraphics[width=0.9\textwidth]{Figures/RECOFit/wrong/SignalRECO_Canv2.pdf}
%%   \caption{The fitting results of the $q^2$ bin No.2 on the wrongly
%%     tagged RECO  sample. The plots show the projections of the fitting results on
%%     three different angular variables: $cos\theta_l$, $cos\theta_k$
%%     and $\PHI$. The curve is the fitting and the points with error
%%     bars are from RECO sample. } 
%%   \label{fig:wtag-bin2}
%% \end{figure}

\begin{figure}[!hbt]
  \centering
  \includegraphics[width=0.9\textwidth]{Figures/RECOFit/wrong/SignalRECO_Canv3.pdf}
  \caption{Angular variable distributions of the mis-tagged MC events after the selection criteria application, together with the projections of the fitted \pdf, as described in Section~\ref{sec:fitval-reco}, for $q^2$ bin~3.}
  \label{fig:wtag-bin3}
\end{figure}

%% \begin{figure}[!hbt]
%%   \centering
%%   \includegraphics[width=0.9\textwidth]{Figures/RECOFit/wrong/SignalRECO_Canv5.pdf}
%%   \caption{The fitting results of the $q^2$ bin No.5 on the wrongly
%%     tagged RECO  sample. The plots show the projections of the fitting results on
%%     three different angular variables: $cos\theta_l$, $cos\theta_k$
%%     and $\PHI$. The curve is the fitting and the points with error
%%     bars are from RECO sample. } 
%%   \label{fig:wtag-bin5}
%% \end{figure}

%% \begin{figure}[!hbt]
%%   \centering
%%   \includegraphics[width=0.9\textwidth]{Figures/RECOFit/wrong/SignalRECO_Canv7.pdf}
%%   \caption{The fitting results of the $q^2$ bin No.7 on the wrongly
%%     tagged RECO  sample. The plots show the projections of the fitting results on
%%     three different angular variables: $cos\theta_l$, $cos\theta_k$
%%     and $\PHI$. The curve is the fitting and the points with error
%%     bars are from RECO sample. } 
%%   \label{fig:wtag-bin7}
%% \end{figure}

%% \begin{figure}[!hbt]
%%   \centering
%%   \includegraphics[width=0.9\textwidth]{Figures/RECOFit/wrong/SignalRECO_Canv8.pdf}
%%   \caption{The fitting results of the $q^2$ bin No.8 on the wrongly
%%     tagged RECO  sample. The plots show the projections of the fitting results on
%%     three different angular variables: $cos\theta_l$, $cos\theta_k$
%%     and $\PHI$. The curve is the fitting and the points with error
%%     bars are from RECO sample. } 
%%   \label{fig:wtag-bin8}
%% \end{figure}

As a second step, the full MC sample, containing both right-tagged and mis-tagged events, is used in the fit, and both the signal terms of the \pdf are used.
The mistag fraction parameter in the \pdf is fixed to the values reported in Section~\ref{sec:mistag}.
Since the mistag fraction values are computed on the same MC sample used for this fit, there is a statistical correlation between it and the fitted distributions; anyway its effects on the fit results are negligible because of the extremely small statistical uncertainty on this parameter.

The fit is still limited to the three angular variables, because all the parameters of the mass shapes and the relative abundances are fixed in the fit and there is no information to extract from the mass distributions.

An example of the reconstruction-level angular distributions of the signal MC sample and the projections of the angular component of the \pdf, for both right-tagged and mis-tagged events in $q^2$ bin~3, is shown in Figure~\ref{fig:fullreco-bin3}.

%% \begin{figure}[!hbt]
%%   \centering
%%   \includegraphics[width=0.9\textwidth]{Figures/RECOFit/full-reco/SignalRECO_Canv0.pdf}
%%   \caption{The fitting results of the $q^2$ bin No.0 on the full RECO
%%     sample. The plots show the projections of the fitting results on
%%     three different angular variables: $cos\theta_l$, $cos\theta_k$
%%     and $\PHI$. The curve is the fitting and the points with error
%%     bars are from RECO sample. }
%%   \label{fig:fullreco-bin0}
%% \end{figure}


%% \begin{figure}[!hbt]
%%   \centering
%%   \includegraphics[width=0.9\textwidth]{Figures/RECOFit/full-reco/SignalRECO_Canv1.pdf}
%%   \caption{The fitting results of the $q^2$ bin No.1 on the full RECO
%%     sample. The plots show the projections of the fitting results on
%%     three different angular variables: $cos\theta_l$, $cos\theta_k$
%%     and $\PHI$. The curve is the fitting and the points with error
%%     bars are from RECO sample. }
%%   \label{fig:fullreco-bin1}
%% \end{figure}

%% \begin{figure}[!hbt]
%%   \centering
%%   \includegraphics[width=0.9\textwidth]{Figures/RECOFit/full-reco/SignalRECO_Canv2.pdf}
%%   \caption{The fitting results of the $q^2$ bin No.2 on the full RECO
%%     sample. The plots show the projections of the fitting results on
%%     three different angular variables: $cos\theta_l$, $cos\theta_k$
%%     and $\PHI$. The curve is the fitting and the points with error
%%     bars are from RECO sample. }
%%   \label{fig:fullreco-bin2}
%% \end{figure}



\begin{figure}[!hbt]
  \centering
  \includegraphics[width=0.9\textwidth]{Figures/RECOFit/full-reco/SignalRECO_Canv3.pdf}
  \caption{Angular variable distributions of all the MC events after the selection criteria application, together with the projections of the fitted \pdf, as described in Section~\ref{sec:fitval-reco}, for $q^2$ bin~3.}
  \label{fig:fullreco-bin3}
\end{figure}



%% \begin{figure}[!hbt]
%%   \centering
%%   \includegraphics[width=0.9\textwidth]{Figures/RECOFit/full-reco/SignalRECO_Canv5.pdf}
%%   \caption{The fitting results of the $q^2$ bin No.5 on the full RECO
%%     sample. The plots show the projections of the fitting results on
%%     three different angular variables: $cos\theta_l$, $cos\theta_k$
%%     and $\PHI$. The curve is the fitting and the points with error
%%     bars are from RECO sample. }
%%   \label{fig:fullreco-bin5}
%% \end{figure}

%% \begin{figure}[!hbt]
%%   \centering
%%   \includegraphics[width=0.9\textwidth]{Figures/RECOFit/full-reco/SignalRECO_Canv7.pdf}
%%   \caption{The fitting results of the $q^2$ bin No.7 on the full RECO
%%     sample. The plots show the projections of the fitting results on
%%     three different angular variables: $cos\theta_l$, $cos\theta_k$
%%     and $\PHI$. The curve is the fitting and the points with error
%%     bars are from RECO sample. }
%%   \label{fig:fullreco-bin7}
%% \end{figure}


%% \begin{figure}[!hbt]
%%   \centering
%%   \includegraphics[width=0.9\textwidth]{Figures/RECOFit/full-reco/SignalRECO_Canv8.pdf}
%%   \caption{The fitting results of the $q^2$ bin No.8 on the full RECO
%%     sample. The plots show the projections of the fitting results on
%%     three different angular variables: $cos\theta_l$, $cos\theta_k$
%%     and $\PHI$. The curve is the fitting and the points with error
%%     bars are from RECO sample. }
%%   \label{fig:fullreco-bin8}
%% \end{figure}

The results of the fits performed in this section are compared with the results of the generator-level fits.

The results of right-tagged event fit and of the generator-level fit, are shown in Figure~\ref{fig:correct-closure-fl} for the $F_L$ parameter, in Figure~\ref{fig:correct-closure-p5p} for the $P_5'$ parameter, and in Figure~\ref{fig:correct-closure-p1} for the $P_1$ parameter.
The results of mis-tagged event fit and of the generator-level fit, are shown in Figure~\ref{fig:wrong-closure-fl} for the $F_L$ parameter, in Figure~\ref{fig:wrong-closure-p5p} for the $P_5'$ parameter, and in Figure~\ref{fig:wrong-closure-p1} for the $P_1$ parameter.
The results of right-tagged event fit and of the generator-level fit, are shown in Figure~\ref{fig:fullreco-closure-fl} for the $F_L$ parameter, in Figure~\ref{fig:fullreco-closure-p5p} for the $P_5'$ parameter, and in Figure~\ref{fig:fullreco-closure-p1} for the $P_1$ parameter.

\begin{figure}[!hbt]
  \centering
  \includegraphics[width=0.7\textwidth]{Figures/RECOFit/correct/Fl.pdf}
  \caption{Results for the $F_L$ parameter from the reconstruction-level fit to right-tagged event distributions (black) and from the generator-level fit (red), for each $q^2$ bin.
    The vertical shaded regions correspond to the $q^2$ bins dedicated to the $J/\psi$ and $\psi'$ control channels.}
  \label{fig:correct-closure-fl}
\end{figure}


\begin{figure}[!hbt]
  \centering
  \includegraphics[width=0.7\textwidth]{Figures/RECOFit/correct/P5p.pdf}
  \caption{Results for the $P_5'$ parameter from the reconstruction-level fit to right-tagged event distributions (black) and from the generator-level fit (red), for each $q^2$ bin.
    The vertical shaded regions correspond to the $q^2$ bins dedicated to the $J/\psi$ and $\psi'$ control channels.}
  \label{fig:correct-closure-p5p}
\end{figure}

\begin{figure}[!hbt]
  \centering
  \includegraphics[width=0.7\textwidth]{Figures/RECOFit/correct/P1.pdf}
  \caption{Results for the $P_1$ parameter from the reconstruction-level fit to right-tagged event distributions (black) and from the generator-level fit (red), for each $q^2$ bin.
    The vertical shaded regions correspond to the $q^2$ bins dedicated to the $J/\psi$ and $\psi'$ control channels.}
  \label{fig:correct-closure-p1}
\end{figure}

\begin{figure}[!hbt]
  \centering
  \includegraphics[width=0.7\textwidth]{Figures/RECOFit/wrong/Fl.pdf}
  \caption{Results for the $F_L$ parameter from the reconstruction-level fit to mis-tagged event distributions (black) and from the generator-level fit (red), for each $q^2$ bin.
    The vertical shaded regions correspond to the $q^2$ bins dedicated to the $J/\psi$ and $\psi'$ control channels.}
  \label{fig:wrong-closure-fl}
\end{figure}


\begin{figure}[!hbt]
  \centering
  \includegraphics[width=0.7\textwidth]{Figures/RECOFit/wrong/P5p.pdf}
  \caption{Results for the $P_5'$ parameter from the reconstruction-level fit to mis-tagged event distributions (black) and from the generator-level fit (red), for each $q^2$ bin.
    The vertical shaded regions correspond to the $q^2$ bins dedicated to the $J/\psi$ and $\psi'$ control channels.}
  \label{fig:wrong-closure-p5p}
\end{figure}

\begin{figure}[!hbt]
  \centering
  \includegraphics[width=0.7\textwidth]{Figures/RECOFit/wrong/P1.pdf}
  \caption{Results for the $P_1$ parameter from the reconstruction-level fit to mis-tagged event distributions (black) and from the generator-level fit (red), for each $q^2$ bin.
    The vertical shaded regions correspond to the $q^2$ bins dedicated to the $J/\psi$ and $\psi'$ control channels.}
  \label{fig:wrong-closure-p1}
\end{figure}

\begin{figure}[!hbt]
  \centering
  \includegraphics[width=0.7\textwidth]{Figures/RECOFit/full-reco/Fl.pdf}
  \caption{Results for the $F_L$ parameter from the reconstruction-level fit (black) and from the generator-level fit (red), for each $q^2$ bin.
    The vertical shaded regions correspond to the $q^2$ bins dedicated to the $J/\psi$ and $\psi'$ control channels.}
  \label{fig:fullreco-closure-fl}
\end{figure}


\begin{figure}[!hbt]
  \centering
  \includegraphics[width=0.7\textwidth]{Figures/RECOFit/full-reco/P5p.pdf}
  \caption{Results for the $P_5'$ parameter from the reconstruction-level fit (black) and from the generator-level fit (red), for each $q^2$ bin.
    The vertical shaded regions correspond to the $q^2$ bins dedicated to the $J/\psi$ and $\psi'$ control channels.}
  \label{fig:fullreco-closure-p5p}
\end{figure}


\begin{figure}[!hbt]
  \centering
  \includegraphics[width=0.7\textwidth]{Figures/RECOFit/full-reco/P1.pdf}
  \caption{Results for the $P_1$ parameter from the reconstruction-level fit (black) and from the generator-level fit (red), for each $q^2$ bin.
    The vertical shaded regions correspond to the $q^2$ bins dedicated to the $J/\psi$ and $\psi'$ control channels.}
  \label{fig:fullreco-closure-p1}
\end{figure}


%% \subsection{Validation of independent MC samples}
%% \label{sec:fitval-half}

%% Since the efficiency derivation procedures use the same full
%% statistics of RECO level signal simulation sample as we fit
%% described in \ref{sec:fitval-reco}, we need to check whether the
%% efficiency works well with an independent sample. In order to check
%% this, we re-derive the efficiency with half of the RECO sample and
%% perform the fitting on the other half of the sample to check the
%% results.

%% In this section, we show the fitting results thus obtained for the
%% RECO sample and the closure test for the correctly tagged
%% events. Basically these results agree well with those reported in
%% Sections \ref{sec:fitval-reco-full} and \ref{sec:fitval-closure-rtag}
%% respectively. These agreements indicate that there is no bias caused
%% by the possible correlation in derivation of efficiency.

%% \subsubsection{Fitting the independents MC samples}
%% \label{sec:fitval-fitres-half}

%% The fitting results of correctly tagged events from half the RECO
%% samples for all $q^2$ bins are shown in Fig.~\ref{fig:halfrtag-bin0}
%% to Fig.~\ref{fig:halfrtag-bin8}.

%% \begin{figure}[!hbt]
%%   \centering
%%   \includegraphics[width=0.9\textwidth]{Figures/HalfFit/SignalRECO_Canv0.pdf}
%%   \caption{The fitting results of the $q^2$ bin No.0 on the half of the correctly
%%     tagged RECO  sample. The plots show the projections of the fitting results on
%%     three different angular variables: $cos\theta_l$, $cos\theta_k$
%%     and $\PHI$. The curve is the fitting and the points with error
%%     bars are from RECO sample. }
%%   \label{fig:halfrtag-bin0}
%% \end{figure}


%% \begin{figure}[!hbt]
%%   \centering
%%   \includegraphics[width=0.9\textwidth]{Figures/HalfFit/SignalRECO_Canv1.pdf}
%%   \caption{The fitting results of the $q^2$ bin No.1 on the half of the correctly
%%     tagged RECO  sample. The plots show the projections of the fitting results on
%%     three different angular variables: $cos\theta_l$, $cos\theta_k$
%%     and $\PHI$. The curve is the fitting and the points with error
%%     bars are from RECO sample. }
%%   \label{fig:halfrtag-bin1}
%% \end{figure}

%% \begin{figure}[!hbt]
%%   \centering
%%   \includegraphics[width=0.9\textwidth]{Figures/HalfFit/SignalRECO_Canv2.pdf}
%%   \caption{The fitting results of the $q^2$ bin No.2 on the half of the correctly
%%     tagged RECO  sample. The plots show the projections of the fitting results on
%%     three different angular variables: $cos\theta_l$, $cos\theta_k$
%%     and $\PHI$. The curve is the fitting and the points with error
%%     bars are from RECO sample. }
%%   \label{fig:halfrtag-bin2}
%% \end{figure}

%% \begin{figure}[!hbt]
%%   \centering
%%   \includegraphics[width=0.9\textwidth]{Figures/HalfFit/SignalRECO_Canv3.pdf}
%%   \caption{The fitting results of the $q^2$ bin No.3 on the half of the correctly
%%     tagged RECO  sample. The plots show the projections of the fitting results on
%%     three different angular variables: $cos\theta_l$, $cos\theta_k$
%%     and $\PHI$. The curve is the fitting and the points with error
%%     bars are from RECO sample. }
%%   \label{fig:halfrtag-bin3}
%% \end{figure}

%% \begin{figure}[!hbt]
%%   \centering
%%   \includegraphics[width=0.9\textwidth]{Figures/HalfFit/SignalRECO_Canv5.pdf}
%%   \caption{The fitting results of the $q^2$ bin No.5 on the half of the correctly
%%     tagged RECO  sample. The plots show the projections of the fitting results on
%%     three different angular variables: $cos\theta_l$, $cos\theta_k$
%%     and $\PHI$. The curve is the fitting and the points with error
%%     bars are from RECO sample. }
%%   \label{fig:halfrtag-bin5}
%% \end{figure}

%% \begin{figure}[!hbt]
%%   \centering
%%   \includegraphics[width=0.9\textwidth]{Figures/HalfFit/SignalRECO_Canv7.pdf}
%%   \caption{The fitting results of the $q^2$ bin No.7 on the half of the correctly
%%     tagged RECO  sample. The plots show the projections of the fitting results on
%%     three different angular variables: $cos\theta_l$, $cos\theta_k$
%%     and $\PHI$. The curve is the fitting and the points with error
%%     bars are from RECO sample. }
%%   \label{fig:halfrtag-bin7}
%% \end{figure}

%% \begin{figure}[!hbt]
%%   \centering
%%   \includegraphics[width=0.9\textwidth]{Figures/HalfFit/SignalRECO_Canv8.pdf}
%%   \caption{The fitting results of the $q^2$ bin No.8 on the half of the correctly
%%     tagged RECO  sample. The plots show the projections of the fitting results on
%%     three different angular variables: $cos\theta_l$, $cos\theta_k$
%%     and $\PHI$. The curve is the fitting and the points with error
%%     bars are from RECO sample. }
%%   \label{fig:halfrtag-bin8}
%% \end{figure}

%% 

%% \subsubsection{Closure test with the correctly tagged events}
%% \label{sec:fitval-closure-half}


%% The closure test results of the correctly tagged events with
%% independent RECO sample for all $q^2$ bins are shown in
%% Fig.~\ref{fig:hc-closure-fl} to
%% Fig.~\ref{fig:hc-closure-p1}.

%% \begin{figure}[!hbt]
%%   \centering
%%   \includegraphics[width=0.7\textwidth]{Figures/HalfFit/Fl.pdf}
%%   \caption{Closure test of the fitting results of $F_L$ for each $q^2$
%%     bins from half the correctly tagged RECO sample with GEN sample.
%%     The red points are from the GEN fitting and the black points are from
%%     the RECO fitting. The vertical shaded regions correspond to the $J/\psi$ and $\psi'$ resonances. }
%%   \label{fig:hc-closure-fl}
%% \end{figure}


%% \begin{figure}[!hbt]
%%   \centering
%%   \includegraphics[width=0.7\textwidth]{Figures/HalfFit/P5p.pdf}
%%   \caption{Closure test of the fitting results of $P_5'$ for each
%%     $q^2$ bins from half the correctly tagged RECO sample with GEN
%%     sample.  The red points are from the GEN fitting and the black points
%%     are from the RECO fitting. The vertical shaded regions correspond to the $J/\psi$ and $\psi'$ resonances. }
%%   \label{fig:hc-closure-p5p}
%% \end{figure}

%% \begin{figure}[!hbt]
%%   \centering
%%   \includegraphics[width=0.7\textwidth]{Figures/HalfFit/P1.pdf}
%%   \caption{Closure test of the fitting results of $P_1$ for each $q^2$
%%     bins from half the correctly tagged RECO sample with GEN sample.
%%     The red points are from the GEN fitting and the black points are from
%%     the RECO fitting. The vertical shaded regions correspond to the $J/\psi$ and $\psi'$ resonances. }
%%   \label{fig:hc-closure-p1}
%% \end{figure}

\section{Reconstructed level fit to low statistics simulated samples}
\label{sec:datalike-MC}

The fit algorithm is also validated with simulated samples having the same statistics of the real data sample, and containing both the signal and the background components.
The goal is to verify whether the analysis is able to measure the interesting observables, in conditions as close as possible to the real data sample.

The data events are obtained by dividing the MC samples in sub-samples with a number of events exactly equal to the signal yield, as obtained from fitting on data the \PBz mass distribution.
The number of sub-samples that we can produced is limited by the statistics available in the MC sample, and for simplicity has been rounded down to 200 sub-samples, for each $q^2$ bin.

The background distributions are generated with pseudo-experiments, using the \pdf described in Section~\ref{sec:backg}, and parameter values measured with data sidebands.
The number of events generated for each sub-sample are equal to the background yield, as obtained from fitting on data the \PBz mass distribution.
To match the signal MC sub-samples, also for background a total of 200 sets of events have been generated, for each $q^2$ bin.

These samples have been used to validate the fitting procedure, first using only the signal, then merging each signal sub-sample with a background one.
I will refer to a merged sub-sample as ``cocktail'' MC sample.

\subsection{Data-like samples of signal MC events}
\label{sec:Cocktail-MC-pure}

As a first step, the angular distribution of the signal sub-samples is fitted, using only the signal components of the \pdf.
As for the fit to the full MC sample, including the mass in the fitted distributions would not change the result.

Each of the 200 MC sub-samples are fitted.
An example of the distributions of the resulting parameters, for $q^2$ bin~3, is shown in Figure~\ref{fig:closure-signal-cocktail-bin3}.

\begin{figure}[!hbt]
  \centering
  \includegraphics[width=1.0\textwidth]{Figures/cocktail-Fit/bin3-value-s.pdf}
  \caption{Distributions of the results of the fits to the 200 signal MC sub-samples, for $q^2$ bin~3.}
  \label{fig:closure-signal-cocktail-bin3}
\end{figure}

The distributions of these results are compared with the results of the fit to the full MC sample.
Any significant difference can be an hint of biases on the results introduced by the fit procedure.

The mean value of the result distributions are shown in Figure~\ref{fig:sub-samp-FL}, in Figure~\ref{fig:sub-samp-P1}, and in Figure~\ref{fig:sub-samp-P5p}, for the $F_L$, $P_1$, and $P_5'$ parameters, respectively.
Only the results from converging fits are included in this distributions, so in general their number is lower than 200.
The error bars assigned to the mean values are the standard deviations of the result distributions, divided by the square root of the number of results in them. 


\begin{figure}[!hbt]
  \centering
  \includegraphics[width=0.7\textwidth]{Figures/cocktail-Fit/com-reco-Fl.pdf}
  \caption{Average values of the $F_L$ result distribution from the fit to 200 signal MC sub-sample (blue), together with the $F_L$ results of the fit to the full MC sample (red). The error bars associated to the sub-sample fit results represent the statistical uncertainty associated to the arithmetic average of the results, as descibed in Section~\ref{sec:Cocktail-MC-pure}.}
  \label{fig:sub-samp-FL}
\end{figure}


\begin{figure}[!hbt]
  \centering
  \includegraphics[width=0.7\textwidth]{Figures/cocktail-Fit/com-reco-P1.pdf}
  \caption{Average values of the $P_1$ result distribution from the fit to 200 signal MC sub-sample (blue), together with the $F_L$ results of the fit to the full MC sample (red). The error bars associated to the sub-sample fit results represent the statistical uncertainty associated to the arithmetic average of the results, as descibed in Section~\ref{sec:Cocktail-MC-pure}.}
  \label{fig:sub-samp-P1}
\end{figure}

\begin{figure}[!hbt]
  \centering
  \includegraphics[width=0.7\textwidth]{Figures/cocktail-Fit/com-reco-P5p.pdf}
  \caption{Average values of the $P_5'$ result distribution from the fit to 200 signal MC sub-sample (blue), together with the $F_L$ results of the fit to the full MC sample (red). The error bars associated to the sub-sample fit results represent the statistical uncertainty associated to the arithmetic average of the results, as descibed in Section~\ref{sec:Cocktail-MC-pure}.}
  \label{fig:sub-samp-P5p}
\end{figure}



%% Pull distributions from fitting on the data-like simulation samples
%% are reported in App.\ref{sec:cock-signal-pull} for each $q^2$ bin.
%% Ideally the pull distributions should follow
%% a N(0,1) normal distribution if the fitting procedures are unbiased
%% and evaluating the error correctly.

%% Since we use the average values of the fitting as the true values in
%% the construction of the pull, the mean of the pull distributions are
%% not interesting here and the focus is the width of the pull.  The unit
%% width of these distributions indicate the fitting is good and the
%% error estimation from the fitting is reasonable.


\subsection{Data-like ``cocktail'' MC samples}
\label{sec:Cocktail-MC-full}

As a second step, the mass and angular distributions of the 200 ``cocktail'' MC samples are fitted using the full \pdf function.
This fit is very similar to the one performed on data, even if there is no S-wave component in the signal events.
For this reason, this validation check is one of the most important steps of the procedure.
As for the signal MC sub-samples, also here the mean and the standard deviation, divided by the square root of fits, are used to represent the ``cocktail'' fit results.

The comparison between the average results of the 200 ``cocktails'' fit and the fit results of Reco MC, is shown in Figure ~\ref{fig:closure-full-cocktail-p1} and Figure~\ref{fig:closure-full-cocktail-P5'}, for the $P_1$ and $P_5'$  parameters, respectively.

%% In some bins the measurements are biased, which is caused by fitting programs and procedures.
%% We account the discrepancies between average values and the fitting result of Reco-MC as a systematic uncertainties called fitting bias.
%% The discrepancies of every bin are shoen in the Figure ~\ref{fig:closure-full-cocktail-p1} and Figure ~\ref{fig:closure-full-cocktail-P5'}.
%% The detail of fitting bias are in Section~\ref{sec:fitbias-syst}.

\begin{figure}[!hbt]
  \centering
  \includegraphics[width=0.7\textwidth]{Figures/cocktail-Fit/avg-full-P1.pdf}
  \caption{Average values of the $P_1$ result distribution from the fit to 200 ``cocktail'' sample (blue), together with the $F_L$ results of the fit to the full MC sample (red). The error bars associated to the cocktail-MC fit results represent the statistical uncertainty associated to the arithmetic average of the results, as descibed in Section~\ref{sec:Cocktail-MC-full}.}
  \label{fig:closure-full-cocktail-p1}
\end{figure}

\begin{figure}[!hbt]
  \centering
  \includegraphics[width=0.7\textwidth]{Figures/cocktail-Fit/avg-full-P5.pdf}
  \caption{Average values of the $P_5'$ result distribution from the fit to 200 ``cocktail'' sample (blue), together with the $F_L$ results of the fit to the full MC sample (red).The error bars associated to the cocktail-MC fit results represent the statistical uncertainty associated to the arithmetic average of the results, as descibed in Section~\ref{sec:Cocktail-MC-full}.}
  \label{fig:closure-full-cocktail-P5'}
\end{figure}



%% \subsubsection{Comparison of data-like statistics fit with and without background}
%% \label{sec:Cocktail-MC-com}

%% Then we compared the average fitting results with full simulation described in
%% Section~\ref{sec:Cocktail-MC-full}. The comparison are in the Figure~\ref{fig:comparison-cocktail-P1}
%% and Figure~\ref{fig:comparison-cocktail-P5}.
%% When fitting with the background, we fixed the parameter $F_L$ to have a better convergence
%% as we will do for data fit (details in Section~\ref{sec:formula}), so we can
%% compare the value of $P_1$ and $P_5'$.

%% \begin{figure}[!hbt]
%%   \centering
%%   \includegraphics[width=0.7\textwidth]{Figures/cocktail-Fit/com-toy-P1.pdf}
%%   \caption{Comparison between the average results of 200 full cocktail
%%           samples and 200 pure signal cocktail samples of $P1$ for each $q^2$
%%     bins. The red points are average results of full cocktail MC and the blue are
%%      pure signal cocktail MC.}
%%   \label{fig:comparison-cocktail-P1}
%% \end{figure}


%% \begin{figure}[!hbt]
%%   \centering
%%   \includegraphics[width=0.7\textwidth]{Figures/cocktail-Fit/com-toy-P5p.pdf}
%%   \caption{Comparison between the average results of 200 full cocktail
%%           samples and 200 pure signal cocktail samples of $P_5'$ for each $q^2$
%%     bins. The red points are average results of full cocktail MC and the blue are
%%      pure signal cocktail MC.}
%%   \label{fig:comparison-cocktail-P5}
%% \end{figure}

%% \subsection{Error comparison between the blinded data and data-like simulation}
%% \label{sec:error-comparison}

%% This section describes the check on the statistical errors we perform
%% on data in a blinded way. We have fit the data in the signal region,
%% but "blinding" the central values for $P_1$ and $P_5'$. That means we
%% check the size of the statistical errors from the fitting only,
%% without looking at the
%% central values.  The errors are from the preliminary
%% fitting results obtained with "HESSE".  If using "MINOS" and more
%% detailed scanning of the initial values, probably there will be some
%% improvements, but we can only fully tune those options after the green
%% light for unblinding.

%% We compare the errors with what are expected from data-like statistics
%% full simulations which are described in
%% Section~\ref{sec:Cocktail-MC-full}, to verify the error estimation in
%% our analysis. The Figure~\ref{fig:error-com-0} to
%% Figure~\ref{fig:error-com-8} are the comparison of all $q^2$ bins'
%% results. In the figures, the red lines are the errors from the data,
%% and the histograms are the distributions from the simulation. The
%% first subplots are for $P_1$, the second ones are for $P_5'$, the
%% third are for the fraction of signal.


%% \begin{figure}[!hbt]
%%   \centering
%%   \includegraphics[width=1.0\textwidth]{Figures/errorbar/error-0.pdf}
%%   \caption{Validation between errors of 200 full cocktail samples and
%%     data channel for $q^2$ bin0. The red line is the error of data
%%     channel, the histogram is the distribution of simulations. And the
%%     first is $P_1$, the second is $P_5'$, the third is fraction of
%%     signal.}
%%   \label{fig:error-com-0}
%% \end{figure}

%% \begin{figure}[!hbt]
%%   \centering
%%   \includegraphics[width=1.0\textwidth]{Figures/errorbar/error-1.pdf}
%%   \caption{Validation between errors of 200 full cocktail samples and
%%     data channel for $q^2$ bin1. The red line is the error of data
%%     channel, the histogram is the distribution of simulations. And the
%%     first is $P_1$, the second is $P_5'$, the third is fraction of
%%     signal.}
%%   \label{fig:error-com-1}
%% \end{figure}

%% \begin{figure}[!hbt]
%%   \centering
%%   \includegraphics[width=1.0\textwidth]{Figures/errorbar/error-2.pdf}
%%   \caption{Validation between errors of 200 full cocktail samples and
%%     data channel for $q^2$ bin2. The red line is the error of data
%%     channel, the histogram is the distribution of simulations. And the
%%     first is $P_1$, the second is $P_5'$, the third is fraction of
%%     signal.}
%%   \label{fig:error-com-2}
%% \end{figure}

%% \begin{figure}[!hbt]
%%   \centering
%%   \includegraphics[width=1.0\textwidth]{Figures/errorbar/error-3.pdf}
%%   \caption{Validation between errors of 200 full cocktail samples and
%%     data channel for $q^2$ bin3. The red line is the error of data
%%     channel, the histogram is the distribution of simulations. And the
%%     first is $P_1$, the second is $P_5'$, the third is fraction of
%%     signal.}
%%   \label{fig:error-com-3}
%% \end{figure}

%% \begin{figure}[!hbt]
%%   \centering
%%   \includegraphics[width=1.0\textwidth]{Figures/errorbar/error-5.pdf}
%%   \caption{Validation between errors of 200 full cocktail samples and
%%     data channel for $q^2$ bin5. The red line is the error of data
%%     channel, the histogram is the distribution of simulations. And the
%%     first is $P_1$, the second is $P_5'$, the third is fraction of
%%     signal.}
%%   \label{fig:error-com-5}
%% \end{figure}

%% \begin{figure}[!hbt]
%%   \centering
%%   \includegraphics[width=1.0\textwidth]{Figures/errorbar/error-7.pdf}
%%   \caption{Validation between errors of 200 full cocktail samples and
%%     data channel for $q^2$ bin7. The red line is the error of data
%%     channel, the histogram is the distribution of simulations. And the
%%     first is $P_1$, the second is $P_5'$, the third is fraction of
%%     signal.}
%%   \label{fig:error-com-7}
%% \end{figure}

%% \begin{figure}[!hbt]
%%   \centering
%%   \includegraphics[width=1.0\textwidth]{Figures/errorbar/error-8.pdf}
%%   \caption{Validation between errors of 200 full cocktail samples and
%%     data channel for $q^2$ bin8. The red line is the error of data
%%     channel, the histogram is the distribution of simulations. And the
%%     first is $P_1$, the second is $P_5'$, the third is fraction of
%%     signal.}
%%   \label{fig:error-com-8}
%% \end{figure}

\section{Validation with data control channels}
\label{sec:controlchannel}

The analysis technique is validated with the data by means of the control channels.
In this way, the S-wave component of the PDF is tested, and we have a check of the efficiency behaviour on real data.

\subsection{Sideband fit in control regions}
\label{sec:bkgforcc}

The background shape is determined as for the other bins, by using the data sidebands as a function of the angular observables.
Each of the three angular observables is fit with a polynomial with different degrees, specified in Table~\ref{tab:psi-bkg}.
The background distributions and \pdf projections are plotted in Figure~\ref{fig:back-l-bin4} to Figure~\ref{fig:back-phi-bin4}, for the \BtoKstJpsi control channel, and in Figure~\ref{fig:back-l-bin6} to Figure~\ref{fig:back-phi-bin6}, for the \BtoKstpsip control channel.

\begin{table*}[!htb]
  \begin {center}
    %% \begin{small}
      \caption{Degrees of the polynomial functions used for control channel \pdfs.
        \label{tab:psi-bkg}}
      \begin{tabular}{c|c|c|c}
        $q^2$ bin & $B^{\cos\theta_\mathrm{K}}$ & $B^{\cos\theta_l}$ & $B^{\PHI}$ \\
        index & degree & degree & degree \\
        \hline
        \BtoKstJpsi & 4 & 4 & 5 \\
        \BtoKstpsip & 4 & 3 & 5 \\
      \end{tabular}
    %% \end{small}
  \end{center}
\end{table*}

\begin{figure}[!hbt]
  \centering
  \includegraphics[width=0.7\textwidth]{Figures/J/bkg-l.pdf}
  \caption{Distribution of the \cTL variable in mass sidebands of the \BtoKstJpsi control channel and the background \pdf projection.}
  \label{fig:back-l-bin4}
\end{figure}

\begin{figure}[!hbt]
  \centering
  \includegraphics[width=0.7\textwidth]{Figures/J/bkg-k.pdf}
  \caption{Distribution of the \cTK variable in mass sidebands of the \BtoKstJpsi control channel and the background \pdf projection.}
  \label{fig:back-k-bin4}
\end{figure}

\begin{figure}[!hbt]
  \centering
  \includegraphics[width=0.7\textwidth]{Figures/J/bkg-phi.pdf}
  \caption{Distribution of the \PHI variable in mass sidebands of the \BtoKstJpsi control channel and the background \pdf projection.}
  \label{fig:back-phi-bin4}
\end{figure}

\begin{figure}[!hbt]
  \centering
  \includegraphics[width=0.7\textwidth]{Figures/Psi/bkg-l.pdf}
  \caption{Distribution of the \cTL variable in mass sidebands of the \BtoKstpsip control channel and the background \pdf projection.}
  \label{fig:back-l-bin6}
\end{figure}

\begin{figure}[!hbt]
  \centering
  \includegraphics[width=0.7\textwidth]{Figures/Psi/bkg-k.pdf}
  \caption{Distribution of the \cTK variable in mass sidebands of the \BtoKstpsip control channel and the background \pdf projection.}
  \label{fig:back-k-bin6}
\end{figure}

\begin{figure}[!hbt]
  \centering
  \includegraphics[width=0.7\textwidth]{Figures/Psi/bkg-phi.pdf}
  \caption{Distribution of the \PHI variable in mass sidebands of the \BtoKstpsip control channel and the background \pdf projection.}
  \label{fig:back-phi-bin6}
\end{figure}


\subsection{Control channel fit}
\label{sec:fitcc}

The two control channels are fitted, with  $F_s$ and $A_s$ fixed, as described in Section~\ref{sec:fitseq}, while $F_L$ is kept floating.
The projection plots of the fit results are shown Figure~\ref{fig:result-bin4} and Figure~\ref{fig:result-bin6}, for \BtoKstJpsi and \BtoKstpsip channels respectively.

\begin{figure}[!hbt]
  \centering
  \includegraphics[width=1.0\textwidth]{Figures/J/TotalPDF-J.pdf}
  \caption{The fitting results of the control channel \cPJgy on data.
    The plots show the projections of the fitting results on three different angular variables: \PBz mass, \cTL, \cTK and \PHI.}
  \label{fig:result-bin4}
\end{figure}

\begin{figure}[!hbt]
  \centering
  \includegraphics[width=1.0\textwidth]{Figures/Psi/TotalPDF-Psi.pdf}
  \caption{The fitting results of the control channel $\Psi'$ on data.
    The plots show the projections of the fitting results on three different angular variables: \PBz mass, \cTL, \cTK and \PHI.}
  \label{fig:result-bin6}
\end{figure}

The results of the measurements for the control channels \BtoKstJpsi and \BtoKstpsip are summarised in Table~\ref{tab:res-control-channel}.

\begin{table*}[!htb]
  \begin {center}
    \begin{small}
      \caption{Results of the fit to the data control channels, as described in Section~\ref{sec:fitcc}. The reported uncertainty is fully statistical.
        \label{tab:res-control-channel}}
      \begin{tabular}{l|c|c|c|c}
        control channel & $F_L$ & $P_1$ & $P_5'$ & $A_s^5$ \\
        \hline
        \BtoKstJpsi & $0.537 \pm 0.002$ & $-0.081 \pm 0.011$ & $-0.024 \pm 0.007$ & $-0.002 \pm 0.002$ \\
        \BtoKstpsip & $0.538 \pm 0.008$ & $-0.031 \pm 0.001$ & $-0.039 \pm 0.001$ & $0.005 \pm 0.001$  \\
      \end{tabular}
    \end{small}
  \end{center}
\end{table*}


For both the control channels, the $F_L$ parameter has been measured in the previous CMS analysis and also by other experiments.
The results from this work, from the previous CMS analysis and from other experiments are in good agreement, as shown in Table~\ref{tab:com.control channel}.

\begin{table*}[!htb]
  \begin {center}
    \begin{small}
      \caption{Measurements from CMS (both in this and in the previous analysis), PDG, and BaBar\cite{BaBar2} of $F_L$ in the control channels.
        The first uncertainty is statistical and the second is systematic.
        \label{tab:com.control channel}}
      \begin{tabular}{l|c|c|c|c|c|c|c}
        control channel & \multicolumn{3}{|c|}{$B^0 \rightarrow K^{*0}(K^+\pi^-) J/\psi(\mu^+ \mu^-)$} & \multicolumn{3}{|c|}{$B^0 \rightarrow K^{*0}(K^+\pi^-) \psi'(\mu^+ \mu^-)$}\\
        \hline
        Experiment  & $F_L$  &  Err(stat) & Err(syst) & $F_L$  &  Err(stat) & Err(syst)\\
        \hline
        This work & $0.537$ & $\pm0.002$ &  $-$  & $0.538$ & $\pm0.008$ &  $-$\\
        \hline
        CMS   &  $0.537$ & $\pm0.002$ &  $-$    &  $0.538$ & $\pm0.008$ &  $-$  \\
        \hline
        PDG   &  $0.571$ & $\pm0.007$ & $-$      &  $0.463$ & $^{+0.028}_{-0.040} $ & $-$ \\
        \hline
        BaBar &  $0.556$ & $\pm0.009$ & $\pm0.010 $   &  $0.48$ & $\pm0.005$ & $\pm0.002 $\\
      \end{tabular}
    \end{small}
  \end{center}
\end{table*}

As further test, the fit on the two control channels has been repeated with the $F_L$ parameter fixed, coherently with the procedure used on data and described in Section~\ref{sec:fitseq}.
The results of the two kind of fits, with $F_L$ either free to float or fixed, are compared in Table~\ref{tab:flfixed-control-channel}.

\begin{table*}[!htb]
  \begin {center}
    \begin{small}
      \caption{Results of the fit to the data control channels, as described in Section~\ref{sec:fitcc}. The reported uncertainty is fully statistical.
        \label{tab:flfixed-control-channel}}
      \begin{tabular}{l|l|c|c|c}
        control channel & $F_L$ state & $P_1$ & $P_5'$ & $A_s^5$ \\
        \hline
        \BtoKstJpsi & floating & $-0.081 \pm 0.011$ & $-0.024 \pm 0.007$ & $-0.002 \pm 0.002$ \\
                    & fixed    & $-0.082 \pm 0.003$ & $-0.024 \pm 0.002$ & $-0.001 \pm 0.002$ \\
        \hline
        \BtoKstpsip & floating & $-0.031 \pm 0.001$ & $-0.039 \pm 0.001$ & $0.005 \pm 0.001$  \\
                    & fixed    & $-0.033 \pm 0.025$ & $-0.040 \pm 0.031$ & $0.005 \pm 0.011$  \\
      \end{tabular}
    \end{small}
  \end{center}
\end{table*}

The difference between the fit results of the $P_1$ and $P_5'$ parameters are reported in the Table~\ref{tab:difference Fl free or fix}.
Since these differences are very small compared to the statistical errors of the results, this test shows that the bias in the results introduced by the choice of fixing some parameters in the \pdf is negligible in the final result.

\begin{table*}[!htb]
  \begin {center}
    \begin{small}
      \caption{Difference between the $P_1$ and $P_5'$ results obtained from a fit sequence with the $F_L$ parameter fixed or free to float.
        \label{tab:difference Fl free or fix}}
      \begin{tabular}{l|c|c}
        control channel & $P1$ & $P_5'$ \\
        \hline
        \BtoKstJpsi & $0.001$ & $<0.001$ \\
        \BtoKstpsip & $0.002$ & $0.001$  \\
      \end{tabular}
    \end{small}
  \end{center}
\end{table*}
