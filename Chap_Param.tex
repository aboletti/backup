\section{Conventions and formulations}
\label{sec:formula}

\subsection{The probability density function}
\label{sec:TotalPDF}

%% The $q^2$ bins to be used in the analysis are defined in the Table~\ref{tab:q2 bins}.
%% They are chosen in such way to match the measurements performed in previous experiments.

%% Nine bins of $q^2$ are used in the analysis, including two which are dominated
%% by the control samples.
The total $q^2$ range used in this analysis extends from 1\GeV$^2$ to 19\GeV$^2$ and it is divided in nine bins, defined in Table~\ref{tab:q2bins}.
The bin definition is the same used in the previous angular analysis on the same datset and it is the result of a compromise between being coherent with the definition used in the previous measurements and having an expected signal yield homogeneously distributed over the $q^2$ bins.

\begin{table}[!htb]
  \begin{center}
    \begin{small}
      \caption{Range definition of the dimuon invariant mass bins. Both $J/\psi$ and $\psi'$ regions, namely $q^2$ bins No.4 and No.6, are used as control channels.
        \label{tab:q2bins}}
      \begin{tabular}{c|l}
        Bin index & $q^2$ range ($Gev^2/c^4$) \\
        \hline
        0 & 1-2  \\
        1 & 2-4.3  \\
        2 & 4.3-6  \\
        3 & 6-8.68   \\
        4 & 8.68-10.09 ($J/\Psi$ region) \\
        5 & 10.09-12.86\\
        6 & 12.86-14.18 ($\Psi'$ region)\\
        7 & 14.18-16\\
        8 & 16-19\\
      \end{tabular}
    \end{small}
  \end{center}
\end{table}

The selection criteria and the analysis techniques are identical for any $q^2$ bin and in each of them the analysis is performed independently.
The two $q^2$ bins containing the control channel regions are not used to fit the signal events.

The angular parameters are extracted through an unbinned fit, using an extended maximum likelihood estimator.
The probability density function (\pdf) used in the fit has the following expression:
\begin{equation} \label{eq:angALL}
  \begin{split}
    \mathrm{pdf}(m,\theta_\PK,\theta_\ell,\varphi) & = Y^{C}_{S} \biggl[ S^{C}(m)  \, S^a(\theta_\PK,\theta_\ell,\varphi) \, \epsilon^{C}(\theta_\PK,\theta_\ell,\varphi) \biggr. \\
      & \biggl. + \frac{f^{M}}{1-f^{M}}~S^{M}(m) \, S^a(-\theta_\PK,-\theta_\ell,\varphi) \, \epsilon^{M}(\theta_\PK,\theta_\ell,\varphi) \biggr] \\
    & + Y_{B}\,B^m(m) \, B^{\theta_\PK}(\theta_\PK) \, B^{\theta_\ell}(\theta_\ell) \, B^{\varphi}(\varphi), \\
  \end{split}
\end{equation}
%%%%%%%%%%%%%% FROM AN
where:
\begin{description}
    \item[$Y_{S}^C$] is the yields of signal events;
    \item[$Y_{B}$] is the yields of background events;
    \item[$f_i^M$] is the CP-mistag fraction (i.e. the number of mis-tagged signal events divided by the number of total signal events);
    \item[$S_i^R(m)$] describes the shape of the right-tagged signal events as a function of the $K\pi\mu\mu$ invariant mass;
    \item[$S_i^M(m)$] describes the shape of the mis-tagged signal events as a function of the $K\pi\mu\mu$ invariant mass;
    \item[$S_i^a(\cos\theta_l,\cos\theta_\mathrm{K},\phi)$] describes the shape of the signal events as a function of the three angular observables;
    \item[$B_i^m(m) \cdot B_i^{\cos\theta_\mathrm{K}}(\cos\theta_\mathrm{K}) \cdot B_i^{\cos\theta_l}(\cos\theta_l) \cdot B_i^{\phi}(\phi) $] are four functions describeing the shapes of the combinatorial background events, as functions of the $K\pi\mu\mu$ invariant mass and the three angular observables; 
    \item[$\epsilon_i^R(\cos\theta_l,\cos\theta_\mathrm{K},\phi)$] describes the efficiency for right-tagged signal events in the 3D-space of the angular observables;
    \item[$\epsilon_i^M(\cos\theta_l,\cos\theta_\mathrm{K},\phi)$] describes the efficiency for mis-tagged signal events in the 3D-space of the angular observables;
\end{description}

All the subscript $i$ runs over the $q^2$ bins listed in the Table~\ref{tab:q2 bins} if not otherwise specified.

%%%%%%%%%%%%% FROM PAPER
where its three terms correspond to the \pdfs for right-tagged signal, mis-tagged signal, and background events, respectively.

The parameters $Y^{C}_{S}$ and $Y_{B}$ are the yields of right-tagged signal events and background events, respectively, while the parameter $f^{M}$ is the fraction of signal events that are mis-tagged.

The functions $S^{C}(m)$ and $S^{M}(m)$ are the signal \PBz-mass \pdfs, for right-tagged and mis-tagged signal events, respectively.
Each of them is composed as the sum of two Gaussian functions, with a common mean for all four Gaussian functions.

In the fit, the mean, the four Gaussian function's width parameters, and the two fractions specifying the relative contribution of the two Gaussian functions in $S^{C}(m)$ and $S^{M}(m)$ are determined from simulation.
The function $S^a(\theta_\PK,\theta_\ell,\varphi)$ describes the signal in the three-dimensional (3D) space of the angular variables and corresponds to Eq.~(\ref{eq:PDF}).
The combination $B^m(m) \, B^{\theta_\PK}(\theta_\PK) \, B^{\theta_\ell}(\theta_\ell) \, B^{\varphi}(\varphi)$ is obtained from the \PBz\ sideband data in $m$ and describes the background in the space of $(m,\theta_\PK,\theta_\ell,\varphi)$, where $B^m(m)$ is an exponential function, $B^{\theta_\PK}(\theta_\PK)$ and $B^{\theta_\ell}(\theta_\ell)$ are second- to fourth-order polynomials, depending on the $q^2$ bin, and $B^{\varphi}(\varphi)$ is a first-order polynomial.
The factorization assumption of the background pdf in Eq.~(\ref{eq:angALL}) is validated by dividing the range of an angular variable into two at its center point and comparing the distributions of events from the two halves in the other angular variables.
%%%%%%%%%%%%%%%%%%%%%%%%

The three-dimensional functions $\epsilon^{C}(\theta_\PK,\theta_\ell,\varphi)$ and $\epsilon^{M}(\theta_\PK,\theta_\ell,\varphi)$ are the efficiencies for right-tagged and mis-tagged signal events, respectively.
The construction of these functions is described in Section~\ref{sec:eff}.
%%%%%%%%%%%%%%%%%%%%%%%%

\subsection{Correctly and wrongly CP-tagged signal events}
\label{sec:fullform}

The signal and control channels are self-tagging decays.
This means that in principle one can distinguish whether the mother particle is a $B^0$ or $\overline{B}^0$ by simply measuring the charges of the daughter hadrons, but this in turn requires the capability of disentangling kaons and pions.
Unfortunately the CMS experiment does not possess such a capability for charged particles above $\mathcal{O}(1~\GeV)$.
To overcome this deficit an algorithm, based on the analysis of the invariant mass of the two hadrons has been used.
Both the $\pi$ and $K$ mass hypothesis is considered for each of the two hadrons of the decay: the hypothesis with an invariant mass for the two-hadrons closer to the nominal $\PKst$ one is retained.
The algorithm has an intrinsic percentage of failure which is referred to as mistag fraction, $f^M_i$, defined as the ratio of mistagged signal events divided by the total signal events.
The mistag fraction is determined from simulated events, comparing the results of the algorithm to the MC truth.
The mistag fraction is determined by counting the number of correctly and wrongly tagged events, where only truth-matched events are considered: results for each bin are shown in Table~\ref{tab:Mis tag fraction}.

\begin{table*}[!htb]
    \begin{center}
        \begin{small}
            \caption{Number of correctly and wrongly CP-tagged events determined with signal and con-
                trol channel simulation samples for each $q^2$ bin in Table~\ref{tab:q2 bins}.
                \label{tab:Mis tag fraction}}
            \begin{tabular}{|l|c|c|c|c|}
                \hline
                $q^2$ bin index  & Correctly tagged &  Wrongly tagged   & Mistag fraction ($f^M_i$)  &   Error  \\
                \hline
                $ 0$    &  $ 30518 $      &  $ 4312 $  & $ 0.124 $ & $ 0.002 $    \\
                $ 1 $   &  $ 64438$       &  $9542$    & $ 0.129 $ & $ 0.001 $\\
                $ 2$    &  $ 51149 $      &  $7892$    & $ 0.134 $ & $ 0.001 $ \\
                $3 $    &  $ 91065$       &  $13845$   & $ 0.132 $ & $ 0.001 $   \\
                $4  $   &   $ 472326$     &  $75157$   & $0.1373 $ & $ 0.0005$ \\
                $ 5 $   &  $119644 $      &  $18256$   & $0.132  $ & $ 0.001 $\\
                $ 6$    &  $30808$        &  $5013 $   & $0.140 $  & $ 0.002$\\
                $ 7 $   &  $69773$        &  $10623$   & $0.132 $  & $ 0.001$\\
                $ 8 $   &  $72769$        &  $11523$   & $0.137$   & $0.001$\\
                \hline
            \end{tabular}
        \end{small}
    \end{center}
\end{table*}

The mis-tag fraction is fixed in the final fit considering it is small.
The MC estimation of mis-tag fraction may have some differences with respect to the real data case.
The contribution of this effect to the systematic uncertainty, evaluated from the data, is discussed in Sect.~\ref{sec:sys-mistag}.

%% We consider this as one source of
%% uncertainties, which will be evaluated from the data after the
%% unblinding of the analysis, in Sect. \ref{sec:sys-mistag}.




\subsection{The decay rate description in full angular variables}
\label{sec:fullform}

The angle $\theta_l$ is defined as the angle between the
direction of the $\mu^+ $ ($\mu^- $) and the direction opposite that of
the $\text{B}^0$ ($\bar{\text{B}^0}$) in the dimuon rest frame; the
angle $\theta_\mathrm{K} $ is defined as the angle between the direction of the
kaon and the direction opposite that of the $B^0$
($\bar{\text{B}^0}$) in the $\text{K}^{*0}$ rest frame; the angle
$\phi$ is the angle between the plane containing the $\mu^+ $ and
$\mu^- $ and the plane containing the kaon and the pion from the
$\PKst$ decay in the $\text{B}^0$ rest frame.  Using the notation of
\cite{Ball2009}, the differential angular distribution can be written
as:
\begin{equation} \label{eq:Angular}
    \begin{split}
        \frac{1}{\mathrm{d}\Gamma/\mathrm{d}q^2}\frac{\mathrm{d}^4\Gamma}{\mathrm{d}q^2 \mathrm{d}\cos\theta_l \mathrm{d}\cos\theta_\mathrm{K} \mathrm{d}\phi} =&\frac{9}{32\pi}\left[\frac{3}{4}F_\mathrm{T}\sin^2\theta_\mathrm{K} + F_\mathrm{L}\cos^2\theta_\mathrm{K} \right.\\
            &\left.+\left(\frac{1}{4}F_\mathrm{T}\sin^2\theta_\mathrm{K}-F_\mathrm{L}\cos^2\theta_\mathrm{K}\right)\cos2\theta_l+\frac{1}{2}P_1F_\mathrm{T}\sin^2\theta_\mathrm{K}\sin^2\theta_l\cos 2\phi \right.\\
            &+\sqrt{F_\mathrm{T}F_\mathrm{L}}\left(\frac{1}{2}P_4'\sin2\theta_\mathrm{K}\sin2\theta_l\cos\phi+P_5'\sin2\theta_\mathrm{K}\sin\theta_l\cos\phi \right)\\
            &-\sqrt{F_\mathrm{T}F_\mathrm{L}}\left(P_6'\sin2\theta_\mathrm{K}\sin\theta_l\sin\phi-\frac{1}{2}P_8'\sin2\theta_\mathrm{K}\sin2\theta_l\sin\phi \right)\\
            &\left.+2P_2F_\mathrm{T}\sin^2\theta_\mathrm{K}\cos\theta_l-P_3F_\mathrm{T}\sin^2\theta_\mathrm{K}\sin^2\theta_l\sin2\phi \right]
    \end{split}
\end{equation}
where the $q^2$ dependent observables $P_i$ and $P'_i$ are optimized observables built via 
combinations of the $\text{K}^{*0}$ decay amplitudes, as defined in~\cite{Genon:Swave};
$F_\mathrm{L}$ is the longitudinal polarization fraction of the $\text{K}^{*0}$ and
$F_\mathrm{T}=(1-F_\mathrm{L})$.

\subsection{Transformation of pdf formulation used in the fit}
\label{sec:transform}

The angular distribution of the decay
$\text{B}^0 \rightarrow \text{K}^{*0} \mu^+ \mu^-$ can be described by
three angles ($\theta_l $, $\theta_\mathrm{K} $ and $\phi$) and the
invariant mass squared of the dimuon system ($q^2$). Because of the
limited number of signal candidates in the data set, we didn't fit the
data to full differential distribution of Eq.~\ref{eq:Angular}. To
retrieve the interesting variables more effectively and to reduce the
number of fitting parameters, we performed the following transformations of the
decay-rate formulation.


Inspired by the derivations in ref \cite{LHCb2}\cite{Matias2012}, to
reduce the number of parameters in the fit, we "fold" the data
twice. "Folding" means that we divide the decay rate into different
parts, calculate them separately according to some symmetries and then
add them together to obtain the equivalent decay rates. If we take
consecutive steps of ``folding'', the similar expansions are used to
get the full PDFs.

Let us take the first folding as an example.
The first folding is at
$ \phi=0$ (for $\phi<0,\phi\rightarrow-\phi$, the $\phi$'s domain is reduced to
    (0,$\pi$)). To be more clear, we divide the decay rate $d\Gamma$
into two parts corresponding to $\phi>0$ and $\phi<0$,
i.e. $d\Gamma(\phi;\phi>0)$, and $d\Gamma(\phi;\phi<0)$:

\begin{equation} \label{eq:folding}
    \begin{split}
        d\hat{\Gamma} &= d\Gamma(\phi|\phi<0) + d\Gamma(\phi|\phi>0) \\
        & = f_0(\phi|\phi\rightarrow-\phi) + f_0(\phi|\phi>0) \\
        & = f_0(\cos\phi, -\sin\phi) + f_0(\cos\phi, \sin\phi)
    \end{split}
\end{equation}


According to trigonometric identities $\cos(-\phi) = \cos\phi $,
$\sin(-\phi) = -\sin\phi $, we can cancel the terms 
that are odd under this transformation
%% containing $\sin\phi$
. Eq.~\ref{eq:Angular} now reads:

\begin{equation} \label{eq:fold1}
    \begin{split}
        \frac{1}{\mathrm{d}\Gamma/\mathrm{d}q^2}\frac{\mathrm{d}^4\Gamma}{\mathrm{d}q^2 \mathrm{d}\cos\theta_l \mathrm{d}\cos\theta_\mathrm{K} \mathrm{d}\phi} =&\frac{9}{16\pi}\left[\frac{3}{4}F_\mathrm{T}\sin^2\theta_\mathrm{K} + F_\mathrm{L}\cos^2\theta_\mathrm{K} \right.\\
            &\left.+(\frac{1}{4}F_\mathrm{T}\sin^2\theta_\mathrm{K}-F_\mathrm{L}\cos^2\theta_\mathrm{K})\cos2\theta_l+\frac{1}{2}P_1F_\mathrm{T}\sin^2\theta_\mathrm{K}\sin^2\theta_l\cos 2\phi \right.\\
            &+\sqrt{F_\mathrm{T}F_\mathrm{L}}(\frac{1}{2}P_4'\sin2\theta_\mathrm{K}\sin2\theta_l\cos\phi+P_5'\sin2\theta_\mathrm{K}\sin\theta_l\cos\phi )\\
            &\left.+2P_2F_\mathrm{T}\sin^2\theta_\mathrm{K}\cos\theta_l \right]
    \end{split}
\end{equation}

The second folding is performed at $\theta_l = \pi/2$ (for
$\theta_l>\pi/2,\theta_l\rightarrow \pi- \theta_l$). The domain of
$\theta_l$ is reduced to (0,$\pi$/2). According to $\cos(\pi-\theta_l) = -
\cos\theta_l$ and $\sin(\pi-\theta_l) = \sin\theta_l$, we can cancel the terms that are odd under this transformation.
 %% proportional to $P_4'$, which contains $\sin 2\theta_l$.


\begin{equation} \label{eq:fold2}
    \begin{split}
        \frac{1}{\mathrm{d}\Gamma/\mathrm{d}q^2}\frac{\mathrm{d}^4\Gamma}{\mathrm{d}q^2 \mathrm{d}\cos\theta_l \mathrm{d}\cos\theta_\mathrm{K} \mathrm{d}\phi} =&\frac{9}{8\pi}\left[\frac{3}{4}F_\mathrm{T}\sin^2\theta_\mathrm{K} + F_\mathrm{L}\cos^2\theta_\mathrm{K} \right.\\
            &\left.+(\frac{1}{4}F_\mathrm{T}\sin^2\theta_\mathrm{K}-F_\mathrm{L}\cos^2\theta_\mathrm{K})\cos2\theta_l+\frac{1}{2}P_1F_\mathrm{T}\sin^2\theta_\mathrm{K}\sin^2\theta_l\cos 2\phi \right.\\
            &\left.+\sqrt{F_\mathrm{T}F_\mathrm{L}}P_5'\sin2\theta_\mathrm{K}\sin\theta_l\cos\phi  \right]
    \end{split}
\end{equation}

We have studied how to do the foldings in a previous note
AN-2015/172~\cite{AN-15-172}. We have derived two different
formulations with two foldings and three foldings separately. Some
preliminary studies based on a simplified fitting model has some
indications that the two folding formulation would produce more
precise results.  More details could be found in ref~\cite{AN-15-172}.

On the other hand, the two LHCb publications and the recent Belle
paper all use the same two-folding formulation. So we take these
indications here and focus on the fitting with the two-folding
formulation in this note.


% The third folding is at $\phi = \pi/2$ (for $\phi >\pi/2,
% \phi\rightarrow\phi- \pi/2$), the $\phi$'s domain is
% (0,$\pi$/2). According $\cos(\phi- \pi/2) = \sin\phi$ and $\sin(\phi-
% \pi/2) = -\cos\phi$, we can cancel the terms containing $\cos\phi$ and
% $\sin\phi$ at the same time. Then the Eq(3) now reads:

% \begin{equation} \label{eq:fold3}
%   \begin{split}
% \frac{1}{\mathrm{d}\Gamma/\mathrm{d}q^2}\frac{\mathrm{d}^4\Gamma}{\mathrm{d}q^2 \mathrm{d}\cos\theta_l \mathrm{d}\cos\theta_\mathrm{K} \mathrm{d}\phi} =&\frac{9}{8\pi}\left[2(\frac{3}{4}F_\mathrm{T}\sin^2\theta_\mathrm{K} + F_\mathrm{L}\cos^2\theta_\mathrm{K} \right +(\frac{1}{4}F_\mathrm{T}\sin^2\theta_\mathrm{K}-F_\mathrm{L}\cos^2\theta_\mathrm{K})\cos2\theta_l) \right.\\
%  &+\left\sqrt{F_\mathrm{T}F_\mathrm{L}}P_5'\sin2\theta_\mathrm{K}\sin\theta_l(\cos\phi-\sin\phi)\right]
% \end{split}
% \end{equation}

\subsection{Influence of S-wave interference on the angular distributions}
\label{sec:S-waveform}
Although the $K^+\pi^-$ invariant mass must be consistent with a
$\text{K}^{*0}$, there can be contributions from a spinless (S-wave)
$K^+\pi^-$ combination. The presence of a $K^+\pi^-$
system in an S-wave configuration, due to a non-resonant contribution or
to feed through from $K^+\pi^-$ scalar resonances, results in additional
terms in the different angular distribution. Denoting the right-hand side
 of Eq.~\ref{eq:Angular} by $W_p$, the differential decay rate takes the form

\begin{equation} \label{eq:S-wave}
    \begin{split}
    (1-F_\mathrm{S})W_p + (W_s + W_{sp})
    \end{split}
\end{equation}

where 
\begin{equation} \label{eq:S-wave0}
    \begin{split}
      W_s = \frac{3}{16\pi} F_\mathrm{S}\sin^2\theta_l
    \end{split}
\end{equation}

and $W_{sp}$ is given from Eq.(44) in~\cite{Genon:Swave}. 
\begin{equation} \label{eq:S-wave1}
    \begin{split}
      W_{sp}= &\frac{3}{16 \pi}\left[ A_\mathrm{S}\sin^2\theta_l\cos\theta_\mathrm{K}+ A_\mathrm{S}^4\sin\theta_\mathrm{K}\sin2\theta_l\cos\phi\right.\\
            &+\left.A^5_\mathrm{S}\sin\theta_\mathrm{K}\sin\theta_l\cos\phi+A_\mathrm{S}^7\sin\theta_\mathrm{K}\sin\theta_l\sin\phi+A_\mathrm{S}^8\sin\theta_\mathrm{K}\sin2\theta_l\sin\phi\right]
    \end{split}
\end{equation}

where $F_\mathrm{S}$ is the fraction of the S-wave component in the
$\text{K}^{*0}$ mass window, and $W_{sp}$ contains all the
interference terms, $A_\mathrm{S}^i$ are the intererence amplitudes between the
S-wave and the P-wave decays\cite{Genon:Swave}. For S-wave and the
interference terms, we do the same transformation as P-wave, after the
first ``folding'', it can reads:

\begin{equation} \label{eq:S-fold1}
    \begin{split}
        \frac{1}{\mathrm{d}\Gamma/\mathrm{d}q^2}\frac{\mathrm{d}^4\Gamma}{\mathrm{d}q^2 \mathrm{d}\cos\theta_l \mathrm{d}\cos\theta_\mathrm{K} \mathrm{d}\phi} =&\frac{3}{8\pi}\left[F_\mathrm{S}\sin^2\theta_l+ A_\mathrm{S}\sin^2\theta_l\cos\theta_\mathrm{K}\right.\\
            &+\left. A_\mathrm{S}^4\sin\theta_\mathrm{K}\sin2\theta_l\cos\phi + A^5_\mathrm{S}\sin\theta_\mathrm{K}\sin\theta_l\cos\phi\right]
    \end{split}
\end{equation}

After the second ``folding'', it reads:
\begin{equation} \label{eq:S-fold2}
    \begin{split}
        \frac{1}{\mathrm{d}\Gamma/\mathrm{d}q^2}\frac{\mathrm{d}^4\Gamma}{\mathrm{d}q^2 \mathrm{d}\cos\theta_l \mathrm{d}\cos\theta_\mathrm{K} \mathrm{d}\phi} =\frac{3}{4\pi}\left[F_\mathrm{S}\sin^2\theta_l+A_\mathrm{S}\sin^2\theta_l\cos\theta_\mathrm{K}+A^5_\mathrm{S}\sin\theta_\mathrm{K}\sin\theta_l\cos\phi\right]
    \end{split}
\end{equation}

% After the third "folding", Eq(6) finally becomes:
% \begin{equation} \label{eq:S-fold3}
%   \begin{split}
% \frac{1}{\mathrm{d}\Gamma/\mathrm{d}q^2}\frac{\mathrm{d}^4\Gamma}{\mathrm{d}q^2 \mathrm{d}\cos\theta_l \mathrm{d}\cos\theta_\mathrm{K} \mathrm{d}\phi} =\frac{3}{4\pi}\left[2(F_\mathrm{S}\sin^2\theta_l+A_\mathrm{S}\sin^2\cos\theta_\mathrm{K})+\left.A^5_\mathrm{S}\sin\theta_\mathrm{K}\sin\theta_l(\cos\phi-\sin\phi)\right]
% \end{split}
% \end{equation}


\subsection{Formulas after transformation with both P-wave and S-wave components}
\label{sec:finalform}

After the two folding are performed, the angular distribution can be
written, using  Eq.~\ref{eq:fold2} and Eq.~\ref{eq:S-fold2} as:

\begin{equation} \label{eq:PDF-f2}
  \begin{split}
  \frac{1}{\mathrm{d}\Gamma/\mathrm{d}q^2}\frac{\mathrm{d}^4\Gamma}{\mathrm{d}q^2 \mathrm{d}\cos\theta_l \mathrm{d}\cos\theta_\mathrm{K} \mathrm{d}\phi} =&\frac{9}{8\pi}\left\{\frac{2}{3}\left[ (F_\mathrm{S}+A_\mathrm{S}\cos\theta_\mathrm{K})\left(1-\cos^2\theta_l\right) + A^5_\mathrm{S}\sqrt{1-\cos^2\theta_\mathrm{K}}\sqrt{1-\cos^2\theta_l}\cos\phi \right] \right.\\
 & + \left(1 - F_\mathrm{S}\right)\left[2F_\mathrm{L}\cos^2\theta_\mathrm{K}\left(1-\cos^2\theta_l\right)+\frac{1}{2}\left(1-F_\mathrm{L}\right)\left(1-\cos^2\theta_\mathrm{K}\right)\left(1+\cos^2\theta_l\right) \right.\\
 & + \frac{1}{2}P_1(1-F_\mathrm{L})(1-\cos^2\theta_\mathrm{K})(1-\cos^2\theta_l)\cos 2\phi \\
 & \left.\left. + 2P_5'\cos\theta_\mathrm{K}\sqrt{F_\mathrm{L}\left(1-F_\mathrm{L}\right)}\sqrt{1-\cos^2\theta_\mathrm{K}}\sqrt{1-\cos^2\theta_l}\cos\phi\right]\right\}
 \end{split}
\end{equation}

Now we have 6 parameters, they are $F_\mathrm{L}$, $F_S$, $P_1$, $P_5'$, $A_\mathrm{S}$
and $A^5_\mathrm{S}$.  In the six parameters, $F_\mathrm{L}$, $F_S$, and $A_S$ have been
measured by the BPH-13-010 analysis and we fix them in the
fitting if not specified otherwise in the following text. The new parameters are: $P_1$, $P_5'$, $A^5_\mathrm{S}$.

Due to limited number of events, if we fit all of the 6 parameters, the fit
maybe not in ''GOOD'' status. So we decided to get the value of $F_\mathrm{L}$, $F_S$, 
and $A_S$ from BPH-13-010, and we can fix them when fitting data.
We let also $F_\mathrm{L}$ float when statistics are big enough for validation purposes,
such as the control channels, as described in Section.\ref{sec:controlchannel}.

\subsection{Range of definition of interference terms}
\label{sec:As5.range}
Due to their nature, the value of the interference terms $A_s$ and $A_s^5$ is limited by the amplitude of the pure P-wave and S-wave components~\cite{Genon:Swave}. Their allowded ranges are the following:
\begin{equation} \label{eq:As.range}
  |A_s|<2\sqrt{3}\sqrt{F_S(1-F_S)F_L}*F_{theo}
\end{equation}
\begin{equation} \label{eq:As5.range}
  |A^5_s|<\sqrt{3}\sqrt{F_S(1-F_S)F_T(1+P_1)}*F_{theo}
\end{equation}
where $F_{theo}$ is a constant factor that depends on the selection cuts applied to the $K\pi$ system mass, and in this analysis is 0.89.

To make sure that the fitted value of $A_s^5$ is contained in this range, it has been substituted in the PDF by
\begin{equation} \label{eq:As5.subst}
  A^5_s\to f\sqrt{3}\sqrt{F_S(1-F_S)F_T(1+P_1)}*F_{theo}
\end{equation}
where $f$ is a placeholder parameter defined in the range [-1;1]. The fit is then performed with respect to $f$ instead of $A_s^5$.

\subsection{P.D.F validity in the parameter space}
\label{sec:phys.bound}
Using the P.D.F. parametrization described above, it is not guaranteed that it is physical (i.e. positive in the whole ($\cos\theta_K$,$\cos\theta_l$,$\phi$) space).

In order to have a working fit sequence and reliable results, we need to identify which values of the parameters allow the P.D.F. to be physical. Since, as explain in Sec.~\ref{sec:finalform}, the free parameters will be $P_1$, $P_5'$, $A^5_\mathrm{S}$, we will compute only their physical regions, keeping the other parameters fixed at the value from the BPH-13-010 analysis result.

This operation can be done analytically, probing the P.D.F. for some selected values of the angular variables. From this procedure it is possible to get that the P1 physical range is [-1,1]; but no boundary can be extracted for $P_5'$ and $A^5_\mathrm{S}$, for which a numerical computation is needed.

To compute numerically the boundary of the physical region, the $P_1/P_5'$ space has been scanned with a grid of step 0.01 in both directions. For each point of this grid, the values of $\cos\theta_K$, $\cos\theta_l$ and $\phi$ are moved on a 3D grid with step 0.02; if the PDF is positive for all of the points of this second grid, the point in the $P_1/P_5'$ space is inside the physical region, otherwise it is outside. The resulting region is equivalent to a $P_1$-dependent upper boundary for the absolute value of $P_5'$.

For each bin, this phisical region has been computed eight times:
\begin{itemize}
\item once, by requiring that only the P-wave component is positive; in this case the result is independent from the nuisance parameter $A_s^5$.
\item seven times, by requiring the whole PDF to be positive, for different values of $A_s^5$; according the convention defined in Sec.~\ref{sec:As5.range}, the seven values of the placeholder parameter $f$ are {-1, -2/3, -1/3, 0, 1/3, 2/3, 1}.
\end{itemize}

The boundaries of this regions, plotted in the negative $P_5'$ sector only (the boundaries are symmetrical with respect to $P_5'=0$), are shown from Figure~\ref{fig:bound0} to Figure~\ref{fig:bound8}.

\begin{figure}[!hbt]
  \centering
  \includegraphics[width=0.85\textwidth]{Figures/boundaries/bound_b0.pdf}
  \caption{Physical boundaries of the negative $P_5'$ sector of $q^2$ bin 0. Accordingly to the description in Sec.~\ref{sec:phys.bound}, the magenta line is the boundary of the P-wave physical region and the set of gray-scale lines are the boundaries of the total-PDF physical region, for different $A_s^5$ values (black for $f=-1$, lightest gray for $f=1$).}
  \label{fig:bound0}
\end{figure}

\begin{figure}[!hbt]
  \centering
  \includegraphics[width=0.85\textwidth]{Figures/boundaries/bound_b1.pdf}
  \caption{Physical boundaries of the negative $P_5'$ sector of $q^2$ bin 1. Accordingly to the description in Sec.~\ref{sec:phys.bound}, the magenta line is the boundary of the P-wave physical region and the set of gray-scale lines are the boundaries of the total-PDF physical region, for different $A_s^5$ values (black for $f=-1$, lightest gray for $f=1$).}
  \label{fig:bound1}
\end{figure}

\begin{figure}[!hbt]
  \centering
  \includegraphics[width=0.85\textwidth]{Figures/boundaries/bound_b2.pdf}
  \caption{Physical boundaries of the negative $P_5'$ sector of $q^2$ bin 2. Accordingly to the description in Sec.~\ref{sec:phys.bound}, the magenta line is the boundary of the P-wave physical region and the set of gray-scale lines are the boundaries of the total-PDF physical region, for different $A_s^5$ values (black for $f=-1$, lightest gray for $f=1$).}
  \label{fig:bound2}
\end{figure}

\begin{figure}[!hbt]
  \centering
  \includegraphics[width=0.85\textwidth]{Figures/boundaries/bound_b3.pdf}
  \caption{Physical boundaries of the negative $P_5'$ sector of $q^2$ bin 3. Accordingly to the description in Sec.~\ref{sec:phys.bound}, the magenta line is the boundary of the P-wave physical region and the set of gray-scale lines are the boundaries of the total-PDF physical region, for different $A_s^5$ values (black for $f=-1$, lightest gray for $f=1$).}
  \label{fig:bound3}
\end{figure}

\begin{figure}[!hbt]
  \centering
  \includegraphics[width=0.85\textwidth]{Figures/boundaries/bound_b5.pdf}
  \caption{Physical boundaries of the negative $P_5'$ sector of $q^2$ bin 5. Accordingly to the description in Sec.~\ref{sec:phys.bound}, the magenta line is the boundary of the P-wave physical region and the set of gray-scale lines are the boundaries of the total-PDF physical region, for different $A_s^5$ values (black for $f=-1$, lightest gray for $f=1$).}
  \label{fig:bound5}
\end{figure}

\begin{figure}[!hbt]
  \centering
  \includegraphics[width=0.85\textwidth]{Figures/boundaries/bound_b7.pdf}
  \caption{Physical boundaries of the negative $P_5'$ sector of $q^2$ bin 7. Accordingly to the description in Sec.~\ref{sec:phys.bound}, the magenta line is the boundary of the P-wave physical region and the set of gray-scale lines are the boundaries of the total-PDF physical region, for different $A_s^5$ values (black for $f=-1$, lightest gray for $f=1$).}
  \label{fig:bound7}
\end{figure}

\begin{figure}[!hbt]
  \centering
  \includegraphics[width=0.85\textwidth]{Figures/boundaries/bound_b8.pdf}
  \caption{Physical boundaries of the negative $P_5'$ sector of $q^2$ bin 8. Accordingly to the description in Sec.~\ref{sec:phys.bound}, the magenta line is the boundary of the P-wave physical region and the set of gray-scale lines are the boundaries of the total-PDF physical region, for different $A_s^5$ values (black for $f=-1$, lightest gray for $f=1$).}
  \label{fig:bound8}
\end{figure}


\subsection{Background shape descriptions from the data sidebands}
\label{sec:background shape}

After the transformation described in section~\ref{sec:transform} on
data, we can get the background shape from the data sidebands fitting,
which is described in Eq.~\ref{eq:PDF} as:
$$B_i^m \cdot B_i^{\cos\theta_\mathrm{K}}(\cos\theta_\mathrm{K}) \cdot
B_i^{\cos\theta_l}(\cos\theta_l) \cdot B_i^{\phi}(\phi) $$

The different components are described as follow:
$B_i^m$ is described as a single exponential with
translation term tied to the $S^R_i(m)$ mean, and
$B_i^{\cos\theta_\mathrm{K}}(\cos\theta_\mathrm{K}) \cdot
B_i^{\cos\theta_l}(\cos\theta_l) \cdot B_i^{\phi}(\phi)$ are different
degree polynomial depending on $q^2$ bin (in Table~\ref{tab:q2 bins}).
These functional forms are chosen to be good enough to describe the
data and to allow the fitting to converge easily. The details are
listed in the Table~\ref{tab:background shape}. The detail of background 
of $q^2$ bin 0 are from Figure~\ref{fig:bin0-bkg-l} to Figure~\ref{fig:bin0-bkg-phi}.
Other $q^2$ bins are in App.~\ref{sec:background}.

The hypothesis that the angular description of the background is factorisable for each variable has been tested. The results of this test are presented of App.~\ref{sec:app-bkg.fact}.

\begin{table*}[!htb]
    \begin{center}
        \begin{small}
            \caption{Mathematical description of the background parameterization for 
                data as a function for each $q^2$ bin in Table~\ref{tab:q2 bins}.
                \label{tab:background shape}}
            \begin{tabular}{|l|c|c|c|c|}
                \hline
                $q^2$ bin index   & $B_i^m$         & $B_i^{\cos\theta_\mathrm{K}}(\cos\theta_\mathrm{K})$    & $B_i^{\cos\theta_l}(\cos\theta_l)$  &  $B_i^{\phi}(\phi)$   \\
                \hline
                $ 0$    &  $ 1 exp $      &  $ 3^{rd} degree $  & $ 2^{nd} degree $ & $ 1^{st} degree $    \\
                $ 1 $   &  $ 1 exp$       &  $ 4^{th} degree$    & $ 2^{nd} degree $ & $ 1^{st} degree $\\
                $ 2$    &  $ 1 exp $      &  $ 4^{th} degree$    & $ 3^{rd} degree $ & $ 1^{st} degree $ \\
                $3 $    &  $ 1 exp$       &  $ 2^{nd} degree$   & $ 4^{th} degree $ & $ 1^{st} degree $   \\
                $ 5 $   &  $1 exp $      &   $ 4^{th} degree$   & $ 2^{nd} degree  $ & $ 1^{st} degree $\\
                $ 7 $   &  $1 exp$        &  $ 2^{nd} degree$   & $3^{rd} degree $  & $ 1^{st} degree$\\
                $ 8 $   &  $1 exp$        &  $ 2^{nd} degree$   & $2^{nd} degree$   & $1^{st} degree$\\
                \hline
            \end{tabular}
        \end{small}
    \end{center}
\end{table*}

\begin{figure}[!hbt]
  \centering
  \includegraphics[width=0.7\textwidth]{Figures/background/bin0-l.pdf}
  \caption{The data sidebands result of $q^2$ bins in the Table~\ref{tab:q2 bins} bin 0. The figure
    show the projections on $cos\theta_l$.}
  \label{fig:bin0-bkg-l}
\end{figure}


\begin{figure}[!hbt]
  \centering
  \includegraphics[width=0.7\textwidth]{Figures/background/bin0-k.pdf}
  \caption{The data sidebands result of $q^2$ bins in the Table~\ref{tab:q2 bins} bin 0. The figure
    show the projections on $cos\theta_\mathrm{K}$.}
  \label{fig:bin0-bkg-k}
\end{figure}

\begin{figure}[!hbt]
  \centering
  \includegraphics[width=0.7\textwidth]{Figures/background/bin0-phi.pdf}
  \caption{The data sidebands result of $q^2$ bins in the Table~\ref{tab:q2 bins} bin 0. The figure
    show the projections on $\phi$.}
  \label{fig:bin0-bkg-phi}
\end{figure}





