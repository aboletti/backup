\section{Introduction}

%%%%%%%%%%%%%%%%

%%%%%%%%%%%%%%55

%% The $q^2$ bins to be used in the analysis are defined in the Table~\ref{tab:q2 bins}.
%% They are chosen in such way to match the measurements performed in previous experiments.

%% Nine bins of $q^2$ are used in the analysis, including two which are dominated
%% by the control samples.
The $q^2$ range used in this analysis extends from 1\GeV$^2$ to 19\GeV$^2$ and it is divided in nine bins, defined in Table~\ref{tab:q2bins}.
The bin definition is the same used in the previous angular analysis on the same datset and it is the result of a compromise between being coherent with the definition used in the previous measurements and having an expected signal yield homogeneously distributed over the $q^2$ bins.

\begin{table}[!htb]
  \begin{center}
    \begin{small}
      \caption{Range definition of the dimuon invariant mass bins. Both $J/\psi$ and $\psi'$ regions, namely $q^2$ bins No.4 and No.6, are used as control channels.
        \label{tab:q2bins}}
      \begin{tabular}{c|l}
        Bin index & $q^2$ range ($Gev^2/c^4$) \\
        \hline
        0 & 1-2  \\
        1 & 2-4.3  \\
        2 & 4.3-6  \\
        3 & 6-8.68   \\
        4 & 8.68-10.09 ($J/\Psi$ region) \\
        5 & 10.09-12.86\\
        6 & 12.86-14.18 ($\Psi'$ region)\\
        7 & 14.18-16\\
        8 & 16-19\\
      \end{tabular}
    \end{small}
  \end{center}
\end{table}

The selection criteria and the analysis techniques are identical for any $q^2$ bin and in each of them the analysis is performed independently.
The two $q^2$ bins containing the control channel regions are not used to fit the signal events.

